\subsection{Présentation du problème}
	Les problèmes à N corps sont des problèmes physiques extrêmement complexes, mathématiquement: ils font partie d'une
	famille d'équation du second ordre qui n'ont pas de solution pouvant s'exprimer avec les fonctions mathématiques
	usuelles. Le seul cas connu pouvant s'écrire \og simplement\fg est le problème à 2 corps.

	Numériquement, ces problèmes sont très simple à implémenter. De façon basique, bête et méchante:
	\lstset{language=C, label=algo::NBodySimple, frame=shadowbox}
	\begin{lstlisting}
typedef struct {
    double x, y, z, vx, vy, vz, ax, ay, az, m;
} Point;

...

int main(void)
{
    int NbPart = 1e6;
    Point part[NbPart];
    ...
    //Some initialisation
    ...
    while(t < tmax && nbite < itemax)
    {
        for(int i=0; i<NbPart; i++)
        {
            part[i].ax  = part[i].ay  = part[i].az = 0;

            for(int j=0; j<NbPart; j++)
            {
                if( i != j )
                {
                    part[i].ax += -G * part[i].m * part[j].m
			* part[i].x / ( pow(sqrt( (part[i].x
			- part[j].x)*(part[i].x - part[j].x)
			+ (part[i].y - part[j].y)*(part[i].y
			- part[j].y) + (part[i].z - part[j].z)
			*(part[i].z - part[j].z)), 3.0) );
                    part[i].ay += -G * part[i].m * part[j].m
			* part[i].y / ( pow(sqrt( (part[i].x
			- part[j].x)*(part[i].x - part[j].x)
			+ (part[i].y - part[j].y)*(part[i].y
			- part[j].y) + (part[i].z - part[j].z)
			*(part[i].z - part[j].z)), 3.0) );
                    part[i].az += -G * part[i].m * part[j].m
			* part[i].z / ( pow(sqrt( (part[i].x
			- part[j].x)*(part[i].x - part[j].x)
			+ (part[i].y - part[j].y)*(part[i].y
			- part[j].y) + (part[i].z - part[j].z)
			*(part[i].z - part[j].z)), 3.0) );
                }
            }

            part[i].vx += part[i].ax * dt;
            part[i].vy += part[i].ay * dt;
            part[i].vz += part[i].az * dt;

            part[i].x += part[i].vx * dt;
            part[i].y += part[i].vy * dt;
            part[i].z += part[i].vz * dt;
        }
    }
}
	\end{lstlisting}
	Comme vous pouvez le voir, cette algorithme parcours 2 fois le tableau de particule. Par conséquent, le
programme effectue $\mathsf{NbPart}^2$ opérations. Quand le nombre de particules commence à augmenter, le temps de
calcul augmente: pour les ordinateurs actuels, calculer l'évolution d'un système tel un amas globulaire
($\mathsf{NbPart} \thickapprox 1e5 \to 1e6$) sur un temps dynamique (environ $10^6 \mathrm{ans}$) avec un pas de temps
$\mathsf{dt}$ d'environ $1e4 \mathrm{ans}$ prendrait plusieurs mois, voir années. Même en parallélisant le code  et en
le faisant tourner sur un super calculateur le temps de calcul resterait important. Nous avons donc besoin de trouver un
algorithme qui permette de faire ces calculs plus rapidement, mais sans perdre trop de précision sur les calculs. Cet
algorithme, c'est le Tree-Code développé par Barnes \& Hut dans~\cite{1986Natur.324..446B}.

\subsection{Présentation de l'algorithme}
	L'algorithme de Barnes et Hut est relativement simple: au lieu de travailler sur l'espace complet, nous le
séparons en $8$ cube, puis nous séparons à nouveau chacun de ces cubes en $8$ sous cubes, et ainsi de suite jusqu'à ce
que les cubes aient un certain nombre de particules à l'intérieur ou que l'arbre de cubes ainsi créé ait atteint un
certain niveau de raffinement.

	\begin{figure}[h]
		\begin{center}
			\newcommand{\TCarre}[2]{\coordinate (Xside) at (#1, 0);
	\coordinate (Yside) at (0 , #1);

	\coordinate (A) at ($ #2 -1/2*(Xside) - 1/2*(Yside) $);
	\coordinate (B) at ($ (A) + (Xside) $);
	\coordinate (C) at ($ (B) + (Yside) $);
	\coordinate (D) at ($ (C) - (Xside) $);

	\draw (A) -- (B) -- (C) -- (D) -- (A) -- cycle;
}
\begin{tikzpicture}
%	\coordinate (Xside) at (10, 0);
%	\coordinate (Yside) at (0 , 10);
%
%	\coordinate (A) at ($ (0,0) -1/2*(Xside) - 1/2*(Yside) $);
%	\coordinate (B) at ($ (A) + (Xside) $);
%	\coordinate (C) at ($ (B) + (Yside) $);
%	\coordinate (D) at ($ (C) - (Xside) $);
%
%	\draw (A) -- (B) -- (C) -- (D) -- (A) -- cycle;

	\coordinate (P1) at (1.5,2);
	\coordinate (P2) at ($ 5*(rand, rand) - 5*(1,1) $);
	\coordinate (P3) at ($ 5*(rand,rand) - 5*(1,1) $);

	\TCarre{10}{(0,0)}

\end{tikzpicture}

		\end{center}
	\end{figure}
