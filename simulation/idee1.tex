La première idée qu'il nous est venu pour tester le scénario décrit dans le
chapitre~\ref{Sec::ToyModel}, était de prendre un objet suivant la fonction de
distribution de la SIK puis de le placer dans un cube homogène faisant office de
bain thermique.

Le premier souci apparent est que le bain va influer sur la SIK en lui donnant
des particules, et cette dernière va déstabiliser le bain, le poussant à
s'effondrer. Pour palier à ce problème, nous avons utilisé une option de
\textsc{GADGET} qui permet de désactiver, pour un ou plusieurs types de
particules, les interactions gravitationnelles qu'elles subissent.
Ainsi, en activant cette option pour les particules de type 4, par exemple,
elles influencerons les autres particules, mais elles-mêmes se déplacerons en
ligne droite, ne ressentant pas les autres particules.

METTRE QUELQUES GRAPHES

Ces simulations ont échoué parce que le bain n'avait aucune influence sur la
SIK.
