Pour obtenir nos conditions initiales, qui devront suivre un profil de \textsc{King}, nous allons utiliser la méthode de
réjection, l'une des plus facile à mettre en place. Nous allons tirer
aléatoirement la position et la vitesse des particules dans une boîte,
puis nous n'en garderons qu'une partie en utilisant la fonction de distribution
comme densité de probabilité.
Pour commencer, nous utilisons les limites de notre modèle pour restreindre
nos tirages à une boîte de taille \mbox{$\left[ - r_{\mathrm{max}}; r_{\mathrm{max}} \right]$} pour
les distances et \mbox{$\left[ -v_{\mathrm{max}}; v_{\mathrm{max}}\right]$} pour les vitesses :
\begin{itemize}
	\item vitesse maximum : l'énergie totale du système est bornée supérieurement par l'énergie de libération :
		\begin{align}
			E = \dfrac{1}{2} m v_i^2 + m\psi(r) &< E_l \notag \\
			E_l - m\psi(r) &> \dfrac{1}{2} m v_i^2 \notag \\
			v_\mathrm{max}^2 = 2\(\dfrac{E_l}{m} - \psi(r)\) &> v_i^2 \notag \\
			v_{\mathrm{max}} = \sqrt{2\(\dfrac{E_l}{m} - \psi(0)\)} > v_i &> - \sqrt{2\(\dfrac{E_l}{m} - \psi(0)\)} = - v_{\mathrm{max}}
		\end{align}
	\item distance maximum : cette distance est obtenue pour
		$m\psi(r_{\mathrm{max}}) = E_l$. Il nous faut donc connaître le potentiel.
\end{itemize}

Le potentiel n'ayant pas d'expression analytique, nous allons devoir réutiliser
notre algorithme de résolution numérique utilisé dans les chapitres précédents pour
modèliser un King.

Il nous faut aussi pouvoir redimensionner les quantités obtenues. Pour cela, le
programme récupère dans un fichier de configuration les dispersions de vitesse
$\sigma_v^2$, rayon de c\oe ur $r_c$, temps de relaxation $T_c$ et distance au
soleil $R_\odot$ dans les unités du catalogue de \textsc{Harris}. Toutes sont
ensuite transformées en unités SI (~mètre, kilogramme, seconde~). Comme nous
avons pu le voir dans le chapitre~\ref{King::Chapitre} traitant du modèle de
King, la masse $m$ d'une particule n'influence pas le profil de densité final :
\begin{align}
	r_c^2 &= \dfrac{\sigma^2}{4\pi G m \rho_0} \notag \\
	      &= \dfrac{(\sigma_v^2)^2 m}{8\pi G m \rho_0} \notag \\
	      &= \dfrac{(\sigma_v^2)^2}{8\pi G \rho_0} \notag \\
	\Rightarrow \rho_0 &= \dfrac{(\sigma_v^2)^2}{8\pi G r_c^2}
\end{align}
Pour redimensionner la densité, nous n'avons donc pas besoin de connaître la
masse d'une particule. Ce constat nous permet de laisser ce paramètre libre et
de jouer sur le nombre de particules dans le système. En effet, une fois la
densité obtenue, nous pouvons l'intégrer sur le volume de l'amas pour trouver la
masse totale de ce dernier, puis, connaissant le nombre de particules, en déduire
la masse d'une étoile par la relation :
\begin{align}
	m = \dfrac{M_{tot}}{\text{Nombre de particules}}
\end{align}
Déduire le reste des paramètres utiles pour le redimensionnement est ensuite
assez simple.
La distance maximum $r_{\mathrm{max}}$ est déduite de la résolution
numérique des équations.

Pour pourvoir tout redimensionner, nous avons aussi besoin de connaître l'énergie
de libération de l'amas. Pour cela, nous utilisons le théorème de \textsc{Gauss}
et un petit raisonnement simple. Par définition, nous avons :
\begin{align}
	E_\mathrm{min} < E < E_l < 0 \ &\text{ et } \ E_l = \frac{p^2}{2 m} + m \psi(r)
	\intertext{soit :}
	\psi(r) = \frac{1}{m}\(E_l - \frac{p^2}{2 m}\)
	\intertext{La valeur maximale du potentiel est donc atteinte pour $p = 0$ :}
	\psi_\mathrm{max} = \psi(R) = \frac{E_l}{m}
	\intertext{Hors du système, le théorème de \textsc{Gauss} nous dit qu'il peut être vu comme une particule ponctuelle de masse $M$. Le potentiel hors de l'amas s'écrit donc :}
	\psi(r) = -\frac{G M}{r}
	\intertext{Par continuité, nous avons :}
	\psi(R) = \frac{E_l}{m} = -\frac{G M}{R}
\end{align}

Pour générer des nombres aléatoires dans l'intervalle voulu, nous utilisons la
fonction \verb|double ran2(long seed)| tiré de~\cite{NumericalRecipesC}, dont
nous nous servont ainsi :
\lstset{language=C, label=algo::tirage, frame=shadowbox}
\begin{lstlisting}
	double  x  = rmax - 2.0 * rmax * ran2(seed),
		y  = rmax - 2.0 * rmax * ran2(seed),
		z  = rmax - 2.0 * rmax * ran2(seed),
		vx = vmax - 2.0 * vmax * ran2(seed),
		vy = vmax - 2.0 * vmax * ran2(seed),
		vz = vmax - 2.0 * vmax * ran2(seed);
\end{lstlisting}
Notre système étant sphérique, nous ne devons pas avoir de vitesse et de module
de distance supérieurs, respectivement, à $v_{\mathrm{max}}$ et
$r_{\mathrm{max}}$, de plus nous avons une probabilité
\mbox{$f(E)/f(E_\mathrm{min})$}
d'avoir une particule d'énergie $E$. Cette énergie minimale est l'énergie
potentielle d'une particule au centre de l'amas, et de vitesse nulle.
Une fois qu'une particule respecte ces conditions et que \og les probabilités sont
avec elle \fg, nous l'enregistrons.

Le programme écrit au fur et à mesure les coordonnées cartésiennes et vitesses
des particules selectionnées dans un fichier dans les unités standards : mètre
pour les distances et mètre par seconde pour les vitesses.

