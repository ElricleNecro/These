Le code que nous utilisons pour faire évoluer notre système est \textsc{Gadget-2}~\footnote{récupérable sur \url{http://www.mpa-garching.mpg.de/gadget/}}~\cite{gadget2}, écrit par Volker \textsc{Springel}.
Seule une petite partie de ce qu'il fait nous intéresse : nous n'allons utiliser que les options concernant le \og~tree code~\fg.
Toutes les autres options ne concernent que les simulations cosmologiques ou pourraient
induire un comportement du code qui nous ferait perdre de la précision sur les calculs.

Pour fonctionner, \textsc{Gadget-2} a besoin d'un fichier de configuration, dans lequel nous devrons jouer sur certains paramètres, et d'un fichier de conditions initiales respectant un format précis.

\subsection{Fichier de conditions initiales}

	Le fichier de conditions initiales doit avoir le format suivant :
	\begin{enumerate}
		\item un en-tête contenant le nombre de particule de chaque type (~Gaz, Halo, Disque, Bulbe, Étoiles, Bndry~), la masse pour chaque type, divers autres informations utiles essentiellement aux simulations cosmologiques,
		\item les positions de chaque particules,
		\item leurs vitesses,
		\item un identifiant permettant de repérer chaque particule.
	\end{enumerate}
	Chaque bloc devant être encadré par sa taille en mémoire.

\subsection{Fichier de configuration}

	Dans ce fichier, nous n'allons jouer que sur certains paramètres :
	\begin{itemize}
		\item \verb|OmegaLambda| : paramètre cosmologique représentant la densité d'énergie du vide, en le mettant à 0, nous faisons savoir à \textsc{Gadget-2} que nous ne faisons pas de simulation cosmologique.
		\item \verb|UnitLength_in_cm|, \verb|UnitMass_in_g| et \verb|UnitVelocity_in_cm_per_s| sont les unités dans lesquelles sont données, respectivement, les positions, masses et vitesses des particules en
			centimètre, gramme en centimètre par seconde. Ce sont ces facteurs de conversion qui donne l'unité de temps interne à \textsc{Gadget-2}. Nous utilisons les parsecs (~$ 1 pc = 3.086 \times 10^{18} cm$~)
			pour les positions, les kilogrammes (~$1 kg = 1000 g$~) pour la masse, et les mètres par seconde (~$ 1 m.s^{-1} = 10^2 cm.s^{-1}$~) pour les vitesses. Ces unités nous donnent comme unité de temps
			interne :
			\begin{align}
				v &= \frac{d}{t} \notag \\
				t &= \frac{d}{v} \notag \\
				t &= \frac{3.086 \times 10^{18}}{10^2} = 3.086 \times 10^{16} s \notag \\
				t &= 9.77894 \times 10^8 ans
			\end{align}
		\item \verb|SofteningStarsMaxPhys| : paramètre de lissage de la force, permettant d'éviter qu'elle \og~n'explose~\fg~à cause d'une collision entre 2 particules trop proches.
			C'est sur ce paramètre qu'il faut jouer pour assurer la stabilité du système sur un grand nombre de temps dynamiques.
		\item \verb|ErrTolTheta| : représente l'angle d'ouverture, ou résolution angulaire, minimum. Celui-ci est fixé à $0.5$ et n'est plus modifié ensuite.
	\end{itemize}

\subsection{Lissage de la force}
	Nous souhaitons nous placer dans la limite fluide afin de minimiser les effets de relaxations dû aux collisions à deux corps. Pour cela, nous pouvons jouer sur deux paramètres : le nombre de particules que nous mettons dans
	le système, et le paramètre de lissage. Étant limité en temps, nous ne pouvons pas lancer de simulations avec un très grand nombre de corps : les plus grosses simulations que nous lançons ont 100 000 particules, et
	occasionnellement 500 000. Pour une évolution pendant 100 temps dynamiques, elles prennent environ 709 minutes en les faisant tourner sur 8 cœurs, et cela peut aller jusqu'à 1020 minutes.

	Le lissage est une borne, en distance, en deçà de laquelle nous considérons un potentiel minimum valant :
	\begin{align}
		\psi(r_{i} \sim 0) = - G \dfrac{m_i}{r_{i} + \epsilon}
	\end{align}
	où $r_i$ est le module de distance d'une particule numérotée $i$. $\epsilon$ est le paramètre de lissage de la force. Afin de nous placer dans la limite fluide, $\epsilon$ doit être tel qu'il contienne un
	nombre $N_\epsilon$ de particule grand devant 1 mais petit devant le nombre total de particule du système. Habituellement, c'est $N_\epsilon$ qui est choisi, imposant ainsi $\epsilon$, mais nous avons choisi $\epsilon$
	imposant ainsi $N_\epsilon$.

%	Dans un amas, la densité est plus forte au centre que sur le bord. Les effets de relaxation sont donc plus important au centre, le $\epsilon$ doit y minimiser ces effets, même s'il n'a aucun effet sur les bords de l'amas.
	À cause d'instabilité, il est important de décrire la dynamique au centre de l'amas avec la meilleur précision possible. Il est donc nécessaire d'être dans la limite fluide au centre.
	Il est alors intéressant de le relier à la densité centrale, plutôt que d'utiliser une distance inter-particulaire moyenne. Nous définissons donc $\alpha$ tel que :
	\begin{align}
		\alpha = \dfrac{\rho(0)}{\rho_\mathrm{moy}} = \dfrac{\rho(0)}{M} \frac{4}{3} \pi R^3
	\end{align}
	avec $M$ la masse totale de l'amas, $R$ le rayon de l'amas, et $\rho(0)$ la densité centrale de l'amas. La densité d'un volume de taille $\epsilon$ et la densité de l'amas nous donne donc :
	\begin{align}
		\rho_\epsilon &= \alpha \rho_\mathrm{moy}					\notag \\
		N_\epsilon    &= \alpha \frac{4}{3} \pi \epsilon^3 \dfrac{3N}{4\pi R^3}	\notag \\
			      &= \alpha N \( \dfrac{\epsilon}{R} \)^{3}
	\end{align}
	$N$ étant le nombre total de particules de l'amas. $N$, $R$ et $\alpha$ étant fixés, nous sommes ainsi libres de choisir $N_\epsilon$ ou $\epsilon$ pour répondre à nos besoins.

