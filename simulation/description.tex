\chapter{Expériences numériques}
	\minitoc

% Nous allons maintenant nous concentrer sur notre problème de bain (et de baignoire qui fuit). Lors de la recherche de
% comment mettre notre bain en place, nous avons rencontré plusieurs situation.
% Quid:
% \begin{itemize}
% \item Premiers test de bain (boîte sans gravitation, sphère de King...),
% \item Premières conditions semblant cohérente,
% \item Résultat donnés par ces conditions.
% \end{itemize}

Après avoir étudié les propriétés observationnelles des amas globulaires et des galaxies, nous avons mis en évidence un certain nombre
de résultats analytiques sur diverses sphères isothermes. Nous allons maintenant entreprendre des simulations numériques visant
à illustrer ces divers résultats et constats.

Les expériences que nous allons mener consistent à étudier la dynamique d'un système auto-gravitant (\textsc{sag}) placé
dans un bain thermique. Ces conditions respectent celles du problème détaillé dans la section~\ref{Sec::ToyModel}.
% respectant ainsi les conditions du problème détaillé dans la section~\ref{Sec::ToyModel}.
%Nous avons testé plusieurs idées pour mettre en place les conditions décrite dans la section~\ref{Sec::ToyModel}.


Afin de préserver l'intégrité de nos différents systèmes, nous avons imposé aux particules du bain et du \textsc{sag} d'avoir la
même masse.

\section{Description des conditions initiales}

% \subsection{Les différents types de bain}

	Nous avons utilisé principalement deux types de bain:
	\begin{description}

		\item[Un thermostat:] ce type de bain est construit de sorte à se comporter comme un
			thermostat. Il s'agit d'un cube périodique, dans lequel les particules massives sont
			réparties spatialement selon une distribution uniforme. La température est fixée par la
			distribution gaussienne attribuée aux vitesses. Un thermostat maintient une
			température sans être affecté par le système avec lequel il est en contact.
			% pas supposé échangé autre chose que de l'énergie avec les systèmes en contact,
			Nous faisons donc en sorte que les particules de ce bain puissent influer sur les
			particules du \textsc{sag} sans que celles-ci n'affectent celles du bain en utilisant
			une option de Gadget.
			% , mais ces dernières n'ont aucun effet sur le thermostat.

		\item[Un réservoir thermique:] il s'agit d'un système composé de particules massives réparties
			uniformément dans un cube périodique, et dont la température est aussi fixée par une
			distribution gaussienne des vitesses. Contrairement au thermostat, les particules du
			réservoir ressentent l'interaction gravitationnelle des particules du \textsc{sag}.
		% \item[Un cube homogéne:] il s'agit d'une variante du bain précédant où les particules vont pouvoir subir
			% l'influence du \textsc{sag} placé en leur sein.

		% \item[Une sphère isotherme:] plus précisément, nous allons prendre un modèle de King qui, comme
			% nous l'avons vu dans le chapitre~\ref{King::Chapitre}, est isotherme. Il est
			% donc possible de l'utiliser pour créer un bain thermique, notamment en plaçant le
			% \textsc{sag} au sein de son cœur.

	\end{description}

	Les paramètres du réservoir et du thermostat sont les suivants:
	\begin{itemize}
		\item le nombre de particules $N_b$;
		\item la température $T$;
		\item le côté du cube $R_c$.
	\end{itemize}
	Ces paramètres sont indépendants et constituent une partie de l'espace des paramètres de nos
	simulations. La stabilité du réservoir thermique va dépendre des valeurs de ces paramètres, notamment à
	travers le critère de Jeans (voir la section~\ref{Chap::Instabilite::Sec::Jeans}).

	\begin{figure}[hbt]
		\begin{center}
			\begin{tikzpicture}
				\draw[pattern=north east lines] (-2, 2) rectangle (2, -2);
				% \draw (-2, 2) grid[step=0.2] (2, -2);

				\draw[fill=red!70,opacity=0.5] (0, 0) circle (0.5);
				\node[fill=white,opacity=0.8] at (0, 0) {\textsc{sag}};

				\node at (-2, 2) [below right, fill=white,opacity=0.8] {Bain};
				\node at (-2, -2) [above right, fill=white,opacity=0.8] {$T$, $N$};
				\draw[<->] (-2, -2.2) -- (2, -2.2);
				\node at (0, -2.2) [below] {$R_c$};
			\end{tikzpicture}
			\caption{Condition des simulations.\label{Fig::CI::Repr}}
		\end{center}
	\end{figure}

	L'effet d'un thermostat sur le \textsc{sag} ne peut a priori se faire qu'aux travers des collisions~\footnote{par collisions, nous parlons des
	interactions à deux corps.}, les particules du thermostat ne pouvant être capturées par le \textsc{sag}. L'effet correspondant ne peut alors
	apparaître que sur des temps grands devant le temps dynamique du \textsc{sag} (voir-- mettre le lien vers la discussion correspondante, une
	fois écrite). Un réservoir thermique sera a priori plus efficace, les interactions gravitationnelles pouvant amener une partie des particules
	du réservoir à être capturées par le \textsc{sag}, ou réciproquement. Des effets peuvent donc apparaître plus rapidement.

	Nous avons utilisé deux types de \textsc{sag}:
	\begin{itemize}
		\item un modèle de King, dont les paramètres (définis dans le chapitre~\ref{King::Chapitre}) sont $W_0$, $\sigma$, $r_c$ et le nombre
			de particules $N_K$;
		\item une sphère de Hénon, de rayon initial $R$, de masse $M$, de viriel initial $\gamma$ et contenant $N_H$ particules.
	\end{itemize}

	La figure~\ref{Fig::CI::Repr} représente la répartition initiale du bain et du \textsc{sag}. À ce moment, des particules du bain se trouvent déjà dans le \textsc{sag}.

% \subsection{Réalisation du système}

\section{Étude préliminaire}

% \subsection{King dans bain sans interaction}

	Une première série de simulations a consisté à placer un modèle de King, de paramètre $W_0 = 5.2$, $\sigma_c = 2.9\
	\mathrm{km}.\mathrm{s}^{-1}$ et $r_c = 3.5\ \mathrm{pc}$, dans un thermostat. Les paramètres de chaque simulations étaient la
	température du bain, la taille de la boîte et le nombre de particule utilisées dans chacun des deux systèmes. La taille de la boîte
	nous permet de jouer sur la densité du bain. La température était calculée pour être un multiple $k$ de la température moyenne de la
	sphère de King initiale. Nous avons fait varier $k$ dans l'intervalle $10^{-3}$ à $10^5$. Chaque configuration initiale a évolué
	sur des périodes de temps allant de $10T_d$ à $120T_d$, où $T_d$ est le temps dynamique de la sphère de King initiale (voir le
	chapitre~\ref{Chap::TempsCarac})
	% de $10^{-3}T_\mathrm{Bord}$ à $10^5 T_\mathrm{Centre}$.
	La densité du \textsc{sag} n'a jamais montré de véritable évolution, ni les autres paramètres.

	PLACER QUELQUES GRAPHES.

	% Ceci est dû au fait que le bain ne peut changer la dispersion de
	% vitesse des particules du King que lors de collision à plusieurs corps. Or ces collisions sont assez
	% longue à agir.

	% Une première représentation du système consiste à placer un modèle de King dans un thermostat. Les
	% paramètres du King ont été fixé à $W_0 = 5.19773$, $\sigma_c = 2.9\ \mathrm{km}.\mathrm{s}^{-1}$ et $r_c
	% = 3.5\ \mathrm{pc}$. Les paramètres de chaque simulations
	% étaient la température du bain, la taille de la boîte et le nombre de particule utilisé dans chacun des
	% deux systèmes. La taille de la boîte nous permet de jouer sur la densité du bain. La température était
	% calculer pour être un multiple soit de la température du cœur de la sphère de King soit un multiple de
	% sa température moyenne. Nous avons parcouru des température allant de $10^{-3}T_\mathrm{Bord}$ à $10^5
	% T_\mathrm{Centre}$. La densité n'a jamais montré de véritable évolution, ni les autres paramètres.
	% Ceci est dû au fait que le bain ne peut changer la dispersion de vitesse des particules du King que lors
	% de collision à plusieurs corps. Or ces collisions sont assez longue à agir.

	% La première idée que nous avons testé tentait de simuler un véritable bain thermique. Nous placions un
	% modèle de King dans une boîte périodique rempli de particule répartie de façon homogène et placé à une
	% certaine température.
	% Par \og simuler un véritable bain thermique\fg nous entendons que les particules du bain ne peuvent pas
	% se faire capturer par le Kingi, mais elles peuvent modifier la vitesse des particules de ce dernier.
	% % C'est-à-dire que le bain pouvait influer sur la vitesse des
	% Au final, ces simulations n'ont jamais rien donnée car...

% \subsection{King dans un King}

	% Une seconde représentation consiste à placer un modèle de King dans une sphère isotherme qui fera office de
	% bain. Nous faisons en sorte que le King soit contenu dans le cœur du bain, dont les paramètres ont été
	% choisi de sorte à ce que le cœur soit complètement isotherme. Un problème est très vite arrivé: pour que
	% le bain ait un cœur suffisamment grand pour contenir le \textsc{sag}, il fallait qu'il ait une taille
	% tellement grande que le nombre de particule nécessaire pour avoir quelque chose de correct était de
	% l'ordre de $10^{10}$ particules.

	% Nos pérégrination nous ont ensuite mené à tenter de simuler le bain non plus par un cube, mais par une
	% autre sphère isotherme, en l'occurrence un autre King. L'idée était de générer un King dont le cœur
	% était suffisamment grand pour y faire tenir un autre King, et le mettre à la bonne température. Le gros
	% souci auquel nous avons eu à faire, c'est que le King devant servir de bain se trouvait être très dilué.

% \subsection{Hénon dans un King}

	% Une troisième approche nous est venu en se disant que, si le king n'évoluait pas, c'est parce qu'il
	% était trop stable, ou que son temps dynamique était trop grand. Nous somme donc passé sur une autre
	% catégorie de \textsc{sag}: ceux formés suite à l'effondrement d'une sphère de Hénon. Il s'agit donc ici
	% de remplacer nos modèle de King par une sphère de Hénon qui, après effondrement, va devenir un cœur-halo
	% avec une pente de $-4$. Les conditions que nous imposons pour ces simulations font que le bain doit
	% avoir une température égale ou inférieur au \textsc{sag}. Or, le \textsc{ch4} était trop froid pour que
	% nous puissions mettre le King à la bonne température.

	% Après les résultats décevant des CI précédentes, nous nous sommes dit que le King était peut-être trop
	% stable, et du coups l'instabilité se produit après des temps trop long. Nous avons donc choisi de donner
	% sa chance à la sphère de Hénon en espérant que le premier effondrement (dû à l'instabilité de Jeans)
	% permette l'apparition rapide de l'instabilité souhaité. Pour diverses raisons dû à la température, nous
	% avons abandonné cette idée.

% \subsection{Hénon dans un cube}

	% La dernière approche à laquelle nous ayons pensé est de rester avec un Hénon, mais cette fois placé dans
	% un cube homogène. Le bain interagissant pleinement avec le \textsc{sag}, ce dernier va pouvoir accréter
	% ou perdre de la masse en interagissant avec le bain, accélérant par là son évolution. Nous avons pu
	% observer plusieurs phénomènes intéressant que nous allons détailler dans la prochaine section.

	% Une autre approche a été développé, consistant à remplacer le thermostat par un réservoir thermique.
	% Nous allons aussi remplacer le modèle de King par une sphère de Hénon.
	% C'est sur cet ensemble de simulations que nous allons nous concentrer dans la suite.

	Face à ce constat, nous avons augmenté la sensibilité de nos simulations en remplaçant le thermostat par un réservoir thermique et la
	sphère de King par une sphère de Hénon.

	% Nous avons finalement décidé de revenir à l'idée initiale, mais avec deux changements majeurs: cette
	% fois le bain va subir l'influence de la gravitation, et nous avons remplacé la \textsc{sik} par une
	% sphère de Hénon. Pourquoi ce changement pour le bain? Parce que nous avons réalisé que le meilleurs
	% moyen de jouer sur la température de notre \textsc{si} est de la laisser acquérir des particules du bain
	% afin de changer la profondeur de son puits de potentiel et donc sa température.
	% Cette configuration nous a permis de mettre en évidence plusieurs évolutions intéressante, mais toujours
	% pas l'instabilité que nous cherchions.

	% Nous allons détailler dans les sections suivantes les conditions de ce set-up et les premiers résultats
	% obtenu ainsi.

\section{Seconde étude}

% \subsection{Comment elles sont mise en place}

	% Pour commencer, nous nous plaçons dans le même système d'unité que celui du
	% chapitre~\ref{Chap::VlasovGadget}. Une première étape est de chercher quels contrainte nous voulons
	% imposer aux différents objet de la simulation. La première étape est bien entendu de s'assurer que les
	% conditions périodiques n'influe pas trop sur l'évolution. Ensuite, nous aimerions imposer aussi la taille de
	% la boîte de sorte que le bain soit stable. Il se trouve que ces deux conditions sont parfois
	% incompatible, la boîte devant être petite pour assurer la stabilité mais pas assez grande pour éviter
	% les soucis inhérents aux conditions périodiques. D'autant que nous devons conserver la température du
	% bain inférieur à celle de la sphère de Hénon une fois effondré.

	% Nous avons fini par choisir de faire sauter la conditions sur la taille de la boîte, ce qui nous fait
	% gagner un paramètre en plus sur lequel jouer plus facilement. Nous avons ensuite lancé un grand nombre
	% de simulations pour parcours une partie de l'espace des paramètres et voir ce qu'il se passe.

	Une seconde série de simulations est construite en utilisant une sphère Hénon de rayon $R=2$, de masse $M=1$, de viriel $\gamma=-0,5$ et
	contenant $N_H = 30000$ placée dans un réservoir thermique. Nous utiliserons ici le même système d'unité que dans le
	chapitre~\ref{Chap::VlasovGadget}, dans lesquelles le temps dynamique du \textsc{sag} est $T_d^H = 2\pi$. De la même façon que lors de l'étude
	précédente, nous allons utiliser les caractéristiques du réservoir comme paramètres des simulations: le nombre de particules $N_b$, la taille
	du cube $R_c$ et sa température $\sigma_c$. Globalement l'évolution dynamique de ce système montre toujours deux phases successives.

	Dans un premier temps, la sphère de Hénon s'effondre et forme, comme attendu pour $\gamma=-0,5$ (\citet{roy}, \citet{Joyceetal}), une
	structure cœur-halo de pente $\alpha\simeq-4$ (\textsc{ch4}) à l'équilibre ($\gamma=-1$). La durée de cette phase initiale est de l'ordre de
	quelques temps dynamiques de la sphère de Hénon initiale. Cette phase d'effondrement n'est pratiquement pas affectée par les paramètres du
	réservoir. L'intersection des deux systèmes (bain et \textsc{sag}) n'étant pas vide, le nombre de particules du bain contenu dans le
	\textsc{sag} est variable, ce qui explique la légère variation observée des paramètres post-effondrement. La
	figure~\ref{Fig::Simu::EvoParamPostCollapse} illustre l'évolution, après effondrement, des différents rayons, axes d'inertie, viriel,
	température et anisotropie en fonction des paramètres du réservoir thermique. Ces valeurs sont calculées en faisant la moyenne de chaque
	paramètres dans l'intervalle $[1T_d, 1,5T_d]$. Cette phase d'effondrement est la traduction de l'instabilité de Jeans.

	% par le nombre de particules du bain (l'intersection des deux système n'étant pas vide). Globalement, le premier effondrement du Hénon, sous
	% l'influence de l'instabilité de Jeans, se déroule sans aucun effet dû à la présence du Bain.

	L'évolution dynamique post-effondrement de cette structure de type \textsc{ch4} est alors influencée par le bain. Cette influence est modulée
	par les paramètres du réservoir thermique. Nous avons effectué plus de 300 simulations dans l'espaces des paramètres, la
	table~\ref{Tab::SimuZoomRes} résume l'évolution d'une partie représentative de ces simulations.
	% Les notations utilisées sont les suivantes:
	Les observables étudiées sont celles définies dans le chapitre~\ref{Chap::Simu::Analysis}, nous avons regroupé leur évolution dynamique en
	différents classes associées aux symboles suivants:

	\begin{itemize}

		\item les variations des observables $R_{10}$, $R_{50}$ et $R_{90}$ peuvent être regroupées en 3 classes:
			\begin{itemize}

				\item[\accretionpeu{}] $R_{10}$ et $R_{50}$ se diminue (de $5\%$ à $40\%$): le cœur du \textsc{sag} se
					contracte. $R_{90}$ varie de $-15\%$ à $60\%$.

				\item[\accretionmoyen{}] indique une accrétion modérée du bain par le \textsc{sag}. Le rayon $R_{10}$ continue
					à décroitre (de l'ordre de $20\%$) alors que $R_{50}$ et $R_{90}$ croissent. La variation de $R_{50}$ peut
					attendre les $30\%$ tandis que $R_{90}$ peut aller jusqu'à doubler.

				\item[\accretionlot{}] indique une forte accrétion du bain par le \textsc{sag}. Tous les rayons caractéristique sont
					au minimum doublé sur l'évolution dynamique du système.
			\end{itemize}

		\item Les variations des rapports d'axes mettent en évidence 3 évolutions:
			\begin{itemize}

				\item[$\diamondsuit$] est un cas particulier qui sera détaillé un peu plus loin.

				\item[$\spadesuit$] apparition d'une déformation du système au cours de laquelle $a_1 \simeq 1$ et $a_2 \to 0,4$.

				\item[$\emptyset$] le système conserve sa forme sphérique.

			\end{itemize}
	\end{itemize}

	% \begin{itemize}
		% \item[\textbf{les notations}] sont les mêmes que dans le chapitre~\ref{Chap::Simu::Analysis}.
		% \item[$\diamondsuit$] indique une triaxialisation du système.
		% \item[\accretionpeu{}] correspond à la variation suivante des différents rayons: $\Delta \ra<0\%$, $\Delta \rb<0\%$, $\Delta
			% \rc<100\%$. Dans ces conditions, les rayons à $10\%$ et $50\%$ diminue: le \textsc{sag} se contracte;
		% \item[\accretionmoyen{}]  indique une accrétion "moyenne" du bain par le \textsc{sag} et correspond à la variation suivante des
			% différents rayons: $\Delta \ra<0\%$, $0\%<\Delta \rb<100\%$, $\Delta \rc<100\%$. À ce niveau, le \textsc{sag} accréte le bain
			% de façon suffisamment significative pour le destabiliser, et se dilate.
		% \item[\accretionlot{}] indique une forte accrétion du bain par le \textsc{sag} et correspond à la variation suivante des différents
			% rayons: $\Delta \ra>200\%$, $\Delta \rb>200\%$, $\Delta \rc>200\%$. L'accrétion est encore plus forte que dans le cas
			% précédent. Dans ce cas, le bain a été en grande partie cannibalisé par le \textsc{sag}.
		% \item[$\emptyset$] indique que le ou les paramètres n'ont pas évolué de manière significative.
	% \end{itemize}
	% $a_1 = \lambda_1 / \lambda_2$, $a_2 = \lambda_3 / \lambda_2$; $\lambda_1>\lambda_2>\lambda_3$. Accrétion:
	% \begin{description}
		% % \item[\contraction{}] -> $\Delta r_{10\%}<0$, $\Delta r_{50\%}<0$, $\Delta r_{90\%}<0$,
		% \item[\accretionpeu{}] indique un "effondrement" de l'objet et correspond à la variation suivante des différents rayons: $\Delta \ra<0\%$, $\Delta \rb<0\%$, $\Delta \rc<0\%$,
		% \item[\accretionmoyen{}]  indique un "effondrement" de l'objet et correspond à la variation suivante des différents rayons: $0\%<\Delta \ra<200\%$, $0\%<\Delta \rb<200\%$, $0\%<\Delta \rc<200\%$,
		% \item[\accretionlot{}] -> $\Delta \ra>200\%$, $\Delta \rb>200\%$, $\Delta \rc>200\%$.
	% \end{description}

	% Ensuite, les résultats sont assez variables. Toutes les simulations montrent un réchauffement du
	% Hénon, mais relativement peu indiquent une évolution des autres paramètres, tel l'anisotropie où les axes d'inertie. La
	% table~\ref{Tab::SimuZoomRes} montre un résumé de certaines de ces simulations.

	\begin{figure}[htbp]
		\centering \rotatebox{90}{
			\includegraphics[scale=0.8]{graphe/post-collapse.pdf}
		}
		\caption{Évolution des différentes observable après l'effondrement en fonction des paramètres du bain thermique.\label{Fig::Simu::EvoParamPostCollapse}}
	\end{figure}

	\begin{table}[htbp]
		% \begin{tabular}{|m{1.5cm}|m{1cm}|m{2cm}|m{2cm}|m{2.5cm}|m{2.0cm}|m{2cm}|}
	Vérifier la pente du halo et ajouter celle du centre
	\centering
		\begin{tabular}{|c|c|c|c|c|c|c|c|c|c|}
			\hline $N$ & $R_c$ & $\sigma_c$ & $R_{10}$, $R_{50}$, $R_{90}$ & $a_1$, $a_2$ & $\beta$ & $\rho$ & $T$ & $\gamma$ & nom \tabularnewline
			\hline
			\hline \multirow{6}{*}{$1\ 10^5$} & \multirow{3}{*}{$33,3$}
					% & $10^{-3}$ & Accrétion & Triaxialisation du \textsc{sag} & tend vers $-0,5$ & Évolution de la pente du halo \tabularnewline \cline{3-7}
					% & $10^{-3}$ & Accrétion & \begin{tikzpicture}\node[draw,ellipse] at (0,0) {}; \end{tikzpicture}Triaxialisation du \textsc{sag} & tend vers $-0,5$ & Évolution de la pente du halo \tabularnewline \cline{3-7}
			& $10^{-3}$ & \accretionmoyen{} & $\diamondsuit$ & ${}^{-0.06}\searrow_{-0.18}$ & $\alpha\searrow$ & $\nearrow$ & $\searrow$ & $A_{3.1}$ \tabularnewline \cline{3-10}
					& & $10^{-1}$ & \accretionpeu{} & $\emptyset$ & ${}_{-0.04}\nearrow^{0.02}$ & $\emptyset$ & $\nearrow$ & $\emptyset$ & $A_{3.2}$  \tabularnewline \cline{3-10}
					& & $2\ 10^{-1}$ & \accretionpeu{} & $\emptyset$ & ${}_{-0.04}\nearrow^{0.03}$ & $\emptyset$ & $\nearrow$ & $\emptyset$ & $A_{3.3}$  \tabularnewline \cline{2-10}
				& \multirow{3}{*}{$66,6$}
					& $10^{-3}$ & \accretionpeu{} & $\emptyset$ & ${}_{-0.04}\nearrow^{-0.01}$ & $\emptyset$ & $\nearrow$ & $\searrow$ & $A_{6.1}$  \tabularnewline \cline{3-10}
					& & $10^{-1}$ & \accretionpeu{} & $\emptyset$ & ${}_{-0.04}\nearrow^{0.027}$ & $\emptyset$ & $\nearrow$ & $\emptyset$ & $A_{6.2}$  \tabularnewline \cline{3-10}
					& & $2\ 10^{-1}$ & \accretionpeu{} & $\emptyset$ & ${}_{-0.04}\nearrow^{0.02}$ & $\emptyset$ & $\nearrow$ & $\emptyset$ & $A_{6.3}$  \tabularnewline \cline{2-10}
			\hline
			\hline $2\ 10^5$ & $66,6$ & $10^{-3}$ & \accretionpeu{} & $\emptyset$ & ${}_{-0.05}\nearrow^{-0.03}$ & $\alpha\searrow$ & $\nearrow$ & $\emptyset$ & $A_{6.1}^m$  \\
			\hline
			\hline $4\ 10^5$ & $66,6$ & $10^{-3}$ & \accretionpeu{} & $\emptyset$ & ${\scriptstyle -0.049}\to{\scriptstyle -0.046}$ & $\alpha\searrow$ & $\nearrow$ & $\emptyset$ & $A_{6.2}^m$  \\
			\hline
			\hline $1\ 10^6$ & $66,6$ & $10^{-3}$ & \accretionpeu{} & $\emptyset$ & ${}_{-0.05}\nearrow^{-0.06}$ & $\alpha\searrow$ & $\nearrow$ & $\emptyset$ & $A_{6.3}^m$  \\
			\hline
			\hline \multirow{6}{*}{$3,25\ 10^5$} & \multirow{3}{*}{$33,3$}
					& $10^{-3}$ & \accretionlot{} & $\spadesuit$ & ${}_{-0.1}\nearrow^{3\ 10^{ -2 }}$ & $\alpha\searrow$ & $\nearrow$ & $\searrow$ & $B_{3.1}$  \tabularnewline \cline{3-10}
					& & $10^{-1}$ & \accretionpeu{} & $\emptyset$ & ${}_{-0.05}\nearrow^{-8\ 10^{-2}}$ & $\emptyset$ & $\nearrow$ & $\searrow$ & $B_{3.2}$  \tabularnewline \cline{3-10}
					& & $2\ 10^{-1}$ & \accretionpeu{} & $\emptyset$ & ${}_{-0.04}\nearrow^{10^{-2}}$ & $\emptyset$ & $\nearrow$ & $\emptyset$ & $B_{3.3}$  \tabularnewline \cline{2-10}
				& \multirow{3}{*}{$66,6$}
					& $10^{-3}$ & \accretionpeu{} & $\emptyset$ & ${}^{-0.05}\searrow_{-0.1}$ & $\alpha\searrow$ & $\nearrow$ & $\searrow$ & $B_{6.1}$  \tabularnewline \cline{3-10}
					& & $10^{-1}$ & \accretionpeu{} & $\emptyset$ & ${}_{-0.04}\nearrow^{0.02}$ & $\emptyset$ & $\nearrow$ & $\emptyset$ & $B_{6.2}$  \tabularnewline \cline{3-10}
					& & $2\ 10^{-1}$ & \accretionpeu{} & $\emptyset$ & ${}_{-0.04}\nearrow^{0.01}$ & $\emptyset$ & $\nearrow$ & $\emptyset$ & $B_{6.3}$  \tabularnewline \cline{2-10}
			\hline
			\hline \multirow{6}{*}{$5,5\ 10^5$} & \multirow{3}{*}{$33,3$}
					& $10^{-3}$ & \accretionlot{} & $\spadesuit$ & ${}^{-0.14}\searrow_{-0.24}$ & $\alpha\searrow$ & $\nearrow$ & $\searrow$ & $C_{3.4}$  \tabularnewline \cline{3-10}
					& & $10^{-1}$ & \accretionlot{} & $\emptyset$ & ${}^{-0.05}\nearrow^{0.01}$ & $\emptyset$ & $\nearrow$ & $\searrow$ & $C_{3.5}$  \tabularnewline \cline{3-10}
					& & $2\ 10^{-1}$ & \accretionpeu{} & $\emptyset$ & ${}_{-0.04}\nearrow^{2\ 10^{-3}}$ & $\emptyset$ & $\nearrow$ & $\searrow$ & $C_{3.6}$  \tabularnewline \cline{2-10}
				& \multirow{3}{*}{$66,6$}
					& $10^{-3}$ & \accretionmoyen{} & $\spadesuit$ & ${}^{-0.05}\searrow_{-0.45}$ & $\alpha\searrow$ & $\nearrow$ & $\searrow$ & $C_{6.1}$  \tabularnewline \cline{3-10}
					& & $10^{-1}$ & \accretionpeu{} & $\emptyset$ & ${}_{-0.04}\nearrow^{0.01}$ & $\emptyset$ & $\nearrow$ & $\emptyset$ & $C_{6.2}$  \tabularnewline \cline{3-10}
					& & $2\ 10^{-1}$ & \accretionpeu{} & $\emptyset$ & ${}_{-0.04}\nearrow^{0.02}$ & $\emptyset$ & $\nearrow$ & $\emptyset$ & $C_{6.3}$  \tabularnewline \cline{2-10}
			\hline
			\hline \multirow{6}{*}{$1856250$} & \multirow{6}{*}{$50$}
					& $1,25\ 10^{-4}$ & \accretionlot{} & $\spadesuit$ & ${}_{-0.15}\nearrow^{0.03}$ & $\alpha\searrow$ & $\nearrow$ & $\searrow$ & $C_{3.1}^m$  \tabularnewline \cline{3-10}
					& & $2,5\ 10^{-4}$ & \accretionlot{} & $\spadesuit$ & ${}_{-0.15}\nearrow^{0.1}$ & $\alpha\searrow$ & $\nearrow$ & $\searrow$ & $C_{3.2}^m$  \tabularnewline \cline{3-10}
					& & $5\ 10^{-4}$ & \accretionlot{} & $\spadesuit$ & ${}_{-0.14}\nearrow^{-0.003}$ & $\alpha\searrow$ & $\nearrow$ & $\searrow$ & $C_{3.3}^m$  \tabularnewline \cline{3-10}
					& & $10^{-3}$ & \accretionlot{} & $\spadesuit$ & ${}_{-0.14}\nearrow^{0.15}$ & $\alpha\searrow$ & $\nearrow$ & $\searrow$ & $C_{3.4}^m$  \tabularnewline \cline{3-10}
					& & $10^{-1}$ & \accretionlot{} & $\emptyset$ & ${}_{-0.1}\nearrow^{0.15}$ & $\alpha\searrow$ & $\nearrow$ & $\searrow$ & $C_{3.5}^m$  \tabularnewline \cline{3-10}
					& & $2\ 10^{-1}$ & \accretionlot{} & $\emptyset$ & ${}_{-0.09}\nearrow^{0.09}$ & $\emptyset$ & $\nearrow$ & $\searrow$ & $C_{3.6}^m$  \tabularnewline \cline{2-10}
			\hline
		\end{tabular}
	\caption{Résultats des simulations combinant un Hénon et un radiateur\label{Tab::SimuZoomRes}}
\end{table}




	Une première analyse nous apprend:
	\begin{enumerate}

		\item les simulations auxquelles sont associées les symboles \accretionlot{} et $\spadesuit$ présentent également une évolution
			particulière de l'anisotropie. Ce paramètre passe par un minimum au moment où le \textsc{sag} se déforme. Ce type d'évolution
			peut-être interprété dans le contexte de l'instabilité d'orbite radiale.

		\item les simulations \accretionmoyen{} entrent dans une catégorie intermédiaire suggérant que les caractéristiques du bain, comme sa
			densité ou sa température, sont des paramètres important dans son déclenchement.

		\item les simulations \accretionpeu{} forment la dernière catégorie. Ces simulations seront étudiées plus en détails dans la prochaine
			section.

		\item les simulations caractérisées par le symbole $\diamondsuit$ sont des simulations très touchées par les conditions périodiques.
			Ces simulations vont avoir tendance à former des barres aligné avec les coins de la boîte. La
			figure~\ref{Fig::Density::Comp::Periodic} compare deux simulations, l'une subissant les effets de la boîte périodique, l'autre
			ne les subissant pas.
			\begin{figure}[hbtp]
				\begin{minipage}{0.49\linewidth}
					\centering \includegraphics{{graphe/Periodique_Plot/N-100000.0_Rc-33.3333333333_sc-0.001_Soft-0.001}.png}
				\end{minipage}\hfill
				\begin{minipage}{0.50\linewidth}
					\centering \includegraphics{{graphe/Periodique_Plot/N-100000.0_Rc-66.6666666666_sc-0.001_Soft-0.001}.png}
				\end{minipage}
				% \centering \includegraphics{{graphe/Periodique_Plot/N-100000.0_Rc-66.6666666666_sc-0.001_Soft-0.001}.png},c 
				\caption{À gauche une simulation ($A_{3,1}$) sous l'influence des conditions périodiques, à droite une simulation ($A_{6,1}$) correcte.\label{Fig::Density::Comp::Periodic}}
				% \begin{minipage}{0.48\linewidth}
					% \centering \includegraphics{{graphe/N-100000.0_Rc-33.3333333333_sc-0.001_Soft-0.001_100_xy}.png}
				% \end{minipage}\hfill
				% \begin{minipage}{0.48\linewidth}
					% \centering \includegraphics{{graphe/N-100000.0_Rc-33.3333333333_sc-0.001_Soft-0.001_100_yz}.png}
				% \end{minipage}
				% \centering \includegraphics{{graphe/N-100000.0_Rc-33.3333333333_sc-0.001_Soft-0.001_100_xz}.png}
				% \caption{Projection de la simulation $N = 10^5$, $R_c = 33,3$ et $\sigma_c = 10^{-3}$\label{Fig::Projection::TriAxial}}
				% Ajouter les colorbar de chaque graphes.
			\end{figure}

	\end{enumerate}


	% Certaines simulations montrent des effets dus aux cubes périodiques trop petits. Nous observerons par exemple la simulation $N_b = 10^5$, $R_c
	% = 33,3$ et $\sigma_c = 10^{-3}$ qui montre une triaxialisation du \textsc{sag}. La figure~\ref{Fig::Projection::TriAxial} montre l'état de ce
	% système à la fin de la simulation qui a duré $15.9T_d^H$. L'objet montre une forme quasiment cubique et des \og bras\fg partent vers les bords du
	% cube, marque des conditions périodiques. Bien qu'il n'y ait que peu de particules dans ces \og bras\fg, nous nous sommes placé dans des
	% conditions pour lesquelles de tels effet ne se produisent pas.

	% souhaitons tout de mêmes éviter
	% les conditions qui éviteraient de voir apparaître de tel propriété.


	% Les autres simulations peuvent se classer en 3 catégories distinctes représentées dans le tableau~\ref{Tab::SimuZoomRes}. La première
	% catégorie, représentée par le symbole \accretionmoyen{}, concerne les simulations ne montrant pas d'évolution, si ce n'est une faible
	% accrétion.

	% La seconde catégorie, qui fera l'objet d'une étude plus poussée dans la section suivante, concerne les objets dont le cœur semble se
	% contracter. Ils sont représenté par le symbole \accretionpeu{} et montrent tous une décroissance stricte de $r_{10\%}$ et $r_{50\%}$.
	% Ces deux catégories conserve un rapport du viriel autour de $1$, ainsi que leur sphéricité. Idem pour l'isotropie du système. En gros, ils
	% n'évoluent pas.

	% La dernière catégorie, le symbole \accretionlot{}, englobe tous les autres objets présentant une très forte accrétion, la taille de leurs
	% cœurs peut aller jusqu'à sextupler. Toutes ne vont pas développer une \textsc{roi} sur le temps de la simulation, mais elles restent une
	% minorité. Nous étudierons plus tard comment est influencé le déclenchement de cette instabilité par les paramètres du bain.

	% D'autres ont montré des effets beaucoup plus intéressants. Prenons la simulation $N_b = 1856250$, $R_c = 50$, $\sigma_c = 10^{-3}$, dont
	% l'évolution de certains paramètres est représentée sur la figure~\ref{Fig::Simu::p4s125}. L'évolution des rayons à $10\%$, $50\%$ et $90\%$ de
	% masse nous indique une forte accrétion du bain par le \textsc{sag}. Les axes d'inertie semblent indiquer qu'une instabilité d'orbite radiale
	% se produise vers $t=30$ (soit $4T_d^H$).

	% \subsection{Qu'avons-nous vu}

	% Et bien, comme prévu, lors de boîte trop petite nous avons les effets des conditions périodiques. Par
	% contre, dans les cas où la taille de la boîte n'était pas suffisante pour assurer la stabilité du bain,
	% nous avons commencer à observer des effets intéressants lié à l'accrétion du bain par l'objet. Dans
	% d'autre cas, où nous n'avions pas assez de particules pour moyenner les effets de relaxations à
	% $N$-corps, nous avons pu voir une évolution intéressante de l'objet. Dans la plupart des cas, il ne
	% c'est rien passez, ou alors uniquement des effets dû à la boîte périodique \& co.

	% \todo[inline]{
		% Mettre les simulations suivantes:
		% \begin{itemize}
			% \item $N = 550 000$, $R_c = 66.6666$, $\sigma_c = 0.001$;
			% \item $N = 550 000$, $R_c = 33.3333$, $\sigma_c = 0.001$ et variante;
			% \item $N = 325 000$, $33.3333$, $\sigma_c = 0.001$.
		% \end{itemize}
	% }

	Dans la section suivante, nous commencerons par étudier le déclenchement de l'instabilité d'orbite radiale, puis nous chercherons à comprendre
	pourquoi certaines simulations semblent montrent un effondrement du \textsc{sag}.

\section{Focalisation sur les simulations intéressantes}
	\subsection{Instabilité d'orbite radiale}

		Les simulations de la première catégorie présente donc une forte accrétion, les rayons $R_{10}$, $R_{50}$ et $R_{90}$ augmentent de
		façon importante, comme le montre la figure~\ref{Fig::Simu::ROI::Radius::Cm34}.
		\begin{figure}[htpb]
			\centering \includegraphics{{ROI_radius_N-550000.0p1.5p1.5p1.5_Rc-33.3333333333p1.5_sc-0.001_Soft-0.001_new3}.pdf}
			\caption{Évolution des rayons à $10\%$ (en bleu), $50\%$ (en vert) et $90\%$ (en rouge) de masse au cours de la simulation
			$C^m_{3.4}$.\label{Fig::Simu::ROI::Radius::Cm34}}
		\end{figure}
		Au fur et à mesure que l'accrétion se poursuit, l'anisotropie du \textsc{sag} diminue jusqu'à un semblant se trouver dans l'intervalle
		$[-0,6; -0,5]$ minimum puis remonte vers sa position d'équilibre. Dans le même temps, les rapports d'axes nous apprennent que la forme
		de l'objet change: $a_1$ reste constant mais $a_2$ va diminuer jusqu'à une position d'équilibre $a_2 \approx 0,5$.
		\begin{figure}[htpb]
			\centering \includegraphics{{ROI_N-550000.0p1.5p1.5p1.5_Rc-33.3333333333p1.5_sc-0.001_Soft-0.001_new3}.pdf}
			\caption{Évolution des rapports d'axes $a_1$ et $a_2$ et de l'anisotropie au cours de la simulation
			$C^m_{3.4}$.\label{Fig::Simu::ROI::RAA::Cm34}}
		\end{figure}
		Cette évolution est présenté sur la figure~\ref{Fig::Simu::ROI::RAA::Cm34} pour la simulation $C^m_{3.4}$. Cette évolution se retrouve
		sur toute les simulations marquées par le symbole $\spadesuit$. Nous avons vu dans le chapitre~\ref{Chap::Instabilite} que
		l'instabilité d'orbite radiale... L'instabilité semble se produire lorsque l'anisotropie atteint son minimum lorsque $a_2$ commence à
		décroitre de façon significative (variation de l'ordre de $20\%$). En cherchant à savoir à quel moment elle se produit, nous
		remarquons que pour les simulations de la catégorie $C^m_{i,j}$, le changement de température du bain n'influe pas sur son
		déclenchement: toutes les simulations de cette catégorie commencent à subir son influence vers $t=35$. À l'exception de la simulation
		$B_{3.1}$ dont l'instabilité apparait vers $t \approx 50$.

		\begin{figure}[htbp]
			\begin{subfigure}{0.50\linewidth}
				\centering \includegraphics{{ROI_zoom_N-550000.0p1.5p1.5p1.5_Rc-33.3333333333p1.5_sc-0.000125_Soft-0.001_new3}.pdf}
				\caption{Simulation $C^m_{3.1}$.\label{Fig::Simu::ROI::RAA::Cm31}}
			\end{subfigure}\hfill
			\begin{subfigure}{0.50\linewidth}
				\centering \includegraphics{{ROI_zoom_N-550000.0p1.5p1.5p1.5_Rc-33.3333333333p1.5_sc-0.00025_Soft-0.001_new3}.pdf}
				\caption{Simulation $C^m_{3.2}$.\label{Fig::Simu::ROI::RAA::Cm32}}
			\end{subfigure}
			\begin{subfigure}{0.50\linewidth}
				\centering \includegraphics{{ROI_zoom_N-550000.0p1.5p1.5p1.5_Rc-33.3333333333p1.5_sc-0.0005_Soft-0.001_new3}.pdf}
				\caption{Simulation $C^m_{3.3}$.\label{Fig::Simu::ROI::RAA::Cm33}}
			\end{subfigure}\hfill
			\begin{subfigure}{0.50\linewidth}
				\centering \includegraphics{{ROI_zoom_N-550000.0p1.5p1.5p1.5_Rc-33.3333333333p1.5_sc-0.001_Soft-0.001_new3}.pdf}
				\caption{Simulation $C^m_{3.4}$.\label{Fig::Simu::ROI::RAA::Cm34Zoom}}
			\end{subfigure}
			\begin{subfigure}{1.00\linewidth}
				\centering \includegraphics{{ROI_zoom_N-550000.0_Rc-33.3333333333_sc-0.001_Soft-0.001_new-seed}.pdf}
				\caption{Simulation $C_{3.4}$.\label{Fig::Simu::ROI::RAA::C34}}
			\end{subfigure}
			\caption{Évolution des rapports d'axes $a_1$ et $a_2$ et de l'anisotropie au cours de différentes simulations.\label{Fig::Simu::ROI::RAA::All}}
		\end{figure}

		% \begin{figure}[htbp]
			% \begin{minipage}{0.45\linewidth}
				% \centering \includegraphics{{N-550000.0p1.5p1.5p1.5_Rc-33.3333333333p1.5_sc-0.000125_axial}.pdf}
				% \caption{Évolution des rapports d'axes}
			% \end{minipage}\hfill
			% \begin{minipage}{0.45\linewidth}
				% \centering \includegraphics{{N-550000.0p1.5p1.5p1.5_Rc-33.3333333333p1.5_sc-0.000125_rayon}.pdf}
				% \caption{Évolution des rayons à $10\%$, $50\%$ et $90\%$ de masse.}
			% \end{minipage}
			% \caption{Simulation de paramètre $N = 1856250$, $R_c = 49.999999999$, $\sigma_c = 0.000125$.\label{Fig::Simu::p4s125}}
		% \end{figure}
		% \begin{figure}[htbp]
			% \begin{minipage}{0.45\linewidth}
				% \centering \includegraphics{{N-550000.0p1.5p1.5p1.5_Rc-33.3333333333p1.5_sc-0.000125_axial}.pdf}
				% \caption{Évolution des rapports d'axes pour la simulation de paramètre $N = 1856250$, $R_c = 49.999999999$, $\sigma_c = 0.000125$.}
			% \end{minipage}\hfill
			% \begin{minipage}{0.45\linewidth}
				% \centering \includegraphics{{N-550000.0p1.5p1.5p1.5_Rc-33.3333333333p1.5_sc-0.0005_axial}.pdf}
				% \caption{Évolution des rapports d'axes pour la simulation de paramètre $N = 1856250$, $R_c = 49.999999999$, $\sigma_c = 0.0005$.}
			% \end{minipage}
				% \centering \includegraphics{{N-550000.0p1.5p1.5p1.5_Rc-33.3333333333p1.5_sc-0.001_axial}.pdf}
				% \caption{Évolution des rapports d'axes pour la simulation de paramètre $N = 1856250$, $R_c = 49.999999999$, $\sigma_c = 0.001$.}
		% \end{figure}
		% \begin{figure}[htbp]
			% \begin{minipage}{0.45\linewidth}
				% \centering \includegraphics{{N-550000.0p1.5p1.5p1.5_Rc-33.3333333333p1.5_sc-0.000125_rayon}.pdf}
				% \caption{Évolution des rayons à $10\%$, $50\%$ et $90\%$ de masse pour la simulation de paramètre $N = 1856250$, $R_c = 49.999999999$, $\sigma_c = 0.000125$.}
			% \end{minipage}\hfill
			% \begin{minipage}{0.45\linewidth}
				% \centering \includegraphics{{N-550000.0p1.5p1.5p1.5_Rc-33.3333333333p1.5_sc-0.0005_rayon}.pdf}
				% \caption{Évolution des rayons à $10\%$, $50\%$ et $90\%$ de masse pour la simulation de paramètre $N = 1856250$, $R_c = 49.999999999$, $\sigma_c = 0.0005$.}
			% \end{minipage}
				% \centering \includegraphics{{N-550000.0p1.5p1.5p1.5_Rc-33.3333333333p1.5_sc-0.001_rayon}.pdf}
				% \caption{Évolution des rayons à $10\%$, $50\%$ et $90\%$ de masse pour la simulation de paramètre $N = 1856250$, $R_c = 49.999999999$, $\sigma_c = 0.001$.}
		% \end{figure}

	\subsection{Évolution de la pente avec l'âge (effet de relaxation)}

		Nous allons maintenant nous concentrer sur la seconde catégorie de simulations, montrant un effondrement progressif du cœur. Plus
		précisément, nous allons étudier la simulations de paramètres $N_b=10^5$, $R_c = 66.7$ et $\sigma_c = 10^{-3}$. La première chose que
		nous faisons est de regarder si la tendance se conserve lorsque nous étendons la simulation. Nous avons donc étendu la durée de la
		simulation de $15.9T_d^H$ à $47.7T_d^H$. La figure~\ref{Fig::Relax::Data} rassemble les graphiques montrant l'évolution des
		observables les plus intéressantes. Les 2 premières figures (\ref{Fig::Relax::Data::SubAxes} et~\ref{Fig::Relax::Data::SubAniso})
		montrent l'évolution des rapports d'axes de la matrice d'inertie et l'anisotropie. Ces observables sont constante, le \textsc{sag}
		n'est donc pas sous l'influence de \textsc{roi}. De plus, la figure~\ref{Fig::Relax::Data::SubRayon} montre une accrétion faible.
		Par contre, la densité (figure~\ref{Fig::Relax::DensiteEvo}) montre une évolution nette de la densité centrale. La première chose que nous cherchons à vérifier est: est-ce
		un effet du bain?

		\begin{figure}[htbp]
			\centering \includegraphics{{N-100000.0_Rc-66.6666666666_sc-0.001_evo}.pdf}
			\centering \caption{Évolution au cours du temps de la densité. En bleu, $t=0$. En vert, $t=15.9T_d^H$. En rouge,
				$t=31.8T_d^H$. En noir, $t=47.4T_d^H$.\label{Fig::Relax::DensiteEvo}}
		\end{figure}

		\begin{figure}[htbp]
			\begin{subfigure}{0.45\linewidth}
				\centering \includegraphics{{N-100000.0_Rc-66.6666666666_sc-0.001_Soft-0.001_axial}.pdf}
				\centering \caption{Évolution des rapports d'axes.\label{Fig::Relax::Data::SubAxes}}
			\end{subfigure}\hfill
			\begin{subfigure}{0.45\linewidth}
				\centering \includegraphics{{N-100000.0_Rc-66.6666666666_sc-0.001_Soft-0.001_aniso}.pdf}
				\centering \caption{Évolution de l'anisotropie moyenne du système.\label{Fig::Relax::Data::SubAniso}}
			\end{subfigure}

			\begin{subfigure}{1.00\linewidth}
				\centering \includegraphics{{N-100000.0_Rc-66.6666666666_sc-0.001_Soft-0.001_rayon}.pdf}
				\centering \caption{Évolution des rayons à $10\%$, $50\%$ et $90\%$ de masse.\label{Fig::Relax::Data::SubRayon}}
			\end{subfigure}\hfill
			% \begin{subfigure}{0.45\linewidth}
				% \centering \includegraphics{{N-100000.0_Rc-66.6666666666_sc-0.001_Soft-0.001_densite}.pdf}
				% \centering \caption{Évolution de la densité à différent temps.\label{Fig::Relax::Data::SubDensite}}
			% \end{subfigure}
			\caption{Simulation de paramètre $N=100000$, $R_c=66.6$, $\sigma_c=0.001$.\label{Fig::Relax::Data}}
		\end{figure}

		Pour répondre à cette question, nous allons augmenter le nombre de particules dans chaque objet (bain et \textsc{sag}) de la
		simulation, tout en gardant constant tous les paramètres de sorte que les nouvelles simulations soit strictement équivalente. Les
		différentes figures~\ref{Fig::Relax::Comp2} à~\ref{Fig::Relax::Comp10} montrent l'évolution de ces nouvelles simulations (les croix)
		superposées à celle que nous étudions (en trait plein), aux même temps que la figure~\ref{Fig::Relax::DensiteEvo}.
		\begin{figure}[htbp]
			\begin{subfigure}{0.45\linewidth}
				\centering \includegraphics{{N-100000.0_Rc-66.6666666666_sc-0.001_evo_compare2}.pdf}
				\centering \caption{$2$ fois plus de particules pour chaque objet.\label{Fig::Relax::Comp2}}
			\end{subfigure}\hfill
			\begin{subfigure}{0.45\linewidth}
				\centering \includegraphics{{N-100000.0_Rc-66.6666666666_sc-0.001_evo_compare4}.pdf}
				\centering \caption{$4$ fois plus de particules pour chaque objet.\label{Fig::Relax::Comp4}}
			\end{subfigure}

			\begin{subfigure}{1.00\linewidth}
				\centering \includegraphics{{N-100000.0_Rc-66.6666666666_sc-0.001_evo_compare10}.pdf}
				\centering \caption{$10$ fois plus de particules pour chaque objet.\label{Fig::Relax::Comp10}}
			\end{subfigure}

			\caption{Comparaison avec plusieurs simulations possédant plus de particules.\label{Fig::Relax::Comp}}
		\end{figure}
		Nous pouvons voir que plus le nombre de particules augmentent, l'effondrement devient de moins en moins prononcé. Il ne serait donc
		pas un effet dû au bain. Pour en être complètement sûr, nous allons regarder l'évolution du \textsc{sag} isolé. La
		figure~\ref{Fig::Relax::CompH} montre l'évolution d'une sphère de Hénon comprenant $3\ 10^4$ particules
		(graphique~\ref{Fig::Relax::Comp3e4}) et d'une sphère de Hénon de $3\ 10^5$ particules. La conclusion apparaît immédiatement: le
		\textsc{sag} plongé dans un bain et le \textsc{sag} isolé ont la même évolution. Encore pire, si l'on multiplie par $10$ le nombre de
		particules du Hénon (en conservant sa masse totale identique), l'évolution change radicalement et ne montre plus d'effondrement.
		L'effet observé n'est donc pas un effet du bain, mais un effet semblant être dû aux collisions.


		\begin{figure}[htbp]
			\begin{subfigure}{0.45\linewidth}
				\centering \includegraphics{{N-100000.0_Rc-66.6666666666_sc-0.001_evo_isolated}.pdf}
				\centering \caption{Comparaison de l'évolution de la densité avec un Hénon de $N_H=3\ 10^4$ particules.\label{Fig::Relax::Comp3e4}}
			\end{subfigure}\hfill
			\begin{subfigure}{0.45\linewidth}
				\centering \includegraphics{{N-100000.0_Rc-66.6666666666_sc-0.001_evo_isolated10}.pdf}
				\centering \caption{Comparaison de l'évolution de la densité avec un Hénon de $N_H=3\ 10^5$ particules.\label{Fig::Relax::Comp3e5}}
			\end{subfigure}

			\caption{Comparaison avec plusieurs Hénon isolés.\label{Fig::Relax::CompH}}
		\end{figure}











		% 		\begin{table}[htbp]
			\rotatebox{90}{
				\begin{tabular}{|c|c|c|p{2cm}|p{2cm}|p{2cm}|p{2.0cm}|c|c|c|}
				\hline $N$ & $R_c$ & $\sigma_c$ & $r_{10\%}$, $r_{50\%}$, $r_{90\%}$ & Axial ratio & $\gamma$ & $\beta$ & $T$ & $E$ & $\rho$ \tabularnewline
				\hline \multirow{6}{*}{$1\ 10^5$} & \multirow{3}{*}{$33,3$}
						% & $1,25\ 10^{-4}$ & Rien & rien & $-1$ & rien & rien & rien & Rien \tabularnewline \cline{3-10}
						% & & $2,5\ 10^{-4}$ & Rien & rien & $-1$ & rien & rien & rien & Rien \tabularnewline \cline{3-10}
						% & & $5\ 10^{-4}$ & Rien & rien & $-1$ & rien & rien & rien & Rien \tabularnewline \cline{3-10}
						& $10^{-3}$ & Un poil d'accrétion & Triaxialisation du \textsc{sag} & $-1,4$ & tend vers $-0,5$ & rien & rien & Diminution de la pente du halo \tabularnewline \cline{3-10}
						& & $10^{-1}$ & Rien & rien & $-1$ & rien & rien & rien & Rien \tabularnewline \cline{3-10}
						& & $2\ 10^{-1}$ & Rien & rien & $-1$ & rien & rien & rien & Rien \tabularnewline \cline{2-10}
					& \multirow{3}{*}{$66,6$}
						% & $1,25\ 10^{-4}$ & Rien & rien & $-1$ & rien & rien & rien & Rien \tabularnewline \cline{3-10}
						% & & $2,5\ 10^{-4}$ & Rien & rien & $-1$ & rien & rien & rien & Rien \tabularnewline \cline{3-10}
						% & & $5\ 10^{-4}$ & Rien & rien & $-1$ & rien & rien & rien & Rien \tabularnewline \cline{3-10}
						& $10^{-3}$ & Rien & rien & $-1$ & rien & rien & rien & Rien \tabularnewline \cline{3-10}
						& & $10^{-1}$ & Rien & rien & $-1$ & rien & rien & rien & Rien \tabularnewline \cline{3-10}
						& & $2\ 10^{-1}$ & Rien & rien & $-1$ & rien & rien & rien & Rien \tabularnewline \cline{2-10}
				\hline \multirow{6}{*}{$325\ 10^5$} & \multirow{3}{*}{$33,3$}
						% & $1,25\ 10^{-4}$ & Rien & Rien & $-1$ & rien & rien & rien & Rien \tabularnewline \cline{3-10}
						% & & $2,5\ 10^{-4}$ & Rien & rien & $-1$ & rien & rien & rien & Rien \tabularnewline \cline{3-10}
						% & & $5\ 10^{-4}$ & Rien & rien & $-1$ & rien & rien & rien & Rien \tabularnewline \cline{3-10}
						& $10^{-3}$ & Beaucoup d'accrétion & Indique une \textsc{roi} & $-1$ & tend vers $-0,5$ & drôle de variation & rien & Diminution de la pente \tabularnewline \cline{3-10}
						& & $10^{-1}$ & Un poil d'accrétion & rien & $-1$ & rien & rien & rien & Rien \tabularnewline \cline{3-10}
						& & $2\ 10^{-1}$ & Rien & rien & $-2$ & rien & rien & rien & Rien \tabularnewline \cline{2-10}
					& \multirow{3}{*}{$66,6$}
						% & $1,25\ 10^{-4}$ & Rien & rien & $-1,4$ & rien & rien & rien & Rien \tabularnewline \cline{3-10}
						% & & $2,5\ 10^{-4}$ & Rien & rien & $-1$ & rien & rien & rien & Rien \tabularnewline \cline{3-10}
						% & & $5\ 10^{-4}$ & Rien & rien & $-1$ & rien & rien & rien & Rien \tabularnewline \cline{3-10}
						& $10^{-3}$ & Un tout petit peu d'accrétion & rien & $-1$ & rien & rien & rien & Légère augmentation de la pente \tabularnewline \cline{3-10}
						& & $10^{-1}$ & Rien & rien & $-1$ & rien & rien & rien & Rien \tabularnewline \cline{3-10}
						& & $2\ 10^{-1}$ & Rien & rien & $-1$ & rien & rien & rien & Rien \tabularnewline \cline{2-10}
				\hline \multirow{12}{*}{$5,5\ 10^5$} & \multirow{6}{*}{$33,3$}
						& $1,25\ 10^{-4}$ & Forte accrétion & Indique une \textsc{roi} & Constant à $-1,25$ & tend vers $-1$ & Augmente & bof & la pente du halo diminue \tabularnewline \cline{3-10}
						& & $2,5\ 10^{-4}$ & Forte accrétion & Indique une \textsc{roi} & Constant à $-1,25$ & tend vers $-1$ & Augmente & bof & Diminution de la pente du halo \tabularnewline \cline{3-10}
						& & $5\ 10^{-4}$ & Accrétion & Indique une \textsc{roi} & Constant & tend vers $-1$ & Augmente & bof & Diminution de la pente du halo \tabularnewline \cline{3-10}
						& & $10^{-3}$ & Accrétion & Indique une \textsc{roi} & Constant & tend vers $-1$ & Augmente & bof & Diminution de la pente du halo \tabularnewline \cline{3-10}
						& & $10^{-1}$ & Accrétion & Rien & Constant & Rien & Évolue un peu & bof & Diminution de la pente du halo \tabularnewline \cline{3-10}
						& & $2\ 10^{-1}$ & Rien & Rien & Constant & Rien & Rien & conservée & Rien \tabularnewline \cline{2-10}
					& \multirow{6}{*}{$66,6$}
						& $1,25\ 10^{-4}$ & ? & ? & $-1$ & rien & rien & rien & Rien \tabularnewline \cline{3-10}
						& & $2,5\ 10^{-4}$ & ? & ? & $-1$ & rien & rien & rien & Rien \tabularnewline \cline{3-10}
						& & $5\ 10^{-4}$ & ? & ? & $-1$ & rien & rien & rien & Rien \tabularnewline \cline{3-10}
						& & $10^{-3}$ & ? & ? & $-1$ & rien & rien & rien & Rien \tabularnewline \cline{3-10}
						& & $10^{-1}$ & accrétion &  Indique une \textsc{roi} & $-1$ & décroit vers $-0,5$ & rien & rien & faible diminution de la pente du halo \tabularnewline \cline{3-10}
						& & $2\ 10^{-1}$ & Rien & Rien & $-1$ & rien & rien & rien & Rien \tabularnewline \cline{2-10}
				\hline
			\end{tabular}
			}
		\end{table}



