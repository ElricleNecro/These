\section{Conclusion}

Les travaux de cette thèse ont eu pour objet la modélisation dynamique de la structure des amas globulaires et des galaxies, structures
autogravitantes (\textsc{sag}). 



Il est communément accepté que ces objets possèdent globalement un profil de densité volumique de masse caractérisé soit par une structure cœur halo
soit par un simple halo pouvant posséder plusieurs pentes. L'objectif de cette thèse était de comprendre les différentes propriétés et les raisons de
l'évolution dynamique de ces profils. Il a été atteint dans ses grandes lignes.



Le tout début de l'étude a été consacré à réalisation d'une synthèse de l'évolution dynamique du profil de densité des amas globulaires de notre
galaxie. Cette étude a permis de confirmer le fait bien connu que ces profils sont, avant le collapse du cœur, ajustables par un modèle de King. Elle
a aussi et surtout mis en évidence le fait que la pente du halo était une fonction croissante de l'âge de ces objets. Cette étude préliminaire
confirme donc le paradigme suivant lequel le profil de densité est l'un des indicateurs fondamentaux de l'évolution des amas globulaires et par
extension des galaxies.


Après un travail de revue théorique sur les différents types de structures isothermes en domaine borné ou non, nous avons montré que les modèles de
King, et par extension les \textsc{SAG} de type cœur halo de pente inférieur à $-2$, étaient d'excellentes approximations de sphères isothermes. En
utilisant un modèle idéalisé de structure cœur halo isotherme de pente quelconque, nous avons pu étendre certains résultats d'instabilité jusque là
obtenus uniquement dans le cadre des sphères isothermes en domaine borné. Cette instabilité appliquée aux structures cœur halo de pente $-4$
(instabilité \textsc{ch4}), a été mise en contexte  en dehors du scénario évolutif à long terme, pour expliquer l'existence de deux type de profils
pour les \textsc{SAG}: cœur halo (pour les amas globulaires ou les galaxies LSB) ou cuspide (pour les galaxies en général). La présence ou l'absence
(réelle ou effective) d'un bain thermique autour de l'objet en formation permettrait ou non à cette instabilité de se développer et conduirait ou pas
à l'effondrement du cœur de ces objets sur des échelles de temps de l'ordre de quelques temps dynamiques.



Afin d'ajuster les différents paramètres de ce scénario nous avons effectué un certain nombre de simulations numériques. Ces expériences démarrées
dans un contexte ciblé se sont révélées d'une ampleur et d'une portée supérieure aux attentes initiales. 



Face à la non-évolution des profils de King placés dans un thermostat, nous avons augmenté la sensibilité de nos  expériences en imposant un bain
thermique dès la phase initiale d'effondrement d'une sphère de Hénon.



% Toutes nos expériences ont été réalisées avec des bains plus froids que la structure autogravitante d'étude. 
La plupart de nos expériences ont été réalisées avec des bains plus froids que la structure autogravitante d'étude. Mais les bains sont toujours moins
denses.

Les raisons de cette dernière limitation sont simple. La méthode particulaire utilisée pour modéliser notre bain
thermique dans de bonnes conditions se serait révélée trop coûteuse avec un contraste de densité plus faible entre le bain et la structure
autogravitante.
% Les raisons de cette limitation sont doubles. Tout d'abord, il est physiquement raisonnable de penser qu'une structure formée, cohérente et à
% l'équilibre, soit dans ce contexte, plus chaude que son environnement. Par ailleurs, la méthode particulaire utilisée pour modéliser notre bain
% thermique dans de bonnes conditions se serait révélée trop couteuse avec un contraste de température plus faible entre le bain et la structure
% autogravitante.


Malgré cette limitation importante nous avons obtenu deux résultats principaux:


\begin{enumerate}

	\item Dans tous les cas d'accrétion progressive du bain par la structure,
	nous avons observé l'apparition d'une instabilité d'orbite radiale.
	Certains éléments du mécanisme de cette instabilité,
	qui n'avait pas été observée dans ce contexte, ont pu être
	analysés:

	\begin{itemize}
		\item Le contexte général de nos résultats sur cette
		instabilité confirme un résultat déjà rapporté mais non
		généralement reproduit. L'instabilité d'orbite radiale se produit
		par déplacement adiabatique d'un système autogravitant à
		l'équilibre vers les zones radiales de son espace des phases. Cette
		instabilité ne saurait donc se déclencher sans la présence du
		germe constitué par cet équilibre.

		\item La densité du bain, et non pas sa température, semble être
		l'un des paramètres essentiels du déclenchement de cette
		instabilité dans ce contexte. 
	\end{itemize}

	\item Dans un contexte sans accrétion, nous avons pu mettre en
	évidence un effondrement du cœur de l'une de nos familles de SAG. Les
	aspects morphologiques de cet effondrement sont tout à fait comparables
	à ceux observés pour les amas globulaires: augmentation du contraste
	de densité, augmentation de la pente du halo, passage progressif à un
	profil de type cuspide. Le résultat final de ce processus est tout à
	fait comparable au profil de type de Vaucouleurs généralisé ou NFW
	généralement observé pour les structures autogravitantes à
	l'échelle galactique dans les simulations de grandes structures.
	L'échelle de temps sur laquelle se produit cette évolution semble
	cependant bien inférieure à celle nécessaire pour observer des
	effets de relaxation à deux corps: elle ne représente au maximum que
	10\% du temps généralement accepté ($T_{rel}\propto\frac{0.1N}{\ln
	N}T_{d}$) pour ce genre de processus. Les études préliminaires que
	nous avons pu mener semblent toutefois montrer que cet effondrement ne soit
	pas causé par la présence du bain thermique. Dans l'état actuel de
	la situation nous ne sommes donc pas en mesure d'affirmer que nous avons
	reproduit numériquement l'instabilité \textsc{ch4}.
\end{enumerate}



Parallèlement à cette étude dynamique, nous avons conduit une
comparaison entre un code résolvant l'équation de Vlasov pour un
système sphérique et le treecode \textsc{Gadget-2}. Nous avons montré que
l'accord entre ces deux codes était étonnamment bon, même à
faible rapport du viriel,
ce qui montre que les simulations $N$-corps sont capables de suivre avec précision la
structure fine
de l'espace des phases, mais il est nécessaire d'utiliser un grand nombre de particules et cela d'autant plus que le système est froid.


\section{Perspectives}


Nous aimerions pouvoir étendre le domaine temporel de l'étude des
simulations déclenchant l'instabilité d'orbite radiale. Cette
extension permettrait d'une part de raffiner l'étude de l'influence de la
densité du bain sur le déclenchement de cette instabilité, en
fournissant par exemple un critère précis et utilisable dans un
contexte éventuellement cosmologique. Dans le même temps, il s'agirait
de confirmer définitivement le fait que le déclenchement
de l'instabilité d'orbite radiale dans ce contexte ne dépend pas de la
température du bain.



À ce stade, une étude plus avancée de la simulation $A_{6,1}$
conduisant à l'évolution accélérée du profil de
densité vers un effondrement est nécessaire.  Cette étude pourrait
être effectuée en modifiant d'une part les caractéristiques
\og{}numériques\fg du code \textsc{Gadget-2} (angle d'ouverture, gestion du pas de temps,
paramètre d'adoucissement du potentiel), mais aussi en construisant de
nouvelles observables mieux adaptées à la surveillance fine de la
dynamique de ce système (calculs raffinés de temps
caractéristiques, calcul de la densité par résolution de
l'équation de Poisson, etc.).


Finalement, les principales pistes envisageables pour tenter de
mettre en évidence numériquement l'instabilité \textsc{ch4}  est
l'augmentation soit de la température du bain, soit de sa densité dans nos simulations. Cette
augmentation pourra être effectuée dans le cadre particulaire
proposé ici, elle sera alors très coûteuse; elle pourra également
être envisagée dans le cadre d'une méthode numérique
adaptée, elle sera alors moins couteuse mais nécessitera le
développement d'un code spécifique.
% Finalement, la principale piste envisageable pour tenter réellement de
% mettre en évidence numériquement l'instabilité \textsc{ch4}  est
% l'augmentation de la température du bain dans nos expériences.  Cette
% augmentation pourra être effectuée dans le cadre particulaire
% proposé ici, elle sera alors très couteuse; elle pourra également
% être envisagée dans le cadre d'une méthode numérique
% adaptée, elle sera alors moins couteuse mais nécessitera le
% développement d'un code spécifique.

