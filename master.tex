\documentclass[a4paper,11pt,twoside,openright]{report}

\usepackage{ucs}
\usepackage[utf8x]{inputenc}

\usepackage[T1]{fontenc}
\usepackage[frenchb]{babel}
%[francais]{babel}
\usepackage[francais]{layout}

%\usepackage{mathabx}							%Pour les symboles astro

\usepackage{amssymb, amsmath,amsfonts}					%Pour la mise en page des maths
\usepackage{graphicx}							%Pour les opérations sur les images
\usepackage{moreverb}							% ???????
\usepackage{gnuplottex}							%Pour appeler GNUPlot dans Latex (compiler avec l'option -shell-escape)
%\usepackage{listings}							%Pour afficher des fichiers sources

\usepackage[usenames]{xcolor}
\usepackage{tikz}							%Pour faire des schémas et choses du genre de bonne qualité en Latex directement
\usetikzlibrary{decorations,decorations.pathreplacing,decorations.pathmorphing,decorations.pathreplacing}
\usetikzlibrary{shapes}
\usetikzlibrary{calc}
\usetikzlibrary{fit,backgrounds}
\usetikzlibrary{shadows}
\usetikzlibrary{patterns}

\usepackage{placeins}							%Pour forcer la position des images dans une section/sous-section
% \usepackage{lscape}							%Pour changer l'orientation de ce que l'on écrit
\usepackage{rotating}
\usepackage{longtable}							%Pour les tableaux sur plusieurs pages
\usepackage{multirow}							%Pour du multi-ligne dans un tableau
\usepackage{array}							% ??????

%\usepackage{picins}							%Placer les images dans le texte
\usepackage{wrapfig}							%Placer les images dans le texte
\usepackage{caption}
\usepackage{subcaption}

\usepackage{textcomp}							% ??????
\usepackage{color}							%Pour mettre de la couleur

\usepackage{lmodern}							%Pour certaine fonte française
\usepackage[top=2cm, bottom=2cm, left=2cm, right=3cm]{geometry}		%Pour changer la géométrie des pages
\usepackage{listings}

\usepackage{fancyhdr}							%Pour les entêtes
\usepackage[francais]{minitoc}						%Pour les sommaires en début de chapitre/section
\usepackage[toc,lof,lot]{multitoc}					%Pour la table des matières, listes, tables sur plusieurs colonne
%\usepackage[babel=true,kerning=true]{microtype}

%\usepackage{fancybox}							%Pour encadrer les équations

\usepackage{xspace}							%Pour gérer correctement les espaces dans les commandes contenant du texte.

%JP
%\usepackage{ucs}							% ??????

%\usepackage{inputx}							%Pour pouvoir modifier le path dans lequel \input et \include recherche les fichiers.
%\usepackage{subfiles}

\usepackage[french,colorinlistoftodos]{todonotes}
\usepackage{pdfpages}


\usepackage[pdftex]{hyperref}					%Pour les liens hypertexte des glossaires, sommaires, formules, ....

\usepackage{natbib}

\definecolor{couleurPostIt}{rgb}{.9,.9,.35}
\tikzset{fondPostit/.style={color= couleurPostIt}}
\tikzset{ombrePunaise/.style={color={blue!10!gray}}}
\tikzset{ombrePostit/.style={color={black},opacity=.5}}
\tikzset{punaise/.style={ball color=red}}
\newcommand{\epingle}[3]{
\coordinate[rotate=#2,yshift={#3*0.375cm}] (e) at #1;
\coordinate[shift={++(60:0.75)}] (g) at (e);
\begin{scope} [scale=1.5]
 \begin{scope}[rotate=-30]
   \coordinate[shift={++(30:0.75)}] (h) at (e);
   \draw[ombrePunaise,line cap=round,line width=4pt] (e) -- ++(60:0.75);
   \fill [ombrePunaise,rotate=-30,scale=0.5] (h) ellipse (.65 and .3) ;
   \fill [ombrePunaise,rotate=60,scale=0.5] (h) ++(0.4,0) ellipse (.4 and .3);
   \fill [ombrePunaise,rotate=60,scale=0.5] (h) ++(0.8,0) ellipse (.2 and .4);
 \end{scope}
 \draw[line cap=round,line width=4pt] (e) -- ++(60:0.75);
 \fill [punaise,rotate=-30,scale=0.5] (g) ellipse (.65cm and .3cm) ;
 \fill [punaise,rotate=60,scale=0.5] (g) ++(0.4,0) ellipse (.4 and .3);
 \fill [punaise,rotate=60,scale=0.5] (g) ++(0.8,0) ellipse (.2 and .4);
\end{scope}}
\tikzset{zlevel/.style={%
    execute at begin scope={\pgfonlayer{#1}},
        execute at end scope={\endpgfonlayer}
}}
\newlength{\mypostx}
\newcommand{\postit}[2]{%
    \setlength{\mypostx}{\marginparwidth}%
    \marginpar{%
    \begin{tikzpicture}[overlay,remember picture]%
        \begin{scope}%
            \node[opacity=0.7,inner sep=1em,rotate=#2,drop shadow,
                text width=\marginparwidth-2em](block)
                at (0.5\marginparwidth,0.){\color{black}#1};
            \fill[yellow!50!white] (block.north west) -- (block.north east)
                .. controls ($ (block.north east)+1.5*(block.west)$) ..
                  ($ 0.9*(block.south east) $)
                  .. controls ($ (block.south)$) ..
                  (block.south west) -- cycle;
            \node[inner sep=1em,rotate=#2,
                text width=\marginparwidth-2em]
                at (0.5\marginparwidth,0.){\color{black}#1};
            \epingle{($ (block.north)-(0.,1em) $)}{5}{0.5};
        \end{scope}%
    \end{tikzpicture}%
    }
}
\hypersetup{
      backref       = true,
      pagebackref   = true,
      hyperindex    = true,
      colorlinks    = true,
      breaklinks    = true,
      urlcolor      = black,
      linkcolor     = black,
      citecolor     = black,
      bookmarks     = true,
      bookmarksopen = true,
      pdfstartview  = FitH,
      pdfauthor     = {Plum Guillaume},
      pdftitle      = {Systeme auto gravitant},
      pdfsubject    = {Sphère isotherme},
      pdfkeywords   = {sphere, isotherme, diagramme, milne, stabilite, king, model, modèle},
      pdfcreator    = {PDFLaTeX},
      pdfproducer   = {PDFLaTeX}
}

% Title Page
\title{Les systèmes auto-gravitant}
\author{Plum Guillaume}
\date{\today}

%Commande pour tracer avec gnuplot dans latex :
%\makeatletter
%\newcommand{\ExecGnuPlot}[1]{
%	\immediate\write18{/usr/bin/gnuplot #1}}
%\makeatother

%Définition de nouvelle commande afin de faciliter l'écriture :
%\newcommand{\x}[1]{\ensuremath{#1'}}  %\dfrac{d #1}{d x}}}
%\newcommand{\R}{\ensuremath{\mathcal{R}}}
%\renewcommand{\(}{\ensuremath{\left(}}
%\renewcommand{\)}{\ensuremath{\right)}}
\newcommand{\x}[1]{\ensuremath{#1'}}  %\dfrac{d #1}{d x}}}
\newcommand{\R}{\ensuremath{\mathcal{R}}}
\newcommand{\dr}{\ensuremath{\mathrm{dr}}}
\newcommand{\dx}[1]{\ensuremath{\mathrm{d#1}}}
\newcommand{\vdx}[1]{\ensuremath{\mathrm{d}#1}}
%\newcommand{\ddp}{\ensuremath{\mathrm{dp}}}
\newcommand{\vdr}{\ensuremath{\mathrm{d}\vec{r}}}
\newcommand{\vdp}{\ensuremath{\mathrm{d}\vec{p}}}
\newcommand{\erf}{\ensuremath{\mathrm{erf}}}
\newcommand{\pint}{\displaystyle{\int}}
\newcommand{\deriv}[2]{\ensuremath{\dfrac{\mathrm{d}#1}{\mathrm{d#2}}}}
\newcommand{\pderiv}[2]{\ensuremath{\dfrac{\partial#1}{\partial\mathrm{#2}}}}
\newcommand{\pderivn}[2]{\ensuremath{\dfrac{\partial#1}{\partial#2}}}
\newcommand{\King}{\textsc{King}\xspace}

\newcommand{\sm}{saute-mouton\xspace}
\newcommand{\Sm}{Saute-mouton\xspace}

\newcommand{\CH}[1]{\textsc{CH}$#1$\xspace}
\newcommand{\CHa}{\textsc{CH}$\alpha$\xspace}

\newcommand{\ddp}{\ensuremath{\mathrm{dp}}}
\renewcommand{\(}{\ensuremath{\left(}}
\renewcommand{\)}{\ensuremath{\right)}}
% \newcommand{\erf}[1]{\mathrm{erf}\(#1\)}
% \DeclareMathOperator{\erf}{erf}

\newcommand{\mnras}{Mon. Not. R. Astr. Soc.}
\newcommand{\apj}{ApJ}
\newcommand{\apjl}{ApJL}
\newcommand{\apjs}{Astrophysical Journal Supplement}
\newcommand{\pasp}{Publication of the Astronomical Society of the Pacific}
\newcommand{\aap}{Astronomy \& Astrophysics}
\newcommand{\aj}{Astronomical Journal}
\newcommand{\nat}{Nature}
\newcommand{\physrep}{Physics Reports}
\newcommand{\pasj}{PASJ}
\newcommand{\na}{New Ast}

\newcommand{\ra}{\ensuremath{r_{10\%}}}
\newcommand{\rb}{\ensuremath{r_{50\%}}}
\newcommand{\rc}{\ensuremath{r_{90\%}}}
\newcommand{\cercle}[2][0.2]{\tikz[scale=#1] \fill[#2] (0, 0) circle (1);}
\newcommand{\accretionpeu}[1][0.10]{\cercle[#1]{blue}}
\newcommand{\accretionmoyen}[1][0.10]{\cercle[#1]{color=green!40!black}}
\newcommand{\accretionlot}[1][0.10]{\cercle[#1]{red}}
\newcommand{\contraction}[1][0.10]{\cercle[#1]{color=blue!50!white}}
\newcommand{\cerclewave}{%
	\begin{tikzpicture}[scale=0.2]%
		\draw (0, 0) circle (1);%
		% \draw[decorate, decoration={coil}] (1, 0) -- (1.5, 0);%
		\draw[decorate,decoration={expanding waves,angle=7}] (1.5, 0) -- (2.5, 0);%
	\end{tikzpicture}%
}

\newcommand{\refeq}[1]{$\left(\ref{#1}\right)$}


%Quelque compteur pour les tables des matiéres et minitoc
\setcounter{tocdepth}{2}
\setcounter{minitocdepth}{2}
\setcounter{secnumdepth}{3}
%Nombre minimum de ligne à afficher avant de couper le tableau :
\setcounter{LTchunksize}{30}

%\inputpaths{annexe,simulation,theorie,introduction}
\graphicspath{{graphe/}{annexe/}{simulation/}{theorie/}{introduction/}{graphe/vlasov_gadget/}}

\newcommand{\TCarre}[2]{\coordinate (Xside) at (#1, 0);
	\coordinate (Yside) at (0 , #1);

	\coordinate (A) at ($ #2 -1/2*(Xside) - 1/2*(Yside) $);
	\coordinate (B) at ($ (A) + (Xside) $);
	\coordinate (C) at ($ (B) + (Yside) $);
	\coordinate (D) at ($ (C) - (Xside) $);

	\draw (A) -- (B) -- (C) -- (D) -- (A) -- cycle;
}

\newlength{\plarg}
\setlength{\plarg}{14cm}
\newlength{\glarg}
\setlength{\glarg}{17cm}

\begin{document}
	\dominitoc
	\begin{titlepage}
	\thispagestyle{empty}
	\noindent
	\begin{minipage}{\linewidth}
		\includegraphics[scale=0.2]{./logo/logo_ENSTA.png} \hfill \includegraphics[scale=0.1]{./logo/logo-obspm.png} \hfill \includegraphics[scale=0.5]{./logo/logo-iap.png} \\
		{\rule{\linewidth}{1pt}}
		\begin{center}
			\begin{tabular}{p{13cm}r}
				\hspace{-2mm}\textbf{Institut d'Astrophysique de Paris} \\
				\hspace{-2mm}\textbf{Unité de mathématiques appliquées de l'ENSTA} \\
			\end{tabular}
		\end{center}
	\end{minipage}

	\begin{center}
		\begin{minipage}{\plarg}
			\vspace{2cm} \centering
			{\huge\bfseries Dynamique d'une sphère autogravitante isotherme et tronquée }\\ \vspace{1cm}
			Thèse présentée devant\\ \vspace{2mm}
			{\large\bfseries l'Observatoire de Paris-Meudon}\\ \vspace{1cm}
			pour obtenir le grade de \\ \vspace{2mm}
			{\large\bfseries Docteur ès Astrophysique}\\ \vspace{1cm}
			par \\ \vspace{2mm}
			{\large\bfseries Guillaume Plum}\\ \vspace{1cm}
			soutenue le 29 septembre 2014 devant le jury composé de \vspace{2mm}
			{\rule{\plarg}{1pt}} \vspace{3mm}
			\begin{tabular}{p{5cm}p{5cm}p{4cm}}
				\vfill Françoise Combes & \vfill Présidente \\
				Pierre-Henri Chavanis & Rapporteur \\
				Daniel Pfenniger & Rapporteur \\
				Julien Devriendt & Examinateur \\
				Stéphane Colombi & Directeur \\
				Jérôme Pérez & Directeur
			\end{tabular}
		\end{minipage}
	\end{center}

	\vfill \hspace{-1cm}
	\begin{minipage}{\glarg}
		{\rule{\glarg}{1pt}}
		Année 2014 \hfill École Doctorale Astronomie et Astrophysique d'Île-de-France
	\end{minipage}
\end{titlepage}


	\cleardoublepage

	\pagenumbering{roman}

	\chapter*{Remerciements}
	\addstarredchapter{Remerciements}
		Avant tout, je voudrais remercier mes deux directeurs de thèse: Jérôme Pérez et Stéphane Colombi pour
		m'avoir permis de faire cette thèse dont le sujet était si intéressant. Merci à tous les deux de votre
		disponibilité, de votre soutien, de votre aide, et surtout de votre patience lors de la difficile
		rédaction (inventer une nouvelle langue n'est jamais facile)!
		
		Je voudrais ensuite remercier Daniel Pfenniger et Pierre-Henri Chavanis pour avoir accepté d'être
		rapporteurs de ma thèse. Merci à Françoise Combes et Julien Devriendt pour avoir fait partie
		du jury.

		Je remercie les deux maîtres de l'informatique de l'UMA: Christophe Mathulik pour son aide sur le
		cluster et pour toutes nos discussions sur les jeux vidéos, l'informatique, etc; et à Maurice Diamantini
		pour nos discussions portant aussi sur l'informatique, mais pas que, et les différents langages de
		programmation que tu m'as fait découvrir. Un grand merci à mon prédécesseur Nicolas Kielbasiewicz pour
		ses conseils et nos longues discussions.

		Un grand merci à tous les doctorants de l'UMA et de l'IAP pour leur accueil et l'ambiance qu'ils mettent
		dans le laboratoire, il y a trop de monde pour que je puisse tous vous citer, mais je pense à vous tous! Plus
		particulièrement, merci à Geoffrey et Laure pour m'avoir proposé leur soutien lors de cette fin de
		thèse, au moment où j'en avais besoin.

		Merci aux membres de l'UMA pour l'ambiance très agréable et accueillante qui règne dans le laboratoire.

		Je me dois de remercier grandement mon co-bureau de l'IAP, Manuel Duarte, qui passa sa thèse à tenter de
		me traumatiser mais que j'ai irrémédiablement attiré du côté lumineux de la force!

		Un énorme merci à mes vieux amis, que j'affectionne beaucoup, que sont Antoine, Ann'sophie et Virgile
		pour leur soutien indéfectible et toutes nos soirées et sorties!

		Pour terminer, je remercie ma famille qui m'a soutenu et encouragé jusqu'au bout, me permettant de me
		consacrer pleinement à ma thèse!

	\chapter*{Résumé}
	\addstarredchapter{Résumé}
		Le thème général de la thèse est la dynamique des
		systèmes autogravitants tels que les amas globulaires et les galaxies.
		L'objectif est de confronter dans ce contexte, les faits observationnels aux
		modèles théoriques et aux résultats de simulations numériques.
		
		Nous nous sommes intéressés dans un premier temps à l'évolution des
		amas globulaires de notre galaxie. L'étude de l'évolution des profils de
		densité nous a appris
		qu'en plus de l'augmentation bien connue de leur densité centrale, ces
		amas présentent également une évolution de la pente de leur halo au
		cours de leur évolution dynamique dans la galaxie.

		Nous avons alors entrepris l'étude générale de différents modèles
		théoriques de sphères isothermes et plus particulièrement de leur
		stabilité. Cette étude nous a permis d'étendre à un modèle de King
		simplifié certains résultats de stabilité habituellement présentés dans
		le cadre exclusif de la sphère isotherme. Ces tests se sont révélés à
		la fois concluants et instructifs.

		Dans l'objectif de la confirmation de ces résultats analytiques et de ces
		constats observationnels, nous avons entrepris des simulations numériques. Le
		code choisi pour effectuer ces simulations est le treecode public Gadget-2.
		Outre sa grande efficacité et sa compatibilité parfaite avec notre problème,
		nous avons profité de ces simulations pour entreprendre des tests comparatifs
		avec un code Vlasov montrant un accord spectaculaire entre l'approche $N$-corps et l'approche fluide.

		Les simulations que nous avons entreprises consistaient à placer un système
		autogravitant au sein d'un bain thermique, les deux systèmes étant constitués
		de particules et possédant des caractéristiques variables. Le système
		autogravitant d'étude était selon les cas une sphère de Hénon subissant un
		effondrement et poursuivant son évolution sur plusieurs centaines de temps
		dynamiques ou une sphère de King initialement à l'équilibre.

		Bien que nous n'ayons pas réussi pour le moment, d'un point de vue numérique, à mettre réellement en évidence l'instabilité
		suggérée par nos calculs théoriques, nos simulations nous ont permis d'obtenir les résultats suivants:
		\begin{itemize}
			\item nous avons pu reproduire numériquement l'évolution de la pente et l'effondrement du c\oe ur du système en quasi-équilibre obtenu après l'effondrement d'une sphère de Henon dans un bain thermique;
			\item dans certains cas compatibles avec la théorie, nous avons pu faire apparaitre une instabilité d'orbite radiale contrôlée par l'accrétion d'une partie du bain thermique.
		\end{itemize}

		% Malgré certains résultats encourageants,  nous n'avons pas encore réussi, d'un
		% point de vue numérique,  à mettre réellement en évidence l'instabilité suggérée
		% par nos calculs théoriques.
		

	% \pagenumbering{roman}
	\small \tableofcontents
	\addstarredchapter{Table des matières}
	\listoffigures
	\addstarredchapter{Liste des Figures}
	% \listoftables
	% \addstarredchapter{Liste des Tables}
	\normalsize

	\newpage

	% \vspace*{\stretch{1}}
	\section*{Notation}
		Dans ce document, nous noterons:
		\begin{itemize}
			\item $T \equiv <v^2>$ (quand nous parlerons de température, il sera question de la dispersion de vitesse du système),
			\item $\vec{a}$ les vecteurs,
			\item $a$ le module de $\vec{a}$,
			\item $d^3 a$ ou $d^3 \vec{a}$ l'élément de volume associé à $\vec{a}$.
		\end{itemize}
	% \vspace*{\stretch{1}}

	\pagenumbering{arabic}

	\part{Introduction}
	% \addstarredpart{Introduction}
	%Pour le header et les pieds de pages :
	\fancypagestyle{these}{
		% \fancyhf{}
		\fancyhead{}
		\fancyhead[LE]{\rightmark}
		\fancyhead[RO]{\leftmark}
	}
	\pagestyle{these}
	%\lhead{}
	%\chead{}
	%\rhead{}
	%\lfoot{}
	%\cfoot{\thepage}
	%\rfoot{}
	% \renewcommand{\chaptermark}[1]{\markboth{\MakeUppercase{\chaptername}\ \thechapter.\ #1}{}}
	%\renewcommand{\sectionmark}[1]{\markboth{\MakeUppercase{\sectionname}\ \small \textsc{\thesection}.\ #1}{}}

%		\addstarredpart{Introduction}
		Décrire ce qui a été fait, ce qui est à faire, comment tout à été fait, ...

En gros, faire un état des lieux en utilisant la bibliographie.
\normalsize

\chapter{Propriétés générales des systèmes auto-gravitants}

	J'ai mis ce chapitre dans l'intro, il permet de décrire ce que l'on sait des
	amas globulaires. Il faut éviter toute référence précise au modèle de King.
	L'idée est de montrer l'évolution de la pente avec l'âge, et les deux
	catégories d'amas avec ou sans c\oe ur. Il faudrait rajouter une section sur
	les galaxies, voire les amas de galaxies en utilisant l'article de revue de
	Merritt.

	\minitoc
	\section{Amas Globulaire}
		\subsection{Historique}
			Les premiers amas d'étoiles ont été catalogués par \textsc{Messier} en 1784,
			mais ils n'ont été identifiés comme tel qu'en 1814 par William
			\textsc{Herschel}. Étudiés depuis cette époque, notre connaissance
			observationnelle à leur propos n'a pas cessé de s'améliorer avec la progression
			des techniques d'observation. Le premier comptage d'étoiles complet fut
			effectué par \textsc{Bailey} en 1893. En 1905, \textsc{Plummer} et \textsc{von
			Zeipel}	ont utilisé les observations pour remonter à la distribution radiale
			des étoiles. \textsc{Von Zeipel} fit alors le rapprochement entre un amas et
			une sphère de gaz à l'équilibre isotherme. Parallèle encore utilisé
			aujourd'hui, bien qu'il soit contestable sur au moins un point : le libre
			parcours moyen d'une étoile est grand devant la taille du système, alors que
			pour une sphère de gaz c'est l'inverse.

		\subsection{Définition d'un amas globulaire}
			Dans notre galaxie, un amas globulaire est, en général, décrit comme un
			très vieil amas d'étoiles, âgé d'environ 10 milliards d'années. Mais
			l'âge absolu de ces objets est très difficile à mesurer: c'est un sujet
			toujours controversé.

			Les amas de notre galaxie présentent des caractéristiques très variées. Par
			exemple, l'amas le plus massif de notre galaxie --~$\omega$ Centauri~--
			possède une masse d'environ $5.10^6 M_\odot$ tandis que celle du moins
			massif --~AM-4~--est d'environ $10^3 M_\odot$. La magnitude
			absolue~\footnote{De $M_V = -10.1$ pour $\omega$ Centauri à $M_V = -1.7$
			pour AM-4} de ces objets et leur distance au centre de la
			galaxie~\footnote{environ 20 kpc pour les plus proche du centre à 120 kpc
			pour AM-1} varie également dans de grandes proportions.

		\subsection{Répartition}

			Nous avons dénombré quelque(s) centaines d'amas globulaires dans
			notre galaxie, et nous continuons à en découvrir (voir par
			exemple~\cite{2014ApJ...786L...3L}) dans notre galaxie, mais aussi dans la galaxie d'Andromède.

			Les amas sont essentiellement répartis dans les régions proches du
			bulbe de la Voie Lactée ou dans son Halo.
			La figure~\ref{Fig::Intro::repartition} montre la répartition des amas
			globulaires se trouvant dans le catalogue de référence à leur propos: le
			catalogue de Harris~\cite{Harris}. Ce catalogue recense tout les paramètres
			de 150 amas de notre galaxie.
			%Selon~\cite{MH-AAR1997}, ils semblent se répartir en 2 groupes:
			%\begin{itemize}
				%\item le premier formant un halo autour de la galaxie,
				%\item le second formant plutôt un disque.
			%\end{itemize}

			\begin{figure}[h]
				\begin{minipage}{0.32\textwidth}
					\begin{center}
						\includegraphics[width=\linewidth]{plan_xOy_GC.pdf}
					\end{center}
				\end{minipage}\hfill
				\begin{minipage}{0.32\textwidth}
					\begin{center}
						\includegraphics[width=\linewidth]{plan_xOz_GC.pdf}
					\end{center}
				\end{minipage}\hfill
				\begin{minipage}{0.32\textwidth}
					\begin{center}
						\includegraphics[width=\linewidth]{plan_yOz_GC.pdf}
					\end{center}
				\end{minipage}
				\caption{\label{Fig::Intro::repartition}Répartition des amas globulaires connus dans notre galaxie. Les graphiques utilisent le système de coordonné galactique (et donc centré sur le soleil).}
				\todo[inline]{Il y a aussi la figure 4.1 du Meylan \& al, mais on y
					voit pas de structure tel qu'ils les décrivent...}
			\end{figure}


		\subsection{Profil}

			L'étude des propriétés physiques des amas globulaires s'étale aujourd'hui
			sur plus d'un siècle d'observation et de modélisation. Une caractéristique
			commune aux amas est leur forme sphérique, l'ellipticité maximale observée
			étant de l'ordre de 10\% et s'explique par une faible rotation solide. Un
			consensus est maintenant établi sur le fait que leur profil de densité
			volumique de masse (abrégé densité et noté $\rho(r)$) est un excellent
			traceur de leur évolution. Les amas globulaires se répartissent en deux
			catégorie: 80\% des amas présentent une densité caractérisée par 2 régions:
			le cœur pour lequel la densité est quasiment constante ($\rho(r) \approx
			\mathrm{cte}$) et le halo dans lequel la densité évolue globalement comme
			une loi de puissance ($\rho(r) \propto r^{-\alpha}$), ils ont un profil de
			type cœur-halo (core-halo dans la littérature anglo-saxonne). La densité du
			cœur des 20\% restant est remplacé par une loi de puissance. Ils sont dit
			cœur effondré (core-collapsed).

			La plupart des modèles d'amas globulaire sont issus de la sphère isotherme. Nous en
			aborderont certains au cours de ce document. Parmi ces modèles, un en particulier
			permet d'ajuster le profil de densité des amas globulaires durant une grande
			partie de leur évolution: le modèle de \textsc{King}
			(voir~\cite{1966AJ.....71...64K} et le chapitre~\ref{King::Chapitre}).

			\begin{figure}[h]
				\begin{center}
					\includegraphics[width=\linewidth]{gc_photo}
				\end{center}
				\begin{minipage}{0.45\textwidth}
					\begin{center}
						\includegraphics[width=\linewidth]{M13}
					\end{center}
				\end{minipage}\hfill
				\begin{minipage}{0.45\textwidth}
					\begin{center}
						\includegraphics[width=\linewidth]{M15}
					\end{center}
				\end{minipage}
				\caption{\label{Fig::Intro::images}Deux amas globulaires et leurs profils de luminosité (tiré de~\cite{2010A&A...522A..71J}). (a)M13, un amas ayant un cœur, (b) M15, un amas ayant un cœur dit effondré}
			\end{figure}

		\subsection{Simulation}
			Dès les années 1960, des simulations numériques sont utilisées en plus de la
			théorie afin de comprendre comment évoluent ces objets.
			Plusieurs approches ont été utilisées, les principales sont:
			\begin{itemize}
					\item les simulations Monte-Carlo, elles permettent de
						simuler l'évolution d'un amas de plusieurs millions
						d'étoile sur une grande période de temps en quelques
						jours, en adaptant les propriétés globales des
						orbites des étoiles;
					\item les simulations N-corps modifient à chaque pas de
						temps les positions et vitesses des étoiles. Ces simulations sont beaucoup plus lente.
			\end{itemize}
			La figure~\ref{Fig::Intro::HeggieFigure}, récupéré de la présentation de
			D.~Heggie à Gravasco, montre les amas du catalogue de
			\textsc{Harris} tracés dans le plan temps de relaxation - magnitude intégré
			en bande V. Il a été ajouté par dessus un ensemble de droite qui indique le
			temps que durerait une simulation N-Corps ou Monte-Carlo d'un amas
			globulaire avec ces paramètres. Par exemple, une simulation N-Corps de M4
			prendrais environ 300 ans tandis que son équivalent Monte-Carlo ne prendrais
			que une journée.

			\begin{figure}[h]
				\centering \includegraphics[width=\linewidth]{Heggie_figure.pdf}
				\caption{\label{Fig::Intro::HeggieFigure}...}
			\end{figure}

			Les simulations ont permit de prendre de l'avance sur les observations en
			permettant de confirmer certains phénomènes prévu par la théorie, notamment
			les oscillation gravothermal (voir~\cite{1996ApJ...471..796M}). Ce phénomène
			intervient après l'effondrement du cœur, la taille de ce dernier oscille.

		\subsection[Lien]{Lien entre données et paramètres\label{amas}}
			\subsubsection{Préliminaire}
			%	Ce qui va nous intéresser ici, c'est de pouvoir lier à chaque étape d'évolution d'un amas, et donc à son âge,
			%	une pente.
				%Notre objectif est de trouver une relation entre l'âge d'un amas et la pente de son halo.
				Nous avons parlé dans la section précédente du profil de densité
				comme d'un marqueur de l'évolution des amas globulaires, nous allons
				voir en quoi.
				S'il est possible de trouver, à partir de ces profils, plusieurs
				marqueur d'évolution, celui qui nous intéresse plus particulièrement
				est la pente du halo. Pour vérifier que cette pente évolue bien avec
				l'âge dynamique de l'objet, et comment, nous avons utilisé les
				données du catalogue de \textsc{Harris}~\cite{Harris} qui nous a
				permis d'obtenir l'âge et le profil de densité d'une centaine d'amas
				globulaire.
				
				%Nous avons donc besoin d'une relation entre cette quantité et l'âge de l'amas, si une telle relation existe.
				Nous avons alors utilisé le temps de relaxation donné dans le catalogue de \textsc{Harris}~\cite{Harris}.
				Pour obtenir les pentes des amas, nous avons utilisé les relevés observationnels~\cite{Trager-graphe}. % (~les données ainsi obtenues et utilisées sont dans la table~\ref{pente-Tc:BSP}~).
				Nous avons commencé par calculer les pentes directement sur les courbes avec un double décimètre, n'ayant alors pas pu obtenir les données correspondant aux graphiques.
				Après avoir tracé la pente mesurée en fonction du temps de relaxation à 2 corps, nous avons ajusté la
				courbe ainsi obtenue par une droite d'équation $ \alpha = \mathrm{pente} = a \log_{10}(T_c) + b$ (~graphe~\ref{Pente-lin}~).
				\begin{figure}[hbt!]
					\centering \includegraphics[scale=0.9]{graphe/Pente-Tc.pdf}
					\caption{Évolution des pentes pour différents âges}
					\label{Pente-lin}
				\end{figure}
			%	\begin{table}[hbt!]
			%		\begin{center}
			%			\begin{tabular}{|c|c|c|}
			%				\hline
			%				Coefficient & Valeur & Erreur \\
			%				\hline
			%				\hline
			%				$a$       &        $-1.19506$   &   $\pm 0.1576$ (~$13.19\%$~) \\
			%				\hline
			%				$b$       &        $2.52082$     &   $\pm 1.213$   (~$48.11\%$~) \\
			%				\hline
			%			\end{tabular}
			%		\end{center}
			%		\caption{Valeur des coefficients donnée par l'ajustement pour les pentes}
			%		\label{pente-lin-coeff}
			%	\end{table}

				Cette approche indiquant clairement une relation linéaire entre ces deux paramètres, nous avons décidé d'entreprendre une démarche plus globale et automatique.
				Pour commencer, nous avons donc récupéré les données auprès des auteurs de l'article sur~\cite{TragerTable}. Nous nous sommes alors confronté au problème des unités.

			\subsubsection{Retraitement}
				Le graphique~\ref{Pente-lin} a été obtenu en utilisant~\cite{Trager-graphe} avec ses unités. % (~les données permettant de retracer, nous les avons finalement trouvé,
			%	les graphiques de cet article sont données dans~\cite{TragerTable}~).
				La pente donnée ici n'est donc pas la pente de la densité, mais de la brillance de surface de l'objet en fonction d'un rayon en seconde d'arc.
				Le premier traitement à effectuer consiste donc à transformer cette brillance de surface par arc seconde carrée en une densité (~kilogramme par kilomètre au cube~).

				Une définition de cette brillance de surface, telle qu'elle semble avoir été utilisée, peut-être trouvé dans~\cite{SBP}.
				Comme indiqué dans ce texte, la brillance de surface va s'écrire :
				\begin{align}
					\mu_V = \mu_{\mathrm{ref}} - 2.5 \log_{10}\(\frac{f/\Omega}{f_{\mathrm{ref}}/\Omega_{\mathrm{ref}}}\)
					\label{mu_V}
				\end{align}
				avec $\Omega$ l'angle solide, exprimé en seconde d'arc au carrée, sous lequel nous voyons l'objet.
			%	\begin{align}
			%		\mu_V = m_V - 2.5 \log_{10}\(\frac{(1\mathrm{"})^2}{\Omega}\)
			%		\label{mu-astuce}
			%	\end{align}
			%	avec $m_V$ la magnitude apparente de l'objet sur une seconde d'arc au carrée.

				Cette définition rappelle celle de~\cite{Trager-graphe}, section~3.2.3 :
				\begin{align}
					\mu = -2.5 \log\(\frac{10^{-0.4 m_2} - 10^{-0.4 m_1}}{\pi \(r_2^2 - r_1^2\)}\)
					\label{Trager-eq}
				\end{align}
			%	où ils prendraient comme référence les points autour de celui considéré (~ou quelque chose comme ça~),
				$m_i$ étant la magnitude en un point de l'objet et $r_i$ le rayon en seconde d'arc pour ce point.
				Par conséquent, les unités de $\mu$ sont un flux par arc seconde carrée en échelle logarithmique :
				\begin{enumerate}
					\item $10^{-0.4 m_i}$ est proportionnel à un flux, % (~ou au rapport d'un flux par rapport à un flux de référence !?!~),
					\item $r_i$ est le rayon de l'objet au point $i$ en seconde d'arc,
					\item[$\Rightarrow$] nous avons donc bien notre flux par seconde d'arc au carrée.
				\end{enumerate}
				Pour pouvoir revenir aux quantités que nous cherchons, il va falloir calculer un peu :
				\begin{align}
					\mu = -2.5 \log\(\frac{10^{-0.4 m_2} - 10^{-0.4 m_1}}{\pi \(r_2^2 - r_1^2\)}\) &\equiv -2.5 \log\(\frac{F}{\pi \(r_2^2 - r_1^2\)}\) \text{avec $F$ le flux}\notag \\
					\intertext{par rapport à toute la documentation que j'ai trouvé, le $\log$ correspond ici à $\log_{10}$ et non à $\ln$.}
					\Rightarrow \frac{F}{\pi \(r_2^2 - r_1^2\)} &= 10^{-\mu/2.5} \label{F--mu} \\
					\intertext{Grâce à~\cite{McL}, nous avons les rapports masse luminosité :}
					\Upsilon &= \frac{M}{L} \label{M/L}\\
					\intertext{or}
					F &= \frac{L}{4\pi D^2} \label{def-F}\\
					\intertext{avec $D$ la distance soleil--amas. D'où}
					\frac{L}{4 \pi^2 \(r_2^2 - r_1^2\) D^2} &= \frac{M}{4 \pi^2 \Upsilon \(r_2^2 - r_1^2\) D^2} = 10^{-\mu/2.5} \notag \\
					\intertext{en combinant~\ref{F--mu}, \ref{M/L} et~\ref{def-F}.}
					\Rightarrow \frac{M}{4\pi \(r_2^2 - r_1^2\)} &= \pi\Upsilon D^2 10^{-\mu/2.5} \notag \\
					\intertext{Mais $r_2^2 - r_1^2 \propto r_i^2$ doit être converti en mètre :}
					%r_2^2 - r_1^2 &\propto r_i^2 \Rightarrow \(D \tan(r_i)\)^2 \varpropto \(D r_i\)^2 \notag \\
					\Rightarrow \frac{M}{4\pi \(D \tan(r_i/3600)\)^2} &= \pi\Upsilon D^2 10^{-\mu/2.5} \label{M-don}
				\end{align}
				Normalement, nous avons maintenant une masse surfacique donnée par~\ref{M-don}.
			%	mais, étonnamment, la conversion de seconde d'arc à mètre n'a fait paraître
			%	aucun facteur numérique !!! J'ai peut-être un peu trop truandé.

				Les rapports masse-luminosité peuvent être trouvés dans~\cite{McL}, mais cette article ne contient que 40 des 140 amas de notre galaxie.
			%	mais il n'y a qu'une quarantaine de rapport comparé à notre échantillon d'environ 140 amas.
				Par contre, il est aisé de remarquer que ces rapports sont
				en moyenne très peu différents de la valeur $\Upsilon = 2$ avec une valeur minimum de $1.87$ pour NGC 4147 et une valeur maximum de
				$2.66$ pour NGC 6441. Dans la suite, nous prendrons donc $\Upsilon = 2$.
			%	Nous avons alors utilisé les données du catalogue pour redimensionner le tout, mais pour passer à la densité, nous avons besoin du rapport masse
			%	sur luminosité de l'amas qui peut-être trouvé dans~\cite{McL}.
			%	Cet article ne donne qu'une quarantaine d'amas sur les 150 de notre galaxie, mais les valeurs de ces rapports étant compris entre $1.87$ pour NGC 4147
			%	(~et quelques autres~) et $2.66$ pour NGC 6441, et tournant surtout autour de $2$, nous pouvons supposer que ces rapports sont les mêmes pour chaque amas et
			%	valent $\Upsilon = \frac{M}{L} = 2$.
			%	Selon~\cite{Trager-graphe}, nous avons donc :
			%	\begin{align}
			%		\mu_V &\propto \text{Flux par $m^{-2}$} = \frac{L}{4\pi D^2} \\
			%		\intertext{or}
			%		L &= \frac{M}{\Upsilon} \notag \\
			%		\intertext{donc}
			%		\mu_V &= \frac{M}{4\pi D^2 \Upsilon} \notag \\
			%		M &= \Upsilon \mu_V 4\pi D^2
			%	\end{align}
			%	avec $D$ la distance entre nous et l'amas. Nous pouvons supposer que cette distance est la même quelque soit l'étoile considéré dans l'amas
			%	(~$D_{min} = 3000\mathrm{pc} \gg R_{amas} \approx 10\mathrm{pc}$~).

			%	Du fait des rapports masse-luminosité manquant, notre échantillon d'une quarantaine d'amas est passé à seulement 12 amas.

			\subsubsection{Résultat}
				Le calcul de pente a été refait en utilisant : les données de l'article~\cite{TragerTable}, le traitement décrit ci-dessus, et en ajustant une
				droite pour des densités comprises entre $10^{-4}$ et 0,1.
				Nous obtenons alors le graphique~\ref{Pente-lin_dim}, l'ajustement utilisant la fonction
				$f(T_c) = a \log(T_c) + b$ (~les coefficients ont été donnés dans la table~\ref{pente-lin-coeff_dim}~).
				Nous avons donc maintenant une relation linéaire entre l'âge de l'amas et la pente du halo.

				\begin{figure}[hbt!]
			%		\centering \includegraphics[scale=0.9]{../Amas/Graphe_Rapport/pente-Tc.pdf}
					\centering \includegraphics[scale=0.9]{graphe/pente-Tc.pdf}
					\caption{Évolution des pentes pour différents âges}
					\label{Pente-lin_dim}
				\end{figure}
				\begin{table}[hbt!]
					\begin{center}
						\begin{tabular}{|c|c|}%c|}
							\hline
							Coefficient & Valeur \\ %& Erreur \\
							\hline
							\hline
							$a$       &         $-2.3341$   \\ %&    $\pm 0.2075$      (~$32.34\%$~) \\
							\hline
							$b$       &         $16.913$   \\  %&    $\pm 1.638$       (~$199.5\%$~) \\
							\hline
						\end{tabular}
					\end{center}
					\caption{Valeurs des coefficients donnée par l'ajustement pour les pentes pour le temps de croisement}
					\label{pente-lin-coeff_dim}
				\end{table}

				Nous avons également utilisé le catalogue de \textsc{Harris} pour obtenir le lien entre la pente du halo et le temps dynamique
				grâce à l'équation~\ref{Td:sig}. Nous obtenons alors le graphique~\ref{Pente-Td-lin}.
				\begin{figure}[hbt!]
			%		\centering \includegraphics[scale=0.9]{../Amas/Graphe_Rapport/pente-Td.pdf}
					\centering \includegraphics[scale=0.9]{graphe/pente-Td.pdf}
					\caption{Évolution des pentes pour différents âges dynamiques}
					\label{Pente-Td-lin}
				\end{figure}
				\begin{table}[hbt!]
					\begin{center}
						\begin{tabular}{|c|c|}%c|}
							\hline
							Coefficient & Valeur \\ %& Erreur \\
							\hline
							\hline
							$a$       &        $-1.6567$   \\ %&   $\pm 0.6523$      (~$278.9\%$~) \\
							\hline
							$b$       &        $1.8927$     \\ %&   $\pm 2.825$       (~$88.57\%$~) \\
							\hline
						\end{tabular}
					\end{center}
					\caption{Valeurs des coefficients donnée par l'ajustement pour les pentes pour le temps dynamique}
					\label{pente-Td-lin-coeff}
				\end{table}

				Sur la figure~\ref{Pente-lin_dim}, nous pouvons noter la présence de 2 points avec des pentes inférieures à $-10$. Ces points correspondent aux amas NGC 5024 et NGC 5139
				(~leurs densités sont donnée en annexe, page~\pageref{Graphe-bofbof}~) pour lesquelles la détermination des pentes n'a pas été très concluante.
				\FloatBarrier

		%\subsection{Préliminaire}
%	Ce qui va nous intéresser ici, c'est de pouvoir lier à chaque étape d'évolution d'un amas, et donc à son âge,
%	une pente.
	Notre objectif est de trouver une relation entre l'âge d'un amas et la pente de son halo.
	%Nous avons donc besoin d'une relation entre cette quantité et l'âge de l'amas, si une telle relation existe.
	Nous avons alors utilisé le temps de relaxation donné dans le catalogue de \textsc{Harris}~\cite{Harris}.
	Pour obtenir les pentes des amas, nous avons utilisé les relevés observationnels~\cite{Trager-graphe}. % (~les données ainsi obtenues et utilisées sont dans la table~\ref{pente-Tc:BSP}~).
	Nous avons commencé par calculer les pentes directement sur les courbes avec un double décimètre, n'ayant alors pas pu obtenir les données correspondant aux graphiques.
	Après avoir tracé la pente mesurée en fonction du temps de relaxation à 2 corps, nous avons ajusté la
	courbe ainsi obtenue par une droite d'équation $ \alpha = \mathrm{pente} = a \log_{10}(T_c) + b$ (~graphe~\ref{Pente-lin}~).
	\begin{figure}[hbt!]
		\centering \includegraphics[scale=0.9]{graphe/Pente-Tc.pdf}
		\caption{Évolution des pentes pour différents âges}
		\label{Pente-lin}
	\end{figure}
%	\begin{table}[hbt!]
%		\begin{center}
%			\begin{tabular}{|c|c|c|}
%				\hline
%				Coefficient & Valeur & Erreur \\
%				\hline
%				\hline
%				$a$       &        $-1.19506$   &   $\pm 0.1576$ (~$13.19\%$~) \\
%				\hline
%				$b$       &        $2.52082$     &   $\pm 1.213$   (~$48.11\%$~) \\
%				\hline
%			\end{tabular}
%		\end{center}
%		\caption{Valeur des coefficients donnée par l'ajustement pour les pentes}
%		\label{pente-lin-coeff}
%	\end{table}

	Cette approche indiquant clairement une relation linéaire entre ces deux paramètres, nous avons décidé d'entreprendre une démarche plus globale et automatique.
	Pour commencer, nous avons donc récupéré les données auprès des auteurs de l'article sur~\cite{TragerTable}. Nous nous sommes alors confronté au problème des unités.

\subsection{Retraitement}
	Le graphique~\ref{Pente-lin} a été obtenu en utilisant~\cite{Trager-graphe} avec ses unités. % (~les données permettant de retracer, nous les avons finalement trouvé,
%	les graphiques de cet article sont données dans~\cite{TragerTable}~).
	La pente donnée ici n'est donc pas la pente de la densité, mais de la brillance de surface de l'objet en fonction d'un rayon en seconde d'arc.
	Le premier traitement à effectuer consiste donc à transformer cette brillance de surface par arc seconde carrée en une densité (~kilogramme par kilomètre au cube~).

	Une définition de cette brillance de surface, telle qu'elle semble avoir été utilisée, peut-être trouvé dans~\cite{SBP}.
	Comme indiqué dans ce texte, la brillance de surface va s'écrire :
	\begin{align}
		\mu_V = \mu_{\mathrm{ref}} - 2.5 \log_{10}\(\frac{f/\Omega}{f_{\mathrm{ref}}/\Omega_{\mathrm{ref}}}\)
		\label{mu_V}
	\end{align}
	avec $\Omega$ l'angle solide, exprimé en seconde d'arc au carrée, sous lequel nous voyons l'objet.
%	\begin{align}
%		\mu_V = m_V - 2.5 \log_{10}\(\frac{(1\mathrm{"})^2}{\Omega}\)
%		\label{mu-astuce}
%	\end{align}
%	avec $m_V$ la magnitude apparente de l'objet sur une seconde d'arc au carrée.

	Cette définition rappelle celle de~\cite{Trager-graphe}, section~3.2.3 :
	\begin{align}
		\mu = -2.5 \log\(\frac{10^{-0.4 m_2} - 10^{-0.4 m_1}}{\pi \(r_2^2 - r_1^2\)}\)
		\label{Trager-eq}
	\end{align}
%	où ils prendraient comme référence les points autour de celui considéré (~ou quelque chose comme ça~),
	$m_i$ étant la magnitude en un point de l'objet et $r_i$ le rayon en seconde d'arc pour ce point.
	Par conséquent, les unités de $\mu$ sont un flux par arc seconde carrée en échelle logarithmique :
	\begin{enumerate}
		\item $10^{-0.4 m_i}$ est proportionnel à un flux, % (~ou au rapport d'un flux par rapport à un flux de référence !?!~),
		\item $r_i$ est le rayon de l'objet au point $i$ en seconde d'arc,
		\item[$\Rightarrow$] nous avons donc bien notre flux par seconde d'arc au carrée.
	\end{enumerate}
	Pour pouvoir revenir aux quantités que nous cherchons, il va falloir calculer un peu :
	\begin{align}
		\mu = -2.5 \log\(\frac{10^{-0.4 m_2} - 10^{-0.4 m_1}}{\pi \(r_2^2 - r_1^2\)}\) &\equiv -2.5 \log\(\frac{F}{\pi \(r_2^2 - r_1^2\)}\) \text{avec $F$ le flux}\notag \\
		\intertext{par rapport à toute la documentation que j'ai trouvé, le $\log$ correspond ici à $\log_{10}$ et non à $\ln$.}
		\Rightarrow \frac{F}{\pi \(r_2^2 - r_1^2\)} &= 10^{-\mu/2.5} \label{F--mu} \\
		\intertext{Grâce à~\cite{McL}, nous avons les rapports masse luminosité :}
		\Upsilon &= \frac{M}{L} \label{M/L}\\
		\intertext{or}
		F &= \frac{L}{4\pi D^2} \label{def-F}\\
		\intertext{avec $D$ la distance soleil--amas. D'où}
		\frac{L}{4 \pi^2 \(r_2^2 - r_1^2\) D^2} &= \frac{M}{4 \pi^2 \Upsilon \(r_2^2 - r_1^2\) D^2} = 10^{-\mu/2.5} \notag \\
		\intertext{en combinant~\ref{F--mu}, \ref{M/L} et~\ref{def-F}.}
		\Rightarrow \frac{M}{4\pi \(r_2^2 - r_1^2\)} &= \pi\Upsilon D^2 10^{-\mu/2.5} \notag \\
		\intertext{Mais $r_2^2 - r_1^2 \propto r_i^2$ doit être converti en mètre :}
		%r_2^2 - r_1^2 &\propto r_i^2 \Rightarrow \(D \tan(r_i)\)^2 \varpropto \(D r_i\)^2 \notag \\
		\Rightarrow \frac{M}{4\pi \(D \tan(r_i/3600)\)^2} &= \pi\Upsilon D^2 10^{-\mu/2.5} \label{M-don}
	\end{align}
	Normalement, nous avons maintenant une masse surfacique donnée par~\ref{M-don}.
%	mais, étonnamment, la conversion de seconde d'arc à mètre n'a fait paraître
%	aucun facteur numérique !!! J'ai peut-être un peu trop truandé.

	Les rapports masse-luminosité peuvent être trouvés dans~\cite{McL}, mais cette article ne contient que 40 des 140 amas de notre galaxie.
%	mais il n'y a qu'une quarantaine de rapport comparé à notre échantillon d'environ 140 amas.
	Par contre, il est aisé de remarquer que ces rapports sont
	en moyenne très peu différents de la valeur $\Upsilon = 2$ avec une valeur minimum de $1.87$ pour NGC 4147 et une valeur maximum de
	$2.66$ pour NGC 6441. Dans la suite, nous prendrons donc $\Upsilon = 2$.
%	Nous avons alors utilisé les données du catalogue pour redimensionner le tout, mais pour passer à la densité, nous avons besoin du rapport masse
%	sur luminosité de l'amas qui peut-être trouvé dans~\cite{McL}.
%	Cet article ne donne qu'une quarantaine d'amas sur les 150 de notre galaxie, mais les valeurs de ces rapports étant compris entre $1.87$ pour NGC 4147
%	(~et quelques autres~) et $2.66$ pour NGC 6441, et tournant surtout autour de $2$, nous pouvons supposer que ces rapports sont les mêmes pour chaque amas et
%	valent $\Upsilon = \frac{M}{L} = 2$.
%	Selon~\cite{Trager-graphe}, nous avons donc :
%	\begin{align}
%		\mu_V &\propto \text{Flux par $m^{-2}$} = \frac{L}{4\pi D^2} \\
%		\intertext{or}
%		L &= \frac{M}{\Upsilon} \notag \\
%		\intertext{donc}
%		\mu_V &= \frac{M}{4\pi D^2 \Upsilon} \notag \\
%		M &= \Upsilon \mu_V 4\pi D^2
%	\end{align}
%	avec $D$ la distance entre nous et l'amas. Nous pouvons supposer que cette distance est la même quelque soit l'étoile considéré dans l'amas
%	(~$D_{min} = 3000\mathrm{pc} \gg R_{amas} \approx 10\mathrm{pc}$~).

%	Du fait des rapports masse-luminosité manquant, notre échantillon d'une quarantaine d'amas est passé à seulement 12 amas.

\subsection{Résultat}
	Le calcul de pente a été refait en utilisant : les données de l'article~\cite{TragerTable}, le traitement décrit ci-dessus, et le critère donné dans la section~\ref{pente-critére}.
	Nous obtenons alors le graphique~\ref{Pente-lin_dim}, l'ajustement utilisant la fonction
	$f(T_c) = a \log(T_c) + b$ (~les coefficients ont été donnés dans la table~\ref{pente-lin-coeff_dim}~).
	Nous avons donc maintenant une relation linéaire entre l'âge de l'amas et la pente du halo.

	\begin{figure}[hbt!]
%		\centering \includegraphics[scale=0.9]{../Amas/Graphe_Rapport/pente-Tc.pdf}
		\centering \includegraphics[scale=0.9]{graphe/pente-Tc.pdf}
		\caption{Évolution des pentes pour différents âges}
		\label{Pente-lin_dim}
	\end{figure}
	\begin{table}[hbt!]
		\begin{center}
			\begin{tabular}{|c|c|}%c|}
				\hline
				Coefficient & Valeur \\ %& Erreur \\
				\hline
				\hline
				$a$       &         $-2.3341$   \\ %&    $\pm 0.2075$      (~$32.34\%$~) \\
				\hline
				$b$       &         $16.913$   \\  %&    $\pm 1.638$       (~$199.5\%$~) \\
				\hline
			\end{tabular}
		\end{center}
		\caption{Valeurs des coefficients donnée par l'ajustement pour les pentes pour le temps de croisement}
		\label{pente-lin-coeff_dim}
	\end{table}

	Nous avons également utilisé le catalogue de \textsc{Harris} pour obtenir le lien entre la pente du halo et le temps dynamique
	grâce à l'équation~\ref{Td:sig}. Nous obtenons alors le graphique~\ref{Pente-Td-lin}.
	\begin{figure}[hbt!]
%		\centering \includegraphics[scale=0.9]{../Amas/Graphe_Rapport/pente-Td.pdf}
		\centering \includegraphics[scale=0.9]{graphe/pente-Td.pdf}
		\caption{Évolution des pentes pour différents âges dynamiques}
		\label{Pente-Td-lin}
	\end{figure}
	\begin{table}[hbt!]
		\begin{center}
			\begin{tabular}{|c|c|}%c|}
				\hline
				Coefficient & Valeur \\ %& Erreur \\
				\hline
				\hline
				$a$       &        $-1.6567$   \\ %&   $\pm 0.6523$      (~$278.9\%$~) \\
				\hline
				$b$       &        $1.8927$     \\ %&   $\pm 2.825$       (~$88.57\%$~) \\
				\hline
			\end{tabular}
		\end{center}
		\caption{Valeurs des coefficients donnée par l'ajustement pour les pentes pour le temps dynamique}
		\label{pente-Td-lin-coeff}
	\end{table}

	Sur la figure~\ref{Pente-lin_dim}, nous pouvons noter la présence de 2 points avec des pentes inférieures à $-10$. Ces points correspondent aux amas NGC 5024 et NGC 5139
	(~leurs densités sont donnée en annexe, page~\pageref{Graphe-bofbof}~) pour lesquelles la détermination des pentes n'a pas été très concluante.
	\FloatBarrier



	%\section{Interprétation}

%Nous avons maintenant une équation liant la valeur de la pente au temps de relaxation : $ \mathrm{pente} = \alpha = d * \log_{10}(T_c) + e $ (~coefficients table~\ref{pente-lin-coeff_dim}~)
%et une autre liant la pente à $W_0$ : $ \alpha = a e^{ b W_0 } + c $.
%En combinant ces deux équations, nous pouvons obtenir une relation entre le temps de
%relaxation et $W_0$ :
%\begin{align}
	%\mathrm{pente} &= d \log_{10}(T_c) + e = a e^{b W_0} + c \notag \\
	%\Rightarrow W_0 &= \frac{1}{b} \ln\( \frac{d \log_{10}(T_c) + e - c}{a} \) \label{Tc:W0->fct}
%\end{align}
%avec :
%\begin{table}[h!]
	%\begin{center}
		%\begin{tabular}{|c|r|}
			%\hline
			%Coefficient	&	Valeur \\
			%\hline
			%\hline
			%$a$		&	$ -10.0698 $ \\
				%\hline
			%$b$		&	$ 0.220152 $ \\
			%\hline
			%$c$		&	$ -1.53409 $ \\
			%\hline
			%$d$		&	$ -2.3341 $ \\
			%\hline
			%$e$		&	$ 16.913 $ \\
			%\hline
		%\end{tabular}
	%\end{center}
%\end{table}

%Le comportement que nous avions observé en étudiant le modèle de \textsc{King}, à savoir une évolution de la pente avec $W_0$, se retrouve avec l'âge de notre amas. Ce qui nous a permis de relier
%l'âge et $W_0$.
%Il nous reste à interpreter ces résultats, ce que nous allons faire dans la section suivante.
		%Nous avons maintenant une équation liant la valeur de la pente au temps de relaxation : $ \mathrm{pente} = \alpha = d * \log_{10}(T_c) + e $ (~coefficients table~\ref{pente-lin-coeff_dim}~)
et une autre liant la pente à $W_0$ : $ \alpha = a e^{ b W_0 } + c $.
En combinant ces deux équations, nous pouvons obtenir une relation entre le temps de
relaxation et $W_0$ :
\begin{align}
	\mathrm{pente} &= d \log_{10}(T_c) + e = a e^{b W_0} + c \notag \\
	\Rightarrow W_0 &= \frac{1}{b} \ln\( \frac{d \log_{10}(T_c) + e - c}{a} \) \label{Tc:W0->fct}
\end{align}
avec :
\begin{table}[h!]
	\begin{center}
		\begin{tabular}{|c|r|}
			\hline
			Coefficient	&	Valeur \\
			\hline
			\hline
			$a$		&	$ -10.0698 $ \\
				\hline
			$b$		&	$ 0.220152 $ \\
			\hline
			$c$		&	$ -1.53409 $ \\
			\hline
			$d$		&	$ -2.3341 $ \\
			\hline
			$e$		&	$ 16.913 $ \\
			\hline
		\end{tabular}
	\end{center}
\end{table}

Le comportement que nous avions observé en étudiant le modèle de \textsc{King}, à savoir une évolution de la pente avec $W_0$, se retrouve avec l'âge de notre amas. Ce qui nous a permis de relier
l'âge et $W_0$.
Il nous reste à interpreter ces résultats, ce que nous allons faire dans la section suivante.


	\section{Petit Scénario\label{petit_scenar}}

%Nous avons vu que les courbes~\ref{King_Modele-test} peuvent être divisées en deux parties : l'une, caractérisée par une densité constante, constituant un cœur, et l'autre constituant un halo.
%Nous allons donc considérer un amas d'étoile composé d'un cœur dense et d'un halo.
%Le halo a peu de gravité et se comporte comme un gaz parfait.
%Le comportement observé dans notre étude est résumé sur le schéma~\ref{schema-effondrement} :
%\input{figure_tikz-pente.tex}
%(~la pente $-4$ avec laquelle commence le schéma vient des résultats de simulations numériques~).
%La question à laquelle nous souhaitons répondre est : comment expliquer qu'un amas évolu en augmentant la densité de son cœur et en diminuant la pente avec laquelle le halo décroit ?

%Pour commencer, un amas doit, lorsqu'il est à l'équilibre, suivre un modèle de \textsc{King} ou de sphère isotherme en boîte. Par conséquent, notre amas est au Viriel.
%De plus, le cœur est un systéme auto-gravitant décrit par la thermodynamique. L'un de ses propriétés thermodynamique les plus importante à ce niveau est sa capacité calorifique. En effet :
%Nous cherchons à concentrer la matière dans le cœur de l'amas, et donc y rajouter de la matière à partir du halo (~qui va se diluer~).
%Pour faire diminuer le rayon de l'orbite d'un astre, il faut augmenter sa vitesse~\footnote{rappel : $v^2 = K\left(\frac{2}{r} - \frac{1}{a}\right)$ avec $K = G\left(M+m\right)$
%pour des interactions à deux corps}, et donc sa température. En se rappelant les résultats sur les diagrammes d'énergie de la sphère isotherme, nous nous rendons bien compte que, si la température
%augmente trop, le cœur va passer dans la zone instable, et s'effondrer.

%Le processus pour faire augmenter la température auquel nous nous attèlerons dans la suite est assez simple : l'amas évolue dans le potentiel galactique ; il va en traverser le disque de façon périodique.
%Moments pendant lesquels il va se faire harceler par des forces de marée qui vont lui faire perdre des étoiles. En considérant la description micro canonique, l'énergie est fixé ; or perdre une étoile
%diminue l'énergie potentielle de l'amas. La température va devoir augmenter pour conserver l'énergie totale constante.

%La perte d'étoile apparaît alors comme un processus intéressant pour expliquer l'évolution observée. Les chapitres suivants nous permettront de déterminer, à l'aide de simulation numérique, si la perte
%d'étoile par effet de marée est suffisante pour l'expliquer.








		Cette dernière étude nous montre que plus un amas est âgé, plus la pente de son halo est importante.
		Dans le chapitre précédent, nous avons relié la pente au paramètre $W_0$ qui, en plus de jouer sur les pentes, joue sur la densité centrale, comme le montre la figure~\ref{King_Modele-test}.
		Par ailleurs, des simulations partant d'un nuage homogène nous apprennent que la pente du halo après sa formation vaut $-4$.
		Un amas va donc partir d'une structure cœur-halo avec un cœur de taille importante, un halo de pente $-4$ et va tendre, en vieillissant, vers un amas ayant une densité centrale très élevée dont la pente tend vers $-2$.
		Puis après un temps infini, l'amas va tendre vers une sphère isotherme.
		C'est ce que représente le schéma suivant.

		\input{theorie/figure_tikz-pente.tex}

		Pour expliquer cette évolution, nous devons trouver quels phénomènes, de dynamique gravitationnelle, pourraient diluer le halo et concentrer de la matière au centre de l'amas. Le phénomène qui vient le plus naturellement à l'esprit est la perte d'étoiles, pouvant s'effectuer par 2 scénarios :
		\begin{enumerate}
			\item des collisions internes qui éjectent des étoiles de l'amas, permettant ainsi au cœur de s'effondrer en essayant de compenser cette perte d'énergie.
			\item des interactions avec un autre objet massif qui va retirer par effet de marée des étoiles à l'amas, causant l'effondrement de son cœur.
		\end{enumerate}
		C'est ce deuxième scénario que nous allons tenter d'étudier.
















%Si la température cinétique du halo augmente, celle du
%cœur va changer pour s'adapter. La capacité calorifique à volume
%constant est négative pour le cœur (~et pour tout système
%auto gravitant~) :
%\begin{align*}
%	E_p &= -2E_c & \text{(~Viriel~)} \\
%	\Rightarrow E &= E_p + E_c = -E_c \\
%	\intertext{or}
%	E_c &\varpropto T \Rightarrow E\varpropto -T \\
%	\intertext{donc}
%	\Rightarrow C_v &= \frac{\partial E}{\partial T} < 0
%\end{align*}
%Cela implique que notre système ne peut que prendre de l'énergie.

%Si la température du cœur augmente, la vitesse de rotation des
%étoiles va augmenter (~$E_c\varpropto T\varpropto v$~), et donc
%leur demi grand axe va diminuer~\footnote{rappel : $v^2 =
%K\left(\frac{2}{r} - \frac{1}{a}\right)$ avec $K = G\left(M+m\right)$
%pour des interactions à deux corps}.
%En conséquent la densité au centre du cœur augmente, augmentant
%ainsi le contraste de densité. Si la température augmente
%suffisamment, le contraste de densité du halo va dépasser la
%valeur critique $\R_c^H$ (~l'énergie est fixé, seul la
%température change, nous utilisons donc la description micro
%canonique~). Le cœur va alors devenir instable.


%	L'interprétation semble simple ici : par exemple : quand les étoiles
%	ont des vitesses de rotation élevées la sphère a une température élevée,
%	et donc sa capacité calorifique à volume constant, $C_v = \frac{\partial H}{\partial T} < 0$
%	(~négatif car $E_p=-2E_c\Rightarrow H=E_c+E_p=-E_c\varpropto-T$~),
%	tend vers $0$ : elle ne peut plus acquérir d'énergie.
%	Si on dépasse la limite de température, elle va devoir se \og~réorganiser~\fg pour
%	rester à l'équilibre et donc pour retourner à un contraste de densité $\R < \R^\beta_c$.
%	Le raisonnement est le même pour la limite en énergie.
		%%Nous avons vu que les courbes~\ref{King_Modele-test} peuvent être divisées en deux parties : l'une, caractérisée par une densité constante, constituant un cœur, et l'autre constituant un halo.
%Nous allons donc considérer un amas d'étoile composé d'un cœur dense et d'un halo.
%Le halo a peu de gravité et se comporte comme un gaz parfait.
%Le comportement observé dans notre étude est résumé sur le schéma~\ref{schema-effondrement} :
%\input{figure_tikz-pente.tex}
%(~la pente $-4$ avec laquelle commence le schéma vient des résultats de simulations numériques~).
%La question à laquelle nous souhaitons répondre est : comment expliquer qu'un amas évolu en augmentant la densité de son cœur et en diminuant la pente avec laquelle le halo décroit ?

%Pour commencer, un amas doit, lorsqu'il est à l'équilibre, suivre un modèle de \textsc{King} ou de sphère isotherme en boîte. Par conséquent, notre amas est au Viriel.
%De plus, le cœur est un systéme auto-gravitant décrit par la thermodynamique. L'un de ses propriétés thermodynamique les plus importante à ce niveau est sa capacité calorifique. En effet :
%Nous cherchons à concentrer la matière dans le cœur de l'amas, et donc y rajouter de la matière à partir du halo (~qui va se diluer~).
%Pour faire diminuer le rayon de l'orbite d'un astre, il faut augmenter sa vitesse~\footnote{rappel : $v^2 = K\left(\frac{2}{r} - \frac{1}{a}\right)$ avec $K = G\left(M+m\right)$
%pour des interactions à deux corps}, et donc sa température. En se rappelant les résultats sur les diagrammes d'énergie de la sphère isotherme, nous nous rendons bien compte que, si la température
%augmente trop, le cœur va passer dans la zone instable, et s'effondrer.

%Le processus pour faire augmenter la température auquel nous nous attèlerons dans la suite est assez simple : l'amas évolue dans le potentiel galactique ; il va en traverser le disque de façon périodique.
%Moments pendant lesquels il va se faire harceler par des forces de marée qui vont lui faire perdre des étoiles. En considérant la description micro canonique, l'énergie est fixé ; or perdre une étoile
%diminue l'énergie potentielle de l'amas. La température va devoir augmenter pour conserver l'énergie totale constante.

%La perte d'étoile apparaît alors comme un processus intéressant pour expliquer l'évolution observée. Les chapitres suivants nous permettront de déterminer, à l'aide de simulation numérique, si la perte
%d'étoile par effet de marée est suffisante pour l'expliquer.








Cette dernière étude nous montre que plus un amas est âgé, plus la pente de son halo est importante.
Dans le chapitre précédent, nous avons relié la pente au paramètre $W_0$ qui, en plus de jouer sur les pentes, joue sur la densité centrale, comme le montre la figure~\ref{King_Modele-test}.
Par ailleurs, des simulations partant d'un nuage homogène nous apprennent que la pente du halo après sa formation vaut $-4$.
Un amas va donc partir d'une structure cœur-halo avec un cœur de taille importante, un halo de pente $-4$ et va tendre, en vieillissant, vers un amas ayant une densité centrale très élevée dont la pente tend vers $-2$.
Puis après un temps infini, l'amas va tendre vers une sphère isotherme.
C'est ce que représente le schéma suivant.

\input{theorie/figure_tikz-pente.tex}

Pour expliquer cette évolution, nous devons trouver quels phénomènes, de dynamique gravitationnelle, pourraient diluer le halo et concentrer de la matière au centre de l'amas. Le phénomène qui vient le plus naturellement à l'esprit est la perte d'étoiles, pouvant s'effectuer par 2 scénarios :
\begin{enumerate}
	\item des collisions internes qui éjectent des étoiles de l'amas, permettant ainsi au cœur de s'effondrer en essayant de compenser cette perte d'énergie.
	\item des interactions avec un autre objet massif qui va retirer par effet de marée des étoiles à l'amas, causant l'effondrement de son cœur.
\end{enumerate}
C'est ce deuxième scénario que nous allons tenter d'étudier.
















%Si la température cinétique du halo augmente, celle du
%cœur va changer pour s'adapter. La capacité calorifique à volume
%constant est négative pour le cœur (~et pour tout système
%auto gravitant~) :
%\begin{align*}
%	E_p &= -2E_c & \text{(~Viriel~)} \\
%	\Rightarrow E &= E_p + E_c = -E_c \\
%	\intertext{or}
%	E_c &\varpropto T \Rightarrow E\varpropto -T \\
%	\intertext{donc}
%	\Rightarrow C_v &= \frac{\partial E}{\partial T} < 0
%\end{align*}
%Cela implique que notre système ne peut que prendre de l'énergie.

%Si la température du cœur augmente, la vitesse de rotation des
%étoiles va augmenter (~$E_c\varpropto T\varpropto v$~), et donc
%leur demi grand axe va diminuer~\footnote{rappel : $v^2 =
%K\left(\frac{2}{r} - \frac{1}{a}\right)$ avec $K = G\left(M+m\right)$
%pour des interactions à deux corps}.
%En conséquent la densité au centre du cœur augmente, augmentant
%ainsi le contraste de densité. Si la température augmente
%suffisamment, le contraste de densité du halo va dépasser la
%valeur critique $\R_c^H$ (~l'énergie est fixé, seul la
%température change, nous utilisons donc la description micro
%canonique~). Le cœur va alors devenir instable.


%	L'interprétation semble simple ici : par exemple : quand les étoiles
%	ont des vitesses de rotation élevées la sphère a une température élevée,
%	et donc sa capacité calorifique à volume constant, $C_v = \frac{\partial H}{\partial T} < 0$
%	(~négatif car $E_p=-2E_c\Rightarrow H=E_c+E_p=-E_c\varpropto-T$~),
%	tend vers $0$ : elle ne peut plus acquérir d'énergie.
%	Si on dépasse la limite de température, elle va devoir se \og~réorganiser~\fg pour
%	rester à l'équilibre et donc pour retourner à un contraste de densité $\R < \R^\beta_c$.
%	Le raisonnement est le même pour la limite en énergie.


	\section{Les Galaxies}%{Que peut-on dire des autres ...!}
		


	\part{Théorie}
%		\minitoc
		\chapter{Le problème de la sphère isotherme}
		\minitoc
		
		\section{La sphère isotherme et son problème intrinsèque}
	La sphère isotherme est la solution classique fournie par une approche thermodynamique du problème de l'équilibre d'un système autogravitant.
	Elle apparait lorsque nous cherchons un maximum de l'entropie statistique de Boltzmann $S(f) = -k_B \int f\ln(f) d\vec{r}d\vec{p}$, où $k_B$ est la constante de Boltzmann.
	En imposant les contraintes d'une masse fixée:
	\begin{align}
		M = \int \rho(\vec{r},t) d\vec{r}
	\end{align}
	et une énergie totale fixée:
	\begin{align}
		H = \int \frac{\vec{p}\,^2}{2m} f d\vec{r}d\vec{p}-\frac{Gm^2}{2}\int \int 
		\frac{f(\vec{r},\vec{p})f(\vec{r}\,^{\prime},\vec{p}\,^{\prime})}
		      {\left|\vec{r}-\vec{r}\,^{\prime}\right|}
		d^3\vec{r}\,^{\prime}d^3\vec{p}\,^{\prime}
	\end{align}
	Pour le système sa fonction de distribution s'écrit :
	\begin{align}
		f(\vec{r},\vec{p})=f(E) = \(\frac{2\pi \alpha^2 m}{\beta}\)^{-3/2} e^{-\beta E}
		\label{sph-iso}
	\end{align}
	La fonction $E=\frac{\vec{p}\,^2}{2m}+m\psi(\vec{r})$ caractérise l'énergie d'une particule test.
	Les constantes $\beta = \frac{1}{k_B T}$ et $\alpha$ sont des multiplicateurs de \textsc{Lagrange} associés aux contraintes imposées lors de la recherche de l'extremum de l'entropie :
	$\beta \Leftrightarrow H = \mathrm{cte}$ et $\alpha \Leftrightarrow M = \mathrm{cte}$.

	Les différents problèmes posés par cette solution ont émaillé la recherche dans ce domaine tout au long du
	\textsc{xx}$^{e}$ siècle d'\cite{emden07} à \cite{1989ApJS...71..651P}  en passant par
	\cite{chandra39}, la revue de \cite{2006IJMPB..20.3113C} pourra être consulté à ce sujet.
	% on pourra consulter à ce sujet la revue de \cite{2006IJMPB..20.3113C}.
	Nous en reprendrons les éléments principaux.
	
\section{Formulation générale du problème}
	
	La densité de la sphère isotherme se calcule directement à partir de la définition \ref{sph-iso}:
	\begin{align}
		\rho(r) &= m \int \(\frac{2\pi \alpha^2 m}{\beta}\)^{-3/2} e^{-\beta \(\frac{p^2}{2m} + m\psi\)} d^3 p \notag \\
			&= 4\pi m \(\frac{2\pi \alpha^2 m}{\beta}\)^{-3/2} \int_0^\infty e^{-\beta \(\frac{p^2}{2m} + m\psi\)} p^2 dp \notag \\
			&= \frac{m}{\alpha^3} e^{-m\beta\psi}
	\end{align}
	Nous avons $\rho(\vec{r}) = \rho(\psi)$, le théorème Gidas-Ni-Niremberg s'applique (voir~\cite{CoursJP}) et le système est donc à symétrie sphérique $\rho(\vec{r}) = \rho(r)$  dans l'espace des positions.
	Avec cette symétrie radiale l'équation de Poisson s'écrit:
	\begin{align}
		\frac{1}{r^2}\frac{d}{dr}\(r^2\frac{d\psi(r)}{dr}\) &= 4\pi G \rho(r) = 4\pi G \frac{m}{\alpha^3} e^{-m\beta\psi(r)} \notag \\
		\intertext{En introduisant les variables $y = m\beta\psi$ et $x = r/r_0$, nous obtenons :}
		\frac{1}{x^2}\frac{d}{dx}\(x^2\frac{d y}{dx}\) &=  \frac{4\pi G m r_0^2}{\alpha^3} e^{-y} \notag \\
		\intertext{Nous pouvons alors choisir $r_0^2 = \frac{\alpha^3}{4\pi G m r_0^2}$ afin d'adimensionner l'équation de Poisson sous la forme:}
		\frac{1}{x^2}\frac{d}{dx}\(x^2\frac{d y}{dx}\) &= e^{-y} \label{Pois:sis}
	\end{align}
	
	Nous pouvons chercher dans un premier temps  des solutions autosimilaires pour cette équation, elles sont de la forme $\Tilde{y}$ telles que :
	\begin{align}
		\Tilde{\rho}(r) &= \frac{A}{r^2} = e^{-\Tilde{y}} \quad
		\Rightarrow \quad \Tilde{y} = - \ln\(\frac{A}{x^2}\)
	\end{align}
	Il est facile de vérifier que de telles solutions n'existent que si $A = 2$.
	% , nous avons donc 
	% \begin{align}
		% \Tilde{\rho}(r) &= \frac{m}{\alpha^3} e^{\ln\(\frac{2}{x^2}\)} = \frac{2 m r_0^2}{\alpha^3 r^2}
		% \intertext{La masse $M(r)$ contenue dans la sphère de rayon $r$ incluse dans ce système s'écrit}
		% \Tilde{M}(r)    &= \int_0^{\infty} \Tilde{\rho}(r) d^3 r = \frac{4\pi m r_0^2}{\alpha^3} \int_0^{r} r^2\frac{1}{r^2} dr = \frac{4\pi m r_0^2}{\alpha^3} r
	% \end{align}

	Les propriétés physiques de cette solution autosimilaire sont singulières :
	\begin{itemize}
		\item la densité diverge en zéro :
		\begin{align*}
			\lim\limits_{r \to 0} \Tilde{\rho}(r) &= \infty
		\end{align*}
		\item la masse est infinie si le support du système n'est pas limité (ce qui est inclu dans les hypothèses):
		\begin{align*}
			\lim\limits_{r \to \infty} \Tilde{M}(r) &= \infty
		\end{align*}
	\end{itemize}
	
	Cette solution forme ce que l'on appelle une sphère isotherme singulière (\textsc{sis}), elle ne peut en aucun cas
	correspondre à la solution thermodynamique recherchée qui doit posséder une masse finie.

	Étudions à présent l'existence de solutions plus générales pouvant avoir une densité et une masse qui ne
	diverge pas. Nous pratiquons pour cela un changement de fonction, en introduisant  $\zeta = y - \Tilde{y}$, la
	différence entre la solution générale $y$ du problème et la \textsc{sis}. Il vient successivement
	\begin{align}
		\frac{1}{x^2}\frac{d}{dx}\(x^2\frac{d\zeta}{dx}\) &= e^{-\zeta - \Tilde{y}} - \frac{1}{x^2}\frac{d}{dx}\(x^2\frac{d\Tilde{y}}{dx}\) \notag \\
								  &= e^{-\zeta - \Tilde{y}} - e^{-\Tilde{y}} \notag \\
								  &= \(e^{-\zeta} - 1\)\frac{2}{x^2}
			\end{align}
et en utilisant l'inconnue%
\begin{align*}
t=\ln\left(  x\right)
\end{align*}
il vient%
\begin{align}
\frac{d^{2}\zeta}{dt^{2}}+\frac{d\zeta}{dt}=2\left(  e^{-\zeta}-1\right)
\label{eq_diff_iso}%
\end{align}
Le problème consiste donc à étudier les propriétés générales de cette équation.

\section{Propriétés de la solution générale et conséquences}

Dans la littérature, le seul cas traité explicitement est celui de la linéarisation de l'équation autour de $\zeta=0$
(voir~\citet{chandra39}).
% Notons que les seuls résultats trouvés dans la littérature concernant le comportement asymtotique de ce système
% concernent son linéarisé au voisinage de l'origine (voir \cite{chandra39} qui demeure la référence absolue).
Nous proposons ici une étude plus générale dans la totalité du plan de phase. Posons $\vec{z}=\left[
\zeta,\dot{\zeta}=\frac{d\zeta}{dt}\right]^{\top}$, nous avons:
\begin{align}
\frac{d\vec{z}}{dt} =F\left(  \vec{z}\right)  =
\left[\dot{\zeta},2\left(  e^{-\zeta}-1\right)-\dot{\zeta}\right]^{\top} \label{sysdif}%
\end{align}
Le seul point d'équilibre est l'origine $\vec{z}_0=\left[  0,0\right]
^{\top}$. Considérons à présent:
\begin{align*}
\mathcal{E}(\zeta,\dot{\zeta}) = \frac{1}{2}\dot{\zeta}^2+2(e^{-\zeta}-\zeta-1)
\end{align*}
% Il est clair que :
Il est apparaît que :
\begin{itemize}
\item La fonction $\mathcal{E}$ est nulle en $\vec{z}=\vec{z}_0$ ;
\item La hessienne de $\mathcal{E}$ est définie positive sur $\mathbb{R}^2_*$: 
\begin{align*}
H_{\mathcal{E}}:=
\left[
\begin{array}
[c]{cc}%
\frac{\partial^2 \mathcal{E}}{\partial \zeta^2}         & \frac{\partial^2 \mathcal{E}}{\partial \zeta \partial \dot{\zeta}}\\
 & \\
\frac{\partial^2 \mathcal{E}}{\partial \dot{\zeta} \partial \zeta}& \frac{\partial^2 \mathcal{E}}{\partial \dot{\zeta}^2}
\end{array}
\right]
=\left[\begin{array}
[c]{cc}%
2e^{-\zeta}         & 0\\
0& 1
\end{array}
\right]
\end{align*}
et donc le point $\vec{z}_0$ est un minimum global de $\mathcal{E}$;
\item La fonction $\mathcal{E}$ est strictement décroissante de la variable $t$, en effet:
\begin{align*}
\frac{d\mathcal{E}}{dt}&=\frac{d\zeta}{dt}\frac{\partial \mathcal{E}}{\partial \zeta}+\frac{d\dot{\zeta}}{dt}\frac{\partial \mathcal{E}}{\partial \dot{\zeta}} \\
\\
&=-\dot{\zeta}^2 
\end{align*}

\end{itemize}
Les trois propriétés énumérées ci-dessus font de la fonction $\mathcal{E}$ une fonction de Lyapounov\index{Fonction! de Ljapounov}\index{Ljapounov, fonction de} stricte du système, l'équilibre $\vec{z}_0$ est donc globalement asymptotiquement stable. 

Le comportement asymptotique $\left(t\rightarrow+\infty\right)$ de la
solution $\zeta(t)$ s'obtient en considérant une combinaison linéaire
des exponentielles des valeurs propres de la matrice
\begin{align*}
	A:=D\left[F(\vec{z})\right](\vec{z}_0)=\left[
\begin{array}
[c]{cc}%
0     & 1\\
-2 & -1\\
\end{array}
\right]
\end{align*}
soit $\left(  -1\pm i\sqrt
{7}\right)  /2$, ainsi lorsque $x\rightarrow+\infty$ nous avons
\begin{align*}
\zeta\left(  x\right)
\sim\frac{k_{1}\cos\left[  \ln\left(  x^{\sqrt{7}/2}\right)  \right]
+k_{2}\sin\left[  \ln\left(  x^{\sqrt{7}/2}\right)  \right]  }{\sqrt{x}%
}\ \ \ \text{avec }k_{1},k_{2}\in\mathbb{R}%
\end{align*}
En majorant les fonctions trigonométriques, nous avons donc $\zeta\left(
x\right)  \sim k/x^{1/2}$ en $x\rightarrow+\infty$, soit $y\left(  x\right)
\sim k/x^{-1/2}-\ln\left(  2/x^{2}\right)  $ soit pour la densité en
variable $r$:
\begin{align}
\rho\left(  r\right)  \sim\frac{2mr_{o}^{2}}{\alpha^{3}r^{2}}\left(
1\pm\left(  \frac{r_{k}}{r}\right)  ^{1/2}\right)  \ \ \ \text{quand
}r\rightarrow+\infty\text{.}\label{asymp_sph_iso}%
\end{align}
% La longueur $r_{k}$ se calcule en écrivant proprement la majoration, en
% revenant aux variables physiques et en calculant le laplacien, elle n'a que
% peu d'intér\^{e}t. Le signe $\pm$ provient de l'encadrement des fonctions
% trigonométriques. Ce qu'il faut remarquer dans cette
% relation, c'est que la masse d'une sphère isotherme est \emph{toujours} infinie si
% celle-ci est d'extension infinie $\left(  r\rightarrow+\infty\right)$.
La longueur $r_{k}$ se calcule en écrivant proprement la majoration, en
revenant aux variables physiques et en calculant le laplacien. Le signe $\pm$ provient de l'encadrement des fonctions
trigonométriques. Ce qu'il faut remarquer dans cette
relation, c'est que la masse d'une sphère isotherme est \emph{toujours} infinie si
celle-ci est d'extension infinie $\left(  r\rightarrow+\infty\right)$.

La fonction de distribution (\ref{sph-iso}) de la sphère isotherme n'est donc pas acceptable car elle est en contradiction avec les hypothèses posées pour l'obtenir.

Le problème de l'équilibre thermodynamique d'un système autogravitant est donc posé !

Pour palier à ce problème, plusieurs approches ont été développées au fil des ans :

\begin{enumerate}
	\item Une approche \og{}rigoureuse\fg consiste à chercher le maximum de l'entropie dans un ensemble de fonctions de distribution à support compact :
		% l'extension spatiale du système est alors finie. Ce problème est communément appelé sphère isotherme en boîte (~SIB~). Il sera étudié en détail dans le chapitre \label{SIB::Chapitre}.
		l'extension spatiale du système est alors finie. Ce problème est communément appelé sphère
		isotherme en boîte (\textsc{sib}). Il sera étudié en détail dans le chapitre \ref{SIB::Chapitre}.
		
	\item Une solution \og{}pragmatique\fg: développée par \cite{King-1966AJ} consiste à tronquer à la main
		et après coup la sphère isotherme. Au-dessus d'une certaine énergie, la fonction de
		distribution est nulle. Bien que empirique cette solution est devenue un modèle de choix pour
		l'ajustement du profil de densité de nombreux amas globulaires et de galaxies naines. Ses trois
		paramètres libres rendent en effet son utilisation adaptée à la modélisation de tels objets.
		Nous y consacrerons le chapitre~\ref{King::Chapitre}.
	
	\item Plus récemment une prise en compte fine des spécificités des modèles gravitationnels semble avoir fait progresser les choses :
		il s'agit du paradigme Darkexp (voir~\citet{2010ApJ...722..851H}). En physique statistique, l'approximation $\ln x! = x\ln x
		-x$, valide pour $x\gg1$, est souvent utilisé. Dans ce type de calcul, $x$ est une fonction du nombre de particule $N$
		présentes dans le système. Les auteurs de ce paradigme remarque que dans notre cas (étude des galaxies ou des amas
		globulaire), ce nombre $N$ n'est pas suffisamment grand pour que nous puissions utiliser l'approximation de Stirling. Cette
		correction semble, selon les auteurs de ce paradigme, pouvoir effectuer une coupure rendant la sphère isotherme plus
		compatible avec un modèle physique acceptable.

\end{enumerate}


\chapter{Sphère isotherme en boîte\label{SIB::Chapitre}}
	\minitoc

	\section{Diagramme de \textsc{Milne}}
		\subsection{Présentation du problème}
	Comme nous l'avons vu dans le chapitre précédent, les sphères isothermes ne sont pas une solution satisfaisante, du fait de leurs masses infinies.
	Nous allons présenter dans ce chapitre une solution de sphère isotherme intégrée sur un support borné. Cette solution est toujours obtenue en recherchant des maxima  de l'entropie statistique de Boltzmann:
	\begin{align*}
		S(f) = - k_B \int f \ln f \vdp\vdr
	\end{align*}
	mais pour une fonction $f$ possédant un support borné. Ce dernier est par exemple une boule (souvent appelée boîte, d'où le nom du chapitre) définie comme :
	\begin{align}
		B = \left\{ \vec{r} \in \mathbb{R}^3\ ;\ \left|\vec{r}\right| < R\right\}
	\end{align}
	où $R$ est le rayon de la boule support de $f$. Dans la suite, nous noterons $\mathbb{I}_{B_R}$ la fonction indicatrice sur cette boule.

\subsection{Obtention des équations}
	La fonction de distribution d'une sphère isotherme en boîte, s'écrit (voir par exemple \cite{CoursJP}) :
	\begin{align}
		f^+(E) = \left(\frac{2\pi\alpha^2m}{\beta}\right)^{-3/2}e^{-\beta E}\times\mathbb{I}_{B_R}
	\end{align}
	où $\alpha$, en mètre, et $\beta=(k_B T)^{-1}$, en Joule, sont des multiplicateurs de Lagrange imposés
	par les contraintes respectives de masse $M$ et d'énergie totale $H$ finie que nous imposons toujours au système.
	Ils assurent également la normalisation de la fonction de distribution.
	% La quantité \mbox{$E = \frac{p^2}{2m} - m\psi(r)$} est l'énergie d'une particule test de masse $m$ se déplacant dans
	% le potentiel $\psi$ créé par la sphère isotherme.
	Le calcul de la densité de masse donne alors \mbox{$\rho(r) = \frac{m}{\alpha^3}e^{-\beta m \psi(r)}\times\mathbb{I}_{B_R}$}. % \psi'(r)}$}, dans la suite nous utiliserons :
	%\mbox{$\psi'(r) = \psi(r) - \psi(0)$}~\footnote{pour faciliter les calculs dans les variables Milne définies ci-dessous}.
	L'équation de Poisson s'écrit :
	\begin{align}
		 \frac{1}{x^2}\frac{d}{dx}\left(x^2\frac{d h}{dx}\right) = e^{-h} \times\mathbb{I}_{B_R} \notag
	\end{align}
	où nous avons utilisé l'adimensionnement  :
	\begin{align}
		x &= \frac{r}{r_0}\mathrm{ avec }\ r_0 = \sqrt{\frac{\alpha^3}{4\pi G m^2\beta}}e^{-\frac{m\beta\psi(0)}{2}} \label{toto11} \\
		h &= m\beta\psi(x) - m\beta\psi(0) \label{toto12}
		\end{align}
		Il est important de noter que l'inconnue du problème $h(x)$ est différente de $y(x)$ utilisée pour la présentation du problème de la sphère isotherme illimitée. Nous avons ici $h(0) = 0$ alors que $y(0) = m\beta\psi(0)$.
		
		En notant par un $^\prime$ toutes les dérivées par rapport à la variable $x$, nous introduisons les variables dites de \textsc{Milne :}
		\begin{align}
			\begin{cases}
			v = x \dfrac{d h}{dx} = x \x{h} \\
			\\
			u = \dfrac{e^{-h} x}{h'} = \dfrac{e^{-h} x^2}{v}
		\end{cases}\label{syst_uv}
		\end{align}
		
Un calcul simple donne alors :
\begin{align}
	\dfrac{d(xv)}{dx} = uv \;\;\Rightarrow\;\; \x{v} = \frac{v \left( u - 1\right)}{x} \notag
\end{align}
La dérivée de $u$ s'obtient directement à partir de sa définition, nous trouvons:
	$$\begin{cases}
		\x{v} = \dfrac{v \left( u - 1\right)}{x} \\
		\x{u} = \dfrac{u}{x}\left(3 - v - u\right)
	\end{cases} \label{systdudv}$$
	
	que nous pouvons écrire de façon compacte:
	\begin{align}
		\dfrac{d v}{d u} = \dfrac{v \left( u - 1\right)}{u \left(3 - u - v\right)}
		\label{eqdudv}
	\end{align}
	La courbe résultant de cette équation, tracée dans le plan $\left(u, v\right)$, forme le diagramme
	de Milne. L'étude de ce diagramme permet d'accéder à de nombreuses caractéristiques de la sphère isotherme en
	boîte. L'équation~(\ref{eqdudv}) ne peut être résolue explicitement mais sa solution peut être obtenue
	numériquement: il faut pour cela préciser les conditions au bord et au centre du système; ce qui est l'objet de la
	section suivante.
	

\subsection{Conditions au centre et sur le bords de la sphère}
	Pour obtenir les conditions initiales permettant la résolution de l'équation~\ref{eqdudv}, nous devons obtenir les conditions aux limites pour la sphère isotherme en boîte.
\subsubsection{En $x = 0$}
	Pour ce faire, nous nous plaçons dans le voisinage de $x=0$, et faisons donc des développements limités de la variable $h$ représentant le potentiel.
	Cela nous permettra d'obtenir le comportement approché de $u(x)$ et de $v(x)$ au centre.
	Nous avons donc :
	\begin{eqnarray*}
		h(x) = h(0) + \x{h}(0)x + \frac{d^2 h}{dx^2}(0)\frac{x^2}{2!} + \frac{d^3 h}{dx^3}(0)\frac{x^3}{3!} + o(x^3) = ax^2 + bx^3 + o(x^3)
	\end{eqnarray*}
	En effet, \mbox{$h(0) = m\beta\left(\psi(0) - \psi(0)\right) = 0$} et \mbox{$\x{h}(0) = m\beta\x{\psi}(0) = 0$} car \mbox{$\x{\psi} = r_0\frac{d \psi}{dr}\propto F$},
	$F$ étant la force s'appliquant sur la particule test.
	Au centre de la sphère, les forces qui s'appliquent à une particule test s'opposent les unes aux autres, d'où $\x{h}(0) = 0$.

	Les variables $u$ et $v$ vont alors se développer comme :
	\begin{eqnarray}
		v &=& x\x{h} = 2ax^2 + 3bx^3 + o(x^3)\label{vDL} \\
		u &=& \frac{x^2e^{-ax^2 - bx^3 + o(x^3)}}{2ax^2 + 3bx^3 + o(x^3)} \\
		  &=& \frac{1 - ax^2 - bx^3 + o(x^3)}{2a + 3bx +o(x)} \\
		  &=& \frac{1}{2a} - \frac{3b}{4a^2}x + o(x)\label{uDL}
	\end{eqnarray}
	Nous utilisons ensuite l'équation de Poisson pour déterminer, par identification, les valeurs de $a$ et de $b$ :
	\begin{eqnarray}
		\x{\left(xv\right)} = 6ax^2 + 12bx^3 + o(x^3) = uv = x^2 + o(x^3)\label{PoisDL}
	\end{eqnarray}
	D'où :
	\begin{eqnarray}
		\left\{\begin{array}{l}
			a = \frac{1}{6} \\
			b = 0
		\end{array}\right.
	\end{eqnarray}
	Maintenant que nous avons nos développements, nous en déduisons les conditions initiales :
	$$(u,v) \substack{\longrightarrow \\ r\to 0} (3,0)$$

\subsubsection{Sur le bords de la sphère : $r = R$}
	Soit $X = \frac{R}{r_0}$. Nous allons tenter d'exprimer nos variables $u$ et $v$ au point $X$ en fonction des quantités connues du problème,
	telles que l'énergie totale $H$, la masse totale $M$, le rayon de la sphère $R$.
	Et, selon la description canonique ou micro canonique choisie, en fonction de la température cinétique du système.

	On introduit les constantes adimensionnées suivantes :
	\begin{eqnarray*}
		\left\{\begin{array}{l}
			\lambda = - \frac{H R}{G M^2} \\
			\\
			\mu     = \frac{m\beta GM}{R}
		\end{array}\right.
	\end{eqnarray*}
	où $\lambda$ représente l'énergie adimensionnée et $\mu$ l'inverse de la température adimensionnée.

	Au bord du système, nous pouvons écrire que $\frac{d \psi}{dr} \equiv \frac{GM}{R^2}$~\footnote{Théorème de Gauss}. De plus :
	\begin{eqnarray*}
		\x{h}(X) = m\beta r_0 \frac{d \psi}{dr} = m\beta r_0 \frac{GM}{R^2} = \frac{\mu}{X} \Rightarrow \mu = X\x{h}(X)
	\end{eqnarray*}
	Or $v(X) = X\x{h}(X)$, d'où :
	\begin{eqnarray}
		v\left(X\right) := v_m = \mu\label{vmmu}
	\end{eqnarray}

	Il nous reste l'énergie à exprimer.
	Pour cela, nous allons commencer par calculer l'énergie totale $H$ à l'aide du théorème du Viriel adapté à une sphère isotherme en boîte~\footnote{Pour un système tronqué, le calcul permettant d'arriver au théorème du Viriel fait apparaître des termes dû aux intégrations par partie qui vont rester.} :
	\begin{eqnarray}
		2T + W = \frac{4}{3}\pi R^3 P_e
	\end{eqnarray}
	avec $P_e$ la pression qui s'exerce sur le bord de la sphère.
	L'énergie cinétique s'écrit \mbox{$K = \frac{1}{2}mv^2 = \frac{3}{2} N k_B T = \frac{3}{2} \frac{M}{m} k_B T$}.
	Donc :
	\begin{eqnarray*}
		H &=& K + W = \frac{3}{2} \frac{M}{m\beta} + \frac{4}{3}\pi R^3 P_e - 2K \\
		  &=& -\frac{3}{2} \frac{M}{m\beta} + \frac{4}{3}\pi R^3 P_e \\
		\Rightarrow \lambda &=& \frac{3}{2}\frac{MR}{m\beta GM^2} - \frac{4}{3}\pi R^3 P_e \frac{R}{GM^2}
	\end{eqnarray*}
	Une sphère isotherme est un système barotropique. Elle a donc une équation d'état polytropique d'indice 1 : $P = \frac{\rho(r)}{m\beta} \Rightarrow P_e = \frac{\rho(R)}{m\beta}$ que nous remplaçons :
	\begin{eqnarray}
		\lambda &=& \frac{3}{2}\frac{1}{X\x{h}(R)} - \frac{4}{3}\pi R^3 \frac{\rho(R)}{m\beta} \frac{R}{GM^2} \notag \\
			&=& \frac{3}{2}\frac{1}{X\x{h}(R)} - \frac{4}{3}\pi R^3 \frac{e^{-h(R) + m\beta\psi(0)}}{\alpha^3\beta} \frac{R}{GM^2} \notag \\
			&=& \frac{3}{2}\frac{1}{X\x{h}(R)} - \frac{4}{3}\pi R^3 \frac{4\pi G \beta m^2 e^{\beta m \psi(0)}}{\alpha^3} r_0^2\frac{e^{-h(R)}}{(\x{h}(R))^2} \notag \\
			&=& \frac{3}{2}\frac{1}{X\x{h}(R)} - \frac{e^{-h(R)}}{(\x{h}(R))^2}\label{lamum}
	\end{eqnarray}

	Nous avons ensuite, par substitution, les valeurs maximales $u_m$ et $v_m$ de $u$ et $v$ en fonction des paramètres du problème :
	\begin{eqnarray}
		\label{uv_max}
		\fbox{$
		\left\{\begin{array}{l}
			v_m = \mu \\
			u_m = \frac{3}{2} - \lambda \mu
		\end{array}\right.
		$}
	\end{eqnarray}



\subsection{Droites de Padmanabhan et diagramme de Milne}
	% On peut facilement éliminer $\mu$ dans le système~(\ref{uv_max}) et obtenir
	Les $\mu$ dans le système~(\ref{uv_max}) peuvent facilement s'éliminer et nous obtenons:
	\begin{align}
		v_m = \frac{3}{2\lambda} - \frac{u_m}{\lambda}\label{droitePb}
	\end{align}
	Cette relation montre que si l'on ne fixe que la valeur de $\lambda$,  les couples $(u_m,v_m)$  se répartissent
	dans le plan $(u,v)$ le long d'une droite de pente $-1/\lambda$, dite de Padmanabhan en référence à sa remarque
	dans l'article \cite{1989ApJS...71..651P}. Nous remarquons aussi que toutes ces droites passent par le point
	$(\frac{3}{2},0)$.
	
	Le choix de $\lambda$ est indépendant de la valeur de la température, le point $(u_m,v_m)$ associé à la sphère
	isotherme en boite de rayon $R$, de masse $M$, d'énergie $H$ et de température $\beta$ se trouve donc à
	l'intersection de la droite de Padmanabhan fixée par $\lambda$ et de la courbe $v(u)$ solution de
	l'équation~(\ref{eqdudv}). 
	
	
	Cette courbe est obtenue numériquement à l'aide d'un solveur runge-kutta d'ordre 4, en utilisant les variables adimensionnées, le
	seul paramètre physique pour la résolution étant la valeur de $X$ caractérisant le rayon et la température de la
	sphère.  L'ensemble de ces courbes et des droites de Padmanabhan correspondantes sont représentées sur la figure
	\ref{Milne}.
	Nous constatons que plus la valeur de $R$ augmente, plus la courbe
	solution est longue et finit par s'enrouler dans une spirale convergeant vers un point. Ce point correspond au
	cas $R\to\infty$ et donc celui de la \textsc{sis}.
	% Pour une température fixée, nous constatons que plus la valeur de $R$ augmente plus la courbe
	% solution est longue et finit par s'enrouler dans une spirale convergeant vers un point. Ce point correspond au
	% cas $R\to\infty$ et donc celui de la \textsc{sis}.
	
	À l'inspection de ce diagramme et de ces droites, nous comprenons aisément que l'on ne peut pas mettre
	n'importe quelle sphère isotherme dans n'importe quelle boîte. %qu'elle boîte.

	En effet, si l'on impose des valeurs de $R$,
	$M$ et $H$ qui donnent une valeur trop grande $\lambda>\lambda_c$, la droite ne pourra pas avoir d'intersection
	avec la courbe. Cette intersection qui aurait fixé le dernier paramètre, la température, n'existant pas il en va
	de même de la sphère isotherme en boîte possédant ces caractéristiques physiques.
	
	Nous remarquons même l'existence d'une seconde valeur caractéristique $\lambda_0$ telle que :
	\begin{itemize}
		\item si $\lambda < \lambda_0$ : il n'existe qu'une seule et unique sphère isotherme possible;
		\item si $\lambda \in \left[\lambda_0,\lambda_c\right]$ : il existe plusieurs possibilités;
		\item si $\lambda > \lambda_c$ : il n'y a plus d'intersection possible !
	\end{itemize}

Comme nous allons le voir dans la prochaine section, cette multiplicité de possibilités est associée à la stabilité des sphères concernées.

	\begin{figure}[h!]
		\begin{minipage}[b]{0.40\linewidth}
			\centering \includegraphics{graphe/milne_X1.pdf}
		\end{minipage}\hfill
		\begin{minipage}[b]{0.48\linewidth}
			% \centering \includegraphics[scale=0.60]{graphe/r_max-5.pdf}
			\centering \includegraphics{graphe/milne_X5.pdf}
		\end{minipage}
		\begin{minipage}[b]{0.40\linewidth}
			% \centering \includegraphics[scale=0.60]{graphe/r_max-10.pdf}
			\centering \includegraphics{graphe/milne_X10.pdf}
		\end{minipage}\hfill
		\begin{minipage}[b]{0.48\linewidth}
			% \centering \includegraphics[scale=0.60]{graphe/r_max-300.pdf}
			\centering \includegraphics{graphe/milne_X300.pdf}
		\end{minipage}
		\caption{Diagramme de Milne pour $X=1$, $X=5$, $X=10$ et pour $X=300$. La sphère isotherme est
		représenté par la courbe bleue, la courbe de Padmanabhan est en vert et la \textsc{sis} en rouge.}
		\label{Milne}
	\end{figure}


	\section{Courbe calorique}
		\subsection{Motivation et définition}
	
	Les variables $(u,v)$ de Milne ne sont pas directement reliées aux grandeurs physiques caractérisant la sphère
	isotherme en boîte considérée. Nous utilisons souvent plutôt un autre moyen de représentation issu de la
	thermodynamique et appellé courbe calorique. Cette courbe est en fait directement reliée à celle de Milne et
	consiste à représenter $\mu$ en fonction de $\lambda$ ou de son opposé $-\lambda$ car l'énergie d'un système
	autogravitant est souvent négative. En pratique nous résolvons le système différentiel $u(v)$ en fixant $X$,
	nous déterminons les coordonnées du point extrême $(u_m,v_m)$, puis en utilisant (\ref{uv_max}) nous en déduisons la valeur
	du couple $(\lambda,\mu)$ correspondant au choix de $X$.
	
	Contrairement aux variables de Milne qui sont des fonctions de $x$ (et donc de $r$), $\lambda$ et $\mu$ sont des
	constantes qui sont fixées dès que nous avons choisi $H$, $M$, $R$ et $\beta$. Une sphère isotherme dans une boîte
	donnée est donc associée à une courbe de Milne dont l'extrémité est fixée par la droite de Padmanabhan ou bien
	par un point de la courbe calorique. Par contre, chaque point de la courbe calorique correspond à une sphère isotherme
	particulière et l'ensemble de tous les points de la courbe, qui forme lui aussi une spirale comme nous pouvons le
	voir sur la figure \ref{Ener}, représente donc une classe de systèmes physiques.
	
	\begin{figure}[h!]
		% \centering \includegraphics[scale=1.00]{graphe/Energie_tg.pdf}
		\centering \includegraphics{calorique.pdf}
		\caption{Courbe calorique de la sphère isotherme en boîte : $\mu(\lambda)$. La croix verte représente la \textsc{sis}}
		\label{Ener}
	\end{figure}

	Dans le diagramme de Milne, la sphère isotherme singulière correspondait au point $\left(1,2\right)$.
	Dans cette nouvelle représentation elle est associée au point $\left(\lambda = \frac{3/2 - u_m}{v_m} = 1/4, \mu = v_m = 2\right)$.
	De la même manière que sur le diagramme précédent, la courbe tend en spiralant vers la sphère isotherme singulière, mais avant de l'atteindre,
	elle passe par un minimum de température, puis un maximum d'énergie délimitant ainsi des intervalles possibles d'existence d'une sphère gravitationelle isotherme en boîte.

\subsection{Stabilité de la sphère isotherme en boîte}
\subsubsection{Description statistique}
	En physique statistique, il existe deux ensembles capables de décrire la sphère isotherme telle que nous l'avons construite :
	\begin{enumerate}

		\item l'ensemble micro-canonique: nous imposons la valeur de l'énergie totale de la sphère, et
			l'équilibre est atteint pour une certaine valeur de la température. C'est par exemple le cas
			d'une sphère isotherme constituée de particules enfermées dans une boîte aux parois
			réfléchissantes. La dispersion de vitesse de ces particules, ou température cinétique du
			système, s'ajuste pour atteindre l'équilibre du viriel sans dissipation d'énergie.
		
		\item l'ensemble canonique: nous imposons la valeur de la température de la sphère, le système rejoint
			l'équilibre en ajustant son énergie totale. Ce processus peut par exemple se produire en mettant
			la sphère en contact avec un bain thermique qui imposera la température et rendra possible
			l'échange d'énergie susceptible d'atteindre l'état d'équilibre.
		
	\end{enumerate}

	Lorsque l'on progresse le long de la courbe calorique vers la partie en spirale, en partant des petites valeurs
	de $X$ i.e. les grandes valeurs de $-\lambda$, nous passons successivement par les deux points remarquables: le
	premier est caractérisé par une tangente horizontale et correspond à la valeur minimale qu'il est possible d'imposer à
	la température, le second est caractérisé par une tangente verticale et correspond à la valeur minimale de
	l'énergie susceptible d'être atteinte par une sphère isotherme de masse donnée et contenue dans une boîte de
	rayon fixé.
	
	Les deux points correspondants et leurs tangentes ont été représentés sur la courbe calorique figure~\ref{Ener}.
	
	De nombreuses analyses dynamiques ont été menées concernant ces limites
	et ont révélées un lien étroit avec la stabilité du système. Une
	approche simple de cette problématique est d'étudier le constraste de
	densité de la sphère et son influence.
	
\subsection{De l'importance du contraste de densité\label{contraste-dens-SIB}}
	La densité de la sphère isotherme s'écrit :
	\begin{align*}
		\rho(r) = \frac{m}{\alpha^3}e^{-\beta m\psi(r)}
	\end{align*}
	En dehors du cas de la \textsc{sis} cette densité est toujours finie au centre du système, nous pouvons donc former la quantité sans dimension $\rho^s(r) = \frac{\rho(r)}{\rho_0}$. Un rapide calcul montre alors que 
		\begin{align}
		\rho^s(x) = e^{-h(x)}=\frac{u(x) v(x)}{x^2}
	\end{align}
	Par définition, ou en utilisant les expressions asymptotiques de $u$ et $v$ en $x=0$, nous avons $\rho^s(0)=1$.
	Au bord du système, nous avons par contre $\rho^s(X) =u_m v_m x^{-2}$, la relation~(\ref{uv_max}) permet donc d'écrire
	\begin{align*}
		\rho^s(X) = \mu X^{-2}\left(\frac{3}{2} - \lambda\mu\right) \notag
	\end{align*}
	Nous pouvons donc former le contraste de densité entre le centre et le bord du système, il s'écrit :
	\begin{align}
		\R = \frac{\rho(0)}{\rho(R)} = \frac{\rho^s(0)}{\rho^s(X)} = \frac{X^2}{\mu\left(\frac{3}{2} - \lambda\mu\right)}
	\end{align}

	Il est possible de déterminer numériquement les coordonnées des points de tangentes remarquables et donc en déduire les valeurs correspondantes du contraste de densité :
	\begin{itemize}
		\item La tangente horizontale est obtenue pour une sphère telle que $X = 9,00$ avec $\(\lambda, \mu\) = \left(-0,199; 2,518\right)$. Son contraste de densité est de $\R^\beta_c \thickapprox 32,1$.
		\item La tangente verticale est obtenue pour une sphère telle que $X = 34,3$ avec $\(\lambda, \mu\) = \left(-0,335; 2,032\right)$. Son  contraste de densité est de $\R^H_c \thickapprox 709$.
	\end{itemize}
	
	Il est même possible d'obtenir numériquement les courbes $\lambda(\R)$ et $\mu(\R)$, c'est l'objet de la figure \ref{Cal_stab}. L'analyse de stabilité est alors possible. 
	\begin{figure}[h!]
		\centering \includegraphics[scale=1.00]{graphe/calorique_stabilite.pdf}
		\caption{Courbes $\lambda(\R)$ et $\mu(\R)$ tirées de l'article \cite{2011MNRAS.414.2728Y}}
		\label{Cal_stab}
	\end{figure}
	Le maximum de l'entropie dans l'ensemble microcanonique ($\delta S=0$ avec $H=\mathrm{cte}$) dans un domaine
	borné correspond à la sphère isotherme en boite. Tous les points de la courbe $\lambda(\R)$ correspondent à
	$\delta S=0$, au point critique $B$ de coordonnées $\R=709$ et $\lambda=-0.335$ nous avons $\frac{d\lambda}{d\R}=0$ et
	$\delta^2 S=0$ (voir \cite{1968MNRAS.138..495L}, \cite{1989ApJS...71..651P} et \cite{1990PhR...188..285P}). Dans
	ces deux derniers articles Padmanabhan (\cite{1989ApJS...71..651P} et \cite{1990PhR...188..285P}) montre que
	tous les points de la courbe $\lambda(\R)$ situés avant $B$ , i.e. $\R < 709$, correspondent à des situations
	telles que $\delta^2 S>0$, ils correspondent à des maxima locaux d'entropie et sont donc des configurations
	stables. Par contre tout ceux situés après le point $B$, i.e. $\R > 709$, sont toujours instables : soit parce
	que $\delta^2 S<0$ dans un premier temps (\cite{1989ApJS...71..651P}), soit pour des raisons dynamiques plus
	compliquées (voir \cite{Katz-Stab} et \cite{1979MNRAS.189..817K}). 
	
	Comme le montre \cite{2002A&A...381..340C}, dans l'ensemble canonique la situation est différente. En lieu et
	place de l'entropie, le potentiel d'étude de la stabilité est l'énergie libre $F=H-TS$. La courbe à étudier est
	maintenant $\mu(\R)$. Elle fait apparaitre deux régions. Tous les points de cette courbe situés avant le point
	$A$ de coordonnées $\R=32,1$ et $\lambda=2,518$, i.e. $\R<32,1$, sont tels que $\delta^2 F>0$ : ils
	correspondent à des sphères isothermes en boite stables. Au delà du point $A$, i.e. $\R>32,1$, le système est
	instable dans l'ensemble canonique.
	
	De manière concrète nous retiendrons que l'étude de la stabilité d'une sphère isotherme en boite est de manière générale contrôlée par son contraste de densité et l'étude des tangentes horizontales ou verticales de sa courbe calorique.   
	\begin{itemize}

		\item Pour une boite isolée aux parois réfléchissantes (situation microcanonique), le point critique de
			stabilité est associé à la tangente verticale la plus à gauche sur la courbe calorique
			$\mu(-\lambda)$ de la figure \ref{Ener}.

		\item Pour une boite dont les parois (conductrices de chaleur...) sont en contact avec un thermostat, le
			point critique de stabilité est associé à la tangente horizontale la plus en haut sur la courbe
			calorique $\mu(-\lambda)$ de la figure \ref{Ener}.

	\end{itemize}
	


%		\section{Numérique}
%			Écrire partie numérique sur milne, ...


\chapter{Le modèle de \textsc{King}\label{King::Chapitre}}
	\minitoc
Face aux problèmes soulevés par la sphère isotherme singulière et afin de prendre en considération les contingences observationnelles, Ivan King proposa en 1966 un nouveau modèle
qui consiste à limiter par des arguments physiques et \og à la main\fg l'extension la sphère.
Il introduit pour cela une énergie maximum, au-delà de laquelle une particule n'est plus liée au système.
Cette énergie, qui s'interprète comme une énergie de libération, implique une vitesse maximale ainsi qu'un rayon au-delà desquels une particule quitterait le système.

	\section{Équation de \textsc{Poisson}}
		%\subsection{Calcul v2.0}

La fonction de distribution associée au modèle de King s'écrit (voir~\cite{King-1966AJ}):
\begin{equation}
	f_K(E) = \begin{cases}
		\rho_0 \(2\pi m\sigma^2\)^{-3/2}\( e^{\frac{E_l-E}{\sigma^2}} - 1\) & \text{si $E < E_l$} \\
		0 & \text{si $E > E_l$}
	\end{cases}\label{King::Eq::DistribFunc}
\end{equation}
%\begin{equation}
%	f_K(E) = \left\{\begin{array}{l} \rho_0 \(2\pi m\sigma^2\)^{-3/2}\( e^{\frac{E_l-E}{\sigma^2}} - 1\)\ \mathrm{si}\ E < E_l \\
%		\\
%		0\ \mathrm{si}\ E > E_l
%	\end{array}\right.
%\end{equation}
où le paramètre $\sigma^2$ représente la dispersion d'énergie du système.
Le paramètre $\rho_0$ correspond à la densité de masse au centre du système. L'énergie $E_l$  de libération d'une particule s'écrit : $E_l = \frac{p_l^2}{2m} + m\psi(r)$, où $\psi(r)$ est le potentiel gravitationnel de la sphère de King et  $p_l$ l'impulsion de libération fonction de $r$.

La densité de masse s'écrit toujours à partir de la fonction de distribution:
\begin{eqnarray*}
	\rho(r) &=& \int^{p_l}_0\,f_K(E)4\pi p^2dp \\
		&=& \frac{\rho_0}{\(2\pi m\sigma^2\)^{3/2}}\int_0^{p_l}\,\left\{e^{\frac{E_l-E}{\sigma^2}} -1\right\} 4\pi p^2 dp\\
\end{eqnarray*}
L'énergie d'une particule test s'écrit  $E = \frac{p^2}{2m} + m\psi$, nous avons donc:
\begin{eqnarray*}
	\rho(r) &=& \frac{\rho_0}{\(2\pi m\sigma^2\)^{3/2}} \int_0^{p_l}\,\left\{e^{\frac{E_l - \frac{p^2}{2m} - m\psi}{\sigma^2}} -1\right\} 4\pi p^2 dp\\
		&=& \frac{4\pi \rho_0}{\(2\pi m\sigma^2\)^{3/2}} \(e^{\frac{E_l - m\psi}{\sigma^2}} \int_0^{p_l}\,e^{-\frac{p^2}{2m\sigma^2}} p^2 dp - \int_0^{p_l}\,p^2 dp\)\\
\end{eqnarray*}
Une intégration par partie de la première intégrale permet alors d'écrire:
\begin{equation*}
	\rho(r) = \frac{4\pi\rho_0}{\(2\pi m\sigma^2\)^{3/2}} 
	\(e^{\frac{E_l - m\psi}{\sigma^2}} 
	\left[
	-m\sigma^2p_l e^{-\frac{p_l^2}{2m\sigma^2}} + m\sigma^2\int_0^{p_l}\,e^{-\frac{p^2}{2m\sigma^2}} dp
	\right] - \frac{p_l^3}{3}\)
\end{equation*}
En introduisant un nouveau potentiel $\phi(r)$ tel que:
\begin{equation}
	p_l^2 = 2m\(E_l - m\psi(r)\) = 2m\phi(r)
\end{equation}
il vient  maintenant:
\begin{equation*}
	\rho(r) = 
	\rho_0 \(-\sqrt{\frac{4\phi}{\pi\sigma^2}}\(1 + \frac{ 2\phi }{3\sigma^2}\) 
	+ 
	\frac{2e^{\frac{\phi}{\sigma^2}}}{\sqrt{2m\pi\sigma^2}} \int_0^{p_l}\,e^{-\frac{p^2}{2m\sigma^2}} dp\)\\
\end{equation*}
la dernière intégrale s'exprime directement en utilisant la fonction d'erreur:
$$\mathrm{erf}(t) = \displaystyle{\frac{2}{\sqrt{\pi}}\int_0^t e^{-u^2}du}$$
la densité du modèle de King s'écrit donc:
\begin{eqnarray}
	\rho(r) = \rho_0 \(-\sqrt{\frac{4\phi}{\pi\sigma^2}}\(1 + \frac{ 2\phi }{3\sigma^2}\) + e^{\frac{\phi}{\sigma^2}}\mathrm{erf}\(\sqrt{\phi}/\sigma\)\)
	\label{rho_r}
\end{eqnarray}
L'équation de Poisson pour le potentiel $\phi(r) = E_l - m\psi(r)$ s'écrit donc:
\begin{eqnarray}
	\frac{d}{dr}\(r^2\frac{d\phi}{dr}\) = -4m\pi G r^2\rho_0 \left\{-\sqrt{\frac{4\phi}{\pi\sigma^2}}\(1 + \frac{ 2\phi }{3\sigma^2}\) + e^{\frac{\phi}{\sigma^2}}\mathrm{erf}\(\sqrt{\phi}/\sigma\)\right\} \label{King-Pois}
\end{eqnarray}
Malgré le fait que cette équation n'admette pas de solution explicite, son étude numérique ne pose pas de problème majeur.



	\section{Étude numérique}
		\subsection{Adimensionnement de l'équation~\ref{King-Pois}}
%	Pour adimensionner l'énergie, nous allons nous servir de la quantité $\sigma^2$. % qui à la dimension d'une énergie.
%	Nous introduisons la quantité $r_c$, pour adimensionner les distances, puis
%	nous faisons donc le changement de variable suivant :
	L'adimensionnement s'effectue avec $\sigma^2$ pour l'énergie et la longueur $r_c$ pour les distances. Ainsi :
	\begin{eqnarray*}
		\left\{\begin{array}{l}
			\gamma = \frac{\phi}{\sigma^2}\\
			x = \frac{r}{r_c}
		\end{array}\right.
	\end{eqnarray*}

	Il vient alors :
	\begin{equation}
		\frac{d}{dx}\(x^2\frac{d\gamma}{dx}\) = -\frac{4m\pi G r_c^2 \rho_0}{\sigma^2}x^2 \left\{-\sqrt{\frac{4\gamma}{\pi}}\(1 + \frac{ 2\gamma }{3}\) + e^{\gamma}\mathrm{erf}\(\sqrt{\gamma}\)\right\}
	\end{equation}

	Afin de compléter l'adimensionnement, nous prenons :
	\begin{equation}
		r_c^2 = \frac{\sigma^2}{4m\pi G\rho_0}
		\label{r_c}
	\end{equation}
	L'équation adimensionnée s'écrit alors :
	\begin{equation}
		\frac{d}{dx}\(x^2\frac{d\gamma}{dx}\) = x^2\frac{d^2\gamma}{dx^2} +
		2x\frac{d\gamma}{dx} = -x^2 \left\{-\sqrt{\frac{4\gamma}{\pi}}\(1 + \frac{ 2\gamma }{3}\) + e^{\gamma}\mathrm{erf}\(\sqrt{\gamma}\)\right\}
		\label{Pois-no_dim}
	\end{equation}

\subsection{Conditions aux limites}
	\subsubsection{Conditions au centre : $r=0$}
		\paragraph{Condition sur la dérivée :}
			Au centre de l'objet, toutes les forces ont des directions opposées --~le système étant isolé, son centre d'inertie est au repos~-- la
			somme des forces $\vec{F}=\sum_i\vec{f_i}$ est donc nulle. Ceci implique :
			\begin{equation}
				\sum_i\vec{f_i}(0)\cdot\vec{e_r} = m\frac{d\psi}{dr}(0) = 0
			\end{equation}
			Donc :
			\begin{equation}
				\left.\frac{d\gamma}{dr}\right|_{r=0} = \left.\frac{d}{dr}\(\frac{E_l - m\psi}{\sigma^2}\)\right|_{r=0} = -\frac{m}{\sigma^2}\left.\frac{d\psi}{dr}\right|_{r=0} = 0
			\end{equation}
		\paragraph{Condition sur la valeur du potentiel :}
%			L'énergie potentielle $m\psi(0)$ est l'énergie d'une particule du système n'ayant
%			pas de vitesse, il s'agit donc de l'énergie minimale d'une particule.
			En l'absence de vitesse, l'énergie minimale est atteinte pour la plus petit valeur de $m\psi(r)$, soit $m\psi(0)$.

			La quantité $\phi=E_l-m\psi$, où $E_l$ est l'énergie de libération,
			représente la quantité d'énergie à fournir à une particule à la distance $r$
			du centre du système pour qu'elle sorte du système. De plus, $\psi <0$, donc
			$\phi$ est positive ou nulle. Au centre, cette quantité vaudra :
			$$\phi(0)=E_l-m\psi(0)$$

			Nous prendrons donc la quantité initiale adimensionnée :
			\begin{equation}
				W_0 = \gamma(0)=\frac{E_l - m\psi(0)}{\sigma^2} > 0
				\label{W_0}
			\end{equation}

			Nous éviterons de la prendre nulle, car, dans ce cas, le système contient au
			plus une étoile.

	\subsubsection{Condition au bord : $r=R$}
		Au bord du système, les étoiles sont proches ou ont atteint la vitesse de
		libération, par conséquent nous avons :
		\begin{equation}
			\phi(X=R/r_c) = E_l - m\psi(X) \Rightarrow \gamma(X) = 0
		\end{equation}

		Le rayon $R$ de l'objet est donc fixé par le potentiel initial choisi.

\subsection{Paramètre du modèle}
	À ce point de l'étude, il devient évident que la physique du système va être dictée par le paramètre $W_0$ qui représente la quantité d'énergie qu'une particule va
	pouvoir gagner avant de ne plus être lié au système. Par conséquent, les paramètres $E_l$ et $\sigma^2$ vont être très importants : plus $\sigma^2$ (~la dispersion
	d'énergie~) est important, plus il va être facile de faire quitter le système à une partie des particules liées. Le raisonnement est similaire avec l'énergie de
	libération $E_l$ : plus cette énergie est proche de la valeur centrale du potentiel, plus le système va être petit, car l'énergie à récupérer pour sortir du système
	sera faible.

	Il n'existe que deux façons pour faire gagner de l'énergie à un particule en utilisant la gravitation : les collisions et le harcèlement de l'amas par un potentiel extérieur.
	Or le modèle de \textsc{King} doit être solution de l'équation de \textsc{Vlasov-Poisson}, il n'y a donc pas de collision. Un profil de \textsc{King} isolé ne doit donc pas évoluer.

	Dans la suite, nous considérerons donc le paramètre $W_0$ comme paramètre du système. Les quantités $E_l$ et $\sigma^2$ ne prendront toute leur importance que lorsque
	nous redimensionneront le problème.

\subsection{Adaptation à l'algorithme de résolution}
	L'algorithme que nous utiliserons pour résoudre cette équation est un \textsc{Runge-Kutta}
	d'ordre $4$ (~RK4~). Nous devons écrire l'équation différentielle comme un système d'ordre 1, ce que nous faisons en posant $u = \frac{d\gamma}{dx}$ :
	\begin{equation}
		\left\{\begin{array}{l}
			u = \x{\gamma}\\
			\\
			\x{u} = -\left\{-\sqrt{\frac{4\gamma}{\pi}}\(1 + \frac{ 2\gamma }{3}\) + e^{\gamma}\mathrm{erf}\(\sqrt{\gamma}\)\right\} - 2\frac{u}{x}
		\end{array}\right.
	\end{equation}

%	Euh .... Outre un problème évident de division par zéro en $x=0$, qui peut-être résolu à
%	l'aide d'un développement limité du potentiel au centre, lors de l'intégration, la solution
%	pour le potentiel est positive !!! \textcolor{red}{$\Rightarrow$ Normal, j'ai
%	oublié de repasser au potentiel !!! Je trace $\gamma$, et non $\psi$, qui est forcément
%	positif !!!}
	Un terme en $1/x$ est apparu, l'équation semble donc singulière en $0$. Un rapide
	développement limité de la fonction $\gamma$ autour de $0$ nous apprend qu'il n'en est rien, il n'y a
	pas de divergence. Pour éviter tout problème numérique, nous allons faire un développement limité d'ordre 6 autour de $0$.
	Les coefficients obtenu par ce développement sont :
%	Mais ce point va poser un problème numériquement. Nous allons donc
%	utiliser un  développement limité quand nous sommes autour de $0$. La calcul a été poussé à
%	l'ordre 6 sans preuve que les termes suivants soit négligeable. Le calcul des termes du
%	développement limité étant assez lourd à cause de la fonction $\mathrm{erf}$, il a été mené
%	avec l'aide de \textsc{Maple}. Voici les termes du développement ainsi obtenu :
	\begin{eqnarray}
		\left\{\begin{array}{l}
			\gamma(0) = W_0 \\
			\gamma'(0) = 0 \\
			\gamma''(0) = -\frac{\rho(W_0)}{3\rho_0} \\
			%\gamma''(0) = \frac{1}{3}\(-e^{W_0}erf(\sqrt{W_0})+2\sqrt{\frac{W_0}{\pi}}+\frac{4}{3}\sqrt{\frac{W_0}{\pi}}W_0\) = -\frac{\rho(0)}{3\rho_0} \\
			\gamma'''(0) = 0 \\
			%\gamma^{(4)}(0) = -\frac{6\gamma''(0)}{5}\left\{-\sqrt{\frac{W_0}{\pi}} -
			%\frac{\sqrt{W_0}}{2 W_0\sqrt{\pi}} + \frac{e^{W_0}\mathrm{erf}\(\sqrt{W_0}\)}{2}
			%+ \frac{1}{2\sqrt{\pi W_0}}\right\} = \frac{2\rho(0)}{5\rho_0}\left\{\sqrt{\frac{W_0}{\pi}} -
			%\frac{e^{W_0}\mathrm{erf}(\sqrt{W_0})}{2}\right\} \\
			\gamma^{(4)}(0) = \frac{2\rho(W_0)}{5\rho_0}\left\{\sqrt{\frac{W_0}{\pi}} -
			\frac{e^{W_0}\mathrm{erf}(\sqrt{W_0})}{2}\right\} \\
			\gamma^{(5)}(0) = 0
		\end{array}\right.
	\end{eqnarray}
	Nous avons alors les développements suivant autour de $0$ :
	\begin{align}
		\Rightarrow \gamma(x) &=& W_0 - \frac{\rho(W_0)}{3\rho_0}\frac{x^2}{2!} + \frac{2\rho(0)}{5\rho_0}\left\{\sqrt{\frac{W_0}{\pi}} -
		\frac{e^{W_0}\mathrm{erf}(\sqrt{W_0})}{2}\right\}\frac{x^4}{4!} + o(x^6) \notag \\
		\Rightarrow \frac{d\gamma(x)}{dx} &=& u = - \frac{\rho(W_0)}{3\rho_0}x + \frac{2\rho(0)}{5\rho_0}\left\{\sqrt{\frac{W_0}{\pi}} -
		\frac{e^{W_0}\mathrm{erf}(\sqrt{W_0})}{2}\right\}\frac{x^3}{3!} + o(x^5) \notag \\
		\Rightarrow \frac{d^2\gamma(x)}{dx^2} &=& \x{u} = \frac{\rho(W_0)}{3\rho_0} + \frac{2\rho(0)}{5\rho_0}\left\{\sqrt{\frac{W_0}{\pi}} -
		\frac{e^{W_0}\mathrm{erf}(\sqrt{W_0})}{2}\right\}\frac{x^2}{2!} + o(x^4) \notag
	\end{align}

	La résolution des équations nous permet alors d'avoir les graphiques de la
	figure~\ref{King_Modele-test} (~les graphes sont adimensionnés~).% mais pas normalisé~).
%	\paragraph{Problème : Résolu :}
%	Tant que la variable $x$ est proche de $0$, j'utilise les développements limités, puis passé
%	un certain seuil, je réutilise l'équation différentielle. Le résultat obtenu est la figure~\ref{King_Modele-err}
%	au lieu de la figure~\ref{King_Modele} (~figure pour laquelle je démarre à $x=1.10^{-6}$ au
%	lieu de $x=0$~). Il semblerait que mon algorithme de RK4 à pas variable ait du mal à suivre,
%	en plus d'avoir, apparemment, une erreur d'adimensionnement/normalisation.
%	\subparagraph{Solution pour le RK4 ne suivant pas :} J'utilisais le développement de $\gamma(x)$ pour calculer la pente
%	et celui de $\x{\gamma(x)}$ pour calculer la dérivée seconde.

%	\textcolor{red}{RAAAAAAHHHHHHHHH !!!!!!!!!! Plein d'erreur de signe dans les calculs et le
%	programme !!!!!!!!!!! Faut tout revérifier !!!!!!!!!!!!!!!!!!!}

%	\begin{figure}[ht!]
%			\begin{minipage}[b]{0.40\linewidth}
%				\centering \includegraphics[scale=0.40]{graphe/erreur_king.png}
%			\end{minipage}\hfill
%			\begin{minipage}[b]{0.48\linewidth}
%				\centering \includegraphics[scale=0.40]{graphe/erreur-pot_king.png}
%			\end{minipage}
%			\caption{Densité et potentiel d'un modèle de \textsc{King} pour les
%			conditions initiales (~au centre~) : $\gamma(0) = \frac{E_l -
%			m\psi(0)}{\sigma^2} = 9$}
%			\label{King_Modele-err}
%	\end{figure}
%	\begin{figure}[ht!]
%			\begin{minipage}[b]{0.40\linewidth}
%				\centering \includegraphics[scale=0.60]{graphe/densite_king.pdf}
%			\end{minipage}\hfill
%			\begin{minipage}[b]{0.48\linewidth}
%				\centering \includegraphics[scale=0.60]{graphe/potentiel_king.pdf}
%			\end{minipage}
%			\caption{Densité et potentiel d'un modèle de \textsc{King} pour les
%			conditions initiales (~au centre~) : $\gamma(0) = \frac{E_l - m\psi(0)}{\sigma^2} = 3$, $\x{\gamma}(0) = 0$}
%			\label{King_Modele}
%	\end{figure}

	\begin{figure}[ht!]
			\begin{minipage}[b]{0.40\linewidth}
				\centering \includegraphics[scale=0.60]{graphe/densite_pluri-king.pdf}
			\end{minipage}\hfill
			\begin{minipage}[b]{0.48\linewidth}
				\centering \includegraphics[scale=0.60]{graphe/densite_pluri_limite-king.pdf}
			\end{minipage}
			\caption{Densité d'un modèle de \textsc{King} pour les conditions initiales
			(~au centre~) : $\gamma(0) = 1$ (~courbe rouge~), $5$ (~courbe verte~), $9$ (~courbe bleu~), $14$ (~courbe violette~), $\x{\gamma}(0) = 0$}
			\label{King_Modele-test}
	\end{figure}

	La figure~\ref{King_Modele-test} montre l'aspect de la densité d'un modèle de \textsc{King} en fonction de $W_0$ et nous indique aussi que ce modèle posséde une structure cœur-halo
	(~une partie quasiment constante suivi d'une décroissance auto-similaire de pente $\alpha$~).
	Plus $\phi$ est grand, plus le cœur est dense (~le graphe de droite trace $\rho(x)/\rho(0)$~). Nous pouvons aussi observer que la pente n'est pas la même :
	elle diminue (~la courbe tend plus vite vers $0$~). Ce constat nous amène à nous poser une question : existe-t-il un
	lien entre la pente et le rayon du cœur ?
	\FloatBarrier


%%		\newpage
	%\section[Relation entre les paramètres]{Relation entre les différents paramètres d'un modèle de \textsc{King}\label{pente-coeff_sec}}
		%	Nous nous intéressons maintenant à l'existence d'un lien entre le rayon du cœur et la pente
du halo. Il n'y a apparemment aucun moyen de faire ça analytiquement (~en tout cas, je n'ai pas encore trouvé~).
Par contre, lorsque je traçais des graphes, obtenu avec le code abordé ci-dessus,
j'ai remarqué une forte dépendance\footnote{logique~!?!} entre la pente, en $\log-\log$, et $W_0$
(~voir graphe de droite de la figure~\ref{King_Modele-test}~). De même pour le rayon à $10\%$
de l'objet étudié. L'idée que j'ai utilisé pour faire le lien entre les deux quantités a été de
regarder leur comportement en fonction des conditions initiales puis de combiner ensuite les 2
comportements pour obtenir une relation empirique entre ces deux paramètres. %courbe les reliant.

\subsection{Calcul des pentes pour différentes conditions initiales\label{pente-critére}}

	Pour obtenir les pentes, nous traçons dans un diagramme $\log-\log$ la densité, puis, à l'aide
du logiciel \textsc{GNUPlot}\footnote{\url{http://www.gnuplot.info/}} nous ajustons une équation du type $a x+b$ à la partie linéaire de la
courbe, le coefficient $a$ représentant la pente.

	Le problème avec cette méthode, c'est que, pour certaines conditions initiales, il n'y a
presque pas de halo : la densité d'étoile chute brutalement (~voir figure~\ref{ci-pente_1}
page~\pageref{ci-pente_1} en annexe~). Les valeurs des pentes pour des systèmes ayant une énergie de
libération et une énergie minimale très proche, ou une grande dispersion d'énergie, sont donc assez
peu fiables. Cette brusque pente vient de la faible quantité d'énergie à fournir pour faire
s'échapper une ou plusieurs particules du système.

	Le traitement a ensuite été automatisé en considérant que le halo correspond à la zone :
	\begin{align}
		10^{-4} < \dfrac{\rho(r)}{\rho(0)} < 0.1 \label{pente::critere}
	\end{align}

	Pour obtenir les points rouge de la figure~\ref{coeff_evo}, nous avons fait varier $\gamma(0) = W_0 = 1$ à $W_0 = 21$.
	La courbe verte est obtenu en ajustant aux points rouge, et en respectant le critère~\ref{pente::critere}, une équation du type $a e^{b x} + c$. %$a x^2 + b x + c$.
	Les coefficients obtenus sont donnés sur la courbe~\ref{coeff_evo}.
%	\begin{table}[hbt!]
%		\begin{center}
%			\begin{tabular}{|c|c|c|}
%				\hline
%				Coefficient & Valeur & Erreur \\
%				\hline
%				\hline
%				$a$       &         -10.0698      &  $\pm 0.2423$       (~$2.406\%$~) \\
%				\hline
%				$b$       &         0.220152      &  $\pm 0.01075$      (~$4.883\%$~) \\
%				\hline
%				$c$       &         -1.63409      &  $\pm 0.09393$      (~$5.748\%$~) \\
%				$a$       &        $-0.0157022$   &   $\pm 0.002226$ (~$14.18\%$~) \\
%				$b$       &        $0.443128$     &   $\pm 0.03663$  (~$8.266\%$~) \\
%				$c$       &        $-5.33431$     &   $\pm 0.1274$   (~$2.388\%$~) \\
%				\hline
%			\end{tabular}
%		\end{center}
%		\caption{Valeur des coefficients donnée par l'ajustement pour les pentes}
%		\label{pente-fit}
%	\end{table}
	\begin{figure}[hbt!]
%		\centering \includegraphics[scale=1.00]{../Resol_King/img-king/pente-w0.pdf} %{graphe/evo-coeff_ci.pdf}
		\centering \includegraphics[scale=1.00]{graphe/pente-w0.pdf}%{img-king/pente-w0.pdf} %{graphe/evo-coeff_ci.pdf}
		\caption{Évolution des pentes pour différentes conditions initiales}
		\label{coeff_evo}
	\end{figure}

	Il est réconfortant de remarquer sur cette courbe la présence d'une asymptote horizontale autour de $-2$ correspondant à la pente d'une sphère isotherme singulière.
	En effet, pour $W_0 \to \infty$ le modèle de \textsc{King} se rapproche d'une sphère isotherme de rayon infini dont le halo est très proche de celui d'une SIS.
%	\FloatBarrier

\subsection{Calcul du rayon à $10\%$ pour différentes conditions initiales\label{r_10}}

	L'un des paramètres libres les plus utilisés du modèle de \textsc{King} est le rayon du cœur (formule~\ref{r_c}, page~\pageref{r_c}~).
%	Il serait ainsi intéressant de prendre cette valeur et de la tracer en fonction de la condition initiale $W_0$,
%	ce qui nous permettrait ainsi de la relier à la pente.
	Il est donc intéressant de relier les paramètres $r_c$ et $W_0$. Ainsi, en utilisant la relation que nous avons déterminé entre $W_0$ et la pente $\alpha$ du halo, nous obtiendrons une relation entre $r_c$ et $\alpha$

	Mais deux problèmes se posent :
	\begin{enumerate}
		\item de la même manière que dans le paragraphe précédent, la détermination de $r_c$ a un problème : quand
	$W_0$ devient trop grand, le système se \og~dilue~\fg, c'est-à-dire que la différence entre
	cœur et halo est de moins en moins évidente, comme sur la figure~\ref{w_0-5_10}
	page~\pageref{w_0-5_10} (~nous sommes dans le cas d'un amas effondré~),
		\item le programme de résolution qui nous permet d'avoir ces courbes est adimensionné par rapport à ce rayon, rendant ainsi très difficile
			le redimensionnement des résultats et leur comparaison avec les données observationnelles.
	\end{enumerate}

	Pour palier à ce dernier problème, il est possible de prendre, plutôt que le rayon du cœur, le rayon à $10\%$.
%	Le rayon à $10\%$ est le rayon à partir duquel $10\%$ de la masse totale de l'amas se trouve à l'intérieur.
	Il est défini comme :
	\begin{quote}
		Le rayon à $10\%$ du système est atteint lorsque, du centre vers le bord du
		système, la densité a diminué de $10\%$ :
		\begin{align}
			\frac{\rho(x_{10})}{\rho(0)} = 0.1
		\end{align}
	\end{quote}

%	Le rayon à $10\%$ est plus délicat à obtenir.
%	Pour obtenir la figure~\ref{coeur_evo}, nous avons considéré que le rayon à $10\%$ correspondait à la distance pour laquelle la densité
%	ne pouvait plus être considérée comme constante : nous avons placé, à la main, la valeur du
%	rayon au moment où la pente devenait nette\footnote{cette contrainte dépend des gens et n'est donc
%	pas très adapté à ce que nous faisons}.

	Une fois ces valeurs de rayon mesurées, nous avons tracé la courbe rouge de la
	figure~\ref{coeur_evo2} que nous avons ajustée avec la fonction $a e^{b x} + c$ et avons
	obtenu les coefficients indiqués sur la courbe.

%	\begin{table}[hbt!]
%		\begin{center}
%			\begin{tabular}{|c|c|c|}
%				\hline
%				Coefficient & Valeur & Erreur \\
%				\hline
%				\hline
%				$d$       &       $1.86332$      &   $\pm 0.04078$ 	(~$2.189\%$~)\\
%				\hline
%				$e$       &       $-0.635746$    &   $\pm 0.01585$    	(~$2.494\%$~)\\
%				\hline
%				$f$       &       $0.0049355$    &   $\pm 0.003953$     (~$80.1\%$~)\\
%				\hline
%			\end{tabular}
%		\end{center}
%		\caption{Valeur des coefficients donnée par l'ajustement pour les rayons à $10\%$}
%		\label{coeur-fit}
%	\end{table}
%	\begin{figure}[hbt!]
%		\centering \includegraphics[scale=1.00]{graphe/evo-coeur_ci.pdf}
%		\caption{Évolution du rayon à $10\%$ calculé à la main pour différents $W_0$}
%		\label{coeur_evo}
%	\end{figure}

%	\textcolor{red}{\underline{Attention :}} il peut être intéressant de refaire ces courbes en
%	incluant le calcul du rayon à $10\%$ directement dans le code (~selon la définition choisi~).

%	Le calcul a été refait en sortant le rayon à $10\%$ directement du code résolvant les
%	équations. De cette manière, la courbe reste sensiblement la même, et les coefficients ont par contre
%	bien changé, comme nous pouvons le voir sur la courbe~\ref{coeur_evo2}.
	\begin{figure}[hbt!]
		\centering \includegraphics[scale=1.00]{graphe/evo-coeur_ci2.pdf}
		\caption{Évolution du rayon à $10\%$ calculé lors de la résolution numérique pour différents $W_0$}
		\label{coeur_evo2}
	\end{figure}
%	\begin{table}[hbt!]
%		\begin{center}
%			\begin{tabular}{|c|c|c|}
%				\hline
%				Coefficient & Valeur & Erreur \\
%				\hline
%				\hline
%				$d$        &       $7.36877$      &   $\pm 0.2352$       (~$3.191\%$~)\\
%				\hline
%				$e$        &       $-0.633846$    &   $\pm 0.02309$      (~$3.642\%$~)\\
%				\hline
%				$f$        &       $0.0419222$    &   $\pm 0.02291$      (~$54.66\%$~)\\
%				\hline
%			\end{tabular}
%		\end{center}
%		\caption{Valeur des coefficients donnée par l'ajustement pour les rayons à $10\%$
%		(~v2.0~)}
%		\label{coeur-fit2}
%	\end{table}
	\FloatBarrier

\subsection{Lien entre les deux\label{ssec::LinkBetween}}

	Maintenant que nous avons la dépendance des rayons à $10\%$ et des pentes en fonction de la
	condition initiale $W_0$, nous pouvons regarder comment varie la pente en fonction de la
	taille de l'amas. Sur la figure~\ref{coeff-coeur2}, nous avons tracé en vert les données, et
	en rouge la courbe formée par nos deux expressions obtenues par ajustement (~les coefficients
	utilisés sont les mêmes que ceux donnés sur les figures~\ref{coeff_evo} et~\ref{coeur_evo2}~).
%	\begin{figure}[hbt!]
%		\centering \includegraphics[scale=1.00]{graphe/evo-coeff_coeur.pdf}
%		\caption{Évolution de la pente calculé à la main en fonction du rayon à $10\%$}
%		\label{coeff-coeur}
%	\end{figure}

	En supposant que nos ajustements sont valables pour toutes les valeurs possibles de $W_0$,
	nous pouvons tenter d'obtenir une expression analytique reliant nos deux quantités. En
	partant de :
	\begin{align}
		\left\{\begin{array}{l}
			f(W_0) = \alpha = a e^{b W_0} + c \\ %a W_0^2 + b W_0 + c \\
			g(W_0) = x_{10\%} = d e^{e W_0} + f
		\end{array}\right.
	\end{align}
	Et en les combinant :
	\begin{align}
		\Rightarrow e^{e W_0} &= \frac{x_{10\%} - f}{d} \notag\\
				W_0   &= \frac{1}{e}\ln\(\frac{x_{10\%} - f}{d}\) \\
		\Rightarrow \alpha &= a \exp\(\frac{1}{e}\ln\(\frac{x_{10\%} - f}{d}\)\) + c %\frac{a}{e^2}\ln^2\(\frac{x_{10\%} - f}{d}\) +
					%\frac{b}{e}\ln\(\frac{x_{10\%} - f}{d}\) + c \notag \\
%		\intertext{Nous obtenons :}
%		\beta &= a e^{1/e} %\frac{1}{e}\ln\(\frac{x_{10\%} - f}{d}\)\(\frac{a}{e}\ln\(\frac{x_{10\%} - f}{d}\) + b\) + c
	\end{align}

	Cette expression s'ajuste \og~bien~\fg~à la courbe, mais \textsc{GNUPlot} donne des erreurs
	assez élevées.
%	\begin{table}[hbt!]
%		\begin{center}
%			\begin{tabular}{|c|c|c|}
%				\hline
%				Coefficient & Valeur & Erreur \\
%				\hline
%				\hline
%				$q=a/e^2$       &      $-0.0394991$   &   $\pm 0.02115$ (~$53.55\%$~)\\
%				\hline
%				$s=1/d$         &      $0.524033$     &   $\pm 4.538$   (~$865.9\%$~)\\
%				\hline
%				$j=f/d$         &      $0.00254127$   &   $\pm 0.02122$ (~$835.2\%$~)\\
%				\hline
%				$k=b/e$         &      $-0.688944$    &   $\pm 0.8111$  (~$117.7\%$~)\\
%				\hline
%				$l=c$           &      $-5.3267$      &   $\pm 6.127$   (~$115\%$~)\\
%				\hline
%			\end{tabular}
%		\end{center}
%		\caption{Valeur des coefficients donnée par l'ajustement pour les pentes en
%		fonctions des rayons à $10\%$}
%		\label{param-fit}
%	\end{table}
%	Les corrections dans la détermination du rayon à $10\%$ ont amélioré l'ajustement de cette courbe, en plus d'avoir changé l'échelle des $x$. Les changements sont
%	donnés sur la figure~\ref{coeff-coeur2}.
%	\begin{table}[hbt!]
%		\begin{center}
%			\begin{tabular}{|c|c|c|}
%				\hline
%				Coefficient & Valeur & Erreur \\
%				\hline
%				\hline
%				$q$       &       $-0.0526849$    &  $\pm 0.02245$      (~$42.6\%$~)\\
%				\hline
%				$s$       &       $0.211927$      &  $\pm 0.4648$       (~$219.3\%$~)\\
%				\hline
%				$j$       &       $0.0081136$     &  $\pm 0.02144$      (~$264.2\%$~)\\
%				\hline
%				$k$       &       $-0.716362$     &  $\pm 0.1606$       (~$22.42\%$~)\\
%				\hline
%				$l$       &       $-5.01452$      &  $\pm 1.478$        (~$29.47\%$~)\\
%				\hline
%			\end{tabular}
%		\end{center}
%		\caption{Valeur des coefficients donnée par l'ajustement pour les pentes en
%		fonctions des rayons à $10\%$ (~v2.0~)}
%		\label{param-fit2}
%	\end{table}
	\begin{figure}[hbt!]
		\centering \includegraphics[scale=1.00]{graphe/evo-coeff_coeur3.pdf}
		\caption{Évolution de la pente en fonction du rayon à $10\%$}
		\label{coeff-coeur2}
	\end{figure}

%	Ensuite, il est intéressant de comparer les données observationnelle à notre travail. Le problème auquel auquel nous avons fait face est
%	que nous avons adimensionné le problème par rapport au rayon de cœur, qu'il n'est donc pas possible d'obtenir par la résolution numérique.
%	Ce travail n'est donc pas utilisable directement pour une comparaison avec les observations. Par contre, le rayon de cœur est mesurable, comme indiqué dans~\cite{Djo-rc}.
	\FloatBarrier

\subsection{Paramètre de concentration : $c$}
	Le dernier paramètre d'un modèle de \textsc{King} sur lequel nous pouvons jouer est le paramètre de concentration $c$.
	Dans le modèle de King, la concentration s'écrit :
	\begin{align}
		c = \log_{10}\(\frac{R}{r_c}\)
	\end{align}
	avec $R$ la taille du système (~tel que $\rho(R) = 0$~) et $r_c$ le rayon de cœur. Ce paramètre est infini pour la sphère isotherme singulière, comme $W_0$.
	Notons que dans le cas de la SIS $W_0$, $R$ et $c$ sont infini.

	Ce paramètre permet aussi de fixer l'état du système. Trouver une fonction ou une équation permettant de faire le lien entre la concentration et la condition
	initiale $W_0$ nous permettrait de trouver le modèle de King correspondant à un amas (~les paramètres $c$, $r_c$ et $\sigma^2$ le représentant~). Nous avons donc inclus
	dans la résolution des équations un calcul de ce paramètre de concentration. Nous avons ainsi pu obtenir la courbe~\ref{concentre}
	\begin{figure}[hbt!]
		\centering \includegraphics[scale=1.00]{graphe/concentration-king.pdf}
		\caption{Évolution de la concentration avec la condition initiale $W_0$}
		\label{concentre}
	\end{figure}

	Le paramètre de concentration est mesurable pour chaque amas (~en tout cas, c'est un des paramètres du catalogue de \textsc{Harris}~\cite{Harris}~).
	Mais le fait que la courbe ne soit pas bijective nous impose plusieurs modèles pour un seul paramètre de concentration.
	\FloatBarrier


	\section{Température d'un modèle de \King \label{sec::temp}}
		L'une des premières questions que l'on peut se poser, par rapport au modèle précèdent~: le modèle de \King est il isotherme~?

%Dans un premier temps, nous rappelons les expressions des fonctions de distribution et densité pour ce modèle~:
%\begin{align}
%	f_K(E) &= \begin{cases}
%		\rho_0 \(2\pi m\sigma^2\)^{-3/2}\( e^{\frac{E_l-E}{\sigma^2}} - 1\) & \text{si $E < E_l$} \\
%		0 & \text{si $E > E_l$}
%	\end{cases} \\
%	\rho(r) &= \rho_0 \(-\sqrt{\frac{4\phi}{\pi\sigma^2}}\(1 + \frac{ 2\phi }{3\sigma^2}\) + e^{\frac{\phi}{\sigma^2}}\mathrm{erf}\(\sqrt{\phi}/\sigma\)\)
%\end{align}

La température est définit comme étant proportionnelle à la dispersion de vitesse tel que~:
\begin{align}
	T(r) &\propto \dfrac{\int p^2 f_K(E) \vdp}{m^2 \int f_K(E) \vdp} & v^2 = \frac{p^2}{m^2}
\end{align}

\subsection{Évolution de la température avec le rayon}

Le numérateur s'écrit~:
\begin{align}
	\int_0^{p_l} \dfrac{p^2}{m^2} f_K\(E\) \mathrm{d}\vec{p} &= \int_0^{p_l} \dfrac{p^2}{m^2} \rho_0 \(2\pi m\sigma^2\)^{-3/2}\( e^{\frac{E_l-E}{\sigma^2}} - 1\) 4\pi p^2 \ddp \notag \\
	    &= \dfrac{4\pi\rho_0}{\(2\pi m\sigma^2\)^{3/2}}\(\int_0^{p_l} \frac{p^4}{m^2} e^{\frac{E_l-E}{\sigma^2}}\ddp -
	    \int_0^{p_l} \frac{p^4}{m^2}\ddp \) \notag
\intertext{En utilisant $E = \dfrac{p^2}{2m} + m\psi$, et à l'aide d'intégration par partie, il est possible de simplifier plusieurs étapes de l'intégration~:}
	    &= \frac{4\pi\rho_0}{m^2\(2\pi m\sigma^2\)^{3/2}}
	    	\(
			e^{\frac{E_l - m\psi}{\sigma^2}}
			\left[
				3m\sigma^2\int_0^{p_l} p^2 e^{-\frac{p^2}{2m\sigma^2}}\ddp
				- m\sigma^2 p_l^3 e^{-\frac{p_l^2}{2m\sigma^2}}
			\right]
			- \dfrac{p_l^5}{5}
	    	\) \notag \\
	    &= \frac{4\pi\rho_0}{m^2\(2\pi m\sigma^2\)^{3/2}}
		\(
			e^{\frac{E_l - m\psi}{\sigma^2}}
	    		\left[
				  3\(m\sigma^2\)^2 \int_0^{p_l} e^{-\frac{p^2}{2m\sigma^2}} \ddp
				- 3\(m\sigma^2\)^2p_le^{-\frac{p_l^2}{2m\sigma^2}}
			\right. \right. \notag \\
				&\qquad \left.\vphantom{ e^{-\frac{p_l^2}{2m\sigma^2}}}\left.\vphantom{e^{-\frac{p_l^2}{2m\sigma^2}}}
				- m\sigma^2 p_l^3 e^{-\frac{p_l^2}{2m\sigma^2}}
			\right]
	    		- \dfrac{p_l^5}{5}
		\) \notag \\
	    &= \frac{4\pi\rho_0}{m^2\(2\pi m\sigma^2\)^{3/2}}
		\(
			e^{\frac{E_l - m\psi}{\sigma^2}}\(m\sigma^2\)^2
	    		\left[
				  3 \int_0^{p_l} e^{-\frac{p^2}{2m\sigma^2}} \ddp
				- 3p_le^{-\frac{p_l^2}{2m\sigma^2}}
				\(1 + \dfrac{p_l^2}{3m\sigma^2}\)
			\right] \right. \notag \\
	    &\qquad \left.\vphantom{ e^{-\frac{p_l^2}{2m\sigma^2}}}
	    		- \dfrac{p_l^5}{5}
		\) \notag \\
	    &= \frac{4\pi\rho_0}{m^2\(2\pi m\sigma^2\)^{3/2}}
		\(
			e^{\frac{E_l - m\psi}{\sigma^2}}\(m\sigma^2\)^2
	    		\left[
				3 \dfrac{\sqrt{\pi}}{2}\sqrt{2m\sigma^2}\erf\(\dfrac{p_l}{\sqrt{2m\sigma^2}}\)
				- 3p_le^{-\frac{p_l^2}{2m\sigma^2}}
				\(1 + \dfrac{p_l^2}{3m\sigma^2}\)
			\right] \right. \notag \\
	    &\qquad \left.\vphantom{ e^{-\frac{p_l^2}{2m\sigma^2}}}
	    		- \dfrac{p_l^5}{5}
		\) \notag \\
%	    &= \frac{12\pi\rho_0 \sigma^4}{\(2\pi m\sigma^2\)^{3/2}} e^{\frac{E_l - m\psi\(r\)}{\sigma^2}} h\(p_l\)\dr
\intertext{avec $p_l = \sqrt{2m\(E_l - m\psi(r)\)} = \sqrt{2m\phi(r)}$~:}
	    &= \frac{4\pi\rho_0}{m^2\(2\pi m\sigma^2\)^{3/2}}
		\(
			e^{\frac{\phi}{\sigma^2}}\(m\sigma^2\)^2
	    		\left[
				3\dfrac{\sqrt{\pi}}{2} \sqrt{2m\sigma^2}\erf\(\dfrac{\sqrt{2m\phi(r)}}{\sqrt{2m\sigma^2}}\)
			\right. \right. \notag \\
	    &\qquad \left.\vphantom{ e^{-\frac{2m\phi(r)}{2m\sigma^2}}}\left.\vphantom{ e^{-\frac{2m\phi(r)}{2m\sigma^2}}}
			        - 3\sqrt{2m\phi(r)}e^{-\frac{2m\phi(r)}{2m\sigma^2}}
				\(1 + \dfrac{2m\phi(r)}{3m\sigma^2}\)
			\right]
			- \dfrac{\(2m\phi(r)\)^{5/2}}{5}
		\) \notag \\
	    &= \frac{4\pi\rho_0}{m^2\(2\pi m\sigma^2\)^{3/2}}
		\(
			\dfrac{3}{2} \(m\sigma^2\)^2\sqrt{2m\pi\sigma^2}
	    		\left[
				e^{\frac{\phi}{\sigma^2}}\erf\(\sqrt{\dfrac{\phi(r)}{\sigma^2}}\)
			\right. \right. \notag \\
	    &\qquad \left.\vphantom{\(\sqrt{\dfrac{\phi(r)}{\sigma^2}}\) }\left.\vphantom{\(\sqrt{\dfrac{\phi(r)}{\sigma^2}}\)}
				- \sqrt{\dfrac{4\phi(r)}{ \pi\sigma^2}}
				\(1 + \dfrac{2\phi(r)}{3\sigma^2}\)
			\right]
			- \dfrac{\(2m\phi(r)\)^{5/2}}{5}
		\) \notag
\end{align}

La température va alors s'écrire~:
\begin{align}
	T(r) &\propto \frac{4\pi\rho_0}{m^2\(2\pi m\sigma^2\)^{3/2}}
		\(
		\dfrac{
			\dfrac{3}{2} \(m\sigma^2\)^2\sqrt{2m\pi\sigma^2}
	    		\left[
				e^{\frac{\phi}{\sigma^2}}\erf\(\sqrt{\dfrac{\phi(r)}{\sigma^2}}\)
				- \sqrt{\dfrac{4\phi(r)}{ \pi\sigma^2}}
				\(1 + \dfrac{2\phi(r)}{3\sigma^2}\)
			\right]
		}
		{
			\rho_0 \(-\sqrt{\frac{4\phi}{\pi\sigma^2}}\(1 + \frac{ 2\phi }{3\sigma^2}\) + e^{\frac{\phi}{\sigma^2}}\mathrm{erf}\(\sqrt{\phi}/\sigma\)\)
		}
			\right. \notag \\
	    &\qquad \left.
		- \dfrac{
			\(2m\phi(r)\)^{5/2}
		}
		{
			5 \rho_0 \(-\sqrt{\frac{4\phi(r)}{\pi\sigma^2}}\(1 + \frac{ 2\phi(r) }{3\sigma^2}\) + e^{\frac{\phi(r)}{\sigma^2}}\mathrm{erf}\(\sqrt{\phi(r)}/\sigma\)\)
		}
		\) \notag \\
	&\propto \frac{4\pi}{m^2\(2\pi m\sigma^2\)^{3/2}} \(m\sigma^2\)^2
		\(
			\dfrac{3}{2}\sqrt{2m\pi\sigma^2}
		- \dfrac{
			\(2m\phi(r)\)^{5/2} \(m\sigma^2\)^{-2}
		}
		{
			5 \(-\sqrt{\frac{4\phi(r)}{\pi\sigma^2}}\(1 + \frac{ 2\phi(r) }{3\sigma^2}\) + e^{\frac{\phi(r)}{\sigma^2}}\mathrm{erf}\(\sqrt{\phi(r)}/\sigma\)\)
		}
		\) \notag \\
	&\propto \sqrt{\dfrac{2\sigma^2}{m^3\pi}}							%\frac{4\pi}{m^2\(2\pi m\sigma^2\)^{3/2}} \(m\sigma^2\)^2
		\(
			\dfrac{3}{2}\sqrt{2m\pi\sigma^2}
		- \dfrac{
			\(2m\phi(r)\)^{5/2} \(m\sigma^2\)^{-2}
		}
		{
			5 \(-\sqrt{\frac{4\phi(r)}{\pi\sigma^2}}\(1 + \frac{ 2\phi(r) }{3\sigma^2}\) + e^{\frac{\phi(r)}{\sigma^2}}\mathrm{erf}\(\sqrt{\phi(r)}/\sigma\)\)
		}
		\)
\end{align}
Nous obtenons alors une expression de la température qui n'est pas analytique. Pour obtenir la température, nous reprenons le noyau développé autour
du modèle de \King pendant le stage, puis nous appliquons la formule obtenue.
Les courbes obtenues sont données tracé sur les graphes~\ref{courbe::Temp}.
\begin{figure}[H]
	\begin{minipage}[b]{0.40\linewidth}
		\centering \includegraphics[scale=0.60]{theorie/graphe/temperature_0-997636.pdf}
	\end{minipage}\hfill
	\begin{minipage}[b]{0.48\linewidth}
		\centering \includegraphics[scale=0.60]{theorie/graphe/temperature_3-98531.pdf}
	\end{minipage}
	\centering \includegraphics[scale=0.60]{theorie/graphe/temperature_11-1096.pdf}
	\caption{Courbes de température pour différents $W_0$\label{courbe::Temp}}
\end{figure}
%\FloatBarrier

Toutes ces courbes sont normalisées par rapport à la température centrale donnée par~:
\begin{align}
	T(0) &\propto \sqrt{\dfrac{2\sigma^2}{m^3\pi}}								%\frac{4\pi\rho_0}{m^2\(2\pi m\sigma^2\)^{3/2}} \(m\sigma^2\)^2
		\(
			\dfrac{3}{2}\sqrt{2m\pi\sigma^2}
		- \dfrac{
			\(2m\phi(0)\)^{5/2} \(m\sigma^2\)^{-2}
		}
		{
			5 \(-\sqrt{\frac{4\phi(0)}{\pi\sigma^2}}\(1 + \frac{ 2\phi(0) }{3\sigma^2}\) +
			e^{\frac{\phi(0)}{\sigma^2}}\mathrm{erf}\(\sqrt{\phi(0)}/\sigma\)\)
		}
		\) \notag \\
	   &\propto \sqrt{\dfrac{2\sigma^2}{m^3\pi}}								%\frac{4\pi\rho_0}{m^2\(2\pi m\sigma^2\)^{3/2}} \(m\sigma^2\)^2
		\(
			\dfrac{3}{2}\sqrt{2m\pi\sigma^2}
		- \dfrac{
			\(2m\sigma^2W_0\)^{5/2} \(m\sigma^2\)^{-2}
		}
		{
			5 \(-\sqrt{\frac{4W_0}{\pi}}\(1 + \frac{ 2W_0 }{3}\) +
			e^{W_0}\mathrm{erf}\(\sqrt{W_0}\)\)
		}
		\)
\end{align}
%\FloatBarrier

\subsection{Température moyenne d'un modèle de \King}
\label{Calc::Temp}

Au vu de ces courbes, nous pouvons considérer que la sphère de \King est bien isotherme. Il est alors possible de déterminer une température moyenne~:
\begin{align}
%	T_{\mathrm{moy}} &= \dfrac{\displaystyle{\int}_0^R 4\pi r^2 T(r) f_K(E) \dr}{\displaystyle{\int}_0^R 4\pi r^2 \dr} \\
%			 &= \dfrac{\displaystyle{\int}_0^R r^2 T(r) \rho_0 \(2\pi m\sigma^2\)^{-3/2}\( e^{\frac{E_l-E}{\sigma^2}} - 1\)}
%			 	  {\displaystyle{\int}_0^R r^2      \rho_0 \(2\pi m\sigma^2\)^{-3/2}\( e^{\frac{E_l-E}{\sigma^2}} - 1\)}
T_{\mathrm{moy}} &= \dfrac{\displaystyle{\int}\displaystyle{\int}\frac{p^2}{m^2}f_K(E)\vdp\vdr}{\displaystyle{\int}\displaystyle{\int}f_K(E)\vdp\vdr} \notag \\ %\int_0^R 4\pi r^2 T(r) \dr \\
	    &= \frac{4\pi\rho_0}{Mm^2\(2\pi m\sigma^2\)^{3/2}}
		    \displaystyle{\int}_0^R 4\pi r^2\(
			\dfrac{3}{2} \(m\sigma^2\)^2\sqrt{2m\pi\sigma^2}
	    		\left[
				e^{\frac{\phi}{\sigma^2}}\erf\(\sqrt{\dfrac{\phi(r)}{\sigma^2}}\)
			\right. \right. \notag \\
	    &\qquad \left.\vphantom{\(\sqrt{\dfrac{\phi(r)}{\sigma^2}}\) }\left.\vphantom{\(\sqrt{\dfrac{\phi(r)}{\sigma^2}}\)}
				- \sqrt{\dfrac{4\phi(r)}{ \pi\sigma^2}}
				\(1 + \dfrac{2\phi(r)}{3\sigma^2}\)
			\right]
			- \dfrac{\(2m\phi(r)\)^{5/2}}{5}
		\) \dr \notag \\
	    &= \frac{4\pi}{Mm^2\(2\pi m\sigma^2\)^{3/2}}
		    \(
			\dfrac{3}{2} \(m\sigma^2\)^2\sqrt{2m\pi\sigma^2}
	    		\displaystyle{\int}_0^R 4\pi r^2 \rho_0\left[
				e^{\frac{\phi}{\sigma^2}}\erf\(\sqrt{\dfrac{\phi(r)}{\sigma^2}}\)
			\right. \right. \notag \\
	    &\qquad \left.\vphantom{\(\sqrt{\dfrac{\phi(r)}{\sigma^2}}\) }\left.\vphantom{\(\sqrt{\dfrac{\phi(r)}{\sigma^2}}\)}
				- \sqrt{\dfrac{4\phi(r)}{ \pi\sigma^2}}
				\(1 + \dfrac{2\phi(r)}{3\sigma^2}\)
			\right] \dr
			-\displaystyle{\int}_0^R 4\pi r^2 \rho_0 \dfrac{\(2m\phi(r)\)^{5/2}}{5} \dr
		\) \notag \\
	    &= \frac{4\pi}{Mm^2\(2\pi m\sigma^2\)^{3/2}}
		    \(
			\dfrac{3}{2} \(m\sigma^2\)^2\sqrt{2m\pi\sigma^2}
	    		\displaystyle{\int}_0^R 4\pi r^2 \rho(r) \dr
			-\displaystyle{\int}_0^R 4\pi r^2 \rho_0 \dfrac{\(2m\phi(r)\)^{5/2}}{5} \dr
		\) \notag \\
	    &= \frac{4\pi}{Mm^2\(2\pi m\sigma^2\)^{3/2}}
		    \(
			\dfrac{3}{2} \(m\sigma^2\)^2\sqrt{2m\pi\sigma^2}
	    		M
			-\displaystyle{\int}_0^R 4\pi r^2 \rho_0 \dfrac{\(2m\phi(r)\)^{5/2}}{5} \dr
		\) \notag \\
	    &= \frac{4\pi}{Mm^2\(2\pi m\sigma^2\)^{3/2}}
			\dfrac{3}{2} \(m\sigma^2\)^2\sqrt{2m\pi\sigma^2}
	    		M
			-\frac{4\pi}{Mm^2\(2\pi m\sigma^2\)^{3/2}}\displaystyle{\int}_0^R 4\pi r^2 \rho_0 \dfrac{\(2m\phi(r)\)^{5/2}}{5}\dr \notag \\
	    &= \frac{3\sigma^2}{m} - \frac{4\pi}{Mm^2\(2\pi m\sigma^2\)^{3/2}}\displaystyle{\int}_0^R 4\pi r^2 \rho_0
	    \dfrac{\(2m\phi(r)\)^{5/2}}{5}\dr \notag \\
	    &= \frac{3\sigma^2}{m} - \frac{32\pi^2\rho_0}{5 M m\(\pi \sigma^2\)^{3/2}}\displaystyle{\int}_0^R r^2 \(\phi(r)\)^{5/2}\dr
\end{align}
\begin{figure}[h!]
	\centering \includegraphics[width=1.00\textwidth]{theorie/graphe/Temperature_centre-moyenne.pdf}
	\caption{Températures maximums et moyennes en fonction de $W_0$\label{courbe::Moy}}
\end{figure}
La courbe~\ref{courbe::Moy} représente la variation des températures moyennes et centrales pour notre amas. Sur la courbe verte représentant la
température moyenne, un régime assez étrange commence à apparaître vers les $W_0$. Ce nouveau régime peut avoir 2 source possible~:
\begin{enumerate}
	\item une instabilité numérique apparaissant à cause de valeurs trop importante ou trop faible (~présence de \textsc{Not A Number} dans le
		fichier de donnée~),
	\item le $W_0$ à une valeur maximum au-delà de laquelle les équations ne sont plus valable.
\end{enumerate}
La première chose à faire dans ce cas est de tracer le maximum de profil de densité pour ces valeurs, de même il va falloir étudier la variation de
$\rho(r=0)$ en fonction de $W_0$.

Il est utile de rappeler que le but de ces travaux est de trouver les quantités à utiliser pour tracer un diagramme d'énergie similaire à celui de la
sphère isotherme en boîte, et ceci afin d'en étudier la stabilité.

Pour ceci nous avons d'avoir chercher à vérifier que le modèle est isotherme, ce qui est vérifié par les courbes~\ref{courbe::Temp}. Puis nous devions
savoir si la quantités à utiliser comme température est la température moyenne ou la température centrale. Ce dernier point est assez délicat
\FloatBarrier



	\section{Synthèse}
		Nous avons maintenant étudié trois modèles pouvant décrire l'état d'équilibre d'un amas d'étoile :
\begin{description}
	\item[la sphère isotherme (~SI~)] est un modèle thermodynamique étudié sur un support spatial infini. Ce modèle avait l'avantage de posséder une solution analytique,
		mais cette solution possède, comme nous l'avons montré, une masse infinie.
	\item[la sphère isotherme en boîte] est une variante du modèle de SI avec une extension finie. Grâce à ce modèle, nous avons appris que la stabilité pouvait être ramené à un seul paramètre important :
		le contraste de densité $\R_c$.
	\item[le modèle de \textsc{King}] est lui aussi une variante de SI, mais avec une coupure plus physique, effectuée après coups. Il nous a permis de montrer que l'évolution
		de la pente du halo, et donc du contraste de densité, pouvait être liée à un seul paramètre : $W_0$. Cette évolution peut-être décrite par une équation de la forme $\alpha = a e^{b W_0} + c$.
\end{description}



\chapter{Gravitation et instabilités}
Les instabilités dans le domaine de la gravitation ne sont pas très variées. L'instabilité gravitationnelle de base qui fait s'effondrer un système froid et trop dense sous l'effet de son propre poids reçoit l'appellation générale d'instabilité de Jeans. L'instabilité dite d'Antonov ou instabilité gravothermale, que nous avons étudiée dans le cas particuliers de la sphère isotherme en boite au chapitre \ref{SIB::Chapitre}, résulte du fait que les systèmes autogravitants ont une chaleur spécifique négative qui conduit à un réaction d'emballement dans certaines conditions. Enfin, la plus subtile est sans doute l'instabilité d'orbites radiales caractérisant ces systèmes sphériques dont l'anisotropie radiale dans l'espace des vitesses est démesurée.

Dans ce chapitre de référence pour notre propos, nous présenterons les fondamentaux des deux instabilités que nous n'avons pas encore étudié, puis nous mèneront une analyse de stabilité du produit final de l'instabilité de Jeans dans un cadre idéalisé.

En fin de chapitre, nous serons alors en mesure de présenter un scénario dynamique basé sur ces instabilités et visant à rendre compte des différentes observations ou simulations dont nous disposons actuellement.

Dans la partie suivante nous testerons numériquement certains points de ce scénario global à travers un certain nombre de cas test bien choisis.
 
 \section{L'instabilit\'{e} d'orbites radiales}

L'un des vieux probl\`{e}mes de la dynamique galactique est
l'\'{e}tude des syst\`{e}mes autogravitants form\'{e}s de particules
majoritairement en orbite radiale, tr\`{e}s allong\'{e}es et passant donc pr\`{e}s
du centre du syst\`{e}me. Il est tr\`{e}s vite apparu que de tels syst\`{e}mes
initialement sph\'{e}riques pourraient \^{e}tre instables et perdre leur
sym\'{e}trie dans ce qu'il est convenu d'appeler l'instabilit\'{e} d'orbite
radiale IOR \ ou ROI\ en anglais. Ce m\'{e}canisme pourrait m\^{e}me \^{e}tre
\`{a} l'origine de la forme de certains objets \`{a} l'\'{e}chelle galactique
attendu que la gravitation peine \`{a} former des structures poss\'{e}dant une
direction privil\'{e}gi\'{e}e.


\subsection{L'histoire de l'instabilité d'orbites radiales}

\subsubsection{Les pionniers}

Le premier r\'{e}sultat publi\'{e} est l'oeuvre de Vadim Antonov
\cite{antonov}. Il s'agit d'un r\'{e}sultat analytique concernant un syst\`{e}me
constitu\'{e} de $N$ particules.  L'\'{e}quation de Poisson correspondante est
\'{e}tudi\'{e}e en perturbation dans un cas limite correspondant aux orbites
radiales. Un syst\`{e}me diff\'{e}rentiel est alors produit pour un certain
"d\'{e}placement" des orbites. Le r\'{e}sultat est obtenu en exhibant une
fonction de Lyapunov stricte pour ce syst\`{e}me diff\'{e}rentiel. L'article
n'est vraiment pas tr\`{e}s clair ...

La m\^{e}me ann\'{e}e 1973, Michel H\'{e}non \cite{henon} publie l'un des tout
premier r\'{e}sultats num\'{e}rique dans ce domaine. Utilisant $N=1000$
coquilles concentriques distribu\'{e}es selon un mod\`{e}le
polytropique $f\left(  E\right)  \propto E^{n}$, il v\'{e}rifie la
stabilit\'{e} obtenue analytiquement par Antonov des les ann\'{e}es 60 pour ce type de systèmes. Puis,
en \'{e}tendant son mod\`{e}le au cas anisotrope (dans l'espace des vitesse)
des polytropes g\'{e}n\'{e}ralis\'{e}s  $f\left(  E\right)  \propto
E^{n}L^{2m}$, il montre num\'{e}riquement que le syst\`{e}me devient instable
lorsque $m\rightarrow-1$, qu'il identifie \`{a} la fonction de distribution
 $f\left(  E\right)  \propto E^{n}\delta\left(  L^{2}\right) $ et donc aux
syst\`{e}mes pr\'{e}sentant de plus en plus d'orbites radiales.
L'instabilit\'{e} est identifi\'{e}e dans l'espace des phases qui montre une
\'{e}volution du syst\`{e}me d\`{e}s que celui-ci est trop anisotrope dans
l'espace des vitesses. Le m\'{e}canisme de cette instabilit\'{e} n'est pas
pr\'{e}cis\'{e}, l'effet de cette instabilit\'{e} dans l'espace des positions
(la fameuse barre) n'est pas non plus abordé. Cet article ne fait pas
r\'{e}f\'{e}rence au travail d'Antonov \cite{antonov}. L'utilisation de coquilles
sph\'{e}riques ne permet pas de prendre en compte d'\'{e}ventuelles
interactions non radiales.

Un pav\'{e} dans la marre : en utilisant les m\'{e}thodes de Water-Bag, qui
consistent \`{a} d\'{e}composer la fonction de distribution sur une base de
fonctions continues par morceaux, l'\'{e}quipe fran\c{c}aise \cite{waterbag},
aur\'{e}ol\'{e}e par son succ\`{e}s dans le cas isotrope\footnote{Cinq ans
plus t\^{o}t, Doremus, Feix et Baumann avaient obtenu la stabilit\'{e} des
syst\`{e}mes isotropes dans l'espace des vitesses par la m\^{e}me
m\'{e}thode.}, publie un r\'{e}sultat affirmant la stabilit\'{e} de tous les
syst\`{e}mes autogravitants $f\left(  E,L^{2}\right)  $ contre des
perturbations non-sph\'{e}riques\footnote{La seule condition requise est le
fait que $\partial f/\partial E<0$ et $\partial f/\partial L^{2}<0$, ce qui
correspond au cas physique.}. Antonov ayant d\'{e}j\`{a} r\'{e}gl\'{e} le cas
des perturbations \`{a} sym\'{e}trie sph\'{e}rique pour ces m\^{e}me
syst\`{e}mes anisotropes, la boucle \'{e}tait boucl\'{e}e : tout syst\`{e}me
autogravitant sph\'{e}rique (isotrope ou non dans l'espace des vitesses) est
stable contre toutes les formes de perturbations. Le cas des syst\`{e}mes
constitu\'{e}s d'orbites de plus en plus radiales, inclus dans le champ
d'action du r\'{e}sultat de l'\'{e}quipe fran\c{c}aise est donc pr\'{e}dit
stable par cette analyse, en d\'{e}saccord patent avec le r\'{e}sultat
analytique d'Antonov \cite{antonov}, et les simulations de Michel H\'{e}non \cite{henon}.

\subsubsection{Des avancées conséquentes\label{ROIadvances}}

Sans faire r\'{e}f\'{e}rence au r\'{e}sultat de l'équipe française, Polyachenko et
G. Shukhman \cite{polyach} proposent une formulation matricielle du probl\`{e}me de
la stabilit\'{e} (bas\'{e}e sur une d\'{e}composition en s\'{e}rie de Fourier
des pertubations) qui leur permet de prouver qu'une fonction de distribution
de la forme $f\left(  E-\lambda L^{2}/r_{a}^{2}\right)$, mod\`{e}le que l'on appelera plus tard
Ossipkov-Merritt, d\'{e}crit un mod\`{e}le instable si $r_{a}^{2}$ est
suffisament petit. L'article n'est pas d'une grande clart\'{e} mais le
r\'{e}sultat est tout \`{a} fait contraire \`{a} celui propos\'{e} par
\cite{waterbag}, et va dans le sens d'Antonov et H\'{e}non. Cet article contient de
plus un crit\`{e}re de stabilit\'{e} qui sera repris dans le livre de
Friedmann et Polyachenko, et qui affirme qu'un syst\`{e}me sph\'{e}rique
d\'{e}clenche une instabilit\'{e} d'orbite radiale d\`{e}s que le rapport
$2T_{r}/T_{\perp}>1.75\pm0.25$ o\`{u} $T_{r}$ et $T_{\perp}$
repr\'{e}sentent respectivement les \'{e}nergies cin\'{e}tiques radiale et
perpendiculaire totales contenues dans le syst\`{e}me.

La premi\`{e}re \'{e}tude num\'{e}rique globale et à peu près réaliste du probl\`{e}me de
l'effondrement gravitationel est effectu\'{e}e par van Albada au d\'{e}but des
ann\'{e}es 80 \cite{albada}. Cette \'{e}tude consid\`{e}re des ensembles de
$N=5000$ particules de m\^{e}me masse dont les conditions initiales sont r\'{e}parties en 2 cat\'{e}gories : les sph\`{e}res homog\`{e}nes de taille 1
et des syst\`{e}mes compos\'{e}s de 20 sph\`{e}res homog\`{e}nes (clumps)
contenant chacune 250 particules. Ces clumps poss\`{e}dent initialement un
rayon \'{e}gal \`{a} 0,4 et leurs centres sont positionn\'{e}s
uniform\'{e}ment dans une boule de rayon 1$.$ Ces diff\'{e}rentes conditions
initiales sont abandonn\'{e}es \`{a} leur gravit\'{e} dans 3 conditions
initales de vitesse d\'{e}termin\'{e}es par le rapport du viriel initial :
$-2T/U=0.5,~0.2$ et $0.1$. Le r\'{e}sultat est clair : les sph\`{e}res
homog\`{e}nes ne souffrent pas (dans les cas d'effondrement
condid\'{e}r\'{e}s) d'instabilit\'{e} d'orbite radiale. Par contre, les
assemblages de grumeaux qui s'\'{e}ffondrent violemment ($-2T/U=0.1$%
)~produisent un \'{e}quilibre triaxial. M\^{e}me si ce papier ne parle pas de
l'instabilit\'{e} d'orbite radiale, il confirme que les profils (lumi\`{e}re,
densit\'{e}) obtenus num\'{e}riquement sont compatibles avec ceux qui sont
observ\'{e}s pour les galaxies (loi en $r^{1/4}$).

Dans un papier \`{a} la fois num\'{e}rique et analytique, Barnes, Goodmann et
Hut \cite{barneshut} abordent l'une des premi\`{e}res \'{e}tudes globale de
l'instabilit\'{e}. L'\'{e}tude num\'{e}rique consiste \`{a} refaire les
exp\'{e}riences avec des coquilles de Michel H\'{e}non en utilisant maintenant
des techniques \`{a} $N$ corps pour des valeurs de $N$ comprises entre
$10^{3}$ et $10^{4}$. Ils confirment les r\'{e}sultats de leur
pr\'{e}d\'{e}cesseur en faisant une \'{e}tude plus exhaustive dans l'espace
des param\`{e}tres des modèles utilisés par Hénon, Polyachenko et
Shukhman;  ils confirment d'ailleurs
le crit\`{e}re de stabilit\'{e} des russes et proposent une explication
analytique pas tr\`{e}s convaincante qui ferait passer l'instabilit\'{e} d'orbite radiale pour une sorte d'instabilit\'{e} de
Jeans\footnote{Instabilit\'{e} qui survient lorsque la pression cin\'{e}tique
ne suffit plus \`{a} compenser la tendance qu'\`{a} le syst\`{e}me \`{a}
s'effondrer sous l'effet de son propre poids.} : la pression stellaire dans la
direction tangentielle devient insuffisante pour compenser la tendance
naturelle qu'ont les orbites radiales \`{a} se condenser. Il est à noter que
l'\'{e}cole russe propose le m\^{e}me style d'interpr\'{e}tation pour
l'instabilit\'{e} dans le livre de Polyachenko et Friedmann (Vol2, p. 148).

Le travail suivant sur le sujet est publi\'{e} par Merritt et Aguilar
en 1985 \cite{merritt_aguilar}! Bien qu'ant\'{e}rieur en date de publication, il fait
r\'{e}f\'{e}rence aux travaux pr\'{e}c\'{e}dents de Barnes et al., en le
r\'{e}sumant magnifiquement en quelques lignes... Il se concentre sur les
aspects num\'{e}rique et sur l'opportunit\'{e} que repr\'{e}sente cette
instabilit\'{e} dans le contexte de la formation des galaxies, c'est la
premi\`{e}re fois que cette id\'{e}e surgit dans la litt\'{e}rature. Les auteurs
utilisent une premi\`{e}re famille de syst\`{e}mes dont le profil de
densit\'{e} est \og de type galactique\fg\, $:\rho\left(  r\right)  $ $\propto
(r/r_{o})^{-2}(1+r/r_{o})^{-2}$ (c'est le fameux mod\`{e}le de Jaffe qui
est compatible avec le profil de luminosit\'{e} en $r^{1/4}$ ). Il poss\`{e}de
en outre la bonne propri\'{e}t\'{e} d'\^{e}tre facilement transposable en un
mod\`{e}le anisotrope en suivant l'algorithme
d'Ossipkov-Merritt\footnote{Il s'agit de modèles sphériques dont la fonction de distribution étend un modèle isotrope à un système présentant une anisotropie radiale de plus en plus forte en s'éloignant du centre du système.}. Il s'agit d'exp\'{e}riences \`{a}
$N$ corps avec $N=5\cdot10^{3}$, l'\'{e}tat initial est un \'{e}quilibre
dont le degr\'{e} d'anisotropie radiale est contr\^{o}l\'{e}
par la valeur de $r_{o}$. Les conclusions sont les suivantes : la
transition stable/instable est tr\`{e}s rapide et elle se produit pour un
rapport $2T_{r}/T_{\perp}\approx2.5$, soit un peu plus que ce qui est
pr\'{e}vu par le crit\`{e}re russe. L'\'{e}tude de deux familles
compl\'{e}mentaires ( l'une avec une anisotropie ind\'{e}pendante du rayon et
l'autre avec une fonction de distribution d\'{e}croissante en $E$
\emph{et} en $L^{2}$) semble indiquer d'une part que la valeur de
$2T_{r}/T_{\perp}$ n'est pas un crit\`{e}re de stabilit\'{e} et d'autre part
que le r\'{e}sultat \cite{waterbag} de Gillon et al. est d\'{e}finitivement
infirm\'{e}. L'id\'{e}e de l'importance de cette instabilit\'{e} dans le
processus de formation des galaxies est avanc\'{e}e en conclusion de l'article, en reprenant ses termes \og elle ne doit pas \^{e}tre \'{e}cart\'{e}e\fg\,.

Une \'{e}tude purement analytique d'une \'{e}quipe anglaise (Palmer \&
Papaloizou, \cite{palmerpapa}) bas\'{e}e sur une analyse spectrale de la perturbation
d'un syst\`{e}me sph\'{e}rique anisotrope conclut \`{a} l'instabilit\'{e}.
C'est, depuis les r\'{e}sultats russes et hormis le r\'{e}sultat water-bag
de Gillon et al., la seule approche analytique frontale de ce probl\`{e}me et
toujours par des m\'{e}thodes consistant \`{a} d\'{e}composer les
perturbations sur des bases de fonctions orthogonales. Bien que le
r\'{e}sultat soit affirm\'{e}, il semble tr\`{e}s difficilement
v\'{e}rifiable... Deux autres aspects importants de cet article sont
la \og démonstration\fg\,de la non validit\'{e} du crit\`{e}re de stabilit\'{e} russe
déjà \'{e}corn\'{e} par Merritt et Aguilar et la pr\'{e}sentation d'un
nouveau m\'{e}canisme pour la croissance de l'instabilit\'{e} inspir\'{e} d'un
travail de Lynden-Bell \cite{lyndenbell}. Ce dernier point m\'{e}rite une attention
particuli\`{e}re. Le travail de Lynden-Bell \'{e}tudie l'influence d'une
perturbation axisym\'{e}trique dans le plan de sym\'{e}trie d'un potentiel de
galaxie spirale sur l'orbite d'une \'{e}toile. Il tend \`{a} montrer un effet
d'\'{e}longation des orbites qui ont alors tendance \`{a} s'aligner le long de
la perturbation. Cet effet serait \`{a} l'\oe uvre dans la formation des
barres des galaxies spirales. Pour la premi\`{e}re fois dans l'histoire de
l'instabilit\'{e} d'orbites radiales, Palmer et Papaloizou sugg\`{e}rent que
c'est le m\'{e}canisme de Lynden-Bell qui est \`{a} l'\oe uvre.

Une magnifique synth\`{e}se de tous ces r\'{e}sultats est effectu\'{e}e par
D. Merritt  \cite{merritt1987} en 1987. Les 2 m\'{e}canismes sont
d\'{e}taill\'{e}s et expliqu\'{e}s, l'instabilit\'{e} de Jean est
critiqu\'{e}e car elle nécessite un syst\`{e}me homog\`{e}ne ce qui
n'est pas le cas, D. Merritt met donc en avant le m\'{e}canisme de Lynden
Bell qu'il d\'{e}crit remarquablement.
 
Un article très intéressant de N. Katz \cite{katz} propose alors de nouvelles simulations cosmologiques montrant que le processus hiérarchique de formation des structures cosmologiques avec ses effondrements successifs tend à gommer les traces possibles d'une instabilité d'orbite radiale qui aurait pu se produire dans les phases initiales de cette formation. 

En cette
m\^{e}me ann\'{e}e 1991, un article de P. Saha \cite{saha} vient \'{e}tendre la
port\'{e}e des m\'{e}thodes spectrales de modes normaux aux syst\`{e}mes
d'extension infinie, ce qui n'\'{e}tait apparement pas le cas des \'{e}tudes
pr\'{e}c\'{e}dentes. Toujours en 1991, une \'{e}tude de D. Weinberg
\cite{weinberg}, reprend les m\'{e}thodes matricielles initi\'{e}es par
l'\'{e}cole russe de Polyachenko, retrouve des r\'{e}sultats et pr\'{e}sente
apparement dans sa section IV-c une analyse d\'{e}taill\'{e}e du m\'{e}canisme
de Lynden-Bell appliqu\'{e} \`{a} l'instabilit\'{e} d'orbites radiales. Cette analyse est en fait l'objet d'un article
complet et assez clair d'une \'{e}quipe argentine \cite{cincotta} qui \'{e}tudie la transformation
d'orbites de type boucle en type boite, ce qui est dans la veine du
m\'{e}canisme de Lynden-Bell et confirme l'intuition de Merritt \cite{merritt1987}.

\subsubsection{Un regain d'intérêt}


Mettant \`{a} profit certains de leurs r\'{e}sultats analytiques \cite{JPerez96}, 
Perez et al. \cite{perez_et_al} proposent et testent un crit\`{e}re de
stabilit\'{e} pour les syst\`{e}mes auto-gravitants construit sur la nature
des perturbations qu'il re\c{c}oit. Ce crit\`{e}re est valid\'{e} sur des
mod\`{e}le Ossipkov-Merrit appliqu\'{e}s \`{a} des polytropes. le nombre de
particules mis en jeu devient pour lapremi\`{e}re fois raisonnable
$N\approx10^{4}$ pour l'ensemble des simulations. Dans leurs résultats analytiques, ils expliquent la carrence des méthodes de water-bag dans le domaine des orbites radiales, ce qui pourrait expliquer le désomais défunt résultat \cite{waterbag}.

 Une \'{e}tude num\'{e}rique syst\'{e}matique de l'instabilit\'{e} d'orbite radiale, utilisant les machines d\'{e}di\'{e}es "GRAPE",  est produite par
une \'{e}quipe allemande \cite{theis} en 1999. Les simulations effectu\'{e}es sont
des effondrements de sph\`{e}res de Plummer de temp\'{e}rature initiale
variable. Le taux de croissance de ROI est grandement affect\'{e} par le
softening $\epsilon$ du potentiel et tr\`{e}s peu par des variations du nombre de
particules. Ces simulations ont mis en \'{e}vidence une \'{e}volution à
tr\`{e}s long terme (de l'ordre du temps de relaxation à 2 corps) du syst\`{e}me triaxial produit par ROI vers un
syst\`{e}me plus ou moins sph\'{e}rique, selon les auteurs cette transformation est due aux interactions à deux particules.

Un \'{e}tude syst\'{e}matique de l'effondrement gravitationnel (collapse) avec test des
param\`{e}tres num\'{e}riques ($N,\varepsilon$, ...) par Roy et Perez
\cite{roy}, permet, entre autres r\'{e}sultats, de mettre en \'{e}vidence un
aspect pressenti de l'instabilité d'orbites radiales. Son d\'{e}clenchement dans un collapse est
subordonn\'{e} \`{a} la pr\'{e}sence d'inhomog\'{e}n\'{e}it\'{e}s robustes.
Ce ne sont en effet que les effondrements au moins deux phase successives qui sont le si\`{e}ge
de ROI : l'effondrement d'une sphère homogène ne remplit pas cette condition. Ces r\'{e}sultats sont compl\'{e}t\'{e}s et raffin\'{e}s par Boily et
Athanassoula \cite{boily} qui montrent un l\'{e}ger effet du nombre de
particules sur l'\'{e}tat final de ROI.

Bien que le r\^{o}le de ROI dans la formation des structures ait été atténué par le travail de Katz \cite{katz} cit\'{e} plus haut, deux analyses
compl\'{e}mentaires par deux \'{e}quipes allemandes \cite{huss} et canadienne \cite{macmillan} observent le r\'{e}sultat de la formation de structure à moyenne \'{e}chelle par des exp\'{e}riences de collapse en se donnant la possibilit\'{e} de supprimer numériquement l'instabilit\'{e} d'orbite radiale (en retirant la composante radiale de la force de gravitation...). 
Ils remarquent que si l'on empèche ROI dans les phases primordiales, on modifie le profil de densité final de ces structures formées dans un contexte cosmologique : on obtient un profil à deux pentes au lieu de trois dans NFW par exemple). La forme allongée que prend le système à cause de l'instabilité d'orbite radiale serait donc progressivement gommée par le processus de formation hierarchique comme l'a remarqué Katz, mais le profil de densité final garderait subtilement sa trace. Les différents résultats observationnels ou numériques indiquent clairement la présence de cette marque...   

Un regain d'activit\'{e} dans le domaine se manifeste sous l'impulsion d'une
\'{e}quipe dirig\'{e}e par E. Barnes depuis 2005. Ce que l'on peut reternir des articles de cette équipe, notamment \cite{barnes2005} et \cite{ROI_Moderne} est la chose suivante : depuis longtemps on
sait que ROI produit un syst\`{e}me triaxial dans l'espace des positions, mais
il cr\'{e}e aussi une s\'{e}gr\'{e}gation spatiale dans l'espace des vitesse
(centre isotrope et halo radial). C'est cette s\'{e}gr\'{e}gation qui serait
\`{a} l'origine du profil universel que l'on observe dans les grandes
structures. Dans le cadre de ce renouveau, deux articles
\'{e}tudient ROI tout azimuts (\cite{barneslanzel}, \cite{trenti}) avec de nombreux d\'{e}tails 
bien expliqu\'{e}s, des comparaisons de codes,  mais rien de neuf.  L'équipe italienne \cite{trenti} s'\'{e}tonne de ne
pas pouvoir produire d'instabilité d'orbite radiale lors de l'effondrement d'une sphère homogène - dite de Hénon. Leur
explication à ce sujet est discutable  : l'effondrement serait trop rapide ou l'anisotropie ne serait pas
suffisante...). L'explication de ce ph\'{e}nom\`{e}ne avait pourtant \'{e}t\'{e} propos\'{e}e par Roy et Perez
\cite{roy} : un effondrement monolitique (sphère de Hénon) se produit en une seule phase.  Le germe dont à besoin ROI pour se développer n'est donc pas présent dans le cas générique de ce type d'effondrement. 


Enfin ROI aurait \'{e}t\'{e} observ\'{e}e dans un \'{e}quilibre triaxial
\cite{antonini} : elle se produirait lorsque ce dernier serait trop peupl\'{e} d'obites en forme de boites. Le syst\`{e}me deviendrait
alors plus prolate et toujours triaxial.



Comme le montre cette perspective historique, l'instabilité d'orbite radiale a connu des développements controversés. Elle demeure cependant fondamentale dans le processus de formation hiérarchique des structures gravitationnelles. La dernière avancée dans la compréhension de son mécanisme ainsi que la preuve de l'instabilité par des méthodes d'énergie se trouve dans le travail de Perez et Maréchal \cite{future}. Les principaux éléments de ce travail sont présentés dans la section suivante.  

%%%%%%%%%%%%%%%%%%%%%%%%%%%%%%%%%%%%%%%%%%%%%%%%%%%%%%%%%%%%%%%%
\subsection{La méthode symplectique}
%%%%%%%%%%%%%%%%%%%%%%%%%%%%%%%%%%%%%%%%%%%%%%%%%%%%%%%%%%%%%%%%

\subsubsection{Présentation de la méthode}

La méthode symplectique à été introduite dans le contexte de la stabilité des systèmes autogravitants par Bartholomew \cite{bartho}, elle est issue de la physique des plasmas. Elle fut popularisée par le regrété Henry Kandrup \cite{kandrupstability}.

Il s'agit de tirer parti de la structure hamiltonienne associée au système de Vlasov--Poisson. Dans ce contexte, on remarque tout d'abord que l'énergie totale contenue dans le système autogravitant décrit par la fonction de distribution $f$, i.e.
\[
	H \left[ f \right]  =
	\int \mathrm{d} {\Gamma}
		\frac{\mathbf{p}^{2}}{2m} f \left( {\Gamma},t \right)
	- \frac{1}{2} Gm^2 \int \mathrm{d} {\Gamma} \int\mathrm{d}{\Gamma}^{\prime}
		\frac{f \left({\Gamma},t \right) f\left( {\Gamma}^{\prime},t\right)}%
		{\left\vert \mathbf{q} - \mathbf{q}^{\prime} \right\vert }
\]

est telle que sa dérivée fonctionnelle est l'énergie moyenne d'une particule test :
\[
\frac{\delta H}{\delta f}=\lim_{\delta f \to 0}=\dfrac{H\left[ f +\delta f\right]-H \left[ f \right]}{\delta f}
= \frac{\mathbf{p}^{2}}{2m} + m \psi=E
\]
Si $K[f]$ est une fonctionnelle dérivable de la fonction de distribution nous pouvons donc écrire
\begin{equation}
	\frac{\mathrm{d} K[f]}{\mathrm{d} t}
	= \int \frac{\delta K}{\delta f}
		\frac{\partial f}{\partial t} \mathrm{d} \Gamma
	= \int \frac{\delta K}{\delta f} \left\{ E, f \right\} \mathrm{d} \Gamma
	\label{derivk}
\end{equation}
où nous avons utilisé la forme canonique de l'équation de Vlasov $\dot f=\left\{ E, f \right\}$.
On peut alors introduire les crochets popularisés par Phil Morrisson \cite{morrison} :
pour deux fonctionnelles $A$ et $B$ de $f$, on a 
\[
	\left[ A, B \right](f) :=
	\int f \left\{
		\frac{\delta A}{\delta f}, \frac{\delta B}{\delta f}
	\right\} \mathrm{d} \Gamma
\]
Ainsi, la relation (\ref{derivk}) s'écrit
\begin{equation}
	\frac{\mathrm{d} K[f]}{\mathrm{d} t}
	= - \int f \left\{
		\frac{\delta H}{\delta f}, \frac{\delta K}{\delta f}
	\right\} \mathrm{d} \Gamma
	= \left[K, H \right](f)
\end{equation}

Comme cela est lumineusement expliqué par Henry Kandrup  \cite{kandrupstability},
toute perturbation linéaire $f^{(1)}$ pouvant physiquement être reçue par l'état décrit par la fonction de distribution $f_0$ s'écrit
\[
	f^{(1)}\left(  {\Gamma},t \right) = -\left\{ g,f_{0}\right\}
\]
la fonction $g$ est appelée \emph{générateur} de la perturbation. Introduisons la quantité
\[
	G[f] := \int f g \mathrm{d} \Gamma
\]
A partir de cette perturbation, il est possible de construire la variation d'énergie correspondante. Au  premier ordre il vient
\begin{equation*}
	H^{(1)} [f_0] = \left[G, H \right](f_0)
	= - \int g \{ f_0, E \} \mathrm{d} \Gamma
	= 0
\end{equation*}
Si l'état $f_0$ est un équilibre, $\{ f_0, E \} = 0$ et la variation d'énergie est donc nulle à l'ordre 1. En d'autre terme La fonctionnelle $H[f]$ présente un extremum en $f=f_0$, c'est bien la définition d'un état d'équilibre.

La variation de l'énergie au second ordre  
$
	H^{(2)} [f_{0}] = \left[G, [G,H] \right](f_0)
$
ne résiste pas à quelques lignes de calculs, il vient
\begin{eqnarray}
	H^{(2)}[f_{0}]
	& = & - \int \left.\frac{\delta [G,H]}{\delta f}\right|_{f=f_{0}} \{g,f_{0}\} \mathrm{d} \Gamma
	\nonumber \\
	& = & - \int \left(
		\{g,E\} + \int \frac{Gm^2}{|\mathbf{q}-\mathbf{q'}|}
		\{g',f'_{0}\} \mathrm{d} \Gamma'
	\right) \{g,f_{0}\} \mathrm{d} \Gamma
	\nonumber \\
	& = & - \int \{g,E\} \{g,f_{0}\} \mathrm{d} \Gamma
	- G m^2 \int\!\!\!\int \frac{\{g,f_{0}\}\{g',f'_{0}\}}{|\mathbf{q} - \mathbf{q'}|}
	\mathrm{d} \Gamma \mathrm{d} \Gamma'
\end{eqnarray}

L'étude du signe de cette quantité s'est révélé un outil efficace d'investigation de la stabilité des systèmes autogravitants.

%%%%%%%%%%%%%%%%%%%%%%%%%%%%%%%%%%%%%%%

\subsubsection{Des critères de stabilité}

Dans le cas classique d'une particule soumise à l'influence de forces conservatives, la stabilité locale d'un état d'équilibre est directement reliée au signe de la variation d'énergie à l'ordre 2 au voisinage de cet état : une variation positive laisse l'équilbre stable alors qu'une variation négative de l'énergie conduit à une instabilité.

L'extension d'un tel résultat à des situations plus compliquées (forces non conservatives, systèmes de particules, etc...) n'est cependant pas triviale.

Dans le cas qui nous intéresse de la physique statistique des systèmes non collisionnels évoluant dans un champ moyen la situation n'est cependant pas  désespérée.

Si la variation $H^{(2)}$ est positive pour tous les générateurs $g$,
Bartholomew \cite{bartho} montre que le système est stable. Cette méthode a été largement mise en application dans tous les cas favorables par J. Perez et J.-J. Aly \cite{perezaly}. 
Par contre, s'il existe des générateurs $g$ conduisant à des variations $H^{(2)}$ négatives, il n'existe pas de résultat général assurant l'instabilité du système; du moins sans hypothèse complémentaire...


Dans ce contexte, les résultats  \cite{blochmarsden} mais surtout  \cite{krechet} de l'équipe de J. Marsden, permet l'étude de cas assez généraux.

Ces résultats affirment que l'existence de modes d'énergie négative ($g$ tels que $H^{(2)}<0$ dans notre contexte) rendent instables des systèmes hamiltoniens dès lors qu'ils ont la possibilité de dissiper leur énergie. Ces résultats sont d'ailleurs connus sous l'appellation \og d'instabilités dissipatives \fg.


%%%%%%%%%%%%%%%%%%%%%%%%%%%%%%%%%%%%%%%

\subsubsection{L'application à l'instabilité d'orbites radiales}

Il n'est pas question ici de reprendre le détail des calculs présentés par l'article de J. Perez et L. Maréchal \cite{future}, nous présenterons simplement l'idée du résultat.

Un état d'équilibre composé exclusivement de particules en orbite radiale est décrit par une fonction de distribution $f_0(E,L^2) = \varphi(E) \delta(L^2)$ : si $\varphi$ est une fonction acceptable mais quelconque, la distribution de Dirac, qui sélectionne les valeurs $L^2=0$,  assure en effet le caractère purement radial de toutes les orbites.
L'idée mise en \oe uvre dans la preuve est alors claire. Après avoir rappelé le caractère hamiltonien de la dynamique de Vlasov-Poisson 
gravitationnelle, il suffit de calculer la variation de 
\begin{equation}
	H^{(2)} =
	\underbrace{- \int \{g,E\} \{g,f\} \mathrm{d} \Gamma}_{(A)}
	\underbrace{- G m^2 %
		\int\!\!\!\int \frac{\{g,f\}\{g',f'\}}{|\mathbf{q} - \mathbf{q'}|}
		\mathrm{d} \Gamma \mathrm{d} \Gamma'}_{(B)}
		\label{H2AB}
\end{equation}
 pour un équilibre de la forme $f_{0,a}(E,L^2) = \varphi(E) \delta_a (L^2)$ avec une fonction $\delta_a (L^2)$ admettant $\delta(L^2)$ lorsque $a$ tend vers 0. Le terme $(A)$ dans la variation \ref{H2AB} correspond à la variation d'énergie cinétique alors que $(B)$ rend compte de la variation d'énergie potentielle. Cette dernière est toujours négative compte-tenu des propriétés du laplacien.
Il est alors un peu laborieux de montrer qu'il existe toute une classe de perturbations, non radiales, associées à des variations d'énergie négative dès que $a$ est suffisamment petit et donc que le système est suffisamment radial.

Dans ce contexte la présence de dissipation est donc irrémédiablement associée à une instabilité du système. Il est clair qu'il est impossible de garantir la totale conservation de l'énergie pour un système physique réel ou numérique. L'instabilité d'orbite radiale peut donc se développer dans les simulations ou dans la réalité! Le mécanisme est celui par initié par Lynden-Bell (voir section {ROIadvances}) : une perturbation non radiale étire ou compresse certaines régions du système initialement sphérique, le couplage avec les orbites proches conduit à une avalanche dans la direction sélectionnée et une barre se forme. Sans la dissipation, un tel couplage est impossible, l'instabilité d'orbite radiale est bien d'origine dissipative. 

On remarque dans ce processus qu'un état d'équilibre radial est nécessaire afin que puisse se développer cette instabilité. 
C'est l'accumulation d'orbites radiales couplées par une légère dissipation qui permet le déclenchement de l'avalanche.   

 

%%%%%%%%%%%%%%%%%%%%%%%%%%%%%%%%%%%%%%%%%%%%%%%%%%%%%%%%%%%


\section{L'instabilit\'{e} de Jeans}

\subsection{Généralités}
L'instabilité de Jeans est sans aucun doute le véritable moteur de la formation des structures autogravitantes.
C'est elle qui fait s'effondrer les nuages du gaz interstellaire pour former les étoiles mais c'est elle aussi qui déclenche et alimente la formation des grandes structures de l'Univers; elle occupe donc quasiment tout le spectre de taille des objets de l'astrophysique. 
D'un point de vue théorique, elle apparait au tout début du \textsc{xx}$^e$ siècle avec Sir James Jeans dans l'article fondateur \cite{jeans02} publié en 1902; elle fut mise en action dès que l'on compris le rôle de la gravitation dans la formation et l'évolution  des objets astrophysique dans les années 20 à 30 de ce siècle. L'avènement de la physique des plasmas lui donnera un cadre théorique solide dans les années 40 ou elle devient le cas particulier gravitationnel de la théorie de l'amortissement de Landau. Ce modèle linéaire 
sera vite controversé  : anarque de Jeans (voir la section 5.2.2 de \cite{2008gady.book.....B}  ou l'article de M. Kiessling \cite{kiessling}), comportement non linéaire complexe, ... L'histoire semble désormais terminée sur le plan théorique avec la démonstration de sa version non linéaire par Cédric Villani et Clément Mouhot en 2008. En physique on pratique généralement trois approches gigognes pour se convaincre de cette instabilité : 
\begin{itemize}
\item L'approche de plus haut niveau (dite cinétique) consiste à partir du système de Vlasov-Poisson et de le linéariser au voisinage d'une distribution d'équilibre; c'est à ce moment qu'apparait l'arnaque de Jeans car dans cette approche on évacue \og à la main \fg et sans autre forme de procès, un terme génant... On fait alors l'hypothèse d'une distribution d'équilibre isotherme dans l'espace des vitesses et uniforme dans l'espace des positions. Au prix de calculs sérieux on obtient une relation de dispersion faisant apparaître une longueur caractéristique pour l'instabilité.

\item L'approche fluide consiste à partir des équations de la mécanique des fluides  -- conservation de la masse et de la quantité de mouvement (Euler) -- fermées par l'équation de Poisson. Ces deux équations fluides sont directement issues de l'équation de Vlasov par intégration sur les impulsions. On linéarise alors au voisinage d'un équilibre  homogène et stationnaire, on fait apparaitre une vitesse du son constante dans le fluide et l'on obtient une relation de dispersion similaire à celle obtenue dans le cas cinétique. Nous détaillerons cette méthode dans la section suivante.

\item L'approche basique consiste simplement à écrire le théorème du viriel scalaire. Une longueur caractéristique apparait alors; elle correspond  à une figure d'équilibre, un raisonnement physique permet de l'identifier à une longueur caractéristique de stabilité.
Cette approche est contenue dans les deux précédentes car le théorème du viriel est d'origine cinétique; la version scalaire utilisée dans ce contexte est simplement la trace de la version complète obtenue en calculant le moment du viriel $\vartheta=\vec p \cdot \vec r$ dans le cadre de la théorie cinétique.
\end{itemize}

Les trois approches fournissent le même résultat et répondent chacun à un niveau d'exigence mathématique de plus en plus élaboré, elles sont équivalentes. On pourra trouver l'ensemble des calculs avec tous les détails dans le cours de J. Perez \cite{CoursJP}. Nous proposons simplement ici de reprendre rapidement les grandes étapes de l'approche fluide.

\subsection{L'instabilité fluide}

Outre le champ de densité $\rho(\vec r, t)$ et le potentiel gravitationnel $\psi(\vec r, t)$ dont nous avons déjà parlé, la description fluide d'un syst\`{e}me autogravitant est assurée par les grandeurs moyennes suivantes :
\begin{itemize}
\item un champ de vitesse moyenne au point $\vec{r}
$ et \`{a} l'instant $t$
\[
\overline{{\vec{v}}\left(  {\vec{r}},t\right)  }
=
\frac{m}{\rho}\int \frac{\vec{p}}{m} f\left(  \vec{r},\vec{p},t\right)  d\vec{p}
\]
\item un \og tenseur\fg\,de pression au point $\vec{r}$ et \`{a} l'instant $t$ dont les composantes cartésiennes s'écrivent
\[
\mathbb{P}_{ij}
=
\frac{1}{m^{2}}\left(  \int p_{i}p_{j}\,f\,d\vec{p} - \int p_{i}\,f\,d\vec{p}\int p_{j}\,f\,d\vec{p}\right)
\]
\end{itemize}
Pour simplifier les notations et s'il n'y a pas d'ambigüité, nous omettrons la barre de moyenne sur le champ de vitesse dans le fluide. La conservation de la masse et de l'impulsion conduisent alors aux deux équations fondamentales suivantes
\begin{subequations}\label{eq:sys-fluide}
  \begin{eqnarray}
  \rho\left[  \dfrac{\partial\vec{v}}{\partial t}+\left(  \vec{v}\cdot\vec{\nabla}\right)  \vec{v}\right]  &=&-\overrightarrow{\nabla\cdot\mathbb{P}}-\rho
\vec{\nabla}\psi
\label{eulerjeans} \\
\dfrac{\partial\rho}{\partial t}+\vec{v}\cdot\vec{\nabla}\rho&=&-\ \rho
\ \vec{\nabla}\cdot\vec{v}
 \label{continuitejeans} 
\end{eqnarray}
\end{subequations} 

Si l'on fait appara\^{\i}tre une vitesse du son $c_{s}$ supposée constante partout dans le
fluide, les champs de pression et de densit\'{e} sont alors reli\'{e}s par
l'\'{e}quation
\[
\overrightarrow{\nabla\cdot\mathbb{P}}=c_{s}^{2}\, \vec{\nabla}\rho
\]
On peut alors \'{e}tudier la stabilit\'{e} d'un \'{e}quilibre d\'{e}crit par
des champs constants et stationnaires $\vec{v}_{eq}=$
$\vec{v}_{o}$, $\rho_{eq}=\rho_{o}$ et $\psi_{eq}=\psi_{o}$. Pour cela, on
d\'{e}veloppe les divers champs du fluide au voisinage de l'\'{e}quilibre%
\[
\left\{
\begin{array}
[c]{c}%
\vec{v}\left(  \vec{r},t\right)  =\vec{v}_{o}+\varepsilon
\vec{v}_{1}\left(  \vec{r},t\right)  \\
\rho\left(  \vec{r},t\right)  =\rho_{o}+\varepsilon\rho_{1}\left(
\vec{r},t\right)  \\
\psi\left(  \vec{r},t\right)  =\psi_{o}+\varepsilon\psi_{1}\left(
\vec{r},t\right)
\end{array}
\right.  \ \ \ \text{avec }\left\vert \varepsilon\right\vert \ll1
\]
On injecte ces relations dans les deux \'{e}quations fondamentales et l'on ne
conserve que les termes d'ordre $\varepsilon$, en n\'{e}gligeant les suivants,
il vient%
\[
\left\{
\begin{array}
[c]{l}%
\rho_{o}\left[  \dfrac{\partial\vec{v}_{1}}{\partial t}+\left(
\vec{v}_{o}\cdot\vec{\nabla}\right)  \vec{v}_{1}\right]  =-c_{s}%
^{2}\ \vec{\nabla}\rho_{1}-\rho_{o}\vec{\nabla}\psi_{1}\\
\\
\dfrac{\partial\rho_{1}}{\partial t}+\vec{v}_{o}\cdot\vec{\nabla}\rho
_{1}=-\ \rho_{o}\ \vec{\nabla}\cdot\vec{v}_{1}%
\end{array}
\right.
\]
Nous n'avons toujours pas pr\'{e}cis\'{e} le r\'{e}f\'{e}rentiel galil\'{e}en
d'usage, on le choisit donc tel que $\vec{v}_{o}=0$, ainsi il ne
reste plus que%
\[
\left\{
\begin{array}
[c]{l}%
\dfrac{\partial\vec{v}_{1}}{\partial t}=-\dfrac{c_{s}^{2}}{\rho_{o}%
}\ \vec{\nabla}\rho_{1}-\vec{\nabla}\psi_{1}\\
\\
\dfrac{1}{\rho_{o}}\dfrac{\partial\rho_{1}}{\partial t}=-\ \ \vec{\nabla
}\cdot\vec{v}_{1}%
\end{array}
\right.
\]
On d\'{e}rive alors la seconde \'{e}quation par rapport au temps et l'on prend
la divergence de la premi\`{e}re : le terme $\ \vec{\nabla}\cdot\dfrac
{\partial\vec{v}_{1}}{\partial t}=\dfrac{\partial}{\partial t}\left(
\vec{\nabla}\cdot\vec{v}_{1}\right)  $ est alors pr\'{e}sent dans les deux
\'{e}quations, on peut l'\'{e}liminer.\ Il vient%
\begin{equation}
\dfrac{\partial^{2}\rho_{1}}{\partial t^{2}}=c_{s}^{2}\Delta\rho_{1}+\rho
_{o}\Delta\psi_{1}\label{inter1jeans}%
\end{equation}
L'\'{e}quation de Poisson pour le potentiel gravitationnel s'\'{e}crit%
\[
\Delta\psi=4\pi G\rho
\]
au voisinage de l'\'{e}quilibre uniforme et stationnaire on a donc%
\[
\Delta\psi_{1}=4\pi G\rho_{1}%
\]
l'\'{e}quation $\left(  \ref{inter1jeans}\right)  $ devient donc%
\[
\dfrac{\partial^{2}\rho_{1}}{\partial t^{2}}=c_{s}^{2}\Delta\rho_{1}+4\pi
G\rho_{o}\rho_{1}%
\]
Il ne reste plus qu'\`{a} d\'{e}composer $\rho_{1}$ sur une base de modes
normaux%
\[
\rho_{1}\left(  \vec{r},t\right)  =\sum_{\alpha}\rho_{1,\alpha}\exp\left[
i\left(  \vec{k}_{\alpha}\cdot\vec{r}+\omega_{\alpha}t\right)  \right]
\]
et l'on obtient une relation de dispersion pour chaque mode $\alpha$%
\[
\omega_{\alpha}^{2}=c_{s}^{2}\left(  k_{\alpha}^{2}-k_{j}^{2}\right)
\ \ \ \ \ \text{avec }k_{j}^{2}:=\frac{4\pi G\rho_{o}}{c_{s}^{2}}\ \text{\ et
\ }k_{\alpha}=\left\vert \vec{k}_{\alpha}\right\vert
\]
On remarque imm\'{e}diatement que :

\begin{itemize}
\item si $k_{\alpha}>k_{j}$ alors $\omega_{\alpha}^{2}>0$, la pulsation du
mode$\ \alpha$ est r\'{e}elle il s'agit donc d'un mode oscillant.

\item si $k_{\alpha}<k_{j}$ alors $\omega_{\alpha}^{2}<0$, la pulsation du
mode$\ \alpha$ est imaginaire pure : $\omega_{\alpha}=\pm i\Omega_{\alpha}$
avec\ $\Omega_{\alpha}\in\mathbb{R}$.\ Le mode $\alpha$ est donc instable (il
poss\`{e}de une partie exponentiellement croissante).
\end{itemize}

Si l'on introduit la longueur d'onde $\lambda_{\alpha}\ $associ\'{e}e au
nombre d'onde $k_{\alpha}$ par la relation%
\[
k_{\alpha}=\frac{2\pi}{\lambda_{\alpha}}
\]
la condition de stabilit\'{e} du mode $\alpha$ s'\'{e}crit
\[
\frac{2\pi}{\lambda_{\alpha}}>k_{j}\ \ \Rightarrow\ \ \lambda_{\alpha}%
<\lambda_{j}\ \ \ \text{avec\ \ }\lambda_{j}:=\frac{2\pi}{k_{j}}=\sqrt{\pi
}\frac{\ c_{s}}{\sqrt{G\rho_{o}}}
\]
Si un mode est associ\'{e} \`{a} une longueur d'onde plus grande que la
longueur d'onde de Jeans $\lambda_{j}$ du syst\`{e}me, il le
d\'{e}stabilisera.

Le critère ci-dessus fait intervenir une longueur caractéristique on peut aussi faire apparaître une masse correspondante. La masse de Jeans $M_j$ correspond à la masse d'une boule homogène de densité $\rho_o$ dont le rayon est la longueur de Jeans
 \begin{equation}
 M_j = \frac{4\pi}{3}\rho_{o} \lambda_{j}^3 =  \frac{4\pi^{5/2}}{3} \frac{c_{s}^3}{G^{3/2}\sqrt{\rho_{o}}}
 \end{equation}
Une sphère homogène plus lourde que sa masse de Jeans est donc instable.

Pour une sphère isotherme l'équation de température $T$  composée de $N$ particules de masse $m$, l'équation d'état s'écrit $P(r)=\dfrac{k_B T}{m}\rho (r)$, la vitesse du son $c_s$ dans la sphère est donc égale à la vitesse quadratique moyenne de ses particules (écart-type de la gaussienne définissant la distribution des vitesses), elle est constante pour la sphère isotherme et s'écrit
$$
c_s^2= \dfrac{ k_B T}{m}
$$
En supposant la sphère isotherme homogène (ce qui est tout de même un peu hardi...)  et de rayon égal à sa longueur de Jeans (pour des raisons d'équilibre), on peut définir la température de Jeans 
\begin{equation}
T_j= \frac{Gm\rho_{o}\lambda_{j}^2}{\pi k_B}=\frac{3Gm^2N}{4\pi^2 k_B\lambda_{j}}
\end{equation}
Une sphère isotherme de rayon égal à sa longueur de Jeans et plus froide que sa température de Jeans est donc instable.

\subsection{Le produit final}

Que se passe-t-il lorsqu'un système autogravitant homogène dépasse sa longueur de Jeans ?
Comment cela est-il possible ?

Si un système autogravitant homogène ne s'effondre pas sous son autogravitation c'est que l'énergie cinétique qu'il contient, c'est-à-dire la température qui le caractérise lui permet de lutter contre cette autogravitation. Nous décrivons ici un état d'équilibre, cette température lui permet donc exactement de contrebalancer son autogravitation, afin que le système soit \og au viriel \fg, s'il était plus chaud le système s'évaporerait. C'est ce qui arrive aux amas ouvert...

Dans son état homogène il peut devenir plus gros, en absorbant sous l'effet de la gravitation un autre système autogravitant homogène... S'il est composé d'étoiles, l'une d'entre elle peut exploser ce qui peut avoir tendance à augmenter la densité moyenne du système tout entier, il peut alors devenir plus grand que sa longueur de Jeans...

Numériquement la situation est plus simple, pour construire une configuration instable au sens de Jeans, il suffit de
fabriquer une boule homogène de $N$ particules massives et de calculer l'énergie potentielle gravitationnelle totale de
cette boule, $W=-\frac{3}{5}\frac{GM^2}{R}$ si sa masse est $M$ et son rayon $R$. On attribue alors une vitesse à chaque
particule de manière à ce que l'énergie cinétique totale $T=\sum_{i=1}^N\frac{1}{2}m_i v_i^2$, soit telle que le rapport
du viriel $\eta=\left|\frac{2T}{W}\right|$ soit plus petit que 1.  Ce jeu de conditions initiales sera ainsi associé à
un système plus gros que sa longueur de Jeans : il s'effondrera. De nombreuses expériences numériques (voir par exemple
\cite{roy} ou \cite{Joyceetal}) ont été menées pour étudier de telles situations le résultat est assez clair. Dès que le
nombre de particules est suffisant pour éviter les problèmes d'interaction a deux corps trop rapides, le système
s'effondre pour former une structure c\oe ur halo à l'équilibre isotrope dans l'espace des vitesses. Le système final
est sphérique indépendamment de sa forme initiale, de la dispersion de vitesse et même du rapport du viriel initial
$\eta\in[0.1, 0.9]$. L'état d'équilibre est atteint en quelques temps dynamiques, le halo homogène contient
approximativement la moitié de la masse totale, le halo possède une densité autosimilaire $\rho(r)\propto r^{-4}$. Ce
type de système est assez bien ajustable par un modèle de King de paramètre $W_0\simeq 5$.

L'introduction de grumeaux ou d'inhomogénéités au sein du système avant son effondrement peut modifier le résultat. Pour ce type de configurations initiales, le processus d'effondrement conduit à des systèmes ne présentant pratiquement plus de c\oe ur. Lorsque que l'on peut ajuster leur densité  par un profil de King le paramètres $W_0$ est alors beaucoup plus grand, typiquement  $W_0>15$. Ces configuration initialement inhomogènes  possèdent aussi une caractéristique qui les distinguent des effondrements homogènes : les configurations initalement très froides ($\eta<0.15$), c'est-à-dire les effondrements les plus violents, conduisent à des états d'équilibre qui ne sont plus sphériques car ils ont développé l'instabilité d'orbite radiale.  


Dans la prochaine section nous allons tenter de proposer, dans un contexte simplifié, une analyse de stabilité permettant de comprendre la diversité des états post-collapse.


\section{Stabilité de l'état post-collapse\label{Sec::ToyModel}}

Nous avons vu dans la section précédente que l'état final générique de l'effondrement d'un système homogène est une structure cœur-halo ce dernier
étant caractérisé par une densité de la forme $\rho(r)\propto r^{-4}$. Tant qu'elle est isolée une telle structure est stable et n'évolue que sur des
échelles de temps beaucoup plus grandes que le temps dynamique, sous l'effet des collisions ou de divers harcèlements (voir l'évolution des amas
globulaires par exemple).

Une situation plus complexe se présente lors de l'effondrement d'un système présentant des inhomogénéités. Prenons pour simplifier le cas idéalisé
d'un système inhomogène caractérisé par deux structures homogènes emboitées de tailles différentes. L'effondrement du plus petit de ces sous-système
va se produire dans le bain thermique assuré par la présence du plus gros. Si cet effondrement produit une structure cœur-halo, sa stabilité doit
ainsi être envisagée dans l'ensemble canonique. Nous avons vu que ce type d'analyse dans le contexte de la sphère isotherme en boite revenait à
chercher la première tangente horizontale sur la courbe calorique. Nous avons vu par ailleurs que les structures cœur-halo de pente quelconques
sont pratiquement isothermes. Nous proposons une analyse de la courbe calorique d'une structure cœur-halo idéalisée, supposée
isotherme et entourée d'un bain thermique. L'intérêt de l'idéalisation est qu'il permet de mener à la main la totalité des calculs. Nous vérifierons
que dans le cas de l'idéalisation de la sphère isotherme (halo de pente -2), l'analyse de stabilité donne les mêmes résultats que ceux obtenus au
moment de l'étude de la sphère isotherme en boite.

\subsection{Modèle et calculs}
Le système que nous étudions est sphérique de rayon fini $R$, il possède un cœur de rayon $r_0$, le paramètre d'étude sera la quantité sans
dimension:
\begin{align*}
	x=\dfrac{R}{r_0}\,\geq\,1
\end{align*}
La densité idéalisée que nous utilisons est la suivante:
\begin{align}
	\rho(r) &= \begin{cases}
			\rho_0	&	\text{si $r<r_0$}\\
			\rho_0 \(\dfrac{r_0}{rx}\)^4	&	\text{si $R>r > r_0$}\\
			0 & \text{si $r>R$}
	\end{cases}
\end{align}
Nous pouvons avantageusement remplacer la densité centrale par la masse totale du système en utilisant le fait que $\int_0^R \rho(r) = M$, il vient:
\begin{align}
	\rho_0 &= \dfrac{3}{4}\dfrac{M x^4}{R^3 \(4x - 3\) \pi}
\end{align}
La masse contenue dans une sphère de rayon $r$ s'écrit:
\begin{align}
	\mu\(r\) = \begin{cases}
		\dfrac{Mr^3x^4}{R^3\(4x-3\)}	&	\text{si $r < r_0$}\\
		\dfrac{\(3R-4rx\)M}{\(3-4x\)r}	&	\text{si $r>r_0$}
	\end{cases}
\end{align}
L'énergie potentielle totale contenue dans le système est le résultat du calcul d'une intégrale assez simple pour notre modèle:
\begin{align}
	W &= -4\pi G\int_0^R r\rho(r)\mu(r)\mathrm{dr} \\
			 &= -\frac{3 G M^2 \left(5-10 x+6 x^3\right)}{5R (3-4 x)^2}
\end{align}

\subsection{Énergie cinétique}

	La sphère que nous étudions est supposée  isotherme de température $T=(\beta k_B )^{-1}$\footnote{C'est une hypothèse simplificatrice mais
	raisonnable compte-tenu de l'analyse faite sur le modèle de King par exemple}. L'énergie cinétique s'écrit alors:
	\begin{align}
		K &= \dfrac{3M}{2m\beta}
	\end{align}
	L'hypothèse d'isothermie permet de calculer la pression grâce à l'équation d'état $P(r)=\dfrac{\rho(r)}{m\beta}$. Sur le bord de la sphère
	nous avons donc:
	\begin{align}
		P\(R\) = \frac{3 M}{4 m \pi  R^3 (-3+4 x) \beta }
	\end{align}
	Le théorème du Viriel  pour une sphère en boite s'écrit:
	\begin{align}
		4\pi R^3 P(R) &= 2 K + W
	\intertext{grâce aux différentes expression obtenues nous pouvons donc trouver une relation entre $\beta$, $x$ et les caractéristiques de la
	sphère. Il vient:}
		\beta(x) &= \frac{20R \left(3 -7  x+4  x^2\right)}{G m M \left(5-10 x+6 x^3\right)}
	\intertext{Nous pouvons également exprimer l'énergie totale $H=K+W$ contenue dans la sphère en fonction de $x$ et des caractéristiques de
	celle-ci. Finalement, en utilisant les mêmes paramètres sans dimensions ceux utilisés pour la \textsc{sib}, nous obtenons:}
		\mu(x) &= \frac{GMm\beta}{R}=\frac{20 (-1+x) (-3+4 x)}{5-10 x+6 x^3}\\
		\lambda(x) &= -\frac{HR}{GM^2}=\frac{3 (-5+4 x) \left(5-10 x+6 x^3\right)}{40 (3-4 x)^2 (-1+x)}
	\end{align}

	Pour chaque valeur du paramètre $x$ nous avons donc un couple $(\lambda,\mu)$. Le paramètre $x$ varie entre la valeur minimale $x_{\min}=1$
	pour un système de densité constante, et $x_{\max}\to +\infty$ pour une sphère de plus en plus grande avec un cœur occupant
	proportionnellement de moins en moins de place lorsque $x$ grandit. Le contraste de densité est facilement calculable pour ce modèle:
	$\mathcal{R}_4= x^4$. La courbe $(\lambda(x),\mu(x))$ pour $x\in[1,+\infty[$ est donc la courbe calorique de notre sphère. Elle est tracée sur
	la figure \ref{fig::DET}.

	\begin{figure}
		\centering \includegraphics[width=1\linewidth]{theorie/graphe/ToyModel-alpha4_spirale.pdf}
		\caption{Courbe calorique du modèle isotherme idéalisé possédant un halo de pente $-4$ \label{fig::DET}}
	\end{figure}

	Cette courbe présente bel et bien une tangente horizontale, en $x^*_4= \frac{1}{4} \left(5+\sqrt{5}\right) \approx 1.81$, qui correspond à une
	instabilité dans l'ensemble canonique dans le cadre de notre calcul (bain thermique). Le contraste de densité correspondant est
	$\mathcal{R}_4\approx 10.71$

\subsection{Conclusion du modèle}

	Une conclusion raisonnable de cette étude est qu'un système cœur-halo dont le contraste de densité est plus important que $\mathcal{R}_4$ est
	instable s'il est entouré d'un bain thermique de même température que la sphère. Cette instabilité conduit à l'effondrement du cœur du
	système.

	Cette instabilité pourrait être à l'origine de l'absence de cœur dans le résultat de l'effondrement d'un système inhomogène.

	La courbe calorique peut être calculée dans les mêmes conditions pour des systèmes cœur-halo de pentes différentes. Nous les avons rassemblé
	sur la figure \ref{ToyModel::AllAlpha}.
	\begin{figure}
			\centering \includegraphics[width=1\linewidth]{theorie/graphe/ToyModel-alpha_spirale.pdf}
			\caption{Courbe calorique pour des systèmes cœur--halo de pente $\alpha=4, 5, 6, 7, 8$\label{ToyModel::AllAlpha}}
	\end{figure}
	La même instabilité est toujours présente. Dans le cas $\alpha=2$, les calculs sont identiques, il conduisent au couple:
	\begin{eqnarray}
		\mu(x) =& \frac{10 (3x-2) (x-1)}{x(15x-14-10\ln(x))} \\
		\lambda(x) =& -\frac{3x(15x-14-10\ln(x))(x-2)}{20(x-1)(3x-2)^2}
	\end{eqnarray}
	Nous remarquons que $\lim(\lambda,\mu)_{x\to\infty}=(-\frac{1}{4},2)$ correspond bien à la spahère isotherme. La tangente verticale n'a pas
	résisté à l'idéalisation, par contre la tangente horizontale reste présente pour une valeur $x^*_2=3.48$ correspondant à un contraste de
	densité $\mathcal{R}_2= 12.1$. Cette valeur est environ trois fois plus petite que la valeur exacte (32.1) mais la transition stable/instable
	est bien respectée par le modèle approché. Il semble légitime de penser que la valeur de $\mathcal{R}_4$ trouvée pour le modèle cœur-halo de
	pente -4 soit également un peu sous-estimée.





	\part{Simulations}
		\chapter{Algorithme : Tree Code}
	\minitoc
	\section{Principe}
		\subsection{Présentation du problème}
	Les problèmes à N corps sont des problèmes physiques extrêmement complexes, mathématiquement: ils font partie d'une
	famille d'équation du second ordre qui n'ont pas de solution pouvant s'exprimer avec les fonctions mathématiques
	usuelles. Le seul cas connu pouvant s'écrire \og simplement\fg est le problème à 2 corps.

	Numériquement, ces problèmes sont très simple à implémenter. De façon basique, bête et méchante:
	\lstset{language=C, label=algo::NBodySimple, frame=shadowbox}
	\begin{lstlisting}
typedef struct {
    double x, y, z, vx, vy, vz, ax, ay, az, m;
} Point;

...

int main(void)
{
    int NbPart = 1e6;
    Point part[NbPart];
    ...
    //Some initialisation
    ...
    while(t < tmax && nbite < itemax)
    {
        for(int i=0; i<NbPart; i++)
        {
            part[i].ax  = part[i].ay  = part[i].az = 0;

            for(int j=0; j<NbPart; j++)
            {
                if( i != j )
                {
                    part[i].ax += -G * part[i].m * part[j].m
			* part[i].x / ( pow(sqrt( (part[i].x
			- part[j].x)*(part[i].x - part[j].x)
			+ (part[i].y - part[j].y)*(part[i].y
			- part[j].y) + (part[i].z - part[j].z)
			*(part[i].z - part[j].z)), 3.0) );
                    part[i].ay += -G * part[i].m * part[j].m
			* part[i].y / ( pow(sqrt( (part[i].x
			- part[j].x)*(part[i].x - part[j].x)
			+ (part[i].y - part[j].y)*(part[i].y
			- part[j].y) + (part[i].z - part[j].z)
			*(part[i].z - part[j].z)), 3.0) );
                    part[i].az += -G * part[i].m * part[j].m
			* part[i].z / ( pow(sqrt( (part[i].x
			- part[j].x)*(part[i].x - part[j].x)
			+ (part[i].y - part[j].y)*(part[i].y
			- part[j].y) + (part[i].z - part[j].z)
			*(part[i].z - part[j].z)), 3.0) );
                }
            }

            part[i].vx += part[i].ax * dt;
            part[i].vy += part[i].ay * dt;
            part[i].vz += part[i].az * dt;

            part[i].x += part[i].vx * dt;
            part[i].y += part[i].vy * dt;
            part[i].z += part[i].vz * dt;
        }
    }
}
	\end{lstlisting}
	Comme vous pouvez le voir, cette algorithme parcours 2 fois le tableau de particule. Par conséquent, le
programme effectue $\mathsf{NbPart}^2$ opérations. Quand le nombre de particules commence à augmenter, le temps de
calcul augmente: pour les ordinateurs actuels, calculer l'évolution d'un système tel un amas globulaire
($\mathsf{NbPart} \thickapprox 1e5 \to 1e6$) sur un temps dynamique (environ $10^6 \mathrm{ans}$) avec un pas de temps
$\mathsf{dt}$ d'environ $1e4 \mathrm{ans}$ prendrait plusieurs mois, voir années. Même en parallélisant le code  et en
le faisant tourner sur un super calculateur le temps de calcul resterait important. Nous avons donc besoin de trouver un
algorithme qui permette de faire ces calculs plus rapidement, mais sans perdre trop de précision sur les calculs. Cet
algorithme, c'est le Tree-Code développé par Barnes \& Hut dans~\cite{1986Natur.324..446B}.

\subsection{Présentation de l'algorithme}
	L'algorithme de Barnes et Hut est relativement simple: au lieu de travailler sur l'espace complet, nous le
séparons en $8$ cube, puis nous séparons à nouveau chacun de ces cubes en $8$ sous cubes, et ainsi de suite jusqu'à ce
que les cubes aient un certain nombre de particules à l'intérieur ou que l'arbre de cubes ainsi créé ait atteint un
certain niveau de raffinement.

	\begin{figure}[h]
		\begin{center}
			\newcommand{\TCarre}[2]{\coordinate (Xside) at (#1, 0);
	\coordinate (Yside) at (0 , #1);

	\coordinate (A) at ($ #2 -1/2*(Xside) - 1/2*(Yside) $);
	\coordinate (B) at ($ (A) + (Xside) $);
	\coordinate (C) at ($ (B) + (Yside) $);
	\coordinate (D) at ($ (C) - (Xside) $);

	\draw (A) -- (B) -- (C) -- (D) -- (A) -- cycle;
}
\begin{tikzpicture}
%	\coordinate (Xside) at (10, 0);
%	\coordinate (Yside) at (0 , 10);
%
%	\coordinate (A) at ($ (0,0) -1/2*(Xside) - 1/2*(Yside) $);
%	\coordinate (B) at ($ (A) + (Xside) $);
%	\coordinate (C) at ($ (B) + (Yside) $);
%	\coordinate (D) at ($ (C) - (Xside) $);
%
%	\draw (A) -- (B) -- (C) -- (D) -- (A) -- cycle;

	\coordinate (P1) at (1.5,2);
	\coordinate (P2) at ($ 5*(rand, rand) - 5*(1,1) $);
	\coordinate (P3) at ($ 5*(rand,rand) - 5*(1,1) $);

	\TCarre{10}{(0,0)}

\end{tikzpicture}

		\end{center}
	\end{figure}

	\section{Application : Calcul du potentiel}
		\input{simulation/potentiel.tex}
	\section{Application : Calcul du centre de densité}
		\input{simulation/densitycenter.tex}
	\FloatBarrier

\chapter{Étude numérique}
	\minitoc
	\section{Générateur de conditions initiales}
		Pour obtenir nos conditions initiales, qui devront suivre un profil de \textsc{King}, nous allons utiliser la méthode de
réjection, l'une des plus facile à mettre en place. Nous allons tirer
aléatoirement la position et la vitesse des particules dans une boîte,
puis nous n'en garderons qu'une partie en utilisant la fonction de distribution
comme densité de probabilité.
Pour commencer, nous utilisons les limites de notre modèle pour restreindre
nos tirages à une boîte de taille \mbox{$\left[ - r_{\mathrm{max}}; r_{\mathrm{max}} \right]$} pour
les distances et \mbox{$\left[ -v_{\mathrm{max}}; v_{\mathrm{max}}\right]$} pour les vitesses :
\begin{itemize}
	\item vitesse maximum : l'énergie totale du système est bornée supérieurement par l'énergie de libération :
		\begin{align}
			E = \dfrac{1}{2} m v_i^2 + m\psi(r) &< E_l \notag \\
			E_l - m\psi(r) &> \dfrac{1}{2} m v_i^2 \notag \\
			v_\mathrm{max}^2 = 2\(\dfrac{E_l}{m} - \psi(r)\) &> v_i^2 \notag \\
			v_{\mathrm{max}} = \sqrt{2\(\dfrac{E_l}{m} - \psi(0)\)} > v_i &> - \sqrt{2\(\dfrac{E_l}{m} - \psi(0)\)} = - v_{\mathrm{max}}
		\end{align}
	\item distance maximum : cette distance est obtenue pour
		$m\psi(r_{\mathrm{max}}) = E_l$. Il nous faut donc connaître le potentiel.
\end{itemize}

Le potentiel n'ayant pas d'expression analytique, nous allons devoir réutiliser
notre algorithme de résolution numérique utilisé dans les chapitres précédents pour
modèliser un King.

Il nous faut aussi pouvoir redimensionner les quantités obtenues. Pour cela, le
programme récupère dans un fichier de configuration les dispersions de vitesse
$\sigma_v^2$, rayon de c\oe ur $r_c$, temps de relaxation $T_c$ et distance au
soleil $R_\odot$ dans les unités du catalogue de \textsc{Harris}. Toutes sont
ensuite transformées en unités SI (~mètre, kilogramme, seconde~). Comme nous
avons pu le voir dans le chapitre~\ref{King::Chapitre} traitant du modèle de
King, la masse $m$ d'une particule n'influence pas le profil de densité final :
\begin{align}
	r_c^2 &= \dfrac{\sigma^2}{4\pi G m \rho_0} \notag \\
	      &= \dfrac{(\sigma_v^2)^2 m}{8\pi G m \rho_0} \notag \\
	      &= \dfrac{(\sigma_v^2)^2}{8\pi G \rho_0} \notag \\
	\Rightarrow \rho_0 &= \dfrac{(\sigma_v^2)^2}{8\pi G r_c^2}
\end{align}
Pour redimensionner la densité, nous n'avons donc pas besoin de connaître la
masse d'une particule. Ce constat nous permet de laisser ce paramètre libre et
de jouer sur le nombre de particules dans le système. En effet, une fois la
densité obtenue, nous pouvons l'intégrer sur le volume de l'amas pour trouver la
masse totale de ce dernier, puis, connaissant le nombre de particules, en déduire
la masse d'une étoile par la relation :
\begin{align}
	m = \dfrac{M_{tot}}{\text{Nombre de particules}}
\end{align}
Déduire le reste des paramètres utiles pour le redimensionnement est ensuite
assez simple.
La distance maximum $r_{\mathrm{max}}$ est déduite de la résolution
numérique des équations.

Pour pourvoir tout redimensionner, nous avons aussi besoin de connaître l'énergie
de libération de l'amas. Pour cela, nous utilisons le théorème de \textsc{Gauss}
et un petit raisonnement simple. Par définition, nous avons :
\begin{align}
	E_\mathrm{min} < E < E_l < 0 \ &\text{ et } \ E_l = \frac{p^2}{2 m} + m \psi(r)
	\intertext{soit :}
	\psi(r) = \frac{1}{m}\(E_l - \frac{p^2}{2 m}\)
	\intertext{La valeur maximale du potentiel est donc atteinte pour $p = 0$ :}
	\psi_\mathrm{max} = \psi(R) = \frac{E_l}{m}
	\intertext{Hors du système, le théorème de \textsc{Gauss} nous dit qu'il peut être vu comme une particule ponctuelle de masse $M$. Le potentiel hors de l'amas s'écrit donc :}
	\psi(r) = -\frac{G M}{r}
	\intertext{Par continuité, nous avons :}
	\psi(R) = \frac{E_l}{m} = -\frac{G M}{R}
\end{align}

Pour générer des nombres aléatoires dans l'intervalle voulu, nous utilisons la
fonction \verb|double ran2(long seed)| tiré de~\cite{NumericalRecipesC}, dont
nous nous servont ainsi :
\lstset{language=C, label=algo::tirage, frame=shadowbox}
\begin{lstlisting}
	double  x  = rmax - 2.0 * rmax * ran2(seed),
		y  = rmax - 2.0 * rmax * ran2(seed),
		z  = rmax - 2.0 * rmax * ran2(seed),
		vx = vmax - 2.0 * vmax * ran2(seed),
		vy = vmax - 2.0 * vmax * ran2(seed),
		vz = vmax - 2.0 * vmax * ran2(seed);
\end{lstlisting}
Notre système étant sphérique, nous ne devons pas avoir de vitesse et de module
de distance supérieurs, respectivement, à $v_{\mathrm{max}}$ et
$r_{\mathrm{max}}$, de plus nous avons une probabilité
\mbox{$f(E)/f(E_\mathrm{min})$}
d'avoir une particule d'énergie $E$. Cette énergie minimale est l'énergie
potentielle d'une particule au centre de l'amas, et de vitesse nulle.
Une fois qu'une particule respecte ces conditions et que \og les probabilités sont
avec elle \fg, nous l'enregistrons.

Le programme écrit au fur et à mesure les coordonnées cartésiennes et vitesses
des particules selectionnées dans un fichier dans les unités standards : mètre
pour les distances et mètre par seconde pour les vitesses.


	\section{Vérification des résultats\label{Verif_gene}} %du générateur}
		Maintenant que nous avons un générateur de conditions initiales, il convient de
le vérifier. C'est-à-dire d'utiliser les coordonnées, vitesses et masses des
particules pour remonter à des quantités comme la densité ou l'énergie de
l'objet créé, puis de comparer ces quantités aux prédictions théoriques.
Notamment, nous avons généré un profil de King, qui se doit donc d'être au
Viriel, mais aussi d'avoir une certaine pente sur la densité, comme vu dans les
précédents chapitres.

Pour faire les vérifications, nous avons choisi d'utiliser des histogrammes.

\subsection{Masse et densité}

\begin{comment}
\piccaption{Découpage de l'amas généré\label{schema::bin}}
\parpic{
	\begin{tikzpicture}[scale=0.8]
		\draw (2.5,2.5) circle(1);
		\draw (2.5,2.5) circle(1.5);
		\draw (2.5,2.5) circle(2);
		\draw (2.5,2.5) circle(2.5);
		\draw[red] (2.5,2.5) -- (3.5,2.5) node[midway, above] {$\mathrm{dr}$};
		\draw[red] (2.5,2.5) -- ++(50:1.5) node[sloped, above, midway] {$2 \mathrm{dr}$};
		\draw[red] (2.5,2.5) -- ++(-50:2) node[sloped, above, midway] {$3 \mathrm{dr}$};
		\draw (2.5,2.5) node[below]{$M_0$};
		\draw (2.5,1.15) node[below]{$M_1$};
		\draw (2.5,1.65) node[below]{$M_2$};
	\end{tikzpicture}
}
\parpic{
	\begin{tikzpicture}[scale=0.8]
		\draw (2.5,2.5) circle(1);
		\draw (2.5,2.5) circle(1.5);
		\draw (2.5,2.5) circle(2);
		\draw (2.5,2.5) circle(2.5);
		\draw[red] (2.5,2.5) -- (3.5,2.5) node[midway, above] {$\mathrm{dr}$};
		\draw[red] (2.5,2.5) -- ++(50:1.5) node[sloped, above, midway] {$2 \mathrm{dr}$};
		\draw[red] (2.5,2.5) -- ++(-50:2) node[sloped, above, midway] {$3 \mathrm{dr}$};
		\draw (2.5,2.5) node[below]{$M_0$};
		\draw (2.5,1.15) node[below]{$M_1$};
		\draw (2.5,1.65) node[below]{$M_2$};
	\end{tikzpicture}
}\newcaption{Découpage de l'amas généré\label{schema::bin}}
\end{comment}
\begin{wrapfigure}{l}{0.28\textwidth}
%	\begin{center}
		\begin{tikzpicture}[scale=0.8]
			\draw (2.5,2.5) circle(1);
			\draw (2.5,2.5) circle(1.5);
			\draw (2.5,2.5) circle(2);
			\draw (2.5,2.5) circle(2.5);
			\draw[red] (2.5,2.5) -- (3.5,2.5) node[midway, above] {$\mathrm{dr}$};
			\draw[red] (2.5,2.5) -- ++(50:1.5) node[sloped, above, midway] {$2 \mathrm{dr}$};
			\draw[red] (2.5,2.5) -- ++(-50:2) node[sloped, above, midway] {$3 \mathrm{dr}$};
			\draw (2.5,2.5) node[below]{$M_0$};
			\draw (2.5,1.15) node[below]{$M_1$};
			\draw (2.5,1.65) node[below]{$M_2$};
		\end{tikzpicture}
%	\end{center}
	\caption{Découpage de l'amas généré\label{schema::bin}}
\end{wrapfigure}
Le premier histogramme que nous générerons sera celui représentant
la masse en fonction du rayon. Notre objet étant sphérique, nous allons le
découper en intervalles de taille $\mathrm{dr}$ comme sur le schéma ci-contre. La fonction
de masse représente la masse se trouvant dans l'intervalle \mbox{$\left[0; j \mathrm{dr}\right]$}.
Pour la calculer, nous comptons le nombre de particules dans chaque chaque coquille sphérique de largeur $dr$ (~bin~), puis,
après avoir multiplié par la masse d'un particule, nous sommons, pour le bin
$j$, tous les bins inférieurs.

En même temps que nous calculons la fonction de masse, nous pouvons calculer la
densité en divisant la masse dans un bin par le volume du bin :
\begin{align}
	\rho_\mathrm{bin} &= \dfrac{M_\mathrm{bin}}{V_\mathrm{bin}} \notag \\
	\rho_i = \rho\( (i+1) \mathrm{dr}\) &= \dfrac{M_{\mathrm{bin}\ i}}{V_{\mathrm{bin}\ i}} \\
		&= \dfrac{M_{\mathrm{bin}\ i}}{\frac{4}{3}\pi ( (i+1)\mathrm{dr})^3 - \frac{4}{3}\pi ( i\mathrm{dr})^3} \notag \\
		&= \dfrac{3 M_{\mathrm{bin}\ i}}{4 \pi \mathrm{dr}^3 \left[ (i+1)^3 - i^3\right]} \notag \\
		&= \dfrac{3 M_{\mathrm{bin}\ i}}{4 \pi \mathrm{dr}^3 \left[ 3 i^2 + 3 i + 1\right]} \\
		&= \dfrac{3 \(M_i - M_{i-1}\)}{4 \pi \mathrm{dr}^3 \left[ 3 i^2 + 3 i + 1\right]}
\end{align}
avec $M_{-1} = 0$.

La densité obtenue et celle prévue par la résolution numérique sont très proche,
comme le montre le graphique~\ref{Comp_gene-theo}.
\begin{figure}[h!]
	\centering \includegraphics[scale=0.5]{graphe/Comp_dens_gene-theo_5000.pdf}
	\caption{Comparaison entre la résolution numérique et la densité donnée par le générateur\label{Comp_gene-theo}}
\end{figure}

\subsection{Énergie et potentiel}

La partie la plus complexe de la vérification est le calcul de l'énergie. Deux
choix s'offrent à nous :
\begin{itemize}
	\item la méthode force brute : nous calculons l'énergie totale en utilisant
		l'expression newtonienne du potentiel :
		$$
			E_{tot} = \frac{1}{2}\sum_{i = 1}^{N} m_i v_i^2 - G \sum_{i = 1}^{N} \sum_{j < i} \dfrac{m_i m_j}{|| r_i - r_j ||}
		$$
		avec $N$ le nombre de particule Le problème de cette méthode est qu'elle nécessite $N^2$
	opérations et n'est donc pas intéressante lorsque nous travaillons avec un grand
	nombre de particules. De plus, si deux particules sont très proche,
	l'énergie potentielle va diverger.
	\item la réflexion : nous avons déjà calculé la densité, et nous avons
		la fonction de masse, nous avons tout ce qu'il nous faut pour
		avoir le potentiel à partir de l'équation de \textsc{Poisson}.
\end{itemize}

Nous allons calculer le potentiel en résolvant l'équation de \textsc{Poisson}. Voyons
comment la résoudre avec ce que nous avons.
\begin{align}
	\Delta\psi &= \frac{1}{r^2}\dfrac{d}{dr}\( r^2 \dfrac{d \psi(r)}{dr} \) = 4\pi G \rho(r) \notag \\
	r^2 \dfrac{d \psi(r)}{dr} &= 4\pi G \int_0^r \rho(r) r^2 dr = G M(r) \notag \\
	\intertext{La densité est une fonction continue par morceau, nous pouvons donc écrire :}
	M(r)    &= 4\pi \int_0^r \rho(r) r^2 dr \notag \\
		&= 4\pi \sum_{j = 0}^{i - 1} \int_{j \mathrm{dr}}^{(j+1)\mathrm{dr}} \rho_j r^2 dr + 4\pi \int_{r_{i-1}}^r r^2 dr \text{, $r\in\left[ r_{i - 1}; r_i \right]$} \notag \\
		&= 4\pi \sum_{j = 0}^{i - 1} \rho_j \left[ \dfrac{r^3}{3} \right|_{j \mathrm{dr}}^{(j+1)\mathrm{dr}} + 4\pi\rho_i\left[\dfrac{r^3}{3}\right|_{r_{i - 1}}^{r} \notag \\
		&= 4\pi \sum_{j = 0}^{i - 1} \dfrac{\rho_j}{3} \mathrm{dr}^3 \( (j+1)^3 - j^3)\) + \dfrac{4\pi \rho_i}{3} \(r^3 - i^3\mathrm{dr}^3\) \notag \\
		&= M(r_{i-1}) + \dfrac{4\pi \rho_i}{3} \(r^3 - i^3\mathrm{dr}^3\) \notag \\
	\intertext{Ceci nous permet alors d'écrire le potentiel :}
	\psi(r) - \psi(0) &= G \int_0^r \dfrac{M(r)}{r^2} dr \notag \\
	\psi\(r_i\) - \psi(0) &= G \sum_{j = 0}^{i} \left\{\int_{j \mathrm{dr}}^{(j+1)\mathrm{dr}} \dfrac{M(r_{j-1})}{r^2} + \dfrac{4\pi \rho_j}{3 r^2} \(r^3 - j^3\mathrm{dr}^3\) dr\right\} \notag \\
	\intertext{avec $ r_i = (i+1) \mathrm{dr} $}
	\psi(r_i) - \psi(0) &= G \sum_{j = 0}^{i} \left\{M_{j-1} \left[ \dfrac{-1}{r}\right|_{j \mathrm{dr}}^{(j+1)\mathrm{dr}} + \dfrac{4\pi \rho_j}{3} \( \left[ \dfrac{r^2}{2} \right|_{j \mathrm{dr}}^{(j+1)\mathrm{dr}} - j^3\mathrm{dr}^3 \left[ \dfrac{-1}{r}\right|_{j \mathrm{dr}}^{(j+1)\mathrm{dr}} \)\right\} \notag \\
			    &=  G \sum_{j = 0}^{i} \left\{\dfrac{1}{j ( j + 1 ) \mathrm{dr}} \( M_{j - 1} - \dfrac{4\pi \rho_j}{3}j^3\mathrm{dr}^3 \) + \dfrac{4\pi \rho_j}{6}\( 2 j + 1 \)\mathrm{dr}^2\right\}
%			    &=  G \sum_{j = 0}^{i} \left\{\dfrac{1}{j ( j + 1 ) \mathrm{dr}} \( 4\pi \sum_{z = 0}^{j - 1}\( \dfrac{\rho_z}{3} \mathrm{dr}^3 \( (z+1)^3 - z^3\)\) - \dfrac{4\pi \rho_j}{3}j^3\mathrm{dr}^3 \) + \dfrac{4\pi \rho_j}{6}\( 2 j + 1 \)\mathrm{dr}^2\right\} \notag \\
%			    &=  G \sum_{j = 0}^{i} \left\{\dfrac{1}{j ( j + 1 ) \mathrm{dr}} \( 4\pi \sum_{z = 0}^{j - 1}\( \dfrac{\rho_z}{3} \mathrm{dr}^3 \( 3 z^2 + 3 z + 1\)\) - \dfrac{4\pi \rho_j}{3}j^3\mathrm{dr}^3 \) + \dfrac{4\pi \rho_j}{6}\( 2 j + 1 \)\mathrm{dr}^2\right\}
	\intertext{Pour le bin central $j = 0$, nous avons :}
	\psi(dr) - \psi(0)  &=  G \dfrac{4\pi \rho_0}{6}\mathrm{dr}^2
\end{align}

Pour obtenir la constante $\psi(0)$, nous allons nous servir des conditions sur le bord du système. En effet, nous avons vu plus haut que :
\begin{align}
	\psi_\mathrm{max} = \psi(R) &= - \frac{G M}{R} \\
	\intertext{Donc :}
	\psi(R) + \psi(0) - \psi(0) &= - \frac{G M}{R} \\
	\psi(0) &= - \frac{G M}{R} - \(\psi(R) - \psi(0)\) \\
	\psi(0) &= \frac{E_l}{m} - \(\psi(R) - \psi(0)\)
\end{align}

Le graphique~\ref{potentiel_5000} nous montre le potentiel théorique et le potentiel calculé par cette méthode.

\begin{figure}[h!]
	\centering \includegraphics{graphe/Potentiel_ci-100000.pdf}
	\caption{Comparaison entre la résolution numérique et le potentiel donné par le générateur\label{potentiel_5000}}
\end{figure}

%Nous pouvons maintenant calculer la différence de potentiel entre le centre de l'amas et un rayon $r_i = (i+1)\mathrm{dr}$, mais nous ne connaissons pas
%$\psi(0)$ qui nous permettrait de remonter au potentiel qui nous intéresse pour le calcul de l'énergie. L'une des solutions évoqué, et utilisé, est
%d'appliquer le théorème du Viriel :
%\begin{align}
%	2Ec + Ep &=0 \\
%	-\sum_{i = 1}^N m_i\psi(r_i) &= -2 E_c \\
%	\sum_{i = 1}^N m_i\(\psi(r_i) + \psi(0) - \psi(0)\) &= 2 E_c \\
%	M_{tot}\psi(0) &= 2 E_c - \sum_{i = 1}^N m_i \(\psi(r_i) - \psi(0)\)
%\end{align}

\subsection{Forme de l'amas}

Maintenant que la densité et le potentiel de l'amas généré ont été vérifiés, il faut aussi vérifier que l'amas ne change pas de forme : nous générons un amas sphérique, nous devons~\footnote{il a en effet été montré qu'un \textsc{King}
non collisionnel est stable~\cite{JPerez96}} conserver un amas sphérique après l'avoir
fait évoluer. Pour vérifier que la forme de l'amas ne change pas, nous allons regarder comment évoluent les axes principaux d'inertie. Pour ce faire, nous allons calculer les valeurs propres de la matrice d'inertie :
\begin{align}
	\mathfrak{I} &= \(\begin{array}{ccc}
				\int \(y^2 + z^2\) dm & - \int xy dm & - \int xz dm \\
				-\int xy dm & \int \(x^2 + z^2\) dm & - \int yz dm \\
				-\int xz dm & -\int yz dm & \int \(x^2 + y^2\) dm
			\end{array}\) \\
		     &= \(\begin{array}{ccc}
				A & - D & - E \\
				-D & B & - F \\
				-E & -F & C
			\end{array}\)\notag
	\intertext{L'équation aux valeurs propres va alors s'écrire :}
	\left|\mathfrak{I} - \lambda \mathbb{I}\right|  &= \(A - \lambda\)\left[\(B-\lambda\)\( C-\lambda\) - F^2\right] + D \(-D\(C-\lambda\) - F E\) - E \left[ D F + E\(B - \lambda\)\right] \notag %\\
%						0	&= \(A - \lambda\)\(B-\lambda\)\( C-\lambda\) - \(A - \lambda\) F^2 - \(C-\lambda\) D^2 - \(B - \lambda\) E^2 \notag\\
%						0	&= \(A - \lambda\)\(BC - \lambda\(B + C\) + \lambda^2\) - \(A - \lambda\) F^2 - \(C-\lambda\) D^2 - \(B - \lambda\) E^2 \notag\\
%						0	&= A B C - \lambda A \(B+C\) + A \lambda^2 - B C \lambda + \lambda^2\(B+C\) - \lambda^3 - \(A - \lambda\) F^2 - \(C-\lambda\) D^2 - \(B - \lambda\) E^2 \notag\\
%						0	&= -\lambda^3 + \lambda^2\(A + B + C\) - \lambda\( A \(B+C\) + BC - F^2 - D^2 - E^2\) + A B C - A F^2 - C D^2 - B E^2 \notag
\end{align}
polynôme d'ordre 3 que l'on résout avec la méthode de \textsc{Cardan}. %~\footnote{\url{http://fr.wikipedia.org/wiki/Méthode_de_Cardan}}.

Une fois ces valeurs propres obtenues, si elles existent, nous traçons l'évolution des rapports $\lambda_1 / \lambda_2$ et $\lambda_3 / \lambda_2$, les valeurs propres étant numérotées dans l'ordre décroissant
\mbox{$\lambda_1 > \lambda_2 > \lambda_3$}.

	\section{Utilisation du code \textsc{GADGET-2}}
		Le code que nous utilisons pour faire évoluer notre système est \textsc{Gadget-2}~\footnote{récupérable sur \url{http://www.mpa-garching.mpg.de/gadget/}}~\cite{gadget2}, écrit par Volker \textsc{Springel}.
Seule une petite partie de ce qu'il fait nous intéresse : nous n'allons utiliser que les options concernant le \og~tree code~\fg.
Toutes les autres options ne concernent que les simulations cosmologiques ou pourraient
induire un comportement du code qui nous ferait perdre de la précision sur les calculs.

Pour fonctionner, \textsc{Gadget-2} a besoin d'un fichier de configuration, dans lequel nous devrons jouer sur certains paramètres, et d'un fichier de conditions initiales respectant un format précis.

\subsection{Fichier de conditions initiales}

	Le fichier de conditions initiales doit avoir le format suivant :
	\begin{enumerate}
		\item un en-tête contenant le nombre de particule de chaque type (~Gaz, Halo, Disque, Bulbe, Étoiles, Bndry~), la masse pour chaque type, divers autres informations utiles essentiellement aux simulations cosmologiques,
		\item les positions de chaque particules,
		\item leurs vitesses,
		\item un identifiant permettant de repérer chaque particule.
	\end{enumerate}
	Chaque bloc devant être encadré par sa taille en mémoire.

\subsection{Fichier de configuration}

	Dans ce fichier, nous n'allons jouer que sur certains paramètres :
	\begin{itemize}
		\item \verb|OmegaLambda| : paramètre cosmologique représentant la densité d'énergie du vide, en le mettant à 0, nous faisons savoir à \textsc{Gadget-2} que nous ne faisons pas de simulation cosmologique.
		\item \verb|UnitLength_in_cm|, \verb|UnitMass_in_g| et \verb|UnitVelocity_in_cm_per_s| sont les unités dans lesquelles sont données, respectivement, les positions, masses et vitesses des particules en
			centimètre, gramme en centimètre par seconde. Ce sont ces facteurs de conversion qui donne l'unité de temps interne à \textsc{Gadget-2}. Nous utilisons les parsecs (~$ 1 pc = 3.086 \times 10^{18} cm$~)
			pour les positions, les kilogrammes (~$1 kg = 1000 g$~) pour la masse, et les mètres par seconde (~$ 1 m.s^{-1} = 10^2 cm.s^{-1}$~) pour les vitesses. Ces unités nous donnent comme unité de temps
			interne :
			\begin{align}
				v &= \frac{d}{t} \notag \\
				t &= \frac{d}{v} \notag \\
				t &= \frac{3.086 \times 10^{18}}{10^2} = 3.086 \times 10^{16} s \notag \\
				t &= 9.77894 \times 10^8 ans
			\end{align}
		\item \verb|SofteningStarsMaxPhys| : paramètre de lissage de la force, permettant d'éviter qu'elle \og~n'explose~\fg~à cause d'une collision entre 2 particules trop proches.
			C'est sur ce paramètre qu'il faut jouer pour assurer la stabilité du système sur un grand nombre de temps dynamiques.
		\item \verb|ErrTolTheta| : représente l'angle d'ouverture, ou résolution angulaire, minimum. Celui-ci est fixé à $0.5$ et n'est plus modifié ensuite.
	\end{itemize}

\subsection{Lissage de la force}
	Nous souhaitons nous placer dans la limite fluide afin de minimiser les effets de relaxations dû aux collisions à deux corps. Pour cela, nous pouvons jouer sur deux paramètres : le nombre de particules que nous mettons dans
	le système, et le paramètre de lissage. Étant limité en temps, nous ne pouvons pas lancer de simulations avec un très grand nombre de corps : les plus grosses simulations que nous lançons ont 100 000 particules, et
	occasionnellement 500 000. Pour une évolution pendant 100 temps dynamiques, elles prennent environ 709 minutes en les faisant tourner sur 8 cœurs, et cela peut aller jusqu'à 1020 minutes.

	Le lissage est une borne, en distance, en deçà de laquelle nous considérons un potentiel minimum valant :
	\begin{align}
		\psi(r_{i} \sim 0) = - G \dfrac{m_i}{r_{i} + \epsilon}
	\end{align}
	où $r_i$ est le module de distance d'une particule numérotée $i$. $\epsilon$ est le paramètre de lissage de la force. Afin de nous placer dans la limite fluide, $\epsilon$ doit être tel qu'il contienne un
	nombre $N_\epsilon$ de particule grand devant 1 mais petit devant le nombre total de particule du système. Habituellement, c'est $N_\epsilon$ qui est choisi, imposant ainsi $\epsilon$, mais nous avons choisi $\epsilon$
	imposant ainsi $N_\epsilon$.

%	Dans un amas, la densité est plus forte au centre que sur le bord. Les effets de relaxation sont donc plus important au centre, le $\epsilon$ doit y minimiser ces effets, même s'il n'a aucun effet sur les bords de l'amas.
	À cause d'instabilité, il est important de décrire la dynamique au centre de l'amas avec la meilleur précision possible. Il est donc nécessaire d'être dans la limite fluide au centre.
	Il est alors intéressant de le relier à la densité centrale, plutôt que d'utiliser une distance inter-particulaire moyenne. Nous définissons donc $\alpha$ tel que :
	\begin{align}
		\alpha = \dfrac{\rho(0)}{\rho_\mathrm{moy}} = \dfrac{\rho(0)}{M} \frac{4}{3} \pi R^3
	\end{align}
	avec $M$ la masse totale de l'amas, $R$ le rayon de l'amas, et $\rho(0)$ la densité centrale de l'amas. La densité d'un volume de taille $\epsilon$ et la densité de l'amas nous donne donc :
	\begin{align}
		\rho_\epsilon &= \alpha \rho_\mathrm{moy}					\notag \\
		N_\epsilon    &= \alpha \frac{4}{3} \pi \epsilon^3 \dfrac{3N}{4\pi R^3}	\notag \\
			      &= \alpha N \( \dfrac{\epsilon}{R} \)^{3}
	\end{align}
	$N$ étant le nombre total de particules de l'amas. $N$, $R$ et $\alpha$ étant fixés, nous sommes ainsi libres de choisir $N_\epsilon$ ou $\epsilon$ pour répondre à nos besoins.


	\section{Résultat des simulations}
		Les seules simulations que nous avons pu faire dans le cadre du stage sont des tests pour vérifier la stabilité de nos conditions initiales, ce qui représente déjà un travail important.

Nous avons donc commencé par générer des amas comportant un faible nombre de particules (~1000 ou 5000 particules~) afin de vérifier notre générateur, ce sont les résultats montrés par la courbe~\ref{Comp_gene-theo} pour 5000 particules, puis~\ref{potentiel_5000} pour 100000 particules.
Il nous faut maintenant vérifier que ces amas restent stables sur un grand nombre de temps dynamiques. Les simulations présentées dans cette section utilisent un amas réel dont les données ont été récupéré dans le catalogue de
\textsc{Harris} et à partir du travail fait dans les chapitres précédents. Il s'agit de l'amas NGC 288. Pour cet amas, les calculs analytiques donnent $\alpha = 439.72$.

Passons maintenant à l'étude de la stabilité sur le "long" terme de nos amas. Nous montrerons ici les résultats des tests
décrits dans la section~\ref{Verif_gene}.

Dans la table~\ref{eps_Neps}, nous avons indiqué quelles valeurs de $\epsilon$ nous avons utilisées : la valeur de $N_\epsilon$ correspondante, le nombre de particules qui sont au-delà du rayon des conditions initiales de l'objet,
et quelques autres informations utiles pour trouver la valeur qui nous intéresse.
	\begin{table}[h!]
		\begin{center}
			\begin{tabular}{|c|c|c|c|c|c|}
				\hline
				\multirow{2}{1cm}{$\epsilon\ \(pc\)$}	&	\multirow{2}{1cm}{$N_\epsilon$}	&	\multirow{2}{1cm}{$N_\mathrm{out}$}	&	\multirow{2}{3.5cm}{Fraction d'énergie cinétique emportée}	&	\multirow{2}{3.5cm}{Fraction d'énergie potentielle emportée}	&	\multirow{2}{2cm}{Courbes associées} \\
					&	&	&	&	&	\\
				\hline
				\hline
				$0.0194028$	&	$ 0.44 $		&	380	&	$ 0.00013573$			&	$ 0.00081092$	&	\ref{soft::0.019}, \ref{soft::0.019-Ax}\\
				\hline
				$0.05$		&	$ 7.52 $		&	239	&	$ 8.04628\times 10^{-5}$	&	$ 0.000488421$	&	\ref{soft::0.05}, \ref{soft::0.05-Ax}\\
				\hline
				$0.15$		&	$ 203.17 $		&	282	&	$ 9.91201\times 10^{-5}$	&	$ 0.00106241$	&	\ref{soft::0.15}, \ref{soft::0.15-Ax}\\
				\hline
				$0.20$		&	$ 481.58 $		&	653	&	$ 0.000265688$			&	$ 0.00265207$	&	\ref{soft::0.2}, \ref{soft::0.2-Ax}\\
				\hline
				$0.30$		&	$ 1625.34 $		&	919	&	$ 0.000433181$			&	$ 0.00466071$	&	\ref{soft::0.3}, \ref{soft::0.3-Ax}\\
				\hline
			\end{tabular}
		\end{center}
		\caption{Valeurs testés pour $\epsilon$ et $N_\epsilon$\label{eps_Neps}}
	\end{table}

	Discutons maintenant les résultats.

	\begin{description}
	%\paragraph{$\epsilon = 0.0194028$ :}
	\item[$\epsilon = 0.0194028$]
	Ce paramètre donne des résultats satisfaisants : sa densité évolue assez peu sur le temps de la simulation comparé aux valeur de $\epsilon > 0.05$, et ses axes d'inertie restent constant
	(~le bruit visible sur les graphes est dû au bruit statistique, bruit statistique dû au nombre fini de particules~). Par contre, le nombre de particules dans un volume de taille $\epsilon$ est trop petit : nous sommes trop loin de la limite fluide.

	%\paragraph{$\epsilon = 0.05$ :}
	\item[$\epsilon = 0.05$]
	Les fractions d'énergie potentielle et cinétique emportées par les particules sortantes sont minimum pour ce paramètre. De plus, sa densité et ses axes d'inertie évoluent peu, comme pour la valeur précédente.
	Cette fois, nous avons suffisamment de particules dans le volume de taille $\epsilon$.

\begin{figure}[h!]
	\centering \includegraphics[scale=0.5]{graphe/Comp_dens_gene-theo_0-05.pdf}
	\caption{Comparaison entre la densité numérique et la densité après évolution : $\epsilon = 0.05$\label{soft::0.05}}
\end{figure}

\begin{figure}[h!]
	\centering \includegraphics[scale=0.5]{graphe/Axial_ratio_0-05.pdf}
	\caption{Évolution des rapports des axes d'inertie : $\epsilon = 0.05$\label{soft::0.05-Ax}}
\end{figure}

	%\paragraph{$\epsilon = 0.15$ :}
	\item[$\epsilon = 0.15$]
	Pour ce paramètre, la fraction d'énergie potentielle emportée est de l'ordre de $0.1\%$, ce qui représente un changement important pour le rapport du Viriel $2 E_c/E_p$
	(~avec $E_c$ l'énergie cinétique et $E_p$ l'énergie potentielle~) : le profil de densité a complétement changé, l'amas s'est étendu.
	Par contre, il commence à se passer des choses intéressantes au niveau des axes d'inertie : pendant une grande partie de la simulation l'amas conserve sa forme, puis une instabilité arrive et il se déforme. % selon
%	l'un des axes, mais reste constant sur le second.
	\item[$\epsilon > 0.15$]
	Les valeurs supérieurs de $\epsilon$ ne sont alors clairement pas intéressante. De plus, ces valeurs sont trop proches du rayon de cœur de l'amas qui est de l'ordre de $10^{16}\ m = 0.32\ pc$ : en lissant toute
	la partie centrale de l'amas, nous changeons sa dynamique. En effet, il suffit de voir que, en augmentant le paramètre de lissage, l'instabilité menant à une déformation arrive de plus en plus tôt.
	Les déformations deviennent plus violentes.
	\end{description}

	La valeur de lissage optimale se trouve donc entre $\epsilon = 0.05$ et $\epsilon = 0.15$, pour une dizaine de particule (~pour $N_\epsilon = 10$, $\epsilon \sim 0.05497\ pc$~).



\chapter{Simulation numérique}
	\minitoc
	\section[1ère idée]{1ère idée: SIK dans un bain homogène}
		La première idée qu'il nous est venu pour tester le scénario décrit dans le
chapitre~\ref{Sec::ToyModel}, était de prendre un objet suivant la fonction de
distribution de la SIK puis de le placer dans un cube homogène faisant office de
bain thermique.

Le premier souci apparent est que le bain va influer sur la SIK en lui donnant
des particules, et cette dernière va déstabiliser le bain, le poussant à
s'effondrer. Pour palier à ce problème, nous avons utilisé une option de
\textsc{GADGET} qui permet de désactiver, pour un ou plusieurs types de
particules, les interactions gravitationnelles qu'elles subissent.
Ainsi, en activant cette option pour les particules de type 4, par exemple,
elles influencerons les autres particules, mais elles-mêmes se déplacerons en
ligne droite, ne ressentant pas les autres particules.

METTRE QUELQUES GRAPHES

Ces simulations ont échoué parce que le bain n'avait aucune influence sur la
SIK.

	\section[2nde idée]{2nde idée: SIK dans une autre SIK}
		Pour ce jeu de simulation, nous nous servons de ce qui a été démontré dans la section~\ref{sec::temp}:
l'isothermalité de la SIK. De plus, il est aisé de jouer sur les paramètres de l'objet pour lui donner
une pente de $-4$, comme dans avec le toy-model. Nous avons donc la possibilité de mettre un \CH{4}
dans une sphère isotherme.

Le premier problème à apparaître est la température à atteindre. Cette température est assez faible
et pour compenser cette faible température, la SIK servant de bain atteignait des tailles tel qu'elle
était complètement dilué.

	\section[3ème idée]{3ème idée: sphère de Hénon dans une SIK}
		\input{simulation/idee3.tex}
	\section[4ème idée]{4ème idée: sphère de Hénon dans un bain homogène en interaction gravitationnelle}
		\label{Simu::Idee4}
		Pour cette gamme de simulation, nous avons choisi de revenir sur un jeu de conditions proche de notre première idée. Nous allons plonger une sphère de Hénon dans un bain homogène remplissant une boîte.
La première chose à faire est de stabiliser le cube. Mais présentons d'abord le système d'unité utilisé ici.

\subsection{Système d'unité}
	\label{simu::sec::unit}
	Nous nous plaçons dans le système décrit dans \cite{fuji1983}. C'est à dire:
	\begin{itemize}
		\item $\tilde{M_h} = 1$
		\item $\tilde{R_0} = 2$
		\item $\tilde{G} = 1$
	\end{itemize}

	Le changement de variable a effectué pour passer des unités physiques à celle de l'article est le suivant:
	\begin{align}
		\begin{cases}
			\tilde{m} = \frac{m}{M_t} \\
			\\
			\tilde{r} = \frac{r}{R_0} \\
			\\
			\tilde{v} = \frac{v}{\sqrt{\frac{GM_t}{R_0}}}
		\end{cases}
	\end{align}
	Point intéressant, le temps dynamique devient:
	\begin{align}
		T_d = \pi \sqrt{\frac{R_0^3}{2GM}} = 2\pi
	\end{align}

\subsection{Stabilité du cube}
	\begin{wrapfigure}{l}{0.20\textwidth}
		\begin{tikzpicture}[scale=1.5]
			\draw (0, 0) -- (1, 0) -- (1, 1) -- (0, 1) -- (0, 0);
			\draw[<->] (0, -0.1) -- (1, -0.1);
			\draw (0.5, -0.1) node[below] {$\tilde{R_c}$};
			\draw[<->] (-0.5, -0.5) -- (1.5, -0.5);
			\draw (0.5, -0.5) node[below] {$L_\mathrm{Jeans}^c$};
		\end{tikzpicture}
	\end{wrapfigure}
	Vouloir faire une simulation avec un cube interagissant, c'est bien beau, mais un tel objet risque de s'effondrer plutôt vite. Il va nous falloir une condition pour pouvoir le stabiliser.
	Une sphère homogène est stable si sa taille est inférieur à sa longueur de Jeans. Notre critère est donc le suivant:
	\begin{align}
		R_c &\le L_\mathrm{Jeans}^c = \dfrac{\sigma_c^2}{\sqrt{G\rho_c}} \\
		R_c^2 &\leq \dfrac{\sigma_c^2}{G\rho_c} \notag \\
		R_c^2 &\leq \dfrac{\sigma_c^2 R_c^3}{GM_c} \notag \\
		\intertext{Soit $N_c$ le nombre de particules du cube:} \\
		1 &\leq \dfrac{\sigma_c^2R_c}{GN_c m_c} \notag
	\end{align}
	Nous imposons aux particules du cube d'avoir la même masse que celle du Hénon, soit:
	\begin{align*}
		m_c = m_h = m = \dfrac{M_h}{N_h}
	\end{align*}
	où $N_h$ est le nombre de particules du Hénon. Notre critère devient:
	\begin{align}
		1 &\leq \dfrac{\sigma_c^2R_cN_h}{GN_c M_h} \notag \\
		GM_h\dfrac{1}{\sigma_c^2}\dfrac{N_c}{N_h} &\leq R_c \notag
	\end{align}
	en oubliant pas que: $G=1$ et $M_h=1$.
	\begin{align}
		\dfrac{1}{\sigma_c^2} \dfrac{N_c}{N_h} &\leq R_c \label{simu::eq::idee4_jeans}
	\end{align}
	

\subsection{Conditions de la simulation}
	Comme nous cherchons à vérifier notre modèle, nous devons placer le bain à une température
	inférieur à celle que la sphère de Hénon atteint une fois à l'équilibre. Soit $\sigma_h^f$ la
	dispersion de vitesse de la sphère après effondrement, on doit avoir:
	\begin{align}
		\sigma_c &< \sigma_h^f \label{simu::eq::idee4_sig}
	\end{align}

	En parallèle, il serait bon d'éviter que le bain détruise la sphère de Hénon. Il faudrait
	donc avoir:
	\begin{align}
		\rho^f\(R\) \geq \rho_c \label{simu::eq::idee4_dens}
	\end{align}
	où $\rho^f\(R\)$ représente la densité au bord de la sphère, une fois l'équilibre
	atteint.

\subsection{Critères de sélections des paramètres}
	En combinant les équations \ref{simu::eq::idee4_dens}, \ref{simu::eq::idee4_sig} et
	\ref{simu::eq::idee4_jeans}, nous obtenons l'ensemble d'équations suivant:
	\begin{align}
		\begin{cases}
			\dfrac{1}{\sigma_c^2} \dfrac{N_c}{N_h} \leq R_c \\
			\\
			\sigma_c < \sigma_h^f
		\end{cases}
	\end{align}
	%avec $R_h$ le rayon de la sphère une fois l'équilibre atteint, sans le bain.

\subsection{Jeux des paramètres}
	Pour vérifier l'influence du bain, nous devrons jouer sur sa température (sa dispersion de vitesse).
	Mais, nous souhaitons conserver les autres paramètres les moins inchangées possible. Par exemple,
	tel que la simulation est construite, un certains nombres de particules $n_c$ sont directement
	incluse dans le Hénon qui se retrouve alors avec $N_h + n_c$ particules et donc une masse de $M_h + n_c m$.
	De ce fait, pour avoir les simulations les plus similaires possible, nous devons conserver la densité moyenne
	du cube $\rho_c$ constante afin de conserver au mieux $n_c$.

	Voyons comment se comporte notre système:
	\begin{align}
		\rho_c = \dfrac{M_c}{R_c^3} = \dfrac{mN_c}{R_c^3}
	\end{align}
	Nous faisons évoluer notre dispersion de vitesse tel que:
	\begin{align*}
		\sigma_c' = k \sigma_c
	\end{align*}
	en conservant:
	\begin{align}
		\rho_c' = \rho_c \label{simu::eq::idee4_cubedens}
	\end{align}
	Le nouveau rayon s'écrit:
	\begin{align}
		R_c' &= \dfrac{1}{\sigma_c'^2}\dfrac{N_c'}{N_h} = \dfrac{1}{k^2\sigma_c^2}\dfrac{N_c'}{N_h} \notag \\
		\intertext{En utilisant \ref{simu::eq::idee4_cubedens}:}
		\rho_c' &= \rho_c \notag \\
		\dfrac{m N_c}{\(\dfrac{1}{k²\sigma_c²}\dfrac{N_c'}{N_h}\)^3} &= \dfrac{m N_c}{\(\dfrac{1}{\sigma_c²}\dfrac{N_c}{N_h}\)^3} \notag \\
		\dfrac{1}{\dfrac{1}{k^6}N_c'^2} &= \dfrac{1}{N_c^2} \notag \\
		N_c' &= k^3 N_c
	\end{align}
	Ainsi, en augmentant d'un facteur $k$ la dispersion de vitesse, il faut augmenter d'un facteur $k^3$ le nombre de particules dans le cube.

\subsection{Test de stabilité du cube dans ces conditions}
	\begin{figure}
		\begin{center}
			\includegraphics[width=\textwidth]{graphe/dc.png}
			\caption{Carte de l'objet à $t=0$}
			\label{simu::graphe::dccarte0}
		\end{center}
	\end{figure}
	\begin{figure}
		\begin{center}
			\includegraphics[width=\textwidth]{graphe/dc500.png}
			\caption{Carte de l'objet à $t=50$}
			\label{simu::graphe::dccarte50}
		\end{center}
	\end{figure}
	\begin{figure}
			\begin{minipage}[b]{0.40\linewidth}
				\centering \includegraphics[width=\textwidth]{graphe/test_simu_temp.pdf}
				\caption{Température moyenne de l'objet}
				\label{simu::graphe::temp}
			\end{minipage}\hfill
			\begin{minipage}[b]{0.48\linewidth}
				\centering \includegraphics[width=\textwidth]{graphe/test_simu.pdf}
				\caption{Viriel de l'objet}
				\label{simu::graphe::viriel}
			\end{minipage}
	\end{figure}

\subsection{Évolution d'une sphère de Hénon isolée}
	\subsubsection{Évolution dans le vide}
		Des simulations sur l'évolution d'une sphère de Hénon dans le vide on été faîte avec le même
		jeux de paramètres. Ces simulations serviront de référence pour celle avec bain. Toutes ont
		été faîtes jusqu'au temps $t = 50$ dans les unités indiqués section~\ref{simu::sec::unit}.

		\begin{figure}
			\begin{center}
				\includegraphics[width=\textwidth]{graphe/All_parameters.pdf}
				\caption{Évolution dans le temps des paramètres}
				\label{simu::graphe::Parameter}
			\end{center}
		\end{figure}

	\subsection{Effet du nombre de particule}
		Les simulations mentionnées dans le paragraphe précèdent ont été faîtes avec 300 000 et 1 000 000
		de particules. Le graphique~\ref{simu::graphe::densitecomp} compare les profiles de densité volumiques de masse
		à la fin de chacune des simulations.

		\begin{figure}
			\begin{center}
				\includegraphics[width=\textwidth]{graphe/comparison_between_310part.pdf}
				\caption{Comparaison de l'état final entre une simulation de 300 000 particules et une de 1 000 000 de particules.}
				\label{simu::graphe::densitecomp}
			\end{center}
		\end{figure}




	\part*{Conclusion}
		\addstarredpart{Conclusion}
		\chapter{Conclusion}
			% \addstarredchapter{Conclusion}

		% \section{Conclusion}

			% Au cours de la thèse, nous avons montré que les modèle de King, et par extension les \textsc{sag} de type cœur halo de pente
			% $\alpha>2$, étaient des sphères isothermes. Nous avons alors développé un modèle idéalisé étendant l'étude de la \textsc{sib}
			% à ces sphères isothermes, de pente quelconque, plongé dans un bain thermique. Ce modèle conserve l'instabilité associé aux systèmes
			% canoniques. Mais le contraste associé au déclenchement de l'instabilité est sous-estimé pour $\alpha=2$. Nous pouvons supposer
			% qu'il le sera aussi pour les autres systèmes.

			% Nous nous sommes ensuite concentré sur les simulations numériques. Notre but était de valider la présence de cette
			% instabilité. Nous avons testé plusieurs configurations de conditions. La première configuration (un King plongé dans un bain
			% thermique qui ne subissait pas son influence) n'a rien donnée suite à l'absence d'effet de relaxation. La seconde a consisté à
			% plonger une sphère de Hénon dans un réservoir thermique. Cette dernière catégorie a donnée des résultats intéressant.

			% Ces simulations sont constitué de deux systèmes: un bain thermique et un \textsc{sag}. Sur nos deux tentatives,
			% l'une n'a rien donnée. Nous avons trop bien supprimer les collisions à deux corps. L'autre tentative a donnée des résultats
			% plus intéressant. Ces résultats ont été rangé dans deux classes.

			% La première classe concerne les simulations montrant une instabilité d'orbite radiale. Ces simulations nous ont permit
			% de montrer que cette instabilité peut se produire lorsqu'un objet en accrète un autre. Nous avons alors mis en évidence que le
			% paramètre permettant de contrôler le déclenchement de l'instabilité est la densité du bain. Sur les rangs de valeurs essayé,
			% la dispersion de vitesse du bain ne semble pas avoir d'effet sur son déclenchement.

			% La seconde classe de simulation présente une diminution progressive des rayons $R_{10}$ et $R_{50}$, mais aucune accrétion.
			% Ces simulations ont donc montrées un effondrement progressif du cœur, accompagné d'une augmentation de la densité du cœur. Une
			% étude des différentes pentes du systèmes nous a appris qu'elles évoluait de sorte que la densité passe d'un système cœur-halo
			% à un système présentant un cusp, compatible avec les modèles de matière noire. Nous avons mit en évidence que cette évolution
			% était bien conduite par les effets de relaxations et non par le bain, mais elle se produit trop rapidement pour n'être dû
			% qu'aux effets de relaxations à deux corps: le \textsc{sag} s'effondre en moins de moins de $10\%$ du temps de relaxation à
			% deux corps.

			% Au final, nous n'avons pas réussi à observer d'instabilité lié à notre problématique. Dans le cas de la première
			% configurations de conditions initiales, nous n'avons, à priori, pas laissé suffisamment de temps aux collisions à deux corps pour agir et
			% thermaliser complètement le \textsc{sag}. Dans le cas de la seconde configuration, soit le bain n'a aucun effet sur les temps
			% considérés, soit nous somme dominé par l'accrétion du bain par le \textsc{sag}.

			% En parallèle, nous avons conduit une comparaison entre un code résolvant l'équation de Vlasov pour un système sphérique et le
			% \og{}treecode\fg \textsc{gadget-2}. Nous avons montré que l'accord entre ces deux codes était étonnamment bon, même à faible
			% rapport du viriel. (Cette partie de la conclusion est à terminer une fois le chapitre associé terminer).

			% % Ces deux systèmes évolue en interaction l'un avec
			% % l'autre. Au cours de ces simulations, nous avons mis en évidence deux types de comportement:
			% % \begin{itemize}

				% % \item lorsque le bain s'effondrait sur le \textsc{sag}, ce dernier développait une instabilité d'orbite radiale
					% % paramétré par la densité du bain;

				% % \item certaines simulations montraient un effondrement progressif du cœur compatible avec l'action des collisions à
					% % deux corps, mais trop rapide pour ces dernières.

			% % \end{itemize}
			% % Par ailleurs, aucune des simulations effectué n'a montré l'instabilité que nous cherchions.

		% \section{Perspectives}

			% Nous aimerions pouvoir étendre l'étude des simulations présentant l'instabilité d'orbite radiale afin de modéliser plus
			% précisément le déclenchement de cette instabilité en fonction de la densité du bain. Dans le même temps, il serait très
			% intéressant de confirmer ou d'infirmer définitivement la dépendance de \textsc{roi} en fonction de la dispersion de vitesse du
			% bain en effectuant des simulations avec un $\sigma$ plusieurs ordre de grandeur supérieur à ceux déjà testé.

			% Une étude plus poussée de la simulation $A_{6,1}$ est nécessaire. En l'état actuel, nous n'avons aucun indice à propos du processus
			% conduisant l'effondrement de cette simulation. Dans un premier temps, il serait intéressant de la reproduire en faisait
			% évoluer divers paramètre de \textsc{gadget-2} pour tester leurs influences sur le résultat. En parallèle, il serait utile de
			% chercher d'autres observables susceptible de nous en apprendre plus sur la dynamique du système.

			% Enfin, nous avons plusieurs idées de conditions initiales pouvant mettre en évidence l'instabilité recherché.
			% \begin{enumerate}
				% \item jusqu'ici, le bain avait toujours une densité inférieur à celle du Hénon. Il serait intéressant de faire des
					% simulations avec un Hénon et un bain de densité égale. Les simulations de la famille $C^m_{3,i}$ nécessiteraient au moins $60$ fois plus
					% de particules dans le bain.

				% % \item Hénon dans bain même densité;

				% % \item Hénon déjà effondrer pour avoir $r^{-4}$.

				% \item Pour correspondre à notre modèle, nous avons besoin que le \textsc{sag} et le bain soit à la même température.
					% Nous pourrions tenter de génerer directement un \textsc{sag} ayant la même température que le bain, puis jouer
					% sur son contraste de densité pour le placer le plus proche possible de l'instabilité.

			% \end{enumerate}

			\section{Conclusion}

Les travaux de cette thèse ont eu pour objet la modélisation dynamique de la structure des amas globulaires et des galaxies, structures
autogravitantes (\textsc{sag}). 



Il est communément accepté que ces objets possèdent globalement un profil de densité volumique de masse caractérisé soit par une structure cœur halo
soit par un simple halo pouvant posséder plusieurs pentes. L'objectif de cette thèse était de comprendre les différentes propriétés et les raisons de
l'évolution dynamique de ces profils. Il a été atteint dans ses grandes lignes.



Le tout début de l'étude a été consacré à réalisation d'une synthèse de l'évolution dynamique du profil de densité des amas globulaires de notre
galaxie. Cette étude a permis de confirmer le fait bien connu que ces profils sont, avant le collapse du cœur, ajustables par un modèle de King. Elle
a aussi et surtout mis en évidence le fait que la pente du halo était une fonction croissante de l'âge de ces objets. Cette étude préliminaire
confirme donc le paradigme suivant lequel le profil de densité est l'un des indicateurs fondamentaux de l'évolution des amas globulaires et par
extension des galaxies.


Après un travail de revue théorique sur les différents types de structures isothermes en domaine borné ou non. Nous avons montré que les modèles de
King, et par extension les \textsc{SAG} de type cœur halo de pente inférieur à $-2$, étaient d'excellentes approximations de sphères isothermes. En
utilisant un modèle idéalisé de structure cœur halo isotherme de pente quelconque, nous avons pu étendre certains résultats d'instabilité jusque là
obtenus uniquement dans le cadre des sphères isothermes en domaine borné. Cette instabilité appliquée aux structures cœur halo de pente $-4$
(instabilité \textsc{ch4}), a été mise en contexte  en dehors du scénario évolutif à long terme, pour expliquer l'existence de deux type de profils
pour les \textsc{SAG}: cœur halo (pour les amas globulaires ou les galaxies LSB) ou cupside (pour les galaxies en général). La présence ou l'absence
(réelle ou effective) d'un bain thermique autour de l'objet en formation permettrait ou non à cette instabilité de se développer et conduirait ou pas
à l'effondrement du cœur de ces objets sur des échelles de temps de l'ordre de quelques temps dynamiques.



Afin d'ajuster les différents paramètres de ce scénario nous avons effectué un certain nombre de simulations numériques. Ces expériences démarrées
dans un contexte ciblé se sont révélées d'une ampleur et d'une portée supérieure aux attentes initiales. 



Face à la non-évolution des profils de King placés dans un thermostat, nous avons augmenté la sensibilité de nos  expériences en imposant un bain
thermique dès la phase initiale d'effondrement d'une sphère de Hénon.



% Toutes nos expériences ont été réalisées avec des bains plus froids que la structure autogravitante d'étude. 
La plupart de nos expériences ont été réalisées avec des bains plus froids que la structure autogravitante d'étude. Mais les bains sont toujours moins
denses.

Les raisons de cette dernière limitation sont simple. La méthode particulaire utilisée pour modéliser notre bain
thermique dans de bonnes conditions se serait révélée trop couteuse avec un contraste de densité plus faible entre le bain et la structure
autogravitante.
% Les raisons de cette limitation sont doubles. Tout d'abord, il est physiquement raisonnable de penser qu'une structure formée, cohérente et à
% l'équilibre, soit dans ce contexte, plus chaude que son environnement. Par ailleurs, la méthode particulaire utilisée pour modéliser notre bain
% thermique dans de bonnes conditions se serait révélée trop couteuse avec un contraste de température plus faible entre le bain et la structure
% autogravitante.


Malgré cette limitation importante nous avons obtenu deux résultats principaux:


\begin{enumerate}

	\item Dans tous les cas d'accrétion progressive du bain par la structure,
	nous avons observé l'apparition d'une instabilité d'orbite radiale.
	Certains éléments du mécanisme de cette instabilité,
	qui n'avait pas été observée dans ce contexte, ont pu être
	analysés:

	\begin{itemize}
		\item Le contexte général de nos résultats sur cette
		instabilité confirme un résultat déjà rapporté mais non
		généralement reproduit. L'instabilité d'orbite radiale se produit
		par déplacement adiabatique d'un système autogravitant à
		l'équilibre vers les zones radiales de son espace des phases. Cette
		instabilité ne saurait donc se déclencher sans la présence du
		germe constitué par cet équilibre.

		\item La densité du bain, et non pas sa température, semble être
		l'un des paramètres essentiels du déclenchement de cette
		instabilité dans ce contexte. 
	\end{itemize}

	\item Dans un contexte sans accrétion, nous avons pu mettre en
	évidence un effondrement du cœur de l'une de nos familles de SAG. Les
	aspects morphologiques de cet effondrement sont tout à fait comparables
	à ceux observés pour les amas globulaires: augmentation du contraste
	de densité, augmentation de la pente du halo, passage progressif à un
	profil de type cupside. Le résultat final de ce processus est tout à
	fait comparable au profil de type de Vaucouleurs généralisé ou NFW
	généralement observé pour les structures autogravitantes à
	l'échelle galactique dans les simulations de grandes structures.
	L'échelle de temps sur laquelle se produit cette évolution semble
	cependant bien inférieure à celle nécessaire pour observer des
	effets de relaxation à deux corps: elle ne représente au maximum que
	10\% du temps généralement accepté ($T_{rel}\propto\frac{0,1N}{\ln
	N}T_{d}$) pour ce genre de processus. Les études préliminaires que
	nous avons pu mener semblent toutefois montrer que cet effondrement ne soit
	pas causé par la présence du bain thermique. Dans l'état actuel de
	la situation nous ne sommes donc pas en mesure d'affirmer que nous avons
	reproduit numériquement l'instabilité \textsc{ch4}.
\end{enumerate}



Parallèlement à cette étude dynamique, nous avons conduit une
comparaison entre un code résolvant l'équation de Vlasov pour un
système sphérique et le treecode \textsc{Gadget-2}. Nous avons montré que
l'accord entre ces deux codes était étonnamment bon, même à
faible rapport du viriel. 


\section{Perspectives}


Nous aimerions pouvoir étendre le domaine temporel de l'étude des
simulations déclenchant l'instabilité d'orbite radiale. Cette
extension permettrait d'une part de raffiner l'étude de l'influence de la
densité du bain sur le déclenchement de cette instabilité, en
fournissant par exemple un critère précis et utilisable dans un
contexte éventuellement cosmologique. Dans le même temps, il s'agirait
d'autre part de confirmer définitivement le fait que le déclenchement
de l'instabilité d'orbite radiale dans ce contexte ne dépend pas de la
température du bain.



À ce stade, une étude plus avancée de la simulation $A_{6,1}$
conduisant à l'évolution accélérée du profil de
densité vers un effondrement est nécessaire.  Cette étude pourrait
être effectuée en modifiant d'une part les caractéristiques
\og{}numériques\fg du code \textsc{Gadget-2} (angle d'ouverture, gestion du pas de temps,
paramètre d'adoucissement du potentiel), mais aussi en construisant de
nouvelles observables mieux adaptées à la surveillance fine de la
dynamique de ce système (calculs raffinés de temps
caractéristiques, calcul de la densité par résolution de
l'équation de Poisson, etc.).


Finalement, les principales pistes envisageables pour tenter de
mettre en évidence numériquement l'instabilité \textsc{ch4}  est
l'augmentation soit de la température du bain, soit de sa densité dans nos simulations. Cette
augmentation pourra être effectuée dans le cadre particulaire
proposé ici, elle sera alors très couteuse; elle pourra également
être envisagée dans le cadre d'une méthode numérique
adaptée, elle sera alors moins couteuse mais nécessitera le
développement d'un code spécifique.
% Finalement, la principale piste envisageable pour tenter réellement de
% mettre en évidence numériquement l'instabilité \textsc{ch4}  est
% l'augmentation de la température du bain dans nos expériences.  Cette
% augmentation pourra être effectuée dans le cadre particulaire
% proposé ici, elle sera alors très couteuse; elle pourra également
% être envisagée dans le cadre d'une méthode numérique
% adaptée, elle sera alors moins couteuse mais nécessitera le
% développement d'un code spécifique.



			% -> toy model == généralisation sphère iso pour tous sag

			% -> simulation: \\
				% \_ résultat sur roi \\
				% \_ rèsultat étrange relaxation \\
				% \_ échec de montrer généralisation sphère iso.

			% -> perspective:
				% \_ étude plus appronfondie es paramètres de la simu relaxation pour comprendre pk evolution si rapide \\
				% \_ Quelques simulation en plus pour bien mettre en avant le paramètrage de roi. \\
				% \_ Chercher de nouvelle façon de faire un bain \& co pour trouver notre effet.

	\bibliographystyle{plainnat}
	\bibliography{Bibliographie}
	\addstarredchapter{Bibliographie}

	\appendix
	\part*{Article}
	\label{Part::Article}
	\includepdf[pages=-]{Ref/notes_vlasolve2.pdf}
	% \include{}

	% \part{Annexe}
	% \chapter{Algorithme : Tree Code}
			% \label{Chap::Algo::OctTree}
			% \minitoc
			% \section{Principe}
				% \subsection{Présentation du problème}
	Les problèmes à N corps sont des problèmes physiques extrêmement complexes, mathématiquement: ils font partie d'une
	famille d'équation du second ordre qui n'ont pas de solution pouvant s'exprimer avec les fonctions mathématiques
	usuelles. Le seul cas connu pouvant s'écrire \og simplement\fg est le problème à 2 corps.

	Numériquement, ces problèmes sont très simple à implémenter. De façon basique, bête et méchante:
	\lstset{language=C, label=algo::NBodySimple, frame=shadowbox}
	\begin{lstlisting}
typedef struct {
    double x, y, z, vx, vy, vz, ax, ay, az, m;
} Point;

...

int main(void)
{
    int NbPart = 1e6;
    Point part[NbPart];
    ...
    //Some initialisation
    ...
    while(t < tmax && nbite < itemax)
    {
        for(int i=0; i<NbPart; i++)
        {
            part[i].ax  = part[i].ay  = part[i].az = 0;

            for(int j=0; j<NbPart; j++)
            {
                if( i != j )
                {
                    part[i].ax += -G * part[i].m * part[j].m
			* part[i].x / ( pow(sqrt( (part[i].x
			- part[j].x)*(part[i].x - part[j].x)
			+ (part[i].y - part[j].y)*(part[i].y
			- part[j].y) + (part[i].z - part[j].z)
			*(part[i].z - part[j].z)), 3.0) );
                    part[i].ay += -G * part[i].m * part[j].m
			* part[i].y / ( pow(sqrt( (part[i].x
			- part[j].x)*(part[i].x - part[j].x)
			+ (part[i].y - part[j].y)*(part[i].y
			- part[j].y) + (part[i].z - part[j].z)
			*(part[i].z - part[j].z)), 3.0) );
                    part[i].az += -G * part[i].m * part[j].m
			* part[i].z / ( pow(sqrt( (part[i].x
			- part[j].x)*(part[i].x - part[j].x)
			+ (part[i].y - part[j].y)*(part[i].y
			- part[j].y) + (part[i].z - part[j].z)
			*(part[i].z - part[j].z)), 3.0) );
                }
            }

            part[i].vx += part[i].ax * dt;
            part[i].vy += part[i].ay * dt;
            part[i].vz += part[i].az * dt;

            part[i].x += part[i].vx * dt;
            part[i].y += part[i].vy * dt;
            part[i].z += part[i].vz * dt;
        }
    }
}
	\end{lstlisting}
	Comme vous pouvez le voir, cette algorithme parcours 2 fois le tableau de particule. Par conséquent, le
programme effectue $\mathsf{NbPart}^2$ opérations. Quand le nombre de particules commence à augmenter, le temps de
calcul augmente: pour les ordinateurs actuels, calculer l'évolution d'un système tel un amas globulaire
($\mathsf{NbPart} \thickapprox 1e5 \to 1e6$) sur un temps dynamique (environ $10^6 \mathrm{ans}$) avec un pas de temps
$\mathsf{dt}$ d'environ $1e4 \mathrm{ans}$ prendrait plusieurs mois, voir années. Même en parallélisant le code  et en
le faisant tourner sur un super calculateur le temps de calcul resterait important. Nous avons donc besoin de trouver un
algorithme qui permette de faire ces calculs plus rapidement, mais sans perdre trop de précision sur les calculs. Cet
algorithme, c'est le Tree-Code développé par Barnes \& Hut dans~\cite{1986Natur.324..446B}.

\subsection{Présentation de l'algorithme}
	L'algorithme de Barnes et Hut est relativement simple: au lieu de travailler sur l'espace complet, nous le
séparons en $8$ cube, puis nous séparons à nouveau chacun de ces cubes en $8$ sous cubes, et ainsi de suite jusqu'à ce
que les cubes aient un certain nombre de particules à l'intérieur ou que l'arbre de cubes ainsi créé ait atteint un
certain niveau de raffinement.

	\begin{figure}[h]
		\begin{center}
			\newcommand{\TCarre}[2]{\coordinate (Xside) at (#1, 0);
	\coordinate (Yside) at (0 , #1);

	\coordinate (A) at ($ #2 -1/2*(Xside) - 1/2*(Yside) $);
	\coordinate (B) at ($ (A) + (Xside) $);
	\coordinate (C) at ($ (B) + (Yside) $);
	\coordinate (D) at ($ (C) - (Xside) $);

	\draw (A) -- (B) -- (C) -- (D) -- (A) -- cycle;
}
\begin{tikzpicture}
%	\coordinate (Xside) at (10, 0);
%	\coordinate (Yside) at (0 , 10);
%
%	\coordinate (A) at ($ (0,0) -1/2*(Xside) - 1/2*(Yside) $);
%	\coordinate (B) at ($ (A) + (Xside) $);
%	\coordinate (C) at ($ (B) + (Yside) $);
%	\coordinate (D) at ($ (C) - (Xside) $);
%
%	\draw (A) -- (B) -- (C) -- (D) -- (A) -- cycle;

	\coordinate (P1) at (1.5,2);
	\coordinate (P2) at ($ 5*(rand, rand) - 5*(1,1) $);
	\coordinate (P3) at ($ 5*(rand,rand) - 5*(1,1) $);

	\TCarre{10}{(0,0)}

\end{tikzpicture}

		\end{center}
	\end{figure}

			% \section{Application : Calcul du potentiel}
				% \input{simulation/potentiel.tex}
			% \section{Application : Calcul du centre de densité}
				% \input{simulation/densitycenter.tex}
			% \FloatBarrier

		% \chapter{Diverses Démonstrations}
			% \minitoc
			% \section[Théorème du Viriel]{Théorème du Viriel dans une sphère\label{Demo::Viriel}}
	Dans cette partie, la notation $i=\left\{1, 2, 3\right\}$ équivaut, respectivement, à $x, y, z$. Le tenseur de pression
	est noté $\mathbb{P}$, la pression scalaire $P$ et l'impulsion (scalaire) $p$
	\subsection{Tenseur de pression}
		\newcommand{\fd}{\ensuremath{f\left(\vec{x}, \vec{p}\right)}}
		\newcommand{\Pres}{\ensuremath{\mathbb{P}}}
		Le tenseur de pression est défini comme:
		\begin{align}
			\Pres_{ij} &= \dfrac{1}{m^2}\left[\int p_i p_j f\left(\vec{x}, \vec{p}\right)\vdp - \int p_i \fd
			\vdp\int p_j \fd\vdp\right] \\
			\intertext{Dans le cas général, nous pouvons écrire:}
			p^2 &= p_1^2 + p_2^2 + p_3^2 \notag \\
			\intertext{En, moyenne, et pour un objet isotrope:}
			\langle p_1^2 \rangle &= \langle p_2^2 \rangle = \langle p_3^2 \rangle = \dfrac{1}{3}\langle p^2 \rangle \notag \\
			\intertext{Ainsi, nous pouvons écrire:}
			\Pres\(\vec{r}\) &= \dfrac{1}{3} \mathrm{Tr}\(\Pres_{ij}\)
		\end{align}
	\subsection{Tenseur énergie cinétique}
		\newcommand{\Cine}{\ensuremath{\mathbb{K}}}
		Le tenseur énergie cinétique est défini comme:
		\begin{align}
			\Cine_{ij} &= \dfrac{1}{2m}\int p_i p_j \fd \vdp \vdr = \mathrm{Tr}\(K_{ij}\) \\
			\intertext{En utilisant le tenseur de Pression, nous pouvons le réécrire sous la forme:}
			\Cine &= \dfrac{3}{2}\int \Pres\(\vec{r}\)\vdr \\
			\intertext{Les problèmes auxquelles nous nous intéressons ont une symétrie sphérique. Ainsi:}
			K &= \dfrac{3}{2}\int \Pres\(r\) 4 \pi r^2 \dr \notag \\
			  &= 6\pi\int\Pres\(r\)r^2\dr \\
			\intertext{En combinant alors l'équation d'Euler~\footnote{$\vec{\nabla}P = \rho \vec{g}$} et le
			théorème de Gauss~\footnote{$\int\int\vec{g}\mathrm{d}\vec{S} = -4\pi G M_{\mathrm{int}\(r\)}$,
			avec $M_{\mathrm{int}\(r\)}$ la masse à l'intérieur du rayon $r$.}:}
			\deriv{P}{r} &= -\dfrac{GM(r)\rho(r)}{r^2} \notag \\
			\intertext{Que nous substituons dans le tenseur énergie cinétique, puis en effectuant une
			intégration par partie:}
			K &= 2\pi\(P\(r\)r^3\right]_{r=0}^{r=R} - 2\pi\int\deriv{P}{r}r^3\dr \notag \\
			  &= 2\pi R^3 P\(R\) + 2\pi\int GM\(r\)\rho\(r\)r\dr \label{annexe::eq::Kine}
		\end{align}
	\subsubsection{Tenseur énergie potentielle\label{ssub:Tenseur énergie potentielle}}
		\newcommand{\Pot}{\ensuremath{\mathbb{U}}}
		Le tenseur d'énergie potentielle est défini comme:
		\begin{align}
			\Pot_{ij} &= -\int r_j\pderiv{\psi}{r_j}\rho\(\vec{r}\)\vdr \\
			\intertext{Soit, en géométrie sphérique:}
			\Pot &= -4\pi\int r\pderiv{\psi}{r}\rho\(r\)r^2\dr \\
			\intertext{qui, combiné une fois encore avec l'équation d'Euler, fait apparaître la pression:}
			\pderiv{\psi}{r} &= \dfrac{GM\(r\)}{r^2} \Rightarrow U = -4\pi G\int rM\(r\)\rho\(r\)\dr \label{annexe::eq::Pot}
		\end{align}
	\subsubsection{Théorème du Viriel}
		En combinant les équations~\ref{annexe::eq::Kine} et~\ref{annexe::eq::Pot}, nous obtenons le théorème du
		Viriel dans une sphère:
		\begin{align}
			2K + U = 4\pi R^3 P\(R\)
		\end{align}

% \section{Génération de la fonction de \ref{Simu::Idee4}}
	% La fonction de distribution que nous souhaitons générer est:
	% \begin{align}
		% f(E, L^2) = \rho_0 \dfrac{1}{\(2\pi\sigma_0^2\)^{-3/2}} e^{-\frac{u^2+j^2/r^2}{2\sigma_0^2}}
	% \end{align}
	% avec $\rho_0 = \frac{3M}{4\pi R_0^3}$, $j$ le moment angulaire et $u$ la vitesse radiale.


		% \chapter{Graphiques}
			% \minitoc
			% \section{Exemple de graphique pour la section~\ref{pente-coeff_sec}}
	\begin{figure}[h]
		\centering \includegraphics[scale=0.40]{graphe/ci-pente_1.png}
		\caption{Évolution des pentes pour différentes conditions initiales}
		\label{ci-pente_1}
	\end{figure}

	\begin{figure}[h]
		\centering \includegraphics[scale=0.40]{graphe/w_0-5_10.png}
		\caption{Densités pour $W_0 = 5$ et $W_0 = 10$}
		\label{w_0-5_10}
	\end{figure}
	\FloatBarrier

%\section{Tables de données de la section~\ref{pente-coeff_sec}}
%	\markboth{\MakeUppercase{\sectionname}\ \thesection.\ Tables de Données}{}
%	\begin{table}[h]
%		\begin{minipage}[b]{0.30\linewidth}
			%\centering \lstinputlisting{../Res/coeff2.res}
			%\begin{center}
			%	\lstinputlisting{../Res/coeff2.res}
			%\end{center}
			%\caption{Données des graphiques~\ref{coeff_evo} et~\ref{coeur_evo}}
%			\input{table/coeff2.tex}
%		\end{minipage}\hfill
%		\begin{minipage}[b]{0.40\linewidth}
			%\centering \lstinputlisting{../Res/coeff3.res}
			%\begin{center}
			%	\lstinputlisting{../Res/coeff3.res}
			%\end{center}
			%\caption{Données des graphiques~\ref{coeff_evo} et~\ref{coeur_evo2}}
%			\input{table/coeff3.tex}
%		\end{minipage}
%	\end{table}
%	\FloatBarrier
%\newpage
\section{Densité de NGC 5024 et NGC 5139\label{Graphe-bofbof}}
	\begin{figure}[h]
		\centering \includegraphics[scale=1.00]{graphe/NGC5024.pdf}
		\caption{Densité de NGC 5024}
	\end{figure}

	\begin{figure}[h]
		\centering \includegraphics[scale=1.00]{graphe/NGC5139.pdf}
		\caption{Densité de NGC 5139}
	\end{figure}
	\FloatBarrier

%\section{Tables pour la section~\ref{amas}}
%	\markboth{\MakeUppercase{\sectionname}\ \thesection.\ Tables de Données}{}
%	\begin{landscape}
%		\setlongtables
%		\input{../Amas/table-pente.tex}
%	\end{landscape}
%	\input{table/pente-Tc.tex}
%	\input{table/pente-dim.tex}

%	\begin{landscape}
%		\setlongtables
%		\input{../Catalogue_Harris/Cata.tex}
%		\caption{Données utilisées du catalogue de Harris\label{Harris-cat}}
%	\end{landscape}

\section{Résultat de simulation}
\subsection{Profils de densité}
\begin{figure}[h!]
	\centering \includegraphics[scale=0.5]{graphe/Comp_dens_gene-theo_0-019.pdf}
	\caption{Comparaison entre la densité numérique et la densité après évolution : $\epsilon = 0.0194028$\label{soft::0.019}}
\end{figure}


\begin{figure}[h!]
	\centering \includegraphics[scale=0.5]{graphe/Comp_dens_gene-theo_0-15.pdf}
	\caption{Comparaison entre la densité numérique et la densité après évolution : $\epsilon = 0.15$\label{soft::0.15}}
\end{figure}


\begin{figure}[h!]
	\centering \includegraphics[scale=0.5]{graphe/Comp_dens_gene-theo_0-2.pdf}
	\caption{Comparaison entre la densité numérique et la densité après évolution : $\epsilon = 0.2$\label{soft::0.2}}
\end{figure}


\begin{figure}[h!]
	\centering \includegraphics[scale=0.5]{graphe/Comp_dens_gene-theo_0-3.pdf}
	\caption{Comparaison entre la densité numérique et la densité après évolution : $\epsilon = 0.3$\label{soft::0.3}}
\end{figure}

	\FloatBarrier

\subsection{Évolution des axes d'inertie}
\begin{figure}[h!]
	\centering \includegraphics[scale=0.5]{graphe/Axial_ratio_0-019.pdf}
	\caption{Évolution des rapports des axes d'inertie : $\epsilon = 0.0194028$\label{soft::0.019-Ax}}
\end{figure}


\begin{figure}[h!]
	\centering \includegraphics[scale=0.5]{graphe/Axial_ratio_0-15.pdf}
	\caption{Évolution des rapports des axes d'inertie : $\epsilon = 0.15$\label{soft::0.15-Ax}}
\end{figure}


\begin{figure}[h!]
	\centering \includegraphics[scale=0.5]{graphe/Axial_ratio_0-2.pdf}
	\caption{Évolution des rapports des axes d'inertie : $\epsilon = 0.2$\label{soft::0.2-Ax}}
\end{figure}


\begin{figure}[h!]
	\centering \includegraphics[scale=0.5]{graphe/Axial_ratio_0-3.pdf}
	\caption{Évolution des rapports des axes d'inertie : $\epsilon = 0.3$\label{soft::0.3-Ax}}
\end{figure}



\end{document}
