	Il est utile de décrire quelques temps caractéristiques de l'évolution de ces systèmes.
	\subsection{Temps dynamique}
		Le temps dynamique est le temps nécessaire à une particule pour traverser le système et donc se \og mettre au courant \fg~des changements ayant lieu.
		Il s'agit donc de la plus petite échelle de temps que nous puissions considérer pour l'évolution d'un système auto-gravitant.
		Pour l'obtenir, considérons un système de $N$ particules de même masse $m$, et de dispersion de vitesse $\sigma$. Les énergies potentielles et cinétiques du système
		peuvent être évaluées par :
		\begin{align}
			E_c = \frac{N}{2}m\sigma^2\qquad\mathrm{et}\qquad E_p = -\frac{G(Nm)^2}{R}
		\end{align}
		puis en supposant que le système à l'équilibre et en appliquant le théorème du \textsc{Viriel}~\footnote{Rappel : $2 E_c + U = 0$ avec $E_c$
		l'énergie cinétique et $U$ l'énergie potentielle}, nous obtenons :
		\begin{align}
			\sigma = \(\frac{GNm}{R}\)^{1/2}
		\end{align}
		Le temps dynamique est proportionnel au temps de croisement des particules du système. Ce dernier dépend du rayon de l'objet
		considéré et de la vitesse caractéristique des étoiles dans le système, ou encore, de la dispersion de vitesse du système :
		\begin{align}
			T_{cr} \propto T_d = \frac{R}{\sigma} \label{Td:sig}
		\end{align}
		En injectant notre expression pour la dispersion de vitesse, nous avons le temps dynamique :
		\begin{align}
			T_d = R\(\frac{R}{GNm}\)^{1/2} = \sqrt{\frac{R^3}{GM}} \propto \frac{1}{\sqrt{G\rho}}
			\label{Td:rho}
		\end{align}
		avec $M = Nm$ la masse totale du système.
	\subsection[Temps de relaxation]{Temps de collision ou temps de relaxation à 2 corps}
		Le temps de collision est par définition le temps mis par les interactions à 2 corps pour modifier d'un ordre de grandeur la vitesse des particules du système.
		Le détail pour le calcul de ce temps peut-être trouvé dans~\cite{ThNico} et~\cite{CoursJP}. Il s'écrit, en fonction du temps de croisement :
		\begin{align}
			T_{c} = \frac{3N}{8\ln\(\frac{3N}{4\pi}\)}T_{cr}
		\end{align}

		Ce temps est donc proportionnel au temps dynamique $T_d$ par l'intermédiaire du temps de croisement.


