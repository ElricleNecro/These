Décrire ce qui a été fait, ce qui est à faire, comment tout à été fait, ...

En gros, faire un état des lieux en utilisant la bibliographie.
\normalsize

\chapter{Propriétés générales des systèmes auto-gravitants}

	J'ai mis ce chapitre dans l'intro, il permet de décrire ce que l'on sait des
	amas globulaires. Il faut éviter toute référence précise au modèle de King.
	L'idée est de montrer l'évolution de la pente avec l'âge, et les deux
	catégories d'amas avec ou sans c\oe ur. Il faudrait rajouter une section sur
	les galaxies, voire les amas de galaxies en utilisant l'article de revue de
	Merritt.

	\minitoc
	\section{Amas Globulaire}
		\subsection{Historique}
			Les premiers amas d'étoiles ont été catalogués par \textsc{Messier} en 1784,
			mais ils n'ont été identifiés comme tel qu'en 1814 par William
			\textsc{Herschel}. Étudiés depuis cette époque, notre connaissance
			observationnelle à leur propos n'a pas cessé de s'améliorer avec la progression
			des techniques d'observation. Le premier comptage d'étoiles complet fut
			effectué par \textsc{Bailey} en 1893. En 1905, \textsc{Plummer} et \textsc{von
			Zeipel}	ont utilisé les observations pour remonter à la distribution radiale
			des étoiles. \textsc{Von Zeipel} fit alors le rapprochement entre un amas et
			une sphère de gaz à l'équilibre isotherme. Parallèle encore utilisé
			aujourd'hui, bien qu'il soit contestable sur au moins un point : le libre
			parcours moyen d'une étoile est grand devant la taille du système, alors que
			pour une sphère de gaz c'est l'inverse.

		\subsection{Définition d'un amas globulaire}
			Dans notre galaxie, un amas globulaire est, en général, décrit comme un
			très vieil amas d'étoiles, âgé d'environ 10 milliards d'années. Mais
			l'âge absolu de ces objets est très difficile à mesurer: c'est un sujet
			toujours controversé.

			Les amas de notre galaxie présentent des caractéristiques très variées. Par
			exemple, l'amas le plus massif de notre galaxie --~$\omega$ Centauri~--
			possède une masse d'environ $5.10^6 M_\odot$ tandis que celle du moins
			massif --~AM-4~--est d'environ $10^3 M_\odot$. La magnitude
			absolue~\footnote{De $M_V = -10.1$ pour $\omega$ Centauri à $M_V = -1.7$
			pour AM-4} de ces objets et leur distance au centre de la
			galaxie~\footnote{environ 20 kpc pour les plus proche du centre à 120 kpc
			pour AM-1} varie également dans de grandes proportions.

		\subsection{Répartition}

			Nous avons dénombré quelque(s) centaines d'amas globulaires dans
			notre galaxie, et nous continuons à en découvrir (voir par
			exemple~\cite{2014ApJ...786L...3L}) dans notre galaxie, mais aussi dans la galaxie d'Andromède.

			Les amas sont essentiellement répartis dans les régions proches du
			bulbe de la Voie Lactée ou dans son Halo.
			La figure~\ref{Fig::Intro::repartition} montre la répartition des amas
			globulaires se trouvant dans le catalogue de référence à leur propos: le
			catalogue de Harris~\cite{Harris}. Ce catalogue recense tout les paramètres
			de 150 amas de notre galaxie.
			%Selon~\cite{MH-AAR1997}, ils semblent se répartir en 2 groupes:
			%\begin{itemize}
				%\item le premier formant un halo autour de la galaxie,
				%\item le second formant plutôt un disque.
			%\end{itemize}

			\begin{figure}[h]
				\begin{minipage}{0.32\textwidth}
					\begin{center}
						\includegraphics[width=\linewidth]{plan_xOy_GC.pdf}
					\end{center}
				\end{minipage}\hfill
				\begin{minipage}{0.32\textwidth}
					\begin{center}
						\includegraphics[width=\linewidth]{plan_xOz_GC.pdf}
					\end{center}
				\end{minipage}\hfill
				\begin{minipage}{0.32\textwidth}
					\begin{center}
						\includegraphics[width=\linewidth]{plan_yOz_GC.pdf}
					\end{center}
				\end{minipage}
				\caption{\label{Fig::Intro::repartition}Répartition des amas globulaires connus dans notre galaxie. Les graphiques utilisent le système de coordonné galactique (et donc centré sur le soleil).}
				\todo[inline]{Il y a aussi la figure 4.1 du Meylan \& al, mais on y
					voit pas de structure tel qu'ils les décrivent...}
			\end{figure}


		\subsection{Profil}

			L'étude des propriétés physiques des amas globulaires s'étale aujourd'hui
			sur plus d'un siècle d'observation et de modélisation. Une caractéristique
			commune aux amas est leur forme sphérique, l'ellipticité maximale observée
			étant de l'ordre de 10\% et s'explique par une faible rotation solide. Un
			consensus est maintenant établi sur le fait que leur profil de densité
			volumique de masse (abrégé densité et noté $\rho(r)$) est un excellent
			traceur de leur évolution. Les amas globulaires se répartissent en deux
			catégorie: 80\% des amas présentent une densité caractérisée par 2 régions:
			le cœur pour lequel la densité est quasiment constante ($\rho(r) \approx
			\mathrm{cte}$) et le halo dans lequel la densité évolue globalement comme
			une loi de puissance ($\rho(r) \propto r^{-\alpha}$), ils ont un profil de
			type cœur-halo (core-halo dans la littérature anglo-saxonne). La densité du
			cœur des 20\% restant est remplacé par une loi de puissance. Ils sont dit
			cœur effondré (core-collapsed).

			La plupart des modèles d'amas globulaire sont issus de la sphère isotherme. Nous en
			aborderont certains au cours de ce document. Parmi ces modèles, un en particulier
			permet d'ajuster le profil de densité des amas globulaires durant une grande
			partie de leur évolution: le modèle de \textsc{King}
			(voir~\cite{1966AJ.....71...64K} et le chapitre~\ref{King::Chapitre}).

			\begin{figure}[h]
				\begin{center}
					\includegraphics[width=\linewidth]{gc_photo}
				\end{center}
				\begin{minipage}{0.45\textwidth}
					\begin{center}
						\includegraphics[width=\linewidth]{M13}
					\end{center}
				\end{minipage}\hfill
				\begin{minipage}{0.45\textwidth}
					\begin{center}
						\includegraphics[width=\linewidth]{M15}
					\end{center}
				\end{minipage}
				\caption{\label{Fig::Intro::images}Deux amas globulaires et leurs profils de luminosité (tiré de~\cite{2010A&A...522A..71J}). (a)M13, un amas ayant un cœur, (b) M15, un amas ayant un cœur dit effondré}
			\end{figure}

		\subsection{Simulation}
			Dès les années 1960, des simulations numériques sont utilisées en plus de la
			théorie afin de comprendre comment évoluent ces objets.
			Plusieurs approches ont été utilisées, les principales sont:
			\begin{itemize}
					\item les simulations Monte-Carlo, elles permettent de
						simuler l'évolution d'un amas de plusieurs millions
						d'étoile sur une grande période de temps en quelques
						jours, en adaptant les propriétés globales des
						orbites des étoiles;
					\item les simulations N-corps modifient à chaque pas de
						temps les positions et vitesses des étoiles. Ces simulations sont beaucoup plus lente.
			\end{itemize}
			La figure~\ref{Fig::Intro::HeggieFigure}, récupéré de la présentation de
			D.~Heggie à Gravasco, montre les amas du catalogue de
			\textsc{Harris} tracés dans le plan temps de relaxation - magnitude intégré
			en bande V. Il a été ajouté par dessus un ensemble de droite qui indique le
			temps que durerait une simulation N-Corps ou Monte-Carlo d'un amas
			globulaire avec ces paramètres. Par exemple, une simulation N-Corps de M4
			prendrais environ 300 ans tandis que son équivalent Monte-Carlo ne prendrais
			que une journée.

			\begin{figure}[h]
				\centering \includegraphics[width=\linewidth]{Heggie_figure.pdf}
				\caption{\label{Fig::Intro::HeggieFigure}...}
			\end{figure}

			Les simulations ont permit de prendre de l'avance sur les observations en
			permettant de confirmer certains phénomènes prévu par la théorie, notamment
			les oscillation gravothermal (voir~\cite{1996ApJ...471..796M}). Ce phénomène
			intervient après l'effondrement du cœur, la taille de ce dernier oscille.

		\subsection[Lien]{Lien entre données et paramètres\label{amas}}
			\subsubsection{Préliminaire}
			%	Ce qui va nous intéresser ici, c'est de pouvoir lier à chaque étape d'évolution d'un amas, et donc à son âge,
			%	une pente.
				%Notre objectif est de trouver une relation entre l'âge d'un amas et la pente de son halo.
				Nous avons parlé dans la section précédente du profil de densité
				comme d'un marqueur de l'évolution des amas globulaires, nous allons
				voir en quoi.
				S'il est possible de trouver, à partir de ces profils, plusieurs
				marqueur d'évolution, celui qui nous intéresse plus particulièrement
				est la pente du halo. Pour vérifier que cette pente évolue bien avec
				l'âge dynamique de l'objet, et comment, nous avons utilisé les
				données du catalogue de \textsc{Harris}~\cite{Harris} qui nous a
				permis d'obtenir l'âge et le profil de densité d'une centaine d'amas
				globulaire.
				
				%Nous avons donc besoin d'une relation entre cette quantité et l'âge de l'amas, si une telle relation existe.
				Nous avons alors utilisé le temps de relaxation donné dans le catalogue de \textsc{Harris}~\cite{Harris}.
				Pour obtenir les pentes des amas, nous avons utilisé les relevés observationnels~\cite{Trager-graphe}. % (~les données ainsi obtenues et utilisées sont dans la table~\ref{pente-Tc:BSP}~).
				Nous avons commencé par calculer les pentes directement sur les courbes avec un double décimètre, n'ayant alors pas pu obtenir les données correspondant aux graphiques.
				Après avoir tracé la pente mesurée en fonction du temps de relaxation à 2 corps, nous avons ajusté la
				courbe ainsi obtenue par une droite d'équation $ \alpha = \mathrm{pente} = a \log_{10}(T_c) + b$ (~graphe~\ref{Pente-lin}~).
				\begin{figure}[hbt!]
					\centering \includegraphics[scale=0.9]{graphe/Pente-Tc.pdf}
					\caption{Évolution des pentes pour différents âges}
					\label{Pente-lin}
				\end{figure}
			%	\begin{table}[hbt!]
			%		\begin{center}
			%			\begin{tabular}{|c|c|c|}
			%				\hline
			%				Coefficient & Valeur & Erreur \\
			%				\hline
			%				\hline
			%				$a$       &        $-1.19506$   &   $\pm 0.1576$ (~$13.19\%$~) \\
			%				\hline
			%				$b$       &        $2.52082$     &   $\pm 1.213$   (~$48.11\%$~) \\
			%				\hline
			%			\end{tabular}
			%		\end{center}
			%		\caption{Valeur des coefficients donnée par l'ajustement pour les pentes}
			%		\label{pente-lin-coeff}
			%	\end{table}

				Cette approche indiquant clairement une relation linéaire entre ces deux paramètres, nous avons décidé d'entreprendre une démarche plus globale et automatique.
				Pour commencer, nous avons donc récupéré les données auprès des auteurs de l'article sur~\cite{TragerTable}. Nous nous sommes alors confronté au problème des unités.

			\subsubsection{Retraitement}
				Le graphique~\ref{Pente-lin} a été obtenu en utilisant~\cite{Trager-graphe} avec ses unités. % (~les données permettant de retracer, nous les avons finalement trouvé,
			%	les graphiques de cet article sont données dans~\cite{TragerTable}~).
				La pente donnée ici n'est donc pas la pente de la densité, mais de la brillance de surface de l'objet en fonction d'un rayon en seconde d'arc.
				Le premier traitement à effectuer consiste donc à transformer cette brillance de surface par arc seconde carrée en une densité (~kilogramme par kilomètre au cube~).

				Une définition de cette brillance de surface, telle qu'elle semble avoir été utilisée, peut-être trouvé dans~\cite{SBP}.
				Comme indiqué dans ce texte, la brillance de surface va s'écrire :
				\begin{align}
					\mu_V = \mu_{\mathrm{ref}} - 2.5 \log_{10}\(\frac{f/\Omega}{f_{\mathrm{ref}}/\Omega_{\mathrm{ref}}}\)
					\label{mu_V}
				\end{align}
				avec $\Omega$ l'angle solide, exprimé en seconde d'arc au carrée, sous lequel nous voyons l'objet.
			%	\begin{align}
			%		\mu_V = m_V - 2.5 \log_{10}\(\frac{(1\mathrm{"})^2}{\Omega}\)
			%		\label{mu-astuce}
			%	\end{align}
			%	avec $m_V$ la magnitude apparente de l'objet sur une seconde d'arc au carrée.

				Cette définition rappelle celle de~\cite{Trager-graphe}, section~3.2.3 :
				\begin{align}
					\mu = -2.5 \log\(\frac{10^{-0.4 m_2} - 10^{-0.4 m_1}}{\pi \(r_2^2 - r_1^2\)}\)
					\label{Trager-eq}
				\end{align}
			%	où ils prendraient comme référence les points autour de celui considéré (~ou quelque chose comme ça~),
				$m_i$ étant la magnitude en un point de l'objet et $r_i$ le rayon en seconde d'arc pour ce point.
				Par conséquent, les unités de $\mu$ sont un flux par arc seconde carrée en échelle logarithmique :
				\begin{enumerate}
					\item $10^{-0.4 m_i}$ est proportionnel à un flux, % (~ou au rapport d'un flux par rapport à un flux de référence !?!~),
					\item $r_i$ est le rayon de l'objet au point $i$ en seconde d'arc,
					\item[$\Rightarrow$] nous avons donc bien notre flux par seconde d'arc au carrée.
				\end{enumerate}
				Pour pouvoir revenir aux quantités que nous cherchons, il va falloir calculer un peu :
				\begin{align}
					\mu = -2.5 \log\(\frac{10^{-0.4 m_2} - 10^{-0.4 m_1}}{\pi \(r_2^2 - r_1^2\)}\) &\equiv -2.5 \log\(\frac{F}{\pi \(r_2^2 - r_1^2\)}\) \text{avec $F$ le flux}\notag \\
					\intertext{par rapport à toute la documentation que j'ai trouvé, le $\log$ correspond ici à $\log_{10}$ et non à $\ln$.}
					\Rightarrow \frac{F}{\pi \(r_2^2 - r_1^2\)} &= 10^{-\mu/2.5} \label{F--mu} \\
					\intertext{Grâce à~\cite{McL}, nous avons les rapports masse luminosité :}
					\Upsilon &= \frac{M}{L} \label{M/L}\\
					\intertext{or}
					F &= \frac{L}{4\pi D^2} \label{def-F}\\
					\intertext{avec $D$ la distance soleil--amas. D'où}
					\frac{L}{4 \pi^2 \(r_2^2 - r_1^2\) D^2} &= \frac{M}{4 \pi^2 \Upsilon \(r_2^2 - r_1^2\) D^2} = 10^{-\mu/2.5} \notag \\
					\intertext{en combinant~\ref{F--mu}, \ref{M/L} et~\ref{def-F}.}
					\Rightarrow \frac{M}{4\pi \(r_2^2 - r_1^2\)} &= \pi\Upsilon D^2 10^{-\mu/2.5} \notag \\
					\intertext{Mais $r_2^2 - r_1^2 \propto r_i^2$ doit être converti en mètre :}
					%r_2^2 - r_1^2 &\propto r_i^2 \Rightarrow \(D \tan(r_i)\)^2 \varpropto \(D r_i\)^2 \notag \\
					\Rightarrow \frac{M}{4\pi \(D \tan(r_i/3600)\)^2} &= \pi\Upsilon D^2 10^{-\mu/2.5} \label{M-don}
				\end{align}
				Normalement, nous avons maintenant une masse surfacique donnée par~\ref{M-don}.
			%	mais, étonnamment, la conversion de seconde d'arc à mètre n'a fait paraître
			%	aucun facteur numérique !!! J'ai peut-être un peu trop truandé.

				Les rapports masse-luminosité peuvent être trouvés dans~\cite{McL}, mais cette article ne contient que 40 des 140 amas de notre galaxie.
			%	mais il n'y a qu'une quarantaine de rapport comparé à notre échantillon d'environ 140 amas.
				Par contre, il est aisé de remarquer que ces rapports sont
				en moyenne très peu différents de la valeur $\Upsilon = 2$ avec une valeur minimum de $1.87$ pour NGC 4147 et une valeur maximum de
				$2.66$ pour NGC 6441. Dans la suite, nous prendrons donc $\Upsilon = 2$.
			%	Nous avons alors utilisé les données du catalogue pour redimensionner le tout, mais pour passer à la densité, nous avons besoin du rapport masse
			%	sur luminosité de l'amas qui peut-être trouvé dans~\cite{McL}.
			%	Cet article ne donne qu'une quarantaine d'amas sur les 150 de notre galaxie, mais les valeurs de ces rapports étant compris entre $1.87$ pour NGC 4147
			%	(~et quelques autres~) et $2.66$ pour NGC 6441, et tournant surtout autour de $2$, nous pouvons supposer que ces rapports sont les mêmes pour chaque amas et
			%	valent $\Upsilon = \frac{M}{L} = 2$.
			%	Selon~\cite{Trager-graphe}, nous avons donc :
			%	\begin{align}
			%		\mu_V &\propto \text{Flux par $m^{-2}$} = \frac{L}{4\pi D^2} \\
			%		\intertext{or}
			%		L &= \frac{M}{\Upsilon} \notag \\
			%		\intertext{donc}
			%		\mu_V &= \frac{M}{4\pi D^2 \Upsilon} \notag \\
			%		M &= \Upsilon \mu_V 4\pi D^2
			%	\end{align}
			%	avec $D$ la distance entre nous et l'amas. Nous pouvons supposer que cette distance est la même quelque soit l'étoile considéré dans l'amas
			%	(~$D_{min} = 3000\mathrm{pc} \gg R_{amas} \approx 10\mathrm{pc}$~).

			%	Du fait des rapports masse-luminosité manquant, notre échantillon d'une quarantaine d'amas est passé à seulement 12 amas.

			\subsubsection{Résultat}
				Le calcul de pente a été refait en utilisant : les données de l'article~\cite{TragerTable}, le traitement décrit ci-dessus, et en ajustant une
				droite pour des densités comprises entre $10^{-4}$ et 0,1.
				Nous obtenons alors le graphique~\ref{Pente-lin_dim}, l'ajustement utilisant la fonction
				$f(T_c) = a \log(T_c) + b$ (~les coefficients ont été donnés dans la table~\ref{pente-lin-coeff_dim}~).
				Nous avons donc maintenant une relation linéaire entre l'âge de l'amas et la pente du halo.

				\begin{figure}[hbt!]
			%		\centering \includegraphics[scale=0.9]{../Amas/Graphe_Rapport/pente-Tc.pdf}
					\centering \includegraphics[scale=0.9]{graphe/pente-Tc.pdf}
					\caption{Évolution des pentes pour différents âges}
					\label{Pente-lin_dim}
				\end{figure}
				\begin{table}[hbt!]
					\begin{center}
						\begin{tabular}{|c|c|}%c|}
							\hline
							Coefficient & Valeur \\ %& Erreur \\
							\hline
							\hline
							$a$       &         $-2.3341$   \\ %&    $\pm 0.2075$      (~$32.34\%$~) \\
							\hline
							$b$       &         $16.913$   \\  %&    $\pm 1.638$       (~$199.5\%$~) \\
							\hline
						\end{tabular}
					\end{center}
					\caption{Valeurs des coefficients donnée par l'ajustement pour les pentes pour le temps de croisement}
					\label{pente-lin-coeff_dim}
				\end{table}

				Nous avons également utilisé le catalogue de \textsc{Harris} pour obtenir le lien entre la pente du halo et le temps dynamique
				grâce à l'équation~\ref{Td:sig}. Nous obtenons alors le graphique~\ref{Pente-Td-lin}.
				\begin{figure}[hbt!]
			%		\centering \includegraphics[scale=0.9]{../Amas/Graphe_Rapport/pente-Td.pdf}
					\centering \includegraphics[scale=0.9]{graphe/pente-Td.pdf}
					\caption{Évolution des pentes pour différents âges dynamiques}
					\label{Pente-Td-lin}
				\end{figure}
				\begin{table}[hbt!]
					\begin{center}
						\begin{tabular}{|c|c|}%c|}
							\hline
							Coefficient & Valeur \\ %& Erreur \\
							\hline
							\hline
							$a$       &        $-1.6567$   \\ %&   $\pm 0.6523$      (~$278.9\%$~) \\
							\hline
							$b$       &        $1.8927$     \\ %&   $\pm 2.825$       (~$88.57\%$~) \\
							\hline
						\end{tabular}
					\end{center}
					\caption{Valeurs des coefficients donnée par l'ajustement pour les pentes pour le temps dynamique}
					\label{pente-Td-lin-coeff}
				\end{table}

				Sur la figure~\ref{Pente-lin_dim}, nous pouvons noter la présence de 2 points avec des pentes inférieures à $-10$. Ces points correspondent aux amas NGC 5024 et NGC 5139
				(~leurs densités sont donnée en annexe, page~\pageref{Graphe-bofbof}~) pour lesquelles la détermination des pentes n'a pas été très concluante.
				\FloatBarrier

		%\subsection{Préliminaire}
%	Ce qui va nous intéresser ici, c'est de pouvoir lier à chaque étape d'évolution d'un amas, et donc à son âge,
%	une pente.
	Notre objectif est de trouver une relation entre l'âge d'un amas et la pente de son halo.
	%Nous avons donc besoin d'une relation entre cette quantité et l'âge de l'amas, si une telle relation existe.
	Nous avons alors utilisé le temps de relaxation donné dans le catalogue de \textsc{Harris}~\cite{Harris}.
	Pour obtenir les pentes des amas, nous avons utilisé les relevés observationnels~\cite{Trager-graphe}. % (~les données ainsi obtenues et utilisées sont dans la table~\ref{pente-Tc:BSP}~).
	Nous avons commencé par calculer les pentes directement sur les courbes avec un double décimètre, n'ayant alors pas pu obtenir les données correspondant aux graphiques.
	Après avoir tracé la pente mesurée en fonction du temps de relaxation à 2 corps, nous avons ajusté la
	courbe ainsi obtenue par une droite d'équation $ \alpha = \mathrm{pente} = a \log_{10}(T_c) + b$ (~graphe~\ref{Pente-lin}~).
	\begin{figure}[hbt!]
		\centering \includegraphics[scale=0.9]{graphe/Pente-Tc.pdf}
		\caption{Évolution des pentes pour différents âges}
		\label{Pente-lin}
	\end{figure}
%	\begin{table}[hbt!]
%		\begin{center}
%			\begin{tabular}{|c|c|c|}
%				\hline
%				Coefficient & Valeur & Erreur \\
%				\hline
%				\hline
%				$a$       &        $-1.19506$   &   $\pm 0.1576$ (~$13.19\%$~) \\
%				\hline
%				$b$       &        $2.52082$     &   $\pm 1.213$   (~$48.11\%$~) \\
%				\hline
%			\end{tabular}
%		\end{center}
%		\caption{Valeur des coefficients donnée par l'ajustement pour les pentes}
%		\label{pente-lin-coeff}
%	\end{table}

	Cette approche indiquant clairement une relation linéaire entre ces deux paramètres, nous avons décidé d'entreprendre une démarche plus globale et automatique.
	Pour commencer, nous avons donc récupéré les données auprès des auteurs de l'article sur~\cite{TragerTable}. Nous nous sommes alors confronté au problème des unités.

\subsection{Retraitement}
	Le graphique~\ref{Pente-lin} a été obtenu en utilisant~\cite{Trager-graphe} avec ses unités. % (~les données permettant de retracer, nous les avons finalement trouvé,
%	les graphiques de cet article sont données dans~\cite{TragerTable}~).
	La pente donnée ici n'est donc pas la pente de la densité, mais de la brillance de surface de l'objet en fonction d'un rayon en seconde d'arc.
	Le premier traitement à effectuer consiste donc à transformer cette brillance de surface par arc seconde carrée en une densité (~kilogramme par kilomètre au cube~).

	Une définition de cette brillance de surface, telle qu'elle semble avoir été utilisée, peut-être trouvé dans~\cite{SBP}.
	Comme indiqué dans ce texte, la brillance de surface va s'écrire :
	\begin{align}
		\mu_V = \mu_{\mathrm{ref}} - 2.5 \log_{10}\(\frac{f/\Omega}{f_{\mathrm{ref}}/\Omega_{\mathrm{ref}}}\)
		\label{mu_V}
	\end{align}
	avec $\Omega$ l'angle solide, exprimé en seconde d'arc au carrée, sous lequel nous voyons l'objet.
%	\begin{align}
%		\mu_V = m_V - 2.5 \log_{10}\(\frac{(1\mathrm{"})^2}{\Omega}\)
%		\label{mu-astuce}
%	\end{align}
%	avec $m_V$ la magnitude apparente de l'objet sur une seconde d'arc au carrée.

	Cette définition rappelle celle de~\cite{Trager-graphe}, section~3.2.3 :
	\begin{align}
		\mu = -2.5 \log\(\frac{10^{-0.4 m_2} - 10^{-0.4 m_1}}{\pi \(r_2^2 - r_1^2\)}\)
		\label{Trager-eq}
	\end{align}
%	où ils prendraient comme référence les points autour de celui considéré (~ou quelque chose comme ça~),
	$m_i$ étant la magnitude en un point de l'objet et $r_i$ le rayon en seconde d'arc pour ce point.
	Par conséquent, les unités de $\mu$ sont un flux par arc seconde carrée en échelle logarithmique :
	\begin{enumerate}
		\item $10^{-0.4 m_i}$ est proportionnel à un flux, % (~ou au rapport d'un flux par rapport à un flux de référence !?!~),
		\item $r_i$ est le rayon de l'objet au point $i$ en seconde d'arc,
		\item[$\Rightarrow$] nous avons donc bien notre flux par seconde d'arc au carrée.
	\end{enumerate}
	Pour pouvoir revenir aux quantités que nous cherchons, il va falloir calculer un peu :
	\begin{align}
		\mu = -2.5 \log\(\frac{10^{-0.4 m_2} - 10^{-0.4 m_1}}{\pi \(r_2^2 - r_1^2\)}\) &\equiv -2.5 \log\(\frac{F}{\pi \(r_2^2 - r_1^2\)}\) \text{avec $F$ le flux}\notag \\
		\intertext{par rapport à toute la documentation que j'ai trouvé, le $\log$ correspond ici à $\log_{10}$ et non à $\ln$.}
		\Rightarrow \frac{F}{\pi \(r_2^2 - r_1^2\)} &= 10^{-\mu/2.5} \label{F--mu} \\
		\intertext{Grâce à~\cite{McL}, nous avons les rapports masse luminosité :}
		\Upsilon &= \frac{M}{L} \label{M/L}\\
		\intertext{or}
		F &= \frac{L}{4\pi D^2} \label{def-F}\\
		\intertext{avec $D$ la distance soleil--amas. D'où}
		\frac{L}{4 \pi^2 \(r_2^2 - r_1^2\) D^2} &= \frac{M}{4 \pi^2 \Upsilon \(r_2^2 - r_1^2\) D^2} = 10^{-\mu/2.5} \notag \\
		\intertext{en combinant~\ref{F--mu}, \ref{M/L} et~\ref{def-F}.}
		\Rightarrow \frac{M}{4\pi \(r_2^2 - r_1^2\)} &= \pi\Upsilon D^2 10^{-\mu/2.5} \notag \\
		\intertext{Mais $r_2^2 - r_1^2 \propto r_i^2$ doit être converti en mètre :}
		%r_2^2 - r_1^2 &\propto r_i^2 \Rightarrow \(D \tan(r_i)\)^2 \varpropto \(D r_i\)^2 \notag \\
		\Rightarrow \frac{M}{4\pi \(D \tan(r_i/3600)\)^2} &= \pi\Upsilon D^2 10^{-\mu/2.5} \label{M-don}
	\end{align}
	Normalement, nous avons maintenant une masse surfacique donnée par~\ref{M-don}.
%	mais, étonnamment, la conversion de seconde d'arc à mètre n'a fait paraître
%	aucun facteur numérique !!! J'ai peut-être un peu trop truandé.

	Les rapports masse-luminosité peuvent être trouvés dans~\cite{McL}, mais cette article ne contient que 40 des 140 amas de notre galaxie.
%	mais il n'y a qu'une quarantaine de rapport comparé à notre échantillon d'environ 140 amas.
	Par contre, il est aisé de remarquer que ces rapports sont
	en moyenne très peu différents de la valeur $\Upsilon = 2$ avec une valeur minimum de $1.87$ pour NGC 4147 et une valeur maximum de
	$2.66$ pour NGC 6441. Dans la suite, nous prendrons donc $\Upsilon = 2$.
%	Nous avons alors utilisé les données du catalogue pour redimensionner le tout, mais pour passer à la densité, nous avons besoin du rapport masse
%	sur luminosité de l'amas qui peut-être trouvé dans~\cite{McL}.
%	Cet article ne donne qu'une quarantaine d'amas sur les 150 de notre galaxie, mais les valeurs de ces rapports étant compris entre $1.87$ pour NGC 4147
%	(~et quelques autres~) et $2.66$ pour NGC 6441, et tournant surtout autour de $2$, nous pouvons supposer que ces rapports sont les mêmes pour chaque amas et
%	valent $\Upsilon = \frac{M}{L} = 2$.
%	Selon~\cite{Trager-graphe}, nous avons donc :
%	\begin{align}
%		\mu_V &\propto \text{Flux par $m^{-2}$} = \frac{L}{4\pi D^2} \\
%		\intertext{or}
%		L &= \frac{M}{\Upsilon} \notag \\
%		\intertext{donc}
%		\mu_V &= \frac{M}{4\pi D^2 \Upsilon} \notag \\
%		M &= \Upsilon \mu_V 4\pi D^2
%	\end{align}
%	avec $D$ la distance entre nous et l'amas. Nous pouvons supposer que cette distance est la même quelque soit l'étoile considéré dans l'amas
%	(~$D_{min} = 3000\mathrm{pc} \gg R_{amas} \approx 10\mathrm{pc}$~).

%	Du fait des rapports masse-luminosité manquant, notre échantillon d'une quarantaine d'amas est passé à seulement 12 amas.

\subsection{Résultat}
	Le calcul de pente a été refait en utilisant : les données de l'article~\cite{TragerTable}, le traitement décrit ci-dessus, et le critère donné dans la section~\ref{pente-critére}.
	Nous obtenons alors le graphique~\ref{Pente-lin_dim}, l'ajustement utilisant la fonction
	$f(T_c) = a \log(T_c) + b$ (~les coefficients ont été donnés dans la table~\ref{pente-lin-coeff_dim}~).
	Nous avons donc maintenant une relation linéaire entre l'âge de l'amas et la pente du halo.

	\begin{figure}[hbt!]
%		\centering \includegraphics[scale=0.9]{../Amas/Graphe_Rapport/pente-Tc.pdf}
		\centering \includegraphics[scale=0.9]{graphe/pente-Tc.pdf}
		\caption{Évolution des pentes pour différents âges}
		\label{Pente-lin_dim}
	\end{figure}
	\begin{table}[hbt!]
		\begin{center}
			\begin{tabular}{|c|c|}%c|}
				\hline
				Coefficient & Valeur \\ %& Erreur \\
				\hline
				\hline
				$a$       &         $-2.3341$   \\ %&    $\pm 0.2075$      (~$32.34\%$~) \\
				\hline
				$b$       &         $16.913$   \\  %&    $\pm 1.638$       (~$199.5\%$~) \\
				\hline
			\end{tabular}
		\end{center}
		\caption{Valeurs des coefficients donnée par l'ajustement pour les pentes pour le temps de croisement}
		\label{pente-lin-coeff_dim}
	\end{table}

	Nous avons également utilisé le catalogue de \textsc{Harris} pour obtenir le lien entre la pente du halo et le temps dynamique
	grâce à l'équation~\ref{Td:sig}. Nous obtenons alors le graphique~\ref{Pente-Td-lin}.
	\begin{figure}[hbt!]
%		\centering \includegraphics[scale=0.9]{../Amas/Graphe_Rapport/pente-Td.pdf}
		\centering \includegraphics[scale=0.9]{graphe/pente-Td.pdf}
		\caption{Évolution des pentes pour différents âges dynamiques}
		\label{Pente-Td-lin}
	\end{figure}
	\begin{table}[hbt!]
		\begin{center}
			\begin{tabular}{|c|c|}%c|}
				\hline
				Coefficient & Valeur \\ %& Erreur \\
				\hline
				\hline
				$a$       &        $-1.6567$   \\ %&   $\pm 0.6523$      (~$278.9\%$~) \\
				\hline
				$b$       &        $1.8927$     \\ %&   $\pm 2.825$       (~$88.57\%$~) \\
				\hline
			\end{tabular}
		\end{center}
		\caption{Valeurs des coefficients donnée par l'ajustement pour les pentes pour le temps dynamique}
		\label{pente-Td-lin-coeff}
	\end{table}

	Sur la figure~\ref{Pente-lin_dim}, nous pouvons noter la présence de 2 points avec des pentes inférieures à $-10$. Ces points correspondent aux amas NGC 5024 et NGC 5139
	(~leurs densités sont donnée en annexe, page~\pageref{Graphe-bofbof}~) pour lesquelles la détermination des pentes n'a pas été très concluante.
	\FloatBarrier



	%\section{Interprétation}

%Nous avons maintenant une équation liant la valeur de la pente au temps de relaxation : $ \mathrm{pente} = \alpha = d * \log_{10}(T_c) + e $ (~coefficients table~\ref{pente-lin-coeff_dim}~)
%et une autre liant la pente à $W_0$ : $ \alpha = a e^{ b W_0 } + c $.
%En combinant ces deux équations, nous pouvons obtenir une relation entre le temps de
%relaxation et $W_0$ :
%\begin{align}
	%\mathrm{pente} &= d \log_{10}(T_c) + e = a e^{b W_0} + c \notag \\
	%\Rightarrow W_0 &= \frac{1}{b} \ln\( \frac{d \log_{10}(T_c) + e - c}{a} \) \label{Tc:W0->fct}
%\end{align}
%avec :
%\begin{table}[h!]
	%\begin{center}
		%\begin{tabular}{|c|r|}
			%\hline
			%Coefficient	&	Valeur \\
			%\hline
			%\hline
			%$a$		&	$ -10.0698 $ \\
				%\hline
			%$b$		&	$ 0.220152 $ \\
			%\hline
			%$c$		&	$ -1.53409 $ \\
			%\hline
			%$d$		&	$ -2.3341 $ \\
			%\hline
			%$e$		&	$ 16.913 $ \\
			%\hline
		%\end{tabular}
	%\end{center}
%\end{table}

%Le comportement que nous avions observé en étudiant le modèle de \textsc{King}, à savoir une évolution de la pente avec $W_0$, se retrouve avec l'âge de notre amas. Ce qui nous a permis de relier
%l'âge et $W_0$.
%Il nous reste à interpreter ces résultats, ce que nous allons faire dans la section suivante.
		%Nous avons maintenant une équation liant la valeur de la pente au temps de relaxation : $ \mathrm{pente} = \alpha = d * \log_{10}(T_c) + e $ (~coefficients table~\ref{pente-lin-coeff_dim}~)
et une autre liant la pente à $W_0$ : $ \alpha = a e^{ b W_0 } + c $.
En combinant ces deux équations, nous pouvons obtenir une relation entre le temps de
relaxation et $W_0$ :
\begin{align}
	\mathrm{pente} &= d \log_{10}(T_c) + e = a e^{b W_0} + c \notag \\
	\Rightarrow W_0 &= \frac{1}{b} \ln\( \frac{d \log_{10}(T_c) + e - c}{a} \) \label{Tc:W0->fct}
\end{align}
avec :
\begin{table}[h!]
	\begin{center}
		\begin{tabular}{|c|r|}
			\hline
			Coefficient	&	Valeur \\
			\hline
			\hline
			$a$		&	$ -10.0698 $ \\
				\hline
			$b$		&	$ 0.220152 $ \\
			\hline
			$c$		&	$ -1.53409 $ \\
			\hline
			$d$		&	$ -2.3341 $ \\
			\hline
			$e$		&	$ 16.913 $ \\
			\hline
		\end{tabular}
	\end{center}
\end{table}

Le comportement que nous avions observé en étudiant le modèle de \textsc{King}, à savoir une évolution de la pente avec $W_0$, se retrouve avec l'âge de notre amas. Ce qui nous a permis de relier
l'âge et $W_0$.
Il nous reste à interpreter ces résultats, ce que nous allons faire dans la section suivante.


	\section{Petit Scénario\label{petit_scenar}}

%Nous avons vu que les courbes~\ref{King_Modele-test} peuvent être divisées en deux parties : l'une, caractérisée par une densité constante, constituant un cœur, et l'autre constituant un halo.
%Nous allons donc considérer un amas d'étoile composé d'un cœur dense et d'un halo.
%Le halo a peu de gravité et se comporte comme un gaz parfait.
%Le comportement observé dans notre étude est résumé sur le schéma~\ref{schema-effondrement} :
%\input{figure_tikz-pente.tex}
%(~la pente $-4$ avec laquelle commence le schéma vient des résultats de simulations numériques~).
%La question à laquelle nous souhaitons répondre est : comment expliquer qu'un amas évolu en augmentant la densité de son cœur et en diminuant la pente avec laquelle le halo décroit ?

%Pour commencer, un amas doit, lorsqu'il est à l'équilibre, suivre un modèle de \textsc{King} ou de sphère isotherme en boîte. Par conséquent, notre amas est au Viriel.
%De plus, le cœur est un systéme auto-gravitant décrit par la thermodynamique. L'un de ses propriétés thermodynamique les plus importante à ce niveau est sa capacité calorifique. En effet :
%Nous cherchons à concentrer la matière dans le cœur de l'amas, et donc y rajouter de la matière à partir du halo (~qui va se diluer~).
%Pour faire diminuer le rayon de l'orbite d'un astre, il faut augmenter sa vitesse~\footnote{rappel : $v^2 = K\left(\frac{2}{r} - \frac{1}{a}\right)$ avec $K = G\left(M+m\right)$
%pour des interactions à deux corps}, et donc sa température. En se rappelant les résultats sur les diagrammes d'énergie de la sphère isotherme, nous nous rendons bien compte que, si la température
%augmente trop, le cœur va passer dans la zone instable, et s'effondrer.

%Le processus pour faire augmenter la température auquel nous nous attèlerons dans la suite est assez simple : l'amas évolue dans le potentiel galactique ; il va en traverser le disque de façon périodique.
%Moments pendant lesquels il va se faire harceler par des forces de marée qui vont lui faire perdre des étoiles. En considérant la description micro canonique, l'énergie est fixé ; or perdre une étoile
%diminue l'énergie potentielle de l'amas. La température va devoir augmenter pour conserver l'énergie totale constante.

%La perte d'étoile apparaît alors comme un processus intéressant pour expliquer l'évolution observée. Les chapitres suivants nous permettront de déterminer, à l'aide de simulation numérique, si la perte
%d'étoile par effet de marée est suffisante pour l'expliquer.








		Cette dernière étude nous montre que plus un amas est âgé, plus la pente de son halo est importante.
		Dans le chapitre précédent, nous avons relié la pente au paramètre $W_0$ qui, en plus de jouer sur les pentes, joue sur la densité centrale, comme le montre la figure~\ref{King_Modele-test}.
		Par ailleurs, des simulations partant d'un nuage homogène nous apprennent que la pente du halo après sa formation vaut $-4$.
		Un amas va donc partir d'une structure cœur-halo avec un cœur de taille importante, un halo de pente $-4$ et va tendre, en vieillissant, vers un amas ayant une densité centrale très élevée dont la pente tend vers $-2$.
		Puis après un temps infini, l'amas va tendre vers une sphère isotherme.
		C'est ce que représente le schéma suivant.

		\input{theorie/figure_tikz-pente.tex}

		Pour expliquer cette évolution, nous devons trouver quels phénomènes, de dynamique gravitationnelle, pourraient diluer le halo et concentrer de la matière au centre de l'amas. Le phénomène qui vient le plus naturellement à l'esprit est la perte d'étoiles, pouvant s'effectuer par 2 scénarios :
		\begin{enumerate}
			\item des collisions internes qui éjectent des étoiles de l'amas, permettant ainsi au cœur de s'effondrer en essayant de compenser cette perte d'énergie.
			\item des interactions avec un autre objet massif qui va retirer par effet de marée des étoiles à l'amas, causant l'effondrement de son cœur.
		\end{enumerate}
		C'est ce deuxième scénario que nous allons tenter d'étudier.
















%Si la température cinétique du halo augmente, celle du
%cœur va changer pour s'adapter. La capacité calorifique à volume
%constant est négative pour le cœur (~et pour tout système
%auto gravitant~) :
%\begin{align*}
%	E_p &= -2E_c & \text{(~Viriel~)} \\
%	\Rightarrow E &= E_p + E_c = -E_c \\
%	\intertext{or}
%	E_c &\varpropto T \Rightarrow E\varpropto -T \\
%	\intertext{donc}
%	\Rightarrow C_v &= \frac{\partial E}{\partial T} < 0
%\end{align*}
%Cela implique que notre système ne peut que prendre de l'énergie.

%Si la température du cœur augmente, la vitesse de rotation des
%étoiles va augmenter (~$E_c\varpropto T\varpropto v$~), et donc
%leur demi grand axe va diminuer~\footnote{rappel : $v^2 =
%K\left(\frac{2}{r} - \frac{1}{a}\right)$ avec $K = G\left(M+m\right)$
%pour des interactions à deux corps}.
%En conséquent la densité au centre du cœur augmente, augmentant
%ainsi le contraste de densité. Si la température augmente
%suffisamment, le contraste de densité du halo va dépasser la
%valeur critique $\R_c^H$ (~l'énergie est fixé, seul la
%température change, nous utilisons donc la description micro
%canonique~). Le cœur va alors devenir instable.


%	L'interprétation semble simple ici : par exemple : quand les étoiles
%	ont des vitesses de rotation élevées la sphère a une température élevée,
%	et donc sa capacité calorifique à volume constant, $C_v = \frac{\partial H}{\partial T} < 0$
%	(~négatif car $E_p=-2E_c\Rightarrow H=E_c+E_p=-E_c\varpropto-T$~),
%	tend vers $0$ : elle ne peut plus acquérir d'énergie.
%	Si on dépasse la limite de température, elle va devoir se \og~réorganiser~\fg pour
%	rester à l'équilibre et donc pour retourner à un contraste de densité $\R < \R^\beta_c$.
%	Le raisonnement est le même pour la limite en énergie.
		%%Nous avons vu que les courbes~\ref{King_Modele-test} peuvent être divisées en deux parties : l'une, caractérisée par une densité constante, constituant un cœur, et l'autre constituant un halo.
%Nous allons donc considérer un amas d'étoile composé d'un cœur dense et d'un halo.
%Le halo a peu de gravité et se comporte comme un gaz parfait.
%Le comportement observé dans notre étude est résumé sur le schéma~\ref{schema-effondrement} :
%\input{figure_tikz-pente.tex}
%(~la pente $-4$ avec laquelle commence le schéma vient des résultats de simulations numériques~).
%La question à laquelle nous souhaitons répondre est : comment expliquer qu'un amas évolu en augmentant la densité de son cœur et en diminuant la pente avec laquelle le halo décroit ?

%Pour commencer, un amas doit, lorsqu'il est à l'équilibre, suivre un modèle de \textsc{King} ou de sphère isotherme en boîte. Par conséquent, notre amas est au Viriel.
%De plus, le cœur est un systéme auto-gravitant décrit par la thermodynamique. L'un de ses propriétés thermodynamique les plus importante à ce niveau est sa capacité calorifique. En effet :
%Nous cherchons à concentrer la matière dans le cœur de l'amas, et donc y rajouter de la matière à partir du halo (~qui va se diluer~).
%Pour faire diminuer le rayon de l'orbite d'un astre, il faut augmenter sa vitesse~\footnote{rappel : $v^2 = K\left(\frac{2}{r} - \frac{1}{a}\right)$ avec $K = G\left(M+m\right)$
%pour des interactions à deux corps}, et donc sa température. En se rappelant les résultats sur les diagrammes d'énergie de la sphère isotherme, nous nous rendons bien compte que, si la température
%augmente trop, le cœur va passer dans la zone instable, et s'effondrer.

%Le processus pour faire augmenter la température auquel nous nous attèlerons dans la suite est assez simple : l'amas évolue dans le potentiel galactique ; il va en traverser le disque de façon périodique.
%Moments pendant lesquels il va se faire harceler par des forces de marée qui vont lui faire perdre des étoiles. En considérant la description micro canonique, l'énergie est fixé ; or perdre une étoile
%diminue l'énergie potentielle de l'amas. La température va devoir augmenter pour conserver l'énergie totale constante.

%La perte d'étoile apparaît alors comme un processus intéressant pour expliquer l'évolution observée. Les chapitres suivants nous permettront de déterminer, à l'aide de simulation numérique, si la perte
%d'étoile par effet de marée est suffisante pour l'expliquer.








Cette dernière étude nous montre que plus un amas est âgé, plus la pente de son halo est importante.
Dans le chapitre précédent, nous avons relié la pente au paramètre $W_0$ qui, en plus de jouer sur les pentes, joue sur la densité centrale, comme le montre la figure~\ref{King_Modele-test}.
Par ailleurs, des simulations partant d'un nuage homogène nous apprennent que la pente du halo après sa formation vaut $-4$.
Un amas va donc partir d'une structure cœur-halo avec un cœur de taille importante, un halo de pente $-4$ et va tendre, en vieillissant, vers un amas ayant une densité centrale très élevée dont la pente tend vers $-2$.
Puis après un temps infini, l'amas va tendre vers une sphère isotherme.
C'est ce que représente le schéma suivant.

\input{theorie/figure_tikz-pente.tex}

Pour expliquer cette évolution, nous devons trouver quels phénomènes, de dynamique gravitationnelle, pourraient diluer le halo et concentrer de la matière au centre de l'amas. Le phénomène qui vient le plus naturellement à l'esprit est la perte d'étoiles, pouvant s'effectuer par 2 scénarios :
\begin{enumerate}
	\item des collisions internes qui éjectent des étoiles de l'amas, permettant ainsi au cœur de s'effondrer en essayant de compenser cette perte d'énergie.
	\item des interactions avec un autre objet massif qui va retirer par effet de marée des étoiles à l'amas, causant l'effondrement de son cœur.
\end{enumerate}
C'est ce deuxième scénario que nous allons tenter d'étudier.
















%Si la température cinétique du halo augmente, celle du
%cœur va changer pour s'adapter. La capacité calorifique à volume
%constant est négative pour le cœur (~et pour tout système
%auto gravitant~) :
%\begin{align*}
%	E_p &= -2E_c & \text{(~Viriel~)} \\
%	\Rightarrow E &= E_p + E_c = -E_c \\
%	\intertext{or}
%	E_c &\varpropto T \Rightarrow E\varpropto -T \\
%	\intertext{donc}
%	\Rightarrow C_v &= \frac{\partial E}{\partial T} < 0
%\end{align*}
%Cela implique que notre système ne peut que prendre de l'énergie.

%Si la température du cœur augmente, la vitesse de rotation des
%étoiles va augmenter (~$E_c\varpropto T\varpropto v$~), et donc
%leur demi grand axe va diminuer~\footnote{rappel : $v^2 =
%K\left(\frac{2}{r} - \frac{1}{a}\right)$ avec $K = G\left(M+m\right)$
%pour des interactions à deux corps}.
%En conséquent la densité au centre du cœur augmente, augmentant
%ainsi le contraste de densité. Si la température augmente
%suffisamment, le contraste de densité du halo va dépasser la
%valeur critique $\R_c^H$ (~l'énergie est fixé, seul la
%température change, nous utilisons donc la description micro
%canonique~). Le cœur va alors devenir instable.


%	L'interprétation semble simple ici : par exemple : quand les étoiles
%	ont des vitesses de rotation élevées la sphère a une température élevée,
%	et donc sa capacité calorifique à volume constant, $C_v = \frac{\partial H}{\partial T} < 0$
%	(~négatif car $E_p=-2E_c\Rightarrow H=E_c+E_p=-E_c\varpropto-T$~),
%	tend vers $0$ : elle ne peut plus acquérir d'énergie.
%	Si on dépasse la limite de température, elle va devoir se \og~réorganiser~\fg pour
%	rester à l'équilibre et donc pour retourner à un contraste de densité $\R < \R^\beta_c$.
%	Le raisonnement est le même pour la limite en énergie.


	\section{Les Galaxies}%{Que peut-on dire des autres ...!}
		
