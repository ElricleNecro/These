Décrire ce qui a été fait, ce qui est à faire, comment tout à été fait, ...

En gros, faire un état des lieux en utilisant la bibliographie.

\chapter{Introduction}%\markboth{Introduction}{}}
	\minitoc
	\section{Historique}
Les premiers amas d'étoiles ont été catalogués par \textsc{Messier} en 1784, mais ils n'ont été identifiés comme tel qu'en 1814 par William \textsc{Herschel}. Étudiés depuis cette époque, notre connaissance
	observationnelle à leur propos n'a pas cessé de s'améliorer avec la progression des techniques d'observation.
	Le premier comptage d'étoiles complet fut effectué par \textsc{Bailey} en 1893. En 1905, \textsc{Plummer} et \textsc{von Zeipel}	ont utilisé les observations pour
	remonter à la distribution radiale des étoiles. \textsc{Von Zeipel} fit alors le rapprochement entre un amas et une sphère de gaz à l'équilibre isotherme.
	Parallèle encore utilisé aujourd'hui, bien qu'il soit contestable sur au moins un point : le libre parcours moyen d'une étoile est grand devant la taille du
	système, alors que pour une sphère de gaz c'est l'inverse.% ; le mouvement des étoiles et celui des particules d'un gaz sont eux-mêmes assez différent (~l'un consiste
	%en des orbites déterminé par le potentiel gravitationnel des étoiles alentour, l'autre n'est constitué que de ligne droite~).

\section{Définition d'un amas globulaire}
	Dans notre galaxie, un amas globulaire est, en général, décrit comme un très vieil amas d'étoiles, âgé d'environ 10 milliards d'années, que l'on trouve soit près du bulbe galactique,
	soit dans le halo. Mais l'âge absolu de ces objets est très difficile à mesurer.
%	Par ailleurs, pour les amas en dehors du groupe local, cette définition n'est plus suffisante :
%	amas globulaire et amas ouvert ont des âges proches les uns des autres
%	En effet, dés que l'on observe une galaxie un peu lointaine~\footnote{et même dans les nuages de Magellan}, les amas sont de plus en plus jeune~\footnote{dans les nuages de Magellan,
%	les âges varient entre $10^6$ et $10^9$ ans}.

%	Mais restons dans notre galaxie pour le moment : ne serait-ce que dans notre galaxie, les amas globulaire peuvent être très différent les uns des autres :
	Les amas de notre galaxie présentent des caractéristiques assez variées. Par exemple, l'amas le plus massif de notre galaxie --~$\omega$ Centauri~-- posséde une masse d'environ $5.10^6 M_\odot$
	tandis que celle du moins massif --~AM-4~--est d'environ $10^3 M_\odot$.
	Leur magnitude absolue~\footnote{De $M_V = -10.1$ pour $\omega$ Centauri à $M_V = -1.7$ pour AM-4} et distance au centre de la galaxie~\footnote{environ 20 kpc pour les plus proche du centre à 120 kpc pour AM-1} varie également dans de grandes proportions.
	Une partie des amas est concentrée autour du centre galactique tandis que d'autres évoluent dans le halo galactique.



	\section{Temps caractéristiques}
%			Maintenant que nous avons étudié les modèles de \textsc{King} et de la sphère isotherme,
%			nous allons nous intéresser à l'un des sujets principaux du stage : les amas globulaire.
%			L'un des enjeux de cette partie et de faire le lien entre différents modèle de \textsc{King},
%			qui est le modèle préférentielle lors des ajustements de données, et les différentes étapes
%			d'évolution d'un amas globulaire.
%			Toutes les données ont été obtenu grâce à~\cite{Harris}, sauf mention contraire.

%			Avant de détailler le travail qui a été effectué, il va nous falloir définir 2 temps
%			caractéristique : les temps dynamiques et temps de collision.
%		\input{pente-temps.tex}
			Il est utile de décrire quelques temps caractéristiques de l'évolution de ces systèmes.
	\subsection{Temps dynamique}
		Le temps dynamique est le temps nécessaire à une particule pour traverser le système et donc se \og mettre au courant \fg~des changements ayant lieu.
		Il s'agit donc de la plus petite échelle de temps que nous puissions considérer pour l'évolution d'un système auto-gravitant.
		Pour l'obtenir, considérons un système de $N$ particules de même masse $m$, et de dispersion de vitesse $\sigma$. Les énergies potentielles et cinétiques du système
		peuvent être évaluées par :
		\begin{align}
			E_c = \frac{N}{2}m\sigma^2\qquad\mathrm{et}\qquad E_p = -\frac{G(Nm)^2}{R}
		\end{align}
		puis en supposant que le système à l'équilibre et en appliquant le théorème du \textsc{Viriel}~\footnote{Rappel : $2 E_c + U = 0$ avec $E_c$
		l'énergie cinétique et $U$ l'énergie potentielle}, nous obtenons :
		\begin{align}
			\sigma = \(\frac{GNm}{R}\)^{1/2}
		\end{align}
		Le temps dynamique est proportionnel au temps de croisement des particules du système. Ce dernier dépend du rayon de l'objet
		considéré et de la vitesse caractéristique des étoiles dans le système, ou encore, de la dispersion de vitesse du système :
		\begin{align}
			T_{cr} \propto T_d = \frac{R}{\sigma} \label{Td:sig}
		\end{align}
		En injectant notre expression pour la dispersion de vitesse, nous avons le temps dynamique :
		\begin{align}
			T_d = R\(\frac{R}{GNm}\)^{1/2} = \sqrt{\frac{R^3}{GM}} \propto \frac{1}{\sqrt{G\rho}}
			\label{Td:rho}
		\end{align}
		avec $M = Nm$ la masse totale du système.
	\subsection[Temps de relaxation]{Temps de collision ou temps de relaxation à 2 corps}
		Le temps de collision est par définition le temps mis par les interactions à 2 corps pour modifier d'un ordre de grandeur la vitesse des particules du système.
		Le détail pour le calcul de ce temps peut-être trouvé dans~\cite{ThNico} et~\cite{CoursJP}. Il s'écrit, en fonction du temps de croisement :
		\begin{align}
			T_{c} = \frac{3N}{8\ln\(\frac{3N}{4\pi}\)}T_{cr}
		\end{align}

		Ce temps est donc proportionnel au temps dynamique $T_d$ par l'intermédiaire du temps de croisement.




	\section{Sphère isotherme singulière}
		\subsection{Équations de \textsc{Vlasov}-\textsc{Poisson}}
	Nous allons étudier ici des systèmes de $N$ particules en interaction gravitationnelle dans la limite $N$ tend vers l'infini.
	Écrire les équations de \textsc{Newton} et travailler avec des sommes peut
	s'avérer lourd en calcul et compliqué, nous allons donc utiliser le formalisme
	de la physique statistique : nous décrivons le système par une fonction de distribution.
	Cette dernière permet de calculer la probabilité pour une particule $\alpha$ d'être dans un voisinage de la
	position ${}^T\,\left[\vec{r_\alpha},\vec{p_\alpha}\right]$ dans l'espace des phases ; elle
	s'exprime en nombre par unité de volume par impulsion au cube : $\left[ f\right] = m^{-3}p^{-3}$.

	En partant de la conservation des particules et en la combinant avec les équations d'\textsc{Hamilton},
	nous obtenons l'équation de \textsc{Boltzmann} :
	\begin{align}
		\frac{\partial f}{\partial t} +\frac{\vec{p}}{m}\frac{\partial f}{\partial \vec{r}} - m\frac{\partial f}{\partial \vec{p}} \frac{\partial \psi}{\partial \vec{r}} = N^2 G C(\vec{\omega}, t)
		\label{Fok-Plan}
	\end{align}
	(~le détail des calculs peut être trouvé dans le cours de gravitation statistique sur~\cite{CoursJP}~).
	Tant que nous pouvons considérer que les \og collisions \fg~n'affectent pas la dynamique du système, hypothèse qui pourra être discutée plus tard, le
	terme de collision $ C(\vec{\omega}, t) $ est nul. Nous obtenons alors l'équation de
	\textsc{Vlasov-Poisson} :
	\begin{align}
		\frac{\partial f}{\partial t} +\frac{\vec{p}}{m}\frac{\partial f}{\partial \vec{r}} - m\frac{\partial f}{\partial \vec{p}} \frac{\partial \psi}{\partial \vec{r}} = 0
		\label{Vla-Pois}
	\end{align}

	Tout les états stationnaires des systèmes que nous étudierons par la suite seront solution de cette équation.
	La quantité principale qui nous intéressera le long de ce document est la densité, qui s'écrit alors comme une loi marginale de $f$ :
	\begin{align}
		\rho(\vec{r}) = m\int f d\vec{p}
	\end{align}

\subsection{Sphère isotherme}
	Ce modèle est un modèle purement thermodynamique qui a été construit de telle sorte
	qu'il corresponde à un maximum de l'entropie statistique $S(f) = -k_B \int f\ln(f) d\Gamma$, avec $d\Gamma = d\vec{p} d\vec{r}$
	un élément de volume de l'espace des phases ; $k_B$ étant la constante de \textsc{Boltzmann}.
	Nous imposons au modèle d'avoir une masse et une énergie totale finies, et ce quelque soit la taille du système.
%	Une hypothèse en plus est à rajouter : la masse totale du système doit être finit, et ce malgré
%	l'extension infini de ce dernier.
	La fonction de distribution ainsi obtenue est :
	\begin{align}
		f^{eq}(E) = \(\frac{2\pi \alpha^2 m}{\beta}\)^{-3/2} e^{-\beta E}
	\end{align}
	Les constantes $\beta = \frac{1}{k_B T}$ et $\alpha$ sont des multiplicateurs de \textsc{Lagrange} associés aux contraintes imposées lors de la recherche d'un extremum local de l'entropie :
	$\beta \Leftrightarrow E = \mathrm{cte}$ et $\alpha \Leftrightarrow M = \mathrm{cte}$.

	La densité va alors s'écrire :
	\begin{align}
		\rho(r) &= m \int \(\frac{2\pi \alpha^2 m}{\beta}\)^{-3/2} e^{-\beta \(\frac{p^2}{2m} + m\psi\)} d^3 p \notag \\
			&= 4\pi m \(\frac{2\pi \alpha^2 m}{\beta}\)^{-3/2} \int_0^\infty e^{-\beta \(\frac{p^2}{2m} + m\psi\)} p^2 dp \notag \\
			&= \frac{m}{\alpha^3} e^{-m\beta\psi}
	\end{align}
	Nous avons donc $\rho(\vec{r}) = \rho(\psi(\vec{r}))$, le système est donc à symétrie sphérique : $\rho(\vec{r}) = \rho(r)$ (~voir~\cite{CoursJP}~).
	Nous introduisons alors cette expression dans l'équation de \textsc{Poisson} :
	\begin{align}
		\frac{1}{r^2}\frac{d}{dr}\(r^2\frac{d\psi(r)}{dr}\) &= 4\pi G \rho(r) = 4\pi G \frac{m}{\alpha^3} e^{-m\beta\psi(r)} \notag \\
		\intertext{En introduisant les variables $y = m\beta\psi$ et $x = r/r_0$, nous obtenons :}
		\frac{1}{r_0^3 x^2 m \beta}\frac{d}{dx}\(r_0^2 x^2 \frac{d y}{r_0 dx}\) &= \frac{4\pi G m}{\alpha^3} e^{-y} \notag \\
		\frac{1}{x^2}\frac{d}{dx}\(x^2\frac{d y}{dx}\) &=  \frac{4\pi G m r_0^2}{\alpha^3} e^{-y} \notag \\
		\intertext{en posant alors : $r_0^2 = \frac{\alpha^3}{4\pi G m r_0^2}$, l'équation adimensionnée est :}
		\Rightarrow \frac{1}{x^2}\frac{d}{dx}\(x^2\frac{d y}{dx}\) &= e^{-y} \label{Pois:sis}
	\end{align}
	Nous cherchons en premier lieu les solutions autosimilaires :
	\begin{align}
		\Tilde{\rho}(r) &= \frac{A}{r^2} = e^{-\Tilde{y}} \\
		\Rightarrow \Tilde{y} &= - \ln\(\frac{2}{x^2}\)
	\end{align}
	En prenant $A = 2$ pour être solution de~\ref{Pois:sis}.
	Nous avons donc maintenant :
	\begin{align}
		\Tilde{\rho}(r) &= \frac{m}{\alpha^3} e^{\ln\(\frac{2}{x^2}\)} = \frac{2 m r_0^2}{\alpha^3 r^2}
		\intertext{Et :}
		\Tilde{M}(r)    &= \int_0^{\infty} \Tilde{\rho}(r) d^3 r = \frac{4\pi m r_0^2}{\alpha^3} \int_0^{r} r^2\frac{1}{r^2} dr = \frac{4\pi m r_0^2}{\alpha^3} r
	\end{align}

	Cette solution autosimilaire a des propriétés physiques très embêtante :
	\begin{itemize}
		\item la densité diverge en zéro :
		\begin{align*}
			\lim\limits_{r \to 0} \Tilde{\rho}(r) &= \infty
		\end{align*}
		\item la masse est infinie :
		\begin{align*}
			\lim\limits_{r \to \infty} \Tilde{M}(r) &= \infty
		\end{align*}
	\end{itemize}
	et ceci pose problème : si la masse devient infinie, l'une des contraintes est violée.
	Cette solution forme ce que l'on appelle une sphère isotherme singulière (~SIS~) et ne correspond pas à un équilibre.

	Regardons maintenant s'il existe des solutions plus générales pouvant avoir une densité et une masse qui ne diverge pas. Pour cela, nous allons regarder l'évolution de
	la différence entre une solution générale $y$ du problème et notre solution autosimilaire. Posons $\zeta = y - \Tilde{y}$ :
	\begin{align}
		\frac{1}{x^2}\frac{d}{dx}\(x^2\frac{d\zeta}{dx}\) &= e^{-\zeta - \Tilde{y}} - \frac{1}{x^2}\frac{d}{dx}\(x^2\frac{d\Tilde{y}}{dx}\) \notag \\
								  &= e^{-\zeta - \Tilde{y}} - e^{-\Tilde{y}} \notag \\
								  &= \(e^{-\zeta} - 1\)\frac{2}{x^2}
		\intertext{Nous posons alors $t = \ln(x) \rightarrow dt = dx/x$}
		2 x \frac{d \zeta}{dx} + x^2 \frac{d^2 \zeta}{dx^2} &= 2 \(e^{-\zeta} - 1\) \Rightarrow 2 \frac{d \zeta}{dt} + \frac{d^2 \zeta}{dt^2} = 2 \(e^{-\zeta} - 1\) \notag
		\intertext{Pour faciliter la résolution du système, nous allons augmenter sa dimension en posant $v = \frac{d \zeta}{dt}$ :}
		\frac{d}{dt}\left\{\begin{array}{c}
			\zeta \\
			v
		\end{array}\right\} &= \left\{\begin{array}{c}
			v \\
			-2 v + 2 \(e^{-\zeta} - 1\)
		\end{array}\right\}
	\end{align}

	Le seul point fixe de ce système est le point $\(\zeta, v\) = \(0, 0\)$.
	Pour pouvoir étudier la stabilité du système autour de ce point, nous linéarisons nos équations autour
	de ce point en prenant $\(\zeta, v\) \to \(\delta\zeta, \delta v\)$ avec $\delta\zeta, \delta v \ll 1$,
	ce qui nous donne :
	\begin{align}
		\frac{d}{dt}\left\{\begin{array}{c}
			\delta\zeta \\
			\delta v
		\end{array}\right\} &= \left\{\begin{array}{c}
			\delta v \\
			-2 \delta v + 2 \(e^{-\delta\zeta} - 1\)
		\end{array}\right\} \simeq \left\{\begin{array}{c}
			\delta v \\
			-2 \delta v + 2 \delta\zeta
		\end{array}\right\} = D \left\{\begin{array}{c}
			\zeta \\
			v
		\end{array}\right\}
		\intertext{en utilisant $e^{-\delta\zeta} \simeq 1 - \delta\zeta$}
		&= \(\begin{array}{cc}
			0  & 1 \\
			-2 & -2
		\end{array}\) \(\begin{array}{c}
			\delta \zeta \\
			\delta v
		\end{array}\) = A \(\begin{array}{c}
			\delta \zeta \\
			\delta v
		\end{array}\)
		\intertext{Les valeurs propres de $A$ sont alors :}
		&\begin{cases}
			\lambda_1 = -1 - i \\
			\lambda_2 = -1 + i
		\end{cases} \\
		\intertext{ce qui nous amène à la solution :}
		&\begin{cases}
			\delta \zeta = e^{\lambda_1\delta\zeta} = e^{-\delta\zeta} e^{-i\delta\zeta} \\
			\delta  v = e^{\lambda_1\delta v} = e^{-\delta v} e^{-i\delta v}
		\end{cases}
	\end{align}
	Il existe donc des solutions de module réel décroissant (~$e^{- x}$~).
	De plus, le système respecte \mbox{$\mathrm{div}\( D\) < 0$} et n'a donc pas de cycle limite.
	Ainsi, toutes les solutions dans un voisinage proche de la solution $\Tilde{y}$ tendent vers cette dernière.
	Il peut-être démontré, en utilisant les sections de Poincaré, que toutes les solutions de cette équation
	tendent vers $\Tilde{y}$. Ainsi, toutes les solutions ont un problème : leur masse est infini pour $t\to\infty \Rightarrow r\to\infty$.
	Toutes les sphères isothermes sont donc proches de SIS quand $r\to +\infty$ : elles ont toutes des masses infinies.
	Il n'existe donc pas de distribution d'équilibre solution de \textsc{Vlasov-Poisson} maximisant l'entropie et possédant un support illimité.

	Pour palier à ce problème, plusieurs approches ont été développées au fil des ans :
	\begin{enumerate}
		\item Une solution \og rigoureuse \fg~: nous reprenons le problème du début en rajoutant une contrainte : nous cherchons un maximum de l'entropie dans un ensemble de fonctions de distribution compactes :
			l'extension de l'objet est finie. C'est le problème de la sphère isotherme en boîte.
			%nous reprenons les hypothèses de la sphère isotherme et nous supposons en plus qu'elle a une extension fini : c'est la sphère isotherme en boîte.
		\item Une solution \og pragmatique \fg~: développée par \textsc{King}, elle consiste à tronquer à la main et après coup la sphère isotherme : au-dessus d'une certaine énergie, la fonction de distribution est nulle.
			Cette solution est aujourd'hui la solution la plus utilisée par les observateurs. Elle possède d'ailleurs plusieurs paramètres libres pouvant rendre son utilisation pour la modélisation des amas globulaires adaptée
			aux ajustements.
	\end{enumerate}


