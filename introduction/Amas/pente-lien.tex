\subsection{Préliminaire}
%	Ce qui va nous intéresser ici, c'est de pouvoir lier à chaque étape d'évolution d'un amas, et donc à son âge,
%	une pente.
	Notre objectif est de trouver une relation entre l'âge d'un amas et la pente de son halo.
	%Nous avons donc besoin d'une relation entre cette quantité et l'âge de l'amas, si une telle relation existe.
	Nous avons alors utilisé le temps de relaxation donné dans le catalogue de \textsc{Harris}~\cite{Harris}.
	Pour obtenir les pentes des amas, nous avons utilisé les relevés observationnels~\cite{Trager-graphe}. % (~les données ainsi obtenues et utilisées sont dans la table~\ref{pente-Tc:BSP}~).
	Nous avons commencé par calculer les pentes directement sur les courbes avec un double décimètre, n'ayant alors pas pu obtenir les données correspondant aux graphiques.
	Après avoir tracé la pente mesurée en fonction du temps de relaxation à 2 corps, nous avons ajusté la
	courbe ainsi obtenue par une droite d'équation $ \alpha = \mathrm{pente} = a \log_{10}(T_c) + b$ (~graphe~\ref{Pente-lin}~).
	\begin{figure}[hbt!]
		\centering \includegraphics[scale=0.9]{graphe/Pente-Tc.pdf}
		\caption{Évolution des pentes pour différents âges}
		\label{Pente-lin}
	\end{figure}
%	\begin{table}[hbt!]
%		\begin{center}
%			\begin{tabular}{|c|c|c|}
%				\hline
%				Coefficient & Valeur & Erreur \\
%				\hline
%				\hline
%				$a$       &        $-1.19506$   &   $\pm 0.1576$ (~$13.19\%$~) \\
%				\hline
%				$b$       &        $2.52082$     &   $\pm 1.213$   (~$48.11\%$~) \\
%				\hline
%			\end{tabular}
%		\end{center}
%		\caption{Valeur des coefficients donnée par l'ajustement pour les pentes}
%		\label{pente-lin-coeff}
%	\end{table}

	Cette approche indiquant clairement une relation linéaire entre ces deux paramètres, nous avons décidé d'entreprendre une démarche plus globale et automatique.
	Pour commencer, nous avons donc récupéré les données auprès des auteurs de l'article sur~\cite{TragerTable}. Nous nous sommes alors confronté au problème des unités.

\subsection{Retraitement}
	Le graphique~\ref{Pente-lin} a été obtenu en utilisant~\cite{Trager-graphe} avec ses unités. % (~les données permettant de retracer, nous les avons finalement trouvé,
%	les graphiques de cet article sont données dans~\cite{TragerTable}~).
	La pente donnée ici n'est donc pas la pente de la densité, mais de la brillance de surface de l'objet en fonction d'un rayon en seconde d'arc.
	Le premier traitement à effectuer consiste donc à transformer cette brillance de surface par arc seconde carrée en une densité (~kilogramme par kilomètre au cube~).

	Une définition de cette brillance de surface, telle qu'elle semble avoir été utilisée, peut-être trouvé dans~\cite{SBP}.
	Comme indiqué dans ce texte, la brillance de surface va s'écrire :
	\begin{align}
		\mu_V = \mu_{\mathrm{ref}} - 2.5 \log_{10}\(\frac{f/\Omega}{f_{\mathrm{ref}}/\Omega_{\mathrm{ref}}}\)
		\label{mu_V}
	\end{align}
	avec $\Omega$ l'angle solide, exprimé en seconde d'arc au carrée, sous lequel nous voyons l'objet.
%	\begin{align}
%		\mu_V = m_V - 2.5 \log_{10}\(\frac{(1\mathrm{"})^2}{\Omega}\)
%		\label{mu-astuce}
%	\end{align}
%	avec $m_V$ la magnitude apparente de l'objet sur une seconde d'arc au carrée.

	Cette définition rappelle celle de~\cite{Trager-graphe}, section~3.2.3 :
	\begin{align}
		\mu = -2.5 \log\(\frac{10^{-0.4 m_2} - 10^{-0.4 m_1}}{\pi \(r_2^2 - r_1^2\)}\)
		\label{Trager-eq}
	\end{align}
%	où ils prendraient comme référence les points autour de celui considéré (~ou quelque chose comme ça~),
	$m_i$ étant la magnitude en un point de l'objet et $r_i$ le rayon en seconde d'arc pour ce point.
	Par conséquent, les unités de $\mu$ sont un flux par arc seconde carrée en échelle logarithmique :
	\begin{enumerate}
		\item $10^{-0.4 m_i}$ est proportionnel à un flux, % (~ou au rapport d'un flux par rapport à un flux de référence !?!~),
		\item $r_i$ est le rayon de l'objet au point $i$ en seconde d'arc,
		\item[$\Rightarrow$] nous avons donc bien notre flux par seconde d'arc au carrée.
	\end{enumerate}
	Pour pouvoir revenir aux quantités que nous cherchons, il va falloir calculer un peu :
	\begin{align}
		\mu = -2.5 \log\(\frac{10^{-0.4 m_2} - 10^{-0.4 m_1}}{\pi \(r_2^2 - r_1^2\)}\) &\equiv -2.5 \log\(\frac{F}{\pi \(r_2^2 - r_1^2\)}\) \text{avec $F$ le flux}\notag \\
		\intertext{par rapport à toute la documentation que j'ai trouvé, le $\log$ correspond ici à $\log_{10}$ et non à $\ln$.}
		\Rightarrow \frac{F}{\pi \(r_2^2 - r_1^2\)} &= 10^{-\mu/2.5} \label{F--mu} \\
		\intertext{Grâce à~\cite{McL}, nous avons les rapports masse luminosité :}
		\Upsilon &= \frac{M}{L} \label{M/L}\\
		\intertext{or}
		F &= \frac{L}{4\pi D^2} \label{def-F}\\
		\intertext{avec $D$ la distance soleil--amas. D'où}
		\frac{L}{4 \pi^2 \(r_2^2 - r_1^2\) D^2} &= \frac{M}{4 \pi^2 \Upsilon \(r_2^2 - r_1^2\) D^2} = 10^{-\mu/2.5} \notag \\
		\intertext{en combinant~\ref{F--mu}, \ref{M/L} et~\ref{def-F}.}
		\Rightarrow \frac{M}{4\pi \(r_2^2 - r_1^2\)} &= \pi\Upsilon D^2 10^{-\mu/2.5} \notag \\
		\intertext{Mais $r_2^2 - r_1^2 \propto r_i^2$ doit être converti en mètre :}
		%r_2^2 - r_1^2 &\propto r_i^2 \Rightarrow \(D \tan(r_i)\)^2 \varpropto \(D r_i\)^2 \notag \\
		\Rightarrow \frac{M}{4\pi \(D \tan(r_i/3600)\)^2} &= \pi\Upsilon D^2 10^{-\mu/2.5} \label{M-don}
	\end{align}
	Normalement, nous avons maintenant une masse surfacique donnée par~\ref{M-don}.
%	mais, étonnamment, la conversion de seconde d'arc à mètre n'a fait paraître
%	aucun facteur numérique !!! J'ai peut-être un peu trop truandé.

	Les rapports masse-luminosité peuvent être trouvés dans~\cite{McL}, mais cette article ne contient que 40 des 140 amas de notre galaxie.
%	mais il n'y a qu'une quarantaine de rapport comparé à notre échantillon d'environ 140 amas.
	Par contre, il est aisé de remarquer que ces rapports sont
	en moyenne très peu différents de la valeur $\Upsilon = 2$ avec une valeur minimum de $1.87$ pour NGC 4147 et une valeur maximum de
	$2.66$ pour NGC 6441. Dans la suite, nous prendrons donc $\Upsilon = 2$.
%	Nous avons alors utilisé les données du catalogue pour redimensionner le tout, mais pour passer à la densité, nous avons besoin du rapport masse
%	sur luminosité de l'amas qui peut-être trouvé dans~\cite{McL}.
%	Cet article ne donne qu'une quarantaine d'amas sur les 150 de notre galaxie, mais les valeurs de ces rapports étant compris entre $1.87$ pour NGC 4147
%	(~et quelques autres~) et $2.66$ pour NGC 6441, et tournant surtout autour de $2$, nous pouvons supposer que ces rapports sont les mêmes pour chaque amas et
%	valent $\Upsilon = \frac{M}{L} = 2$.
%	Selon~\cite{Trager-graphe}, nous avons donc :
%	\begin{align}
%		\mu_V &\propto \text{Flux par $m^{-2}$} = \frac{L}{4\pi D^2} \\
%		\intertext{or}
%		L &= \frac{M}{\Upsilon} \notag \\
%		\intertext{donc}
%		\mu_V &= \frac{M}{4\pi D^2 \Upsilon} \notag \\
%		M &= \Upsilon \mu_V 4\pi D^2
%	\end{align}
%	avec $D$ la distance entre nous et l'amas. Nous pouvons supposer que cette distance est la même quelque soit l'étoile considéré dans l'amas
%	(~$D_{min} = 3000\mathrm{pc} \gg R_{amas} \approx 10\mathrm{pc}$~).

%	Du fait des rapports masse-luminosité manquant, notre échantillon d'une quarantaine d'amas est passé à seulement 12 amas.

\subsection{Résultat}
	Le calcul de pente a été refait en utilisant : les données de l'article~\cite{TragerTable}, le traitement décrit ci-dessus, et le critère donné dans la section~\ref{pente-critére}.
	Nous obtenons alors le graphique~\ref{Pente-lin_dim}, l'ajustement utilisant la fonction
	$f(T_c) = a \log(T_c) + b$ (~les coefficients ont été donnés dans la table~\ref{pente-lin-coeff_dim}~).
	Nous avons donc maintenant une relation linéaire entre l'âge de l'amas et la pente du halo.

	\begin{figure}[hbt!]
%		\centering \includegraphics[scale=0.9]{../Amas/Graphe_Rapport/pente-Tc.pdf}
		\centering \includegraphics[scale=0.9]{graphe/pente-Tc.pdf}
		\caption{Évolution des pentes pour différents âges}
		\label{Pente-lin_dim}
	\end{figure}
	\begin{table}[hbt!]
		\begin{center}
			\begin{tabular}{|c|c|}%c|}
				\hline
				Coefficient & Valeur \\ %& Erreur \\
				\hline
				\hline
				$a$       &         $-2.3341$   \\ %&    $\pm 0.2075$      (~$32.34\%$~) \\
				\hline
				$b$       &         $16.913$   \\  %&    $\pm 1.638$       (~$199.5\%$~) \\
				\hline
			\end{tabular}
		\end{center}
		\caption{Valeurs des coefficients donnée par l'ajustement pour les pentes pour le temps de croisement}
		\label{pente-lin-coeff_dim}
	\end{table}

	Nous avons également utilisé le catalogue de \textsc{Harris} pour obtenir le lien entre la pente du halo et le temps dynamique
	grâce à l'équation~\ref{Td:sig}. Nous obtenons alors le graphique~\ref{Pente-Td-lin}.
	\begin{figure}[hbt!]
%		\centering \includegraphics[scale=0.9]{../Amas/Graphe_Rapport/pente-Td.pdf}
		\centering \includegraphics[scale=0.9]{graphe/pente-Td.pdf}
		\caption{Évolution des pentes pour différents âges dynamiques}
		\label{Pente-Td-lin}
	\end{figure}
	\begin{table}[hbt!]
		\begin{center}
			\begin{tabular}{|c|c|}%c|}
				\hline
				Coefficient & Valeur \\ %& Erreur \\
				\hline
				\hline
				$a$       &        $-1.6567$   \\ %&   $\pm 0.6523$      (~$278.9\%$~) \\
				\hline
				$b$       &        $1.8927$     \\ %&   $\pm 2.825$       (~$88.57\%$~) \\
				\hline
			\end{tabular}
		\end{center}
		\caption{Valeurs des coefficients donnée par l'ajustement pour les pentes pour le temps dynamique}
		\label{pente-Td-lin-coeff}
	\end{table}

	Sur la figure~\ref{Pente-lin_dim}, nous pouvons noter la présence de 2 points avec des pentes inférieures à $-10$. Ces points correspondent aux amas NGC 5024 et NGC 5139
	(~leurs densités sont donnée en annexe, page~\pageref{Graphe-bofbof}~) pour lesquelles la détermination des pentes n'a pas été très concluante.
	\FloatBarrier

