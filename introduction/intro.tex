\section{Historique}
Les premiers amas d'étoiles ont été catalogués par \textsc{Messier} en 1784, mais ils n'ont été identifiés comme tel qu'en 1814 par William \textsc{Herschel}. Étudiés depuis cette époque, notre connaissance
	observationnelle à leur propos n'a pas cessé de s'améliorer avec la progression des techniques d'observation.
	Le premier comptage d'étoiles complet fut effectué par \textsc{Bailey} en 1893. En 1905, \textsc{Plummer} et \textsc{von Zeipel}	ont utilisé les observations pour
	remonter à la distribution radiale des étoiles. \textsc{Von Zeipel} fit alors le rapprochement entre un amas et une sphère de gaz à l'équilibre isotherme.
	Parallèle encore utilisé aujourd'hui, bien qu'il soit contestable sur au moins un point : le libre parcours moyen d'une étoile est grand devant la taille du
	système, alors que pour une sphère de gaz c'est l'inverse.% ; le mouvement des étoiles et celui des particules d'un gaz sont eux-mêmes assez différent (~l'un consiste
	%en des orbites déterminé par le potentiel gravitationnel des étoiles alentour, l'autre n'est constitué que de ligne droite~).

\section{Définition d'un amas globulaire}
	Dans notre galaxie, un amas globulaire est, en général, décrit comme un très vieil amas d'étoiles, âgé d'environ 10 milliards d'années, que l'on trouve soit près du bulbe galactique,
	soit dans le halo. Mais l'âge absolu de ces objets est très difficile à mesurer.
%	Par ailleurs, pour les amas en dehors du groupe local, cette définition n'est plus suffisante :
%	amas globulaire et amas ouvert ont des âges proches les uns des autres
%	En effet, dés que l'on observe une galaxie un peu lointaine~\footnote{et même dans les nuages de Magellan}, les amas sont de plus en plus jeune~\footnote{dans les nuages de Magellan,
%	les âges varient entre $10^6$ et $10^9$ ans}.

%	Mais restons dans notre galaxie pour le moment : ne serait-ce que dans notre galaxie, les amas globulaire peuvent être très différent les uns des autres :
	Les amas de notre galaxie présentent des caractéristiques assez variées. Par exemple, l'amas le plus massif de notre galaxie --~$\omega$ Centauri~-- posséde une masse d'environ $5.10^6 M_\odot$
	tandis que celle du moins massif --~AM-4~--est d'environ $10^3 M_\odot$.
	Leur magnitude absolue~\footnote{De $M_V = -10.1$ pour $\omega$ Centauri à $M_V = -1.7$ pour AM-4} et distance au centre de la galaxie~\footnote{environ 20 kpc pour les plus proche du centre à 120 kpc pour AM-1} varie également dans de grandes proportions.
	Une partie des amas est concentrée autour du centre galactique tandis que d'autres évoluent dans le halo galactique.

