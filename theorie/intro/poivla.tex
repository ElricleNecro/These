% \chapter{Le syst\`{e}me de Vlasov-Poisson}

\section{{L'\'{e}quation de {Poisson}}}

%%%%%%%%%%%%%%%%%%%%%%%%%%%%%%%%%%%%%%%%%%%%%%%%%%%%%%%%%%%%%%%%%%%%%%%%%%%%%%%%%%%%%%%%%%%%%%%%%%%%%%%%%%%


{Les syst\`{e}mes que nous \'{e}tudions en dynamique stellaire poss\`{e}dent
des tailles carac\-t\'{e}\-ris\-tiques de l'ordre du {parsec} ($3,08\times
10^{16}$ m) pour les amas {globulaire}s et du millier de parsecs pour les
{galaxie}s. En n\'{e}gligeant le r\^{o}le du gaz interstellaire et sur de
telles \'{e}chelles, la seule des quatre interactions pouvant intervenir dans
la dynamique de ces syst\`{e}mes est \'{e}videmment la gravitation. }

{Nous consid\'{e}rons donc un syst\`{e}me $\Omega\subset\mathbb{R}^{3}$ dont
la masse est distribu\'{e}e selon une certaine densit\'{e}
\[
\rho=\left\{
\begin{array}
[c]{l}%
\rho(\mathbf{r}^{\prime})\\
0
\end{array}%
\begin{array}
[c]{l}%
\text{si}\qquad\mathbf{r}^{\prime}\in\Omega\\
\text{si}\qquad\mathbf{r}^{\prime}\notin\Omega
\end{array}
\right\}
\]
La loi de la gravitation de Newton nous indique alors que la force
$\mathbf{f}(\mathbf{r})$ que cr\'{e}e le syst\`{e}me en tout point
$\mathbf{r}\in\mathbb{R}^{3}$ }de masse unit\'{e}{, est obtenue en sommant
toutes les contributions infinit\'{e}simales
\begin{equation}
\delta\mathbf{f}(\mathbf{r})=G\frac{\mathbf{r}^{\prime}-\mathbf{r}}{\left\vert
\mathbf{r}^{\prime}-\mathbf{r}\right\vert ^{3}}\rho(\mathbf{r}^{\prime}%
)\delta^{3}\mathbf{r}^{\prime}\label{poisss}%
\end{equation}
issues de chaque \'{e}l\'{e}ment de volume $\delta^{3}\mathbf{x}^{\prime}$.
Dans l'hypoth\`{e}se d'une distribution continue de mati\`{e}re nous avons
donc
\begin{equation}
\mathbf{f}(\mathbf{r})=G\displaystyle\int\frac{\mathbf{r}^{\prime}-\mathbf{r}%
}{\left\vert \mathbf{r}^{\prime}-\mathbf{r}\right\vert ^{3}}\rho
(\mathbf{r}^{\prime})d^{3}\mathbf{r}^{\prime}\label{eq1}%
\end{equation}
}

{En introduisant le {potentiel} gravitationnel
\begin{equation}
\psi(\mathbf{r})=-G\displaystyle\int\frac{\rho(\mathbf{r}^{\prime}%
)}{\left\vert \mathbf{r}^{\prime}-\mathbf{r}\right\vert }d^{3}\mathbf{r}%
^{\prime}\label{eq2}%
\end{equation}
et en remarquant que
\[
\mathrm{grad}_{\mathbf{r}}\left(  \frac{1}{\left\vert \mathbf{r}^{\prime
}-\mathbf{r}\right\vert }\right)  =\frac{\mathbf{r}^{\prime}-\mathbf{r}%
}{\left\vert \mathbf{r}^{\prime}-\mathbf{r}\right\vert ^{3}}
\]
Nous avons donc
\[
\mathbf{f}(\mathbf{r})=G\displaystyle\int\mathrm{grad}_{\mathbf{r}}\left(
\frac{1}{\left\vert \mathbf{r}^{\prime}-\mathbf{r}\right\vert }\right)
\rho(\mathbf{r}^{\prime})d^{3}\mathbf{r}^{\prime}
\]
pour une large classe de densit\'{e}s suffisamment r\'{e}guli\`{e}res (celles
permettant \`{a} $\left(  \ref{eq1}\right)  $ de converger), la force
d\'{e}rive donc du {potentiel} $\left(  \ref{eq2}\right)  $ et l'on a
\begin{equation}
\mathbf{f}(\mathbf{r})=-\mathrm{grad}_{\mathbf{r}}\psi(\mathbf{r})\label{eq3}%
\end{equation}
le syst\`{e}me est alors dit conservatif. Plut\^{o}t que d'utiliser le champ
vectoriel $\mathbf{f}(\mathbf{r})$, il est pr\'{e}f\'{e}rable de trouver une
relation entre champs scalaires. Ceci est possible en remarquant que
l'\'{e}quation $\left(  \ref{eq2}\right)  $ est une {convolution} :}
\begin{equation}
\psi(\mathbf{r},t)=-G\,\,\,\rho(\mathbf{r},t)\ast\frac{1}{\left\vert
\mathbf{r}\right\vert }\ .\label{poiconvlap}%
\end{equation}
{La deuxi\`{e}me fonction de cette {convolution} est, \`{a} une constante
pr\`{e}s, la fonction de {Green} du laplacien. Un calcul trivial montre en
effet que si }$u${\ est une fonction test au sens des distributions, alors le
produit scalaire}
\[
\left\langle \Delta\left(  \frac{1}{\left\vert \mathbf{r}\right\vert }\right)
,u\right\rangle =-4\pi\,u\left(  0\right)  =\left\langle -4\pi\delta\left(
\mathbf{r}\right)  \text{ },u\right\rangle \text{.}
\]
{Il devient donc \'{e}vident d'appliquer le laplacien \`{a} l'\'{e}quation
}$\left(  \ref{poiconvlap}\right)  ${\ qui s'\'{e}crit alors}
\[
\left\langle \Delta\psi(\mathbf{r},t),u\right\rangle =\left\langle
-G\,\,\Delta\left(  \rho(\mathbf{r},t)\ast\frac{1}{\left\vert \mathbf{r}%
\right\vert }\right)  ,u\right\rangle
\]
{soit }
\[
\left\langle \Delta\psi(\mathbf{r},t),u\right\rangle =\left\langle
-G\,\,\rho(\mathbf{r},t)\ast\Delta\left(  \frac{1}{\left\vert \mathbf{r}%
\right\vert }\right)  ,u\right\rangle =\left\langle 4\pi G\,\,\rho
(\mathbf{r},t)\ast\delta\left(  \mathbf{r}\right)  ,u\right\rangle
\]


{le {Dirac} \'{e}tant l'\'{e}l\'{e}ment neutre de l'alg\`{e}bre de
{convolution}, nous obtenons finalement l'\'{e}quation de {Poisson} }
\[
\Delta\psi(\mathbf{r},t)=4\pi G\,\,\rho(\mathbf{r},t)
\]


{Cette \'{e}quation, seule ou coupl\'{e}e avec d'autres, est \`{a} la base de
nombreux pro\-bl\`{e}\-mes d'astrophysique th\'{e}orique. }

\begin{itemize}
\item {Utilis\'{e}e seule elle n\'{e}cessite la donn\'{e}e (th\'{e}orique ou
exp\'{e}rimentale) de la fonction $\rho(\mathbf{r})$. La r\'{e}solution de
l'\'{e}quation de {Poisson} fournit alors la force g\'{e}n\'{e}r\'{e}e par le
syst\`{e}me en tout point de l'espace, il devient donc formellement possible
de conna\^{\i}tre les propri\'{e}t\'{e}s dynamiques de la trajectoire d'une
particule test \'{e}voluant dans le champ de gravitation cr\'{e}\'{e} par la
distribution $\Omega$. }

\item {Dans le cadre d'une th\'{e}orie fournissant la densit\'{e} du
syst\`{e}me, l'\'{e}quation de {Poisson} sera alors utilis\'{e}e comme
\'{e}quation de fermeture du probl\`{e}me. }
\end{itemize}

\section{Le syst\`{e}me de Vlasov-Poisson}

Pour \'{e}tudier la dynamique des syst\`{e}mes autogravitants comme les
galaxies ou les amas globulaires, la mise en \oe uvre des m\'{e}\-tho\-des de
la m\'{e}canique classique conduit \`{a} \'{e}crire, donc \`{a} r\'{e}soudre,
un nombre d'\'{e}quations diff\'{e}rentielles \'{e}gal au nombre d'\'{e}toiles
composant le syst\`{e}me, ce qui est en pratique irr\'{e}alisable.\ Il
convient d'utiliser des m\'{e}thodes statistiques de champ moyen, dans le cas
gravitationnel la th\'{e}orie de Vlasov-Poisson.\ 

Supposons que le syst\`{e}me que nous consid\'{e}rons poss\`{e}de $3N$
degr\'{e}s de libert\'{e}. Cela signifie que les positions des diff\'{e}rents
points du syst\`{e}me sont d\'{e}finies par $N$ vecteurs $\mathbf{r}_{\alpha}$
comportant chacun 3 composantes, l'indice $\alpha$ prenant toutes les valeurs
$1,2,\cdots,N$. A un instant donn\'{e}, l'\'{e}tat du syst\`{e}me sera
compl\`{e}tement d\'{e}termin\'{e} par les $N$ vecteurs position et les $N$
vecteurs impulsion $\mathbf{p}_{\alpha}=m_{\alpha}\mathbf{\dot{r}}_{\alpha}$
correspondants. On peut repr\'{e}senter math\'{e}matiquement les
diff\'{e}rents \'{e}tats d'un syst\`{e}me par des points dans un espace \`{a}
$6N$ {dimension}s appel\'{e} espace des phases. La trajectoire, ou le lieu de
ces points, permet de repr\'{e}senter de mani\`{e}re univoque l'\'{e}volution
du syst\`{e}me.

En s'int\'{e}ressant \`{a} un volume infinit\'{e}simal $d\Gamma$ de cet espace
des phases
\[
d\Gamma:=d\mathbf{r}_{1}d\mathbf{r}_{2}\cdots d\mathbf{r}_{N}d\mathbf{p}%
_{1}d\mathbf{p}_{2}\cdots d\mathbf{p}_{N}
\]
on peut introduire la probabilit\'{e} $d\Omega$ des \'{e}tats
repr\'{e}sent\'{e}s par des points contenus dans ce volume \`{a} l'instant
$t$. C'est-\`{a}-dire, la probabilit\'{e} pour qu'\`{a} cet instant les
positions $\mathbf{r}_{\alpha}$ et les {impulsion}s $\mathbf{p}_{\alpha}$
soient comprises dans les intervalles infinit\'{e}simaux $[\mathbf{r}_{\alpha
}+d\mathbf{r}_{\alpha}]$ et $[\mathbf{p}_{\alpha}+d\mathbf{p}_{\alpha}]$.
Cette probabilit\'{e} se laisse exprimer par
\[
d\Omega\;=\;f^{(N)}(\mathbf{r}_{1},\cdots,\mathbf{r}_{N},\mathbf{p}_{1}%
,\cdots,\mathbf{p}_{N},t)\,d\Gamma
\]
o\`{u} la quantit\'{e} $f^{(N)}(\mathbf{r}_{1},\cdots,\mathbf{r}%
_{N},\mathbf{p}_{1},\cdots,\mathbf{p}_{N},t)$ est appel\'{e}e fonction de
distribution \`{a} $N$ particules du syst\`{e}me. Cette fonction est
clairement positive et de norme unit\'{e}
\[
1=\int f^{\left(  N\right)  }\left(  \mathbf{w}_{1},...,\mathbf{w}%
_{N},t\right)  \,\,\,d\mathbf{w}_{1}\,...d\mathbf{w}_{N}
\]
{\ }

elle repr\'{e}sente la densit\'{e}, au sens probabiliste, de la variable
al\'{e}atoire $\mathbf{w}=(\mathbf{w}_{1},...,\mathbf{w}_{N})$ avec
$\mathbf{w}_{\alpha}=\left(  \mathbf{r}_{\alpha},\mathbf{p}_{\alpha}\right)
$. Si le nombre de particules est conserv\'{e} au cours de l'\'{e}volution,
cette densit\'{e} ob\'{e}it \`{a} une \'{e}quation de continuit\'{e}
(exprimant le fait que $df$ est une diff\'{e}rentielle totale exacte, ou
physiquement que le nombre de particules est conserv\'{e})
\begin{equation}
\frac{\partial f^{(N)}}{\partial t}\;+\;\mathrm{div}_{\mathbf{w}}%
(f^{(N)}\mathbf{\dot{w}})\;=\;0\label{continu}%
\end{equation}
o\`{u} $\mathbf{\dot{w}}=(\mathbf{\dot{r}}_{1},\mathbf{\dot{p}}_{1}%
,...,\mathbf{\dot{r}}_{N},\mathbf{\dot{p}}_{N})$ repr\'{e}sente la vitesse du
flot des points dans l'espace des phases. En explicitant la divergence de
l'\'{e}quation (\ref{continu}), il vient
\[
\frac{\partial f^{(N)}}{\partial t}\;+\;\sum_{\alpha=1}^{N}\left[
\mathrm{div}_{\mathbf{r}_{\alpha}}\left(  f^{(N)}\mathbf{\dot{r}}_{\alpha
}\right)  +\mathrm{div}_{\mathbf{p}_{\alpha}}\left(  f^{(N)}\mathbf{\dot{p}%
}_{\alpha}\right)  \right]  ~\;=\;0
\]
Apr\`{e}s d\'{e}rivation, nous obtenons
\begin{align*}
&  \frac{\partial f^{(N)}}{\partial t}+\displaystyle\sum\limits_{\alpha=1}%
^{N}\left\{  \mathbf{\dot{r}}_{\alpha}.\mathrm{grad}_{\mathbf{r}_{\alpha}%
}\left(  f^{(N)}\right)  +\mathbf{\dot{p}}_{\alpha}.\mathrm{grad}%
_{\mathbf{p}_{\alpha}}\left(  f^{(N)}\right)  \right. \\
&  \,\;\;\;\;\;\;\;\left.  +~f^{(N)}\left[  \mathrm{div}_{\mathbf{r}_{\alpha}%
}\left(  \mathbf{\dot{r}}_{\alpha}\right)  +\mathrm{div}_{\mathbf{p}_{\alpha}%
}\left(  \mathbf{\dot{p}}_{\alpha}\right)  \right]  \right\}
\end{align*}
On introduit alors les \'{e}quations de {Hamilton} pour chaque particule (qui
reviennent ici \`{a} \'{e}crire le {principe} fondamental de la dynamique),
\[
\forall~1\leq\alpha\leq N~~~~\mathbf{\dot{r}}_{\alpha}=\frac{\partial
H}{\partial\mathbf{p}_{\alpha}}~~~~\text{et}~~~~\mathbf{\dot{p}}_{\alpha
}=-\frac{\partial H}{\partial\mathbf{r}_{\alpha}}
\]
o\`{u} $H$ est le hamiltonien du syst\`{e}me
\[%
\begin{array}
[c]{cccc}%
H= & \displaystyle\sum\limits_{\alpha=1}^{N}\dfrac{\mathbf{p}_{\alpha}^{2}%
}{2m_{\alpha}} & + & \displaystyle\sum\limits_{\alpha\neq\beta=1}^{N,N}%
-\dfrac{G}{2}\dfrac{m_{\alpha}m_{\beta}}{\left\vert \mathbf{r}_{\alpha
}-\mathbf{r}_{\beta}\right\vert }%
\end{array}
\]
on constate alors que
\[
\forall~1\leq\alpha\leq N~~~~\mathrm{div}_{\mathbf{r}_{\alpha}}\left(
\mathbf{\dot{r}}_{\alpha}\right)  =\frac{\partial^{2}H}{\partial
\mathbf{r}_{\alpha}\partial\mathbf{p}_{\alpha}}=-\mathrm{div}_{\mathbf{p}%
_{\alpha}}\left(  \mathbf{\dot{p}}_{\alpha}\right)
\]
Ainsi donc l'\'{e}quation de continuit\'{e} se met sous la forme
\begin{equation}
\frac{\partial f^{(N)}}{\partial t}+\sum_{\alpha=1}^{N}\left[  \mathbf{\dot
{r}}_{\alpha}.\mathrm{grad}_{\mathbf{r}_{\alpha}}\left(  f^{(N)}\right)
+\mathbf{\dot{p}}_{\alpha}.\mathrm{grad}_{\mathbf{p}_{\alpha}}\left(
f^{(N)}\right)  \right]  =0\label{liouville}%
\end{equation}
Il s'agit de l'\'{e}quation de {Liouville}.

Dans notre contexte gravitationnel, la forme du hamiltonien permet de
l'expli\-ci\-ter plus avant
\[
\frac{\partial f^{(N)}}{\partial t}+\sum_{\alpha=1}^{N}\left\{  \dfrac
{\mathbf{p}_{\alpha}}{m_{\alpha}}.\dfrac{\partial f^{(N)}}{\partial
\mathbf{r}_{\alpha}}-\dfrac{\partial U_{\alpha}}{\partial\mathbf{r}_{\alpha}%
}.\dfrac{\partial f^{(N)}}{\partial\mathbf{p}_{\alpha}}\right\}  =0
\]


o\`{u} l'on a pos\'{e}
\[
U_{1}=\displaystyle\sum\limits_{\beta=2}^{N}-G\dfrac{m_{1}m_{\beta}%
}{\left\vert \mathbf{r}_{1}-\mathbf{r}_{\beta}\right\vert }\ ,\ \ U_{2}%
=-G\dfrac{m_{2}m_{1}}{\left\vert \mathbf{r}_{2}-\mathbf{r}_{1}\right\vert
}+\displaystyle\sum\limits_{\beta=3}^{N}-G\dfrac{m_{1}m_{\beta}}{\left\vert
\mathbf{r}_{2}-\mathbf{r}_{\beta}\right\vert }\ ,\ \ \ \text{etc...}
\]
{En pratique, lorsque le syst\`{e}me devient plus grand qu'une paire, cette
\'{e}quation s'av\`{e}re inuti\-lisable, et nous devons faire des
hypoth\`{e}ses de nature statistique. \`{A} partir de la fonction de
distribution \`{a} $N$ particules $f^{(N)}$, nous pouvons construire une
fonction de distribution \`{a} une particule (densit\'{e} marginale),
\[
f^{(1)}=f^{(1)}(\mathbf{w}_{1},t)=\int\cdots\int f^{(N)}d\mathbf{w}_{2}\cdots
d\mathbf{w}_{N}
\]
En int\'{e}grant l'\'{e}quation de {Liouville} sur }$\mathbf{w}_{2}%
\cdots\mathbf{w}_{N}$, il vient
\[%
\begin{array}
[c]{ll}%
\dfrac{\partial f^{(1)}}{\partial t} & +\displaystyle\sum\limits_{\alpha
=1}^{N}\left\{  \displaystyle\int\dfrac{\mathbf{p}_{\alpha}}{m_{\alpha}%
}.\dfrac{\partial f^{(N)}}{\partial\mathbf{r}_{\alpha}}d\mathbf{w}_{2}\cdots
d\mathbf{w}_{N}\right\} \\
& \\
& -\displaystyle\sum\limits_{\alpha=1}^{N}\left\{  \displaystyle\int%
\dfrac{\partial U_{\alpha}}{\partial\mathbf{r}_{\alpha}}.\dfrac{\partial
f^{(N)}}{\partial\mathbf{p}_{\alpha}}d\mathbf{w}_{2}\cdots d\mathbf{w}%
_{N}\right\}  =0
\end{array}
\]
{les deux sommes d'int\'{e}grales se simplifient consid\'{e}rablement en
utilisant le fait que la fonction de distribution s'annule sur le bord du
syst\`{e}me}
\[
\forall\alpha=1,...,N\qquad\lim_{\mathbf{w}_{\alpha}\rightarrow\infty
}f^{\left(  N\right)  }=0
\]
{ainsi toutes les int\'{e}grations sur }$\mathbf{r}_{2},...,\mathbf{r}_{N}%
${\ s'annulent et }
\[%
\begin{array}
[c]{ll}%
\displaystyle\sum\limits_{\alpha=1}^{N}\left\{  \displaystyle\int%
\dfrac{\mathbf{p}_{\alpha}}{m_{\alpha}}.\dfrac{\partial f^{(N)}}%
{\partial\mathbf{r}_{\alpha}}d\mathbf{w}_{2}\cdots d\mathbf{w}_{N}\right\}  &
=\displaystyle\int\dfrac{\mathbf{p}_{1}}{m_{1}}.\dfrac{\partial f^{(N)}%
}{\partial\mathbf{r}_{1}}d\mathbf{w}_{2}\cdots d\mathbf{w}_{N}\\
& \\
& =\dfrac{\mathbf{p}_{1}}{m_{1}}.\dfrac{\partial}{\partial\mathbf{r}_{1}%
}\displaystyle\int f^{(N)}d\mathbf{w}_{2}\cdots d\mathbf{w}_{N}\\
& \\
& =\dfrac{\mathbf{p}_{1}}{m_{1}}.\dfrac{\partial f^{(1)}}{\partial
\mathbf{r}_{1}}%
\end{array}
\]


{pour les m\^{e}mes raisons mais en vitesse... }
\[%
\begin{array}
[c]{ll}%
\displaystyle\sum\limits_{\alpha=1}^{N}\left\{  \displaystyle\int%
\dfrac{\partial U_{\alpha}}{\partial\mathbf{r}_{\alpha}}.\dfrac{\partial
f^{(N)}}{\partial\mathbf{p}_{\alpha}}d\mathbf{w}_{2}\cdots d\mathbf{w}%
_{N}\right\}   & =\displaystyle\int\dfrac{\partial U_{1}}{\partial
\mathbf{r}_{1}}.\dfrac{\partial f^{(N)}}{\partial\mathbf{p}_{1}}%
d\mathbf{w}_{2}\cdots d\mathbf{w}_{N}%
\end{array}
\]
{de l'\'{e}quation de {Liouville} il ne reste alors plus que }
\[
\frac{\partial f^{(1)}}{\partial t}+\dfrac{\mathbf{p}_{1}}{m_{1}}%
.\dfrac{\partial f^{(1)}}{\partial\mathbf{r}_{1}}-\displaystyle\int%
\dfrac{\partial U_{1}}{\partial\mathbf{r}_{1}}.\dfrac{\partial f^{(N)}%
}{\partial\mathbf{p}_{1}}d\mathbf{w}_{2}\cdots d\mathbf{w}_{N}=0
\]
{L'hypoth\`{e}se raisonnable qu'il convient alors de faire est de supposer les
particules (\'{e}toiles) indiscernables. Explicitons cette subtilit\'{e} sur
le calcul de l'\'{e}nergie potentielle}
\begin{align*}
U_{1} &  =U_{1}\left(  \mathbf{r}_{1},...,\mathbf{r}_{N}\right)  =\sum
_{\beta=2}^{N}U_{1\beta}\\
\text{o\`{u}\qquad}U_{\alpha\beta} &  =U_{\alpha\beta}\left(  \mathbf{r}%
_{\alpha},\mathbf{r}_{\beta}\right)  :=-\frac{Gm_{\alpha}m_{\beta}}{\left\vert
\mathbf{r}_{\alpha}-\mathbf{r}_{\beta}\right\vert }%
\end{align*}
Dans cette somme, l'indiscernabilit\'{e} des particules fait jouer \`{a}
toutes le m\^{e}me r\^{o}le. On peut donc en choisir une (la particule 1) pour
repr\'{e}senter toutes les autres et subir leur action globale, et une autre
(la particule 2) pour repr\'{e}senter toutes les particules du syst\`{e}me et
agir sur la particule 1 {de fa\c{c}on globale. Il s'agit de l'hypoth\`{e}se de
champ moyen. Elle est fondamentale. En pratique cela revient \`{a}
consid\'{e}rer que }%
\[
U_{12}=U_{13}=\cdots=U_{1N}\ \ \ \Rightarrow\ U_{1}=\sum_{\beta=2}%
^{N}U_{1\beta}\ =\left(  N-1\right)  U_{12}\ \
\]
ainsi l'int\'{e}grale r\'{e}tive devient{\ }
\[
\displaystyle\int\dfrac{\partial U_{1}}{\partial\mathbf{r}_{1}}.\dfrac
{\partial f^{(N)}}{\partial\mathbf{p}_{1}}d\mathbf{w}_{2}\cdots d\mathbf{w}%
_{N}=\left(  N-1\right)  \displaystyle\int\dfrac{\partial U_{12}}%
{\partial\mathbf{r}_{1}}.\dfrac{\partial f^{(N)}}{\partial\mathbf{p}_{1}%
}d\mathbf{w}_{2}\cdots d\mathbf{w}_{N}%
\]
La particule 1 est appell\'{e}e \og\emph{particule test}\fg, elle\ \'{e}volue
dans le syst\`{e}me {moyen} repr\'{e}sent\'{e} par la particule 2$.$

{En introduisant la densit\'{e} marginale \`{a} 2 particules }
\[
f^{(2)}=f^{(2)}(\mathbf{w}_{1},\mathbf{w}_{2},t)=\int\cdots\int f^{(N)}%
d\mathbf{w}_{3}\cdots d\mathbf{w}_{N}
\]
{et comme }$U_{12}${\ ne d\'{e}pend que de }$\mathbf{w}_{1}${\ et }%
$\mathbf{w}_{2}$, {\ on peut finir l'int\'{e}gration du terme contenant
l'\'{e}nergie {potentiel}le pour obtenir la version int\'{e}gr\'{e}e de
{Liouville} sui\-van\-te}
\begin{equation}
\frac{\partial f^{(1)}}{\partial t}+\dfrac{\mathbf{p}_{1}}{m_{1}}%
.\dfrac{\partial f^{(1)}}{\partial\mathbf{r}_{1}}=\left(  N-1\right)
\displaystyle \int\dfrac{\partial U_{12}}{\partial\mathbf{r}_{1}}%
.\dfrac{\partial f^{(2)}}{\partial\mathbf{p}_{1}}d\mathbf{w}_{2}%
\label{liouint}%
\end{equation}
{Cette \'{e}quation permet donc de calculer }$f^{(1)}${\ \`{a} partir de
}$f^{(2)}${, en poursuivant on obtient }
\[
f^{(1)}\hookleftarrow f^{(2)}\hookleftarrow...\hookleftarrow f^{(N)}
\]


{\noindent Cette cascade d'\'{e}quations est g\'{e}n\'{e}ralement appel\'{e}e
hi\'{e}rarchie {BBGKY}.}

{La technique habituelle consiste \`{a} stopper la hi\'{e}rarchie,
c'est-\`{a}-dire trouver dans quelles conditions }
\[
\exists p<N,\qquad\text{tel que \qquad}\forall n>p\qquad f^{\left(  n\right)
}=0
\]


{Dans ce contexte, l'hypoth\`{e}se de chaos mol\'{e}culaire de {Boltzmann} est
assez op\'{e}rationnelle, elle revient \`{a} poser}
\[
f^{(2)}(\mathbf{w}_{1},\mathbf{w}_{2},t)=f^{(1)}(\mathbf{w}_{1},t)f^{(1)}%
(\mathbf{w}_{2},t)+g(\mathbf{w}_{1},\mathbf{w}_{2},t)
\]
{la fonction }$g(\mathbf{w}_{1},\mathbf{w}_{2},t)${\ d\'{e}crivant les
corr\'{e}lations ou interactions binaires.}

{Les particules \'{e}tant indiscernables, il convient de poser}
\[
f\left(  \mathbf{w},t\right)  =N\,f^{(1)}(\mathbf{w},t)\qquad\text{ainsi\qquad
}\int f\left(  \mathbf{w},t\right)  d\mathbf{w=}N
\]
{Sous toutes ces hypoth\`{e}ses, la version int\'{e}gr\'{e}e }$\left(
\ref{liouint}\right)  ${\ de l'\'{e}quation de {Liouville} s'\'{e}crit donc }
\[
\frac{\left(  \frac{\partial f}{\partial t}+\frac{\mathbf{p_{1}}}{m_{1}}%
.\frac{\partial f}{\partial\mathbf{r}}\right)  }{{\ N}}=\left(  {\ N-1}%
\right)  \displaystyle\int\tfrac{\partial U_{12}}{\partial\mathbf{r}_{1}%
}.\frac{{\ \partial}\left(  \frac{f\left(  \mathbf{w}_{1},t\right)  }{N}%
\frac{f\left(  \mathbf{w}_{2},t\right)  }{N}+{\ g}\right)  }{{\ \partial
}\mathbf{p}_{1}}d\mathbf{w}_{2}%
\]
{soit }
\[%
\begin{array}
[c]{ll}%
\dfrac{\partial f}{\partial t}+\dfrac{\mathbf{p_{1}}}{m_{1}}.\dfrac{\partial
f}{\partial\mathbf{r}}= & \dfrac{\left(  N-1\right)  }{N}\dfrac{\partial
f\left(  \mathbf{w}_{1},t\right)  }{\partial\mathbf{p}_{1}}\dfrac{\partial
}{\partial\mathbf{r}_{1}}\displaystyle\int U_{12}f\left(  \mathbf{w}%
_{2},t\right)  d\mathbf{w}_{2}\\
& \\
& +N\left(  N-1\right)  \displaystyle\int\dfrac{\partial U_{12}}%
{\partial\mathbf{r}_{1}}.\dfrac{\partial g}{\partial\mathbf{p}_{1}}%
d\mathbf{w}_{2}%
\end{array}
\]
un petit rappel s'impose alors
\[
\displaystyle\int U_{12}f\left(  \mathbf{w}_{2},t\right)  d\mathbf{w}%
_{2}=-Gm_{1}m_{2}\displaystyle\int\dfrac{f\left(  \mathbf{w}_{2},t\right)
}{\left\vert \mathbf{r}_{1}-\mathbf{r}_{2}\right\vert }d\mathbf{r}%
_{2}d\mathbf{p}_{2}%
\]
{dans le second membre de cette relation, l'int\'{e}gration sur les
{impulsion}s nous fait reconna\^{\i}tre la densit\'{e} de masse}
\[
\rho\left(  \mathbf{r_{2}},t\right)  =m_{2}\displaystyle\int f\left(
\mathbf{w_{2}},t\right)  d\mathbf{p}%
\]
{Nous avons donc}
\[
\displaystyle\int U_{12}f\left(  \mathbf{w}_{2},t\right)  d\mathbf{w}%
_{2}=-Gm_{1}\displaystyle\int\dfrac{\rho\left(  \mathbf{r}_{2},t\right)
}{\left\vert \mathbf{r}_{1}-\mathbf{r}_{2}\right\vert }d\mathbf{r}_{2}%
\]
{o\`{u} l'on voit sourdre le {potentiel} gravitationnel impliqu\'{e} dans
l'\'{e}quation de
{Poisson} :}
\[
\displaystyle{\ \int U_{12}f\left(  \mathbf{w}_{2},t\right)  d\mathbf{w}%
_{2}=m_{1}\psi\left(  \mathbf{r}_{1},t\right)  }%
\]
en prenant $N-1\approx N$, et en laissant tomber l'indice 1 on obtient
finalement
\begin{equation}
\frac{\partial f}{\partial t}+\frac{\mathbf{p}}{m}.\frac{\partial f}%
{\partial\mathbf{r}}-m\frac{\partial f}{\partial\mathbf{p}}\frac{\partial\psi
}{\partial\mathbf{r}}=N^{2}GC\left(  \mathbf{w},t\right)  \label{boltzman}%
\end{equation}
{o\`{u} l'on a introduit le terme de corr\'{e}lation }
\[
C\left(  \mathbf{w},t\right)  =\displaystyle\int\dfrac{\partial g\left(
\mathbf{w},\mathbf{w}_{2},t\right)  }{\partial\mathbf{p}}\dfrac{\mathbf{r}%
-\mathbf{r}_{2}}{\left\vert \mathbf{r}-\mathbf{r}_{2}\right\vert ^{3}%
}d\mathbf{w}_{2}%
\]
{L'\'{e}quation }$\left(  \ref{boltzman}\right)  ${\ est appel\'{e}e
\'{e}quation de {Boltzmann}.}

{Dans le cas des syst\`{e}mes gravitationnels, et contrairement au gaz de Van
Der Waals, les {collision}s sont dynamiquement inefficaces pendant de longues
p\'{e}riodes, {Chandrasekhar} montre que pour un syst\`{e}me autogravitant
compos\'{e} de }$N${\ par\-ti\-cu\-les, le temps dynamique }$T_{d}${\ et le
temps de relaxation par {collision}s }$T_{rc}${\ v\'{e}rifient (voir
\ref{chap_trelax})}
\[
T_{rc}\approx N\ln\left(  N\right)  T_{d}%
\]
{Sur une centaine de temps dynamiques, les amas {globulaire}s et {galaxie}s en
tous genres sont dits non {collision}nels, i.e. }$C\left(  \mathbf{w}%
,t\right)  \equiv0${, et l'\'{e}quation de la dynamique des {galaxie}s est
l'\'{e}quation de {Vlasov} ({Boltzmann} sans {collision}s)}
\[
\dfrac{\partial f}{\partial t}+\dfrac{\mathbf{p}}{m}\ \dfrac{\partial
f}{\partial\mathbf{r}}-m\dfrac{\partial f}{\partial\mathbf{p}}\dfrac
{\partial\psi}{\partial\mathbf{r}}=0
\]
{dans cette \'{e}quation le {potentiel} }$\psi${\ est lui-m\^{e}me reli\'{e}
\`{a} la fonction de distribution }$f${\ via l'\'{e}quation de {Poisson}, nous
avons donc le syst\`{e}me int\'{e}gro-diff\'{e}rentiel suivant }
\[
\left\{
\begin{array}
[c]{l}%
\dfrac{\partial f}{\partial t}+\dfrac{\mathbf{p}}{m}\ \dfrac{\partial
f}{\partial\mathbf{r}}-m\dfrac{\partial f}{\partial\mathbf{p}}\dfrac
{\partial\psi}{\partial\mathbf{r}}=0\\
\\
\psi(\mathbf{r},t\mathbf{)}=~-Gm\displaystyle\int\dfrac{f(\mathbf{r}^{\prime
},\mathbf{p}^{\prime},t)}{\mid\mathbf{r}-\mathbf{r}^{\prime}\mid}%
d^{3}\mathbf{r}^{\prime}d^{3}\mathbf{p}^{\prime}%
\end{array}
\right.
\]
{dit syst\`{e}me de {Vlasov}-{Poisson}, en introduisant l'\'{e}nergie
{moyen}ne par particule }
\[
E=\frac{\mathbf{p}^{2}}{2m}+m\psi
\]
{ce syst\`{e}me s'\'{e}crit sous forme {canonique} }
\[
\left\{
\begin{array}
[c]{lll}%
\dfrac{\partial f}{\partial t}=\left\{  \,E\,,\,f\,\right\}   &  &
\text{{\ {Vlasov} : dynamique}}\\
\Delta\psi=~4\pi G\rho &  & \text{{\ {Poisson} : champ {moyen}}}%
\end{array}
\right.
\]


{Ce syst\`{e}me est \`{a} la base de l'\'{e}tude de la dynamique des
{galaxie}s. Nous proposons deux axes principaux d'\'{e}tudes pour ce
probl\`{e}me : }

\begin{itemize}
\item {L'\'{e}tude des solutions stationnaires du syst\`{e}me de
{Vlasov}-{Poisson} qui devrait nous permettre de rendre compte des diverses
propri\'{e}t\'{e}s des {galaxie}s lorsqu'elles peuvent \^{e}tre
consid\'{e}r\'{e}es en \'{e}quilibre. }

\item {L'\'{e}tude de la stabilit\'{e} des solutions stationnaires du
syst\`{e}me de {Vlasov}-{Poisson} afin de conna\^{\i}tre les configurations
d'\'{e}quilibre privil\'{e}gi\'{e}es pour un syst\`{e}me auto-gravitant. }
\end{itemize}


