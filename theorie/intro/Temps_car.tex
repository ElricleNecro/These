\section{Temps caractéristiques\label{chap_trelax}}

Les systèmes autogravitants possèdent plusieurs temps caractéristiques dont la connaissance est importante pour comprendre la dynamique.

\subsection{Le temps de croisement et le temps dynamique}

% Le temps de croisement $T_{cr}$ d'un système physique est défini comme le rapport d'une vitesse caractéristique par une taille caractéristique. La
% vitesse $V$ est par exemple celle d'un constituant représentatif du système, ou une dispersion de vitesse $\sigma=\sqrt{\left\langle
% V^{2}\right\rangle}$:
Pour un système quelconque, le temps de croisement est défini comme le temps caractéristique que mettrait une particule à parcourir le système dans son ensemble. Il est défini
comme le rapport entre une taille caractéristique et une vitesse caractéristique du système, typiquement la dispersion de vitesse du système $\sigma=\sqrt{\left\langle v^2\right\rangle}$:
\begin{align*}
	T_{cr}\propto\frac{R}{\sigma}%
\end{align*}
Le temps dynamique correspond au temps de croisement pour un système autogravitant isolé à l'équilibre. Le théorème du viriel donne:
\begin{align*}
	2E_{c}+E_{p}=0
\end{align*}
Pour un système de taille caractéristique $R$ composé de $N$ particules de masse $m$ et dont la dispersion de vitesse est $\sigma$ nous pouvons écrire:
\begin{align*}
	E_{c}=\frac{1}{2}Nm\sigma^{2}\ \ \ \text{et\ \ \ }E_{p}=-\frac{G\left(Nm\right)^{2}}{R}%
\end{align*}
le viriel donne donc:
\begin{align*}
	\sigma=\left(  \frac{GNm}{R}\right)  ^{1/2}=\left(  \frac{GM}{R}\right) ^{1/2}%
\end{align*}
En remplaçant la dispersion de vitesse $\sigma$ dans le temps de croisement par ce résultat, nous obtenons le temps dynamique:
\begin{align*}
	T_{d}\propto\sqrt{\frac{R^{3}}{GM}}%
\end{align*}
Le rapport $R^{3}/M$ est proportionnel à l'inverse de la densité moyenne de masse $\overline{\rho}$ du système, nous avons alors:
\begin{align}
	T_{d}\propto\left(  G\overline{\rho}\right)  ^{-1/2} \label{def:T-dyn_tcr} % \propto T_{cr}
\end{align}

Le problème de la relation~\refeq{def:T-dyn_tcr} est la définition précise de la densité moyenne. Si l'on se contente d'un ordre de grandeur pour $T_d$,
la question ne se pose pas mais pour l'analyse précise de simulation il faudra faire un choix. Nous aurons l'occasion d'y revenir dans la
section~\ref{Sec::2ndStudy} du chapitre~\ref{Chap::Resultat}.
Une option est de se donner un potentiel ou une répartition de masse et de calculer $\bar{\rho}$. % dans un sens statistique.
Nous pouvons alors écrire l'équation du mouvement d'une particule test à l'intérieur du système.
% Plusieurs temps caractéristiques apparaissent alors:
% \begin{itemize}

	% \item le temps de chute libre est celui pris par une particule pour rejoindre le centre depuis le bord du système;

	% \item le temps orbital est celui mis par une particule pour effectuer une période de son mouvement radial.

% \end{itemize}

% Tous ces temps sont de l'ordre de grandeur de $T_{d}$.

\subsection{Le temps de relaxation à deux corps}

Considérons un système constitué de $N$ particules ponctuelles de masse $m$ avec $N\gg1$. À l'équilibre, et sur de petites échelles de temps (de
l'ordre de $T_{d})$, il semble intéressant de considérer que l'une de ces particules soit en orbite dans le champ moyen créé par toutes les autres. Sur
de plus longues échelles de temps, le passage proche d'une voisine va affecter la trajectoire et modifier la vitesse de cette
particule test. Nous définissons donc le temps de relaxation à 2 corps comme:
\begin{align}
	T_{rel}\propto\left[  \overline{\frac{1}{v^{2}}\left(  \frac{dv^{2}}{dt}\right)  }\right]  ^{-1}. \label{TREL}%
\end{align}
Il s'interprète classiquement comme le temps mis par les interactions à 2 corps\footnote{Les interactions à plus de 2 corps sont négligeables.} pour modifier significativement
$v^{2}$. Le calcul de ce temps ainsi que des définitions plus précises abondent dans la littérature (voir par exemple le chapitre 14 de\ \cite{HH}),
nous privilégions ici l'extraction d'un ordre de grandeur basé sur un raisonnement de S. Chandrasekhar.

Tous les corps ont la même masse $m$. Lors d'un passage à une distance $p$ d'un voisin, une particule test de vitesse $v$ ressent une force de module:
\begin{align*}
	F=\frac{Gm^{2}}{p^{2}}%
\end{align*}
pendant un temps caractéristique:
\begin{align*}
	\delta\tau=\frac{p}{v}
\end{align*}
La variation de vitesse $\delta v$ de la particule test induite par cette rencontre peut s'obtenir en utilisant le principe fondamental de la dynamique:
\begin{align}
	m\frac{\delta v}{\delta\tau}=F\Rightarrow\delta v=\frac{F\delta\tau}{m}=\frac{Gm}{pv} \label{deltavrel}
\end{align}
Le paramètre $p$ est souvent appelé paramètre d'impact de la rencontre qui prend alors le statut de collision. Supposons que la particule test soit
en orbite circulaire de rayon $r$ dans le système de densité $\rho$. Sur une période, le nombre de collisions avec
un paramètre d'impact compris entre $p$ et $p+dp$ s'écrit:
\begin{align}
	dn  &= \left(  2\pi pdp\right)  \times\left(  2\pi r\right)  \times\left(\frac{\rho\left(  r\right)  }{m}\right) \label{dnrel}\\
	    &= \frac{4\pi^{2}}{m}pr\rho\left(  r\right)  dp
\end{align}
Les collisions sont suffisamment aléatoires pour que la moyenne des variations de vitesse $\overline{\delta v}$ soit nulle. Pour obtenir l'action des
collisions sur la vitesse, il faut donc considérer une dispersion de vitesse. Sur une période, la variation $\left(\Delta v^{2}\right)_{orb}$ de
$\delta v^{2}$ s'obtient en combinant~\refeq{deltavrel} et~\refeq{dnrel} et en sommant sur toutes les valeurs possibles de $p$, soit:
\begin{align*}
	\left(\Delta v^{2}\right)_{orb}=\int_{p_{\min}}^{p_{\max}}\delta v^{2}dn=\frac{4\pi^{2}G^{2}mr\rho\left(r\right)}{v^{2}}\int_{p_{\min}}^{p_{\max}}\frac{dp}{p}
\end{align*}
Si l'on veut éviter les infinis, il est nécessaire à ce stade d'introduire une coupure haute et basse pour les valeurs de $p$. Avec nos ordres de
grandeurs et en suivant le calcul de S. Chandrasekhar, nous introduisons le logarithme coulombien:
\begin{align*}
	\ln\Lambda=\ln\left(\frac{p_{\max}}{p_{\min}}\right)
\end{align*}
où $p_\mathrm{max}$ et $p_\mathrm{min}$ vont être des grandeurs caractéristiques du système.
% et utiliseront le fait que $p_{\max}=r$ tandis que $p_{\min}$ peut correspondre au rayon d'une particule test.
% que nous estimons par la relation crue%
% \begin{align*}
	% p_{\min}=\left(\frac{\frac{4}{3}\pi r^{3}}{N}\right)^{1/3}=\left(\frac{4\pi}{3N}\right)  ^{1/3}r
% \end{align*}
% ainsi%
% \begin{align*}
	% \ln\Lambda &=-\frac{1}{3}\ln\left(  \frac{4\pi}{3N}\right)=-\frac{1}{3}\left[\ln\left(\frac{4\pi}{3}\right)-\ln\left(N\right)\right] \\
	     % &\approx\frac{1}{3}\ln N
% \end{align*}
Finalement, sur une période nous obtenons:
\begin{align*}
	\left(\Delta v^{2}\right)_{orb}=4\pi^{2}\frac{G^{2}mr\rho\left(r\right)}{v^{2}}\ln \Lambda
\end{align*}
% Toujours sur une période nous avons vu au paragraphe précédent que:
Le temps dynamique étant à peu près le temps que met une particule à parcourir son orbite, nous pouvons écrire:
\begin{align*}
	\left(  \Delta t\right)  _{orb}\approx T_{d}%
\end{align*}
Ce qui nous permet d'écrire:
% il n'en faut pas plus pour écrire%
\begin{align*}
	\frac{1}{v^{2}}\left(\frac{dv^{2}}{dt}\right)_{orb}\approx4\pi^{2}\frac{G^{2}mr\rho\left(r\right)}{v^{4}}\frac{\ln \Lambda}{T_{d}}%
\end{align*}
La moyenne permettant de calculer le temps de relaxation à 2 corps peut s'envisager sur une période, l'expression~\refeq{TREL} s'écrit donc:
\begin{align*}
	\frac{T_{rel}}{T_{d}}\approx\frac{1}{4\pi^{2}}\frac{v^{4}}{G^{2}mr\rho\left(  r\right)  }\frac{1}{\ln \Lambda}%
\end{align*}
le théorème du viriel, car nous sommes à l'équilibre, nous donne:
\begin{align*}
	v^{4}=\left(\frac{GNm}{r}\right)^{2}%
\end{align*}
ainsi:
\begin{align*}
	\frac{T_{rel}}{T_{d}}\approx\frac{1}{4\pi^{2}}\frac{m}{r^{3}\rho\left(r\right)}\frac{N^{2}}{\ln \Lambda}
\end{align*}
% Au point où nous en sommes on peut toujours tenter
Nous écrivons:
\begin{align*}
	\rho\left(r\right)=\frac{Nm}{\frac{4}{3}\pi r^{3}}%
\end{align*}
et nous obtenons finalement:
\begin{align*}
	\frac{T_{rel}}{T_{d}}\approx\frac{3}{16\pi^{3}}\frac{N}{\ln \Lambda}%
\end{align*}
Dans un amas d'étoiles ou une galaxie, le temps de relaxation à deux corps est donc généralement beaucoup plus grand que le temps dynamique. Le
tableau~\ref{Tab::TempsCarac::OG} donne un ordre de grandeur de ces temps pour différents objets. %résume bien l'affaire
\begin{table}[htbp]
	\centering \begin{tabular}[c]{l|c|c|c|c|c|}
		\cline{2-6} & $N$ & $R\left[  \text{kpc}\right]  $ & $\sigma\left[  \text{km.s}^{-1}\right]$ & $T_{dyn}\left[  \text{Gan}\right]  $ & $T_{rel}\left[\text{Gan}\right]$\\\hline
		\multicolumn{0}{|l|}{\small {Amas ouverts}} & $250$ & $1\times10^{-3}$ & $1$ & $1\times10^{-3}$ & $8\times10^{-4}$\\\hline
		\multicolumn{0}{|l|}{\small {Amas globulaires}} & $5\times10^{5}$ & $1\times10^{-2}$ & $7$ & $1\times10^{-3}$ & $1$\\\hline
		\multicolumn{0}{|l|}{\small {Galaxies elliptiques}} & $10^{11}$ & $5$ & $200$ & $2\times10^{-2}$ & $2\times10^{6}$\\\hline
		\multicolumn{0}{|l|}{\small {Groupes diffus de galaxies}} & $5$ & $400$ & $100$ & $4$ & $2\times 10^{-1}$\\\hline
		\multicolumn{0}{|l|}{\small {Groupes compacts de galaxies}} & $4$ & $40$ & $200$ & $2\times 10^{-1}$ & $1\times10^{-2}$\\\hline
		\multicolumn{0}{|l|}{\small {Amas riches de galaxies}} & $400$ & $1200$ & $700$ & $2$ & $2$\\\hline
	\end{tabular}
	\caption{Ordres de grandeur des temps dynamiques et temps de relaxation à 2 corps pour divers objets.\label{Tab::TempsCarac::OG}}
\end{table}
Avec un univers âgé de $13$ Gan, les galaxies sont très peu influencées par les collisions à deux corps % les galaxies sont loin d'avoir acquis le statut d'objet relaxé par les collisions,
l'hypothèse Vlasov
est donc entièrement justifiée. En ce qui concerne les autres objets autogravitants, il est clair que les amas ouverts relaxent très vite alors que le
statut des amas globulaires est plus discutable: l'importance des collisions dans leur dynamique est différente entre les régions centrales denses et
le halo externe plus diffus.

Notons que le calcul du temps de relaxation à deux corps que nous avons présenté ici est sous-estimé en ce qui concerne les groupes et amas de galaxies,
pour lesquels l'effet de la matière noire doit être pris en compte.

Un phénomène de ségrégation de masse est associé au processus de collision. Une fois encore, ce calcul ne le prend pas en compte. Une rencontre
entre deux étoiles de masses différentes ne produit pas la même variation de vitesse. L'effet net de cette ségrégation est de ralentir les étoiles
les plus lourdes et donc de les faire tomber dans le puits de potentiel formé par le système. Sur l'échelle de temps de la relaxation, nous devrions
observer une organisation des objets dans le système: les plus lourds au centre, les plus légers à l'extérieur.

Dans le cadre de simulations numériques, l'estimation du temps de relaxation est une véritable question. Les valeurs de $p_{\min}$ et $p_{\max}$
utilisées pour le calcul du logarithme coulombien $\ln \Lambda$ doivent s'adapter à la simulation tout comme l'estimation de la densité. Nous
présenterons notre stratégie en la matière dans la section~\ref{Sec::2ndStudy} du chapitre~\ref{Chap::Resultat}.
