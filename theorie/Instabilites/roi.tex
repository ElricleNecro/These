\section{L'instabilit\'{e} d'orbites radiales}

L'un des vieux probl\`{e}mes de la dynamique galactique est
l'\'{e}tude des syst\`{e}mes autogravitants form\'{e}s de particules
majoritairement en orbite radiale, tr\`{e}s allong\'{e}es et passant donc pr\`{e}s
du centre du syst\`{e}me. Il est tr\`{e}s vite apparu que de tels syst\`{e}mes
initialement sph\'{e}riques pourraient \^{e}tre instables et perdre leur
sym\'{e}trie dans ce qu'il est convenu d'appeler l'instabilit\'{e} d'orbite
radiale IOR \ ou ROI\ en anglais. Ce m\'{e}canisme pourrait m\^{e}me \^{e}tre
\`{a} l'origine de la forme de certains objets \`{a} l'\'{e}chelle galactique
attendu que la gravitation peine \`{a} former des structures poss\'{e}dant une
direction privil\'{e}gi\'{e}e.


\subsection{L'histoire de l'instabilité d'orbites radiales}

\subsubsection{Les pionniers}

Le premier r\'{e}sultat publi\'{e} est l'oeuvre de Vadim Antonov
\cite{antonov}. Il s'agit d'un r\'{e}sultat analytique concernant un syst\`{e}me
constitu\'{e} de $N$ particules.  L'\'{e}quation de Poisson correspondante est
\'{e}tudi\'{e}e en perturbation dans un cas limite correspondant aux orbites
radiales. Un syst\`{e}me diff\'{e}rentiel est alors produit pour un certain
"d\'{e}placement" des orbites. Le r\'{e}sultat est obtenu en exhibant une
fonction de Lyapunov stricte pour ce syst\`{e}me diff\'{e}rentiel. L'article
n'est vraiment pas tr\`{e}s clair ...

La m\^{e}me ann\'{e}e 1973, Michel H\'{e}non \cite{henon} publie l'un des tout
premier r\'{e}sultats num\'{e}rique dans ce domaine. Utilisant $N=1000$
coquilles concentriques distribu\'{e}es selon un mod\`{e}le
polytropique $f\left(  E\right)  \propto E^{n}$, il v\'{e}rifie la
stabilit\'{e} obtenue analytiquement par Antonov des les ann\'{e}es 60 pour ce type de systèmes. Puis,
en \'{e}tendant son mod\`{e}le au cas anisotrope (dans l'espace des vitesse)
des polytropes g\'{e}n\'{e}ralis\'{e}s  $f\left(  E\right)  \propto
E^{n}L^{2m}$, il montre num\'{e}riquement que le syst\`{e}me devient instable
lorsque $m\rightarrow-1$, qu'il identifie \`{a} la fonction de distribution
 $f\left(  E\right)  \propto E^{n}\delta\left(  L^{2}\right) $ et donc aux
syst\`{e}mes pr\'{e}sentant de plus en plus d'orbites radiales.
L'instabilit\'{e} est identifi\'{e}e dans l'espace des phases qui montre une
\'{e}volution du syst\`{e}me d\`{e}s que celui-ci est trop anisotrope dans
l'espace des vitesses. Le m\'{e}canisme de cette instabilit\'{e} n'est pas
pr\'{e}cis\'{e}, l'effet de cette instabilit\'{e} dans l'espace des positions
(la fameuse barre) n'est pas non plus abordé. Cet article ne fait pas
r\'{e}f\'{e}rence au travail d'Antonov \cite{antonov}. L'utilisation de coquilles
sph\'{e}riques ne permet pas de prendre en compte d'\'{e}ventuelles
interactions non radiales.

Un pav\'{e} dans la marre : en utilisant les m\'{e}thodes de Water-Bag, qui
consistent \`{a} d\'{e}composer la fonction de distribution sur une base de
fonctions continues par morceaux, l'\'{e}quipe fran\c{c}aise \cite{waterbag},
aur\'{e}ol\'{e}e par son succ\`{e}s dans le cas isotrope\footnote{Cinq ans
plus t\^{o}t, Doremus, Feix et Baumann avaient obtenu la stabilit\'{e} des
syst\`{e}mes isotropes dans l'espace des vitesses par la m\^{e}me
m\'{e}thode.}, publie un r\'{e}sultat affirmant la stabilit\'{e} de tous les
syst\`{e}mes autogravitants $f\left(  E,L^{2}\right)  $ contre des
perturbations non-sph\'{e}riques\footnote{La seule condition requise est le
fait que $\partial f/\partial E<0$ et $\partial f/\partial L^{2}<0$, ce qui
correspond au cas physique.}. Antonov ayant d\'{e}j\`{a} r\'{e}gl\'{e} le cas
des perturbations \`{a} sym\'{e}trie sph\'{e}rique pour ces m\^{e}me
syst\`{e}mes anisotropes, la boucle \'{e}tait boucl\'{e}e : tout syst\`{e}me
autogravitant sph\'{e}rique (isotrope ou non dans l'espace des vitesses) est
stable contre toutes les formes de perturbations. Le cas des syst\`{e}mes
constitu\'{e}s d'orbites de plus en plus radiales, inclus dans le champ
d'action du r\'{e}sultat de l'\'{e}quipe fran\c{c}aise est donc pr\'{e}dit
stable par cette analyse, en d\'{e}saccord patent avec le r\'{e}sultat
analytique d'Antonov \cite{antonov}, et les simulations de Michel H\'{e}non \cite{henon}.

\subsubsection{Des avancées conséquentes\label{ROIadvances}}

Sans faire r\'{e}f\'{e}rence au r\'{e}sultat de l'équipe française, Polyachenko et
G. Shukhman \cite{polyach} proposent une formulation matricielle du probl\`{e}me de
la stabilit\'{e} (bas\'{e}e sur une d\'{e}composition en s\'{e}rie de Fourier
des pertubations) qui leur permet de prouver qu'une fonction de distribution
de la forme $f\left(  E-\lambda L^{2}/r_{a}^{2}\right)$, mod\`{e}le que l'on appelera plus tard
Ossipkov-Merritt, d\'{e}crit un mod\`{e}le instable si $r_{a}^{2}$ est
suffisament petit. L'article n'est pas d'une grande clart\'{e} mais le
r\'{e}sultat est tout \`{a} fait contraire \`{a} celui propos\'{e} par
\cite{waterbag}, et va dans le sens d'Antonov et H\'{e}non. Cet article contient de
plus un crit\`{e}re de stabilit\'{e} qui sera repris dans le livre de
Friedmann et Polyachenko, et qui affirme qu'un syst\`{e}me sph\'{e}rique
d\'{e}clenche une instabilit\'{e} d'orbite radiale d\`{e}s que le rapport
$2T_{r}/T_{\perp}>1.75\pm0.25$ o\`{u} $T_{r}$ et $T_{\perp}$
repr\'{e}sentent respectivement les \'{e}nergies cin\'{e}tiques radiale et
perpendiculaire totales contenues dans le syst\`{e}me.

La premi\`{e}re \'{e}tude num\'{e}rique globale et à peu près réaliste du probl\`{e}me de
l'effondrement gravitationel est effectu\'{e}e par van Albada au d\'{e}but des
ann\'{e}es 80 \cite{albada}. Cette \'{e}tude consid\`{e}re des ensembles de
$N=5000$ particules de m\^{e}me masse dont les conditions initiales sont r\'{e}parties en 2 cat\'{e}gories : les sph\`{e}res homog\`{e}nes de taille 1
et des syst\`{e}mes compos\'{e}s de 20 sph\`{e}res homog\`{e}nes (clumps)
contenant chacune 250 particules. Ces clumps poss\`{e}dent initialement un
rayon \'{e}gal \`{a} 0,4 et leurs centres sont positionn\'{e}s
uniform\'{e}ment dans une boule de rayon 1$.$ Ces diff\'{e}rentes conditions
initiales sont abandonn\'{e}es \`{a} leur gravit\'{e} dans 3 conditions
initales de vitesse d\'{e}termin\'{e}es par le rapport du viriel initial :
$-2T/U=0.5,~0.2$ et $0.1$. Le r\'{e}sultat est clair : les sph\`{e}res
homog\`{e}nes ne souffrent pas (dans les cas d'effondrement
condid\'{e}r\'{e}s) d'instabilit\'{e} d'orbite radiale. Par contre, les
assemblages de grumeaux qui s'\'{e}ffondrent violemment ($-2T/U=0.1$%
)~produisent un \'{e}quilibre triaxial. M\^{e}me si ce papier ne parle pas de
l'instabilit\'{e} d'orbite radiale, il confirme que les profils (lumi\`{e}re,
densit\'{e}) obtenus num\'{e}riquement sont compatibles avec ceux qui sont
observ\'{e}s pour les galaxies (loi en $r^{1/4}$).

Dans un papier \`{a} la fois num\'{e}rique et analytique, Barnes, Goodmann et
Hut \cite{barneshut} abordent l'une des premi\`{e}res \'{e}tudes globale de
l'instabilit\'{e}. L'\'{e}tude num\'{e}rique consiste \`{a} refaire les
exp\'{e}riences avec des coquilles de Michel H\'{e}non en utilisant maintenant
des techniques \`{a} $N$ corps pour des valeurs de $N$ comprises entre
$10^{3}$ et $10^{4}$. Ils confirment les r\'{e}sultats de leur
pr\'{e}d\'{e}cesseur en faisant une \'{e}tude plus exhaustive dans l'espace
des param\`{e}tres des modèles utilisés par Hénon, Polyachenko et
Shukhman;  ils confirment d'ailleurs
le crit\`{e}re de stabilit\'{e} des russes et proposent une explication
analytique pas tr\`{e}s convaincante qui ferait passer l'instabilit\'{e} d'orbite radiale pour une sorte d'instabilit\'{e} de
Jeans\footnote{Instabilit\'{e} qui survient lorsque la pression cin\'{e}tique
ne suffit plus \`{a} compenser la tendance qu'\`{a} le syst\`{e}me \`{a}
s'effondrer sous l'effet de son propre poids.} : la pression stellaire dans la
direction tangentielle devient insuffisante pour compenser la tendance
naturelle qu'ont les orbites radiales \`{a} se condenser. Il est à noter que
l'\'{e}cole russe propose le m\^{e}me style d'interpr\'{e}tation pour
l'instabilit\'{e} dans le livre de Polyachenko et Friedmann (Vol2, p. 148).

Le travail suivant sur le sujet est publi\'{e} par Merritt et Aguilar
en 1985 \cite{merritt_aguilar}! Bien qu'ant\'{e}rieur en date de publication, il fait
r\'{e}f\'{e}rence aux travaux pr\'{e}c\'{e}dents de Barnes et al., en le
r\'{e}sumant magnifiquement en quelques lignes... Il se concentre sur les
aspects num\'{e}rique et sur l'opportunit\'{e} que repr\'{e}sente cette
instabilit\'{e} dans le contexte de la formation des galaxies, c'est la
premi\`{e}re fois que cette id\'{e}e surgit dans la litt\'{e}rature. Les auteurs
utilisent une premi\`{e}re famille de syst\`{e}mes dont le profil de
densit\'{e} est \og de type galactique\fg\, $:\rho\left(  r\right)  $ $\propto
(r/r_{o})^{-2}(1+r/r_{o})^{-2}$ (c'est le fameux mod\`{e}le de Jaffe qui
est compatible avec le profil de luminosit\'{e} en $r^{1/4}$ ). Il poss\`{e}de
en outre la bonne propri\'{e}t\'{e} d'\^{e}tre facilement transposable en un
mod\`{e}le anisotrope en suivant l'algorithme
d'Ossipkov-Merritt\footnote{Il s'agit de modèles sphériques dont la fonction de distribution étend un modèle isotrope à un système présentant une anisotropie radiale de plus en plus forte en s'éloignant du centre du système.}. Il s'agit d'exp\'{e}riences \`{a}
$N$ corps avec $N=5\cdot10^{3}$, l'\'{e}tat initial est un \'{e}quilibre
dont le degr\'{e} d'anisotropie radiale est contr\^{o}l\'{e}
par la valeur de $r_{o}$. Les conclusions sont les suivantes : la
transition stable/instable est tr\`{e}s rapide et elle se produit pour un
rapport $2T_{r}/T_{\perp}\approx2.5$, soit un peu plus que ce qui est
pr\'{e}vu par le crit\`{e}re russe. L'\'{e}tude de deux familles
compl\'{e}mentaires ( l'une avec une anisotropie ind\'{e}pendante du rayon et
l'autre avec une fonction de distribution d\'{e}croissante en $E$
\emph{et} en $L^{2}$) semble indiquer d'une part que la valeur de
$2T_{r}/T_{\perp}$ n'est pas un crit\`{e}re de stabilit\'{e} et d'autre part
que le r\'{e}sultat \cite{waterbag} de Gillon et al. est d\'{e}finitivement
infirm\'{e}. L'id\'{e}e de l'importance de cette instabilit\'{e} dans le
processus de formation des galaxies est avanc\'{e}e en conclusion de l'article, en reprenant ses termes \og elle ne doit pas \^{e}tre \'{e}cart\'{e}e\fg\,.

Une \'{e}tude purement analytique d'une \'{e}quipe anglaise (Palmer \&
Papaloizou, \cite{palmerpapa}) bas\'{e}e sur une analyse spectrale de la perturbation
d'un syst\`{e}me sph\'{e}rique anisotrope conclut \`{a} l'instabilit\'{e}.
C'est, depuis les r\'{e}sultats russes et hormis le r\'{e}sultat water-bag
de Gillon et al., la seule approche analytique frontale de ce probl\`{e}me et
toujours par des m\'{e}thodes consistant \`{a} d\'{e}composer les
perturbations sur des bases de fonctions orthogonales. Bien que le
r\'{e}sultat soit affirm\'{e}, il semble tr\`{e}s difficilement
v\'{e}rifiable... Deux autres aspects importants de cet article sont
la \og démonstration\fg\,de la non validit\'{e} du crit\`{e}re de stabilit\'{e} russe
déjà \'{e}corn\'{e} par Merritt et Aguilar et la pr\'{e}sentation d'un
nouveau m\'{e}canisme pour la croissance de l'instabilit\'{e} inspir\'{e} d'un
travail de Lynden-Bell \cite{lyndenbell}. Ce dernier point m\'{e}rite une attention
particuli\`{e}re. Le travail de Lynden-Bell \'{e}tudie l'influence d'une
perturbation axisym\'{e}trique dans le plan de sym\'{e}trie d'un potentiel de
galaxie spirale sur l'orbite d'une \'{e}toile. Il tend \`{a} montrer un effet
d'\'{e}longation des orbites qui ont alors tendance \`{a} s'aligner le long de
la perturbation. Cet effet serait \`{a} l'\oe uvre dans la formation des
barres des galaxies spirales. Pour la premi\`{e}re fois dans l'histoire de
l'instabilit\'{e} d'orbites radiales, Palmer et Papaloizou sugg\`{e}rent que
c'est le m\'{e}canisme de Lynden-Bell qui est \`{a} l'\oe uvre.

Une magnifique synth\`{e}se de tous ces r\'{e}sultats est effectu\'{e}e par
D. Merritt  \cite{merritt1987} en 1987. Les 2 m\'{e}canismes sont
d\'{e}taill\'{e}s et expliqu\'{e}s, l'instabilit\'{e} de Jean est
critiqu\'{e}e car elle nécessite un syst\`{e}me homog\`{e}ne ce qui
n'est pas le cas, D. Merritt met donc en avant le m\'{e}canisme de Lynden
Bell qu'il d\'{e}crit remarquablement.
 
Un article très intéressant de N. Katz \cite{katz} propose alors de nouvelles simulations cosmologiques montrant que le processus hiérarchique de formation des structures cosmologiques avec ses effondrements successifs tend à gommer les traces possibles d'une instabilité d'orbite radiale qui aurait pu se produire dans les phases initiales de cette formation. 

En cette
m\^{e}me ann\'{e}e 1991, un article de P. Saha \cite{saha} vient \'{e}tendre la
port\'{e}e des m\'{e}thodes spectrales de modes normaux aux syst\`{e}mes
d'extension infinie, ce qui n'\'{e}tait apparement pas le cas des \'{e}tudes
pr\'{e}c\'{e}dentes. Toujours en 1991, une \'{e}tude de D. Weinberg
\cite{weinberg}, reprend les m\'{e}thodes matricielles initi\'{e}es par
l'\'{e}cole russe de Polyachenko, retrouve des r\'{e}sultats et pr\'{e}sente
apparement dans sa section IV-c une analyse d\'{e}taill\'{e}e du m\'{e}canisme
de Lynden-Bell appliqu\'{e} \`{a} l'instabilit\'{e} d'orbites radiales. Cette analyse est en fait l'objet d'un article
complet et assez clair d'une \'{e}quipe argentine \cite{cincotta} qui \'{e}tudie la transformation
d'orbites de type boucle en type boite, ce qui est dans la veine du
m\'{e}canisme de Lynden-Bell et confirme l'intuition de Merritt \cite{merritt1987}.

\subsubsection{Un regain d'intérêt}


Mettant \`{a} profit certains de leurs r\'{e}sultats analytiques \cite{JPerez96}, 
Perez et al. \cite{perez_et_al} proposent et testent un crit\`{e}re de
stabilit\'{e} pour les syst\`{e}mes auto-gravitants construit sur la nature
des perturbations qu'il re\c{c}oit. Ce crit\`{e}re est valid\'{e} sur des
mod\`{e}le Ossipkov-Merrit appliqu\'{e}s \`{a} des polytropes. le nombre de
particules mis en jeu devient pour lapremi\`{e}re fois raisonnable
$N\approx10^{4}$ pour l'ensemble des simulations. Dans leurs résultats analytiques, ils expliquent la carrence des méthodes de water-bag dans le domaine des orbites radiales, ce qui pourrait expliquer le désomais défunt résultat \cite{waterbag}.

 Une \'{e}tude num\'{e}rique syst\'{e}matique de l'instabilit\'{e} d'orbite radiale, utilisant les machines d\'{e}di\'{e}es "GRAPE",  est produite par
une \'{e}quipe allemande \cite{theis} en 1999. Les simulations effectu\'{e}es sont
des effondrements de sph\`{e}res de Plummer de temp\'{e}rature initiale
variable. Le taux de croissance de ROI est grandement affect\'{e} par le
softening $\epsilon$ du potentiel et tr\`{e}s peu par des variations du nombre de
particules. Ces simulations ont mis en \'{e}vidence une \'{e}volution à
tr\`{e}s long terme (de l'ordre du temps de relaxation à 2 corps) du syst\`{e}me triaxial produit par ROI vers un
syst\`{e}me plus ou moins sph\'{e}rique, selon les auteurs cette transformation est due aux interactions à deux particules.

Un \'{e}tude syst\'{e}matique de l'effondrement gravitationnel (collapse) avec test des
param\`{e}tres num\'{e}riques ($N,\varepsilon$, ...) par Roy et Perez
\cite{roy}, permet, entre autres r\'{e}sultats, de mettre en \'{e}vidence un
aspect pressenti de l'instabilité d'orbites radiales. Son d\'{e}clenchement dans un collapse est
subordonn\'{e} \`{a} la pr\'{e}sence d'inhomog\'{e}n\'{e}it\'{e}s robustes.
Ce ne sont en effet que les effondrements au moins deux phase successives qui sont le si\`{e}ge
de ROI : l'effondrement d'une sphère homogène ne remplit pas cette condition. Ces r\'{e}sultats sont compl\'{e}t\'{e}s et raffin\'{e}s par Boily et
Athanassoula \cite{boily} qui montrent un l\'{e}ger effet du nombre de
particules sur l'\'{e}tat final de ROI.

Bien que le r\^{o}le de ROI dans la formation des structures ait été atténué par le travail de Katz \cite{katz} cit\'{e} plus haut, deux analyses
compl\'{e}mentaires par deux \'{e}quipes allemandes \cite{huss} et canadienne \cite{macmillan} observent le r\'{e}sultat de la formation de structure à moyenne \'{e}chelle par des exp\'{e}riences de collapse en se donnant la possibilit\'{e} de supprimer numériquement l'instabilit\'{e} d'orbite radiale (en retirant la composante radiale de la force de gravitation...). 
Ils remarquent que si l'on empèche ROI dans les phases primordiales, on modifie le profil de densité final de ces structures formées dans un contexte cosmologique : on obtient un profil à deux pentes au lieu de trois dans NFW par exemple). La forme allongée que prend le système à cause de l'instabilité d'orbite radiale serait donc progressivement gommée par le processus de formation hierarchique comme l'a remarqué Katz, mais le profil de densité final garderait subtilement sa trace. Les différents résultats observationnels ou numériques indiquent clairement la présence de cette marque...   

Un regain d'activit\'{e} dans le domaine se manifeste sous l'impulsion d'une
\'{e}quipe dirig\'{e}e par E. Barnes depuis 2005. Ce que l'on peut reternir des articles de cette équipe, notamment \cite{barnes2005} et \cite{ROI_Moderne} est la chose suivante : depuis longtemps on
sait que ROI produit un syst\`{e}me triaxial dans l'espace des positions, mais
il cr\'{e}e aussi une s\'{e}gr\'{e}gation spatiale dans l'espace des vitesse
(centre isotrope et halo radial). C'est cette s\'{e}gr\'{e}gation qui serait
\`{a} l'origine du profil universel que l'on observe dans les grandes
structures. Dans le cadre de ce renouveau, deux articles
\'{e}tudient ROI tout azimuts (\cite{barneslanzel}, \cite{trenti}) avec de nombreux d\'{e}tails 
bien expliqu\'{e}s, des comparaisons de codes,  mais rien de neuf.  L'équipe italienne \cite{trenti} s'\'{e}tonne de ne
pas pouvoir produire d'instabilité d'orbite radiale lors de l'effondrement d'une sphère homogène - dite de Hénon. Leur
explication à ce sujet est discutable  : l'effondrement serait trop rapide ou l'anisotropie ne serait pas
suffisante...). L'explication de ce ph\'{e}nom\`{e}ne avait pourtant \'{e}t\'{e} propos\'{e}e par Roy et Perez
\cite{roy} : un effondrement monolitique (sphère de Hénon) se produit en une seule phase.  Le germe dont à besoin ROI pour se développer n'est donc pas présent dans le cas générique de ce type d'effondrement. 


Enfin ROI aurait \'{e}t\'{e} observ\'{e}e dans un \'{e}quilibre triaxial
\cite{antonini} : elle se produirait lorsque ce dernier serait trop peupl\'{e} d'obites en forme de boites. Le syst\`{e}me deviendrait
alors plus prolate et toujours triaxial.



Comme le montre cette perspective historique, l'instabilité d'orbite radiale a connu des développements controversés. Elle demeure cependant fondamentale dans le processus de formation hiérarchique des structures gravitationnelles. La dernière avancée dans la compréhension de son mécanisme ainsi que la preuve de l'instabilité par des méthodes d'énergie se trouve dans le travail de Perez et Maréchal \cite{future}. Les principaux éléments de ce travail sont présentés dans la section suivante.  

%%%%%%%%%%%%%%%%%%%%%%%%%%%%%%%%%%%%%%%%%%%%%%%%%%%%%%%%%%%%%%%%
\subsection{La méthode symplectique}
%%%%%%%%%%%%%%%%%%%%%%%%%%%%%%%%%%%%%%%%%%%%%%%%%%%%%%%%%%%%%%%%

\subsubsection{Présentation de la méthode}

La méthode symplectique à été introduite dans le contexte de la stabilité des systèmes autogravitants par Bartholomew \cite{bartho}, elle est issue de la physique des plasmas. Elle fut popularisée par le regrété Henry Kandrup \cite{kandrupstability}.

Il s'agit de tirer parti de la structure hamiltonienne associée au système de Vlasov--Poisson. Dans ce contexte, on remarque tout d'abord que l'énergie totale contenue dans le système autogravitant décrit par la fonction de distribution $f$, i.e.
\[
	H \left[ f \right]  =
	\int \mathrm{d} {\Gamma}
		\frac{\mathbf{p}^{2}}{2m} f \left( {\Gamma},t \right)
	- \frac{1}{2} Gm^2 \int \mathrm{d} {\Gamma} \int\mathrm{d}{\Gamma}^{\prime}
		\frac{f \left({\Gamma},t \right) f\left( {\Gamma}^{\prime},t\right)}%
		{\left\vert \mathbf{q} - \mathbf{q}^{\prime} \right\vert }
\]

est telle que sa dérivée fonctionnelle est l'énergie moyenne d'une particule test :
\[
\frac{\delta H}{\delta f}=\lim_{\delta f \to 0}=\dfrac{H\left[ f +\delta f\right]-H \left[ f \right]}{\delta f}
= \frac{\mathbf{p}^{2}}{2m} + m \psi=E
\]
Si $K[f]$ est une fonctionnelle dérivable de la fonction de distribution nous pouvons donc écrire
\begin{equation}
	\frac{\mathrm{d} K[f]}{\mathrm{d} t}
	= \int \frac{\delta K}{\delta f}
		\frac{\partial f}{\partial t} \mathrm{d} \Gamma
	= \int \frac{\delta K}{\delta f} \left\{ E, f \right\} \mathrm{d} \Gamma
	\label{derivk}
\end{equation}
où nous avons utilisé la forme canonique de l'équation de Vlasov $\dot f=\left\{ E, f \right\}$.
On peut alors introduire les crochets popularisés par Phil Morrisson \cite{morrison} :
pour deux fonctionnelles $A$ et $B$ de $f$, on a 
\[
	\left[ A, B \right](f) :=
	\int f \left\{
		\frac{\delta A}{\delta f}, \frac{\delta B}{\delta f}
	\right\} \mathrm{d} \Gamma
\]
Ainsi, la relation (\ref{derivk}) s'écrit
\begin{equation}
	\frac{\mathrm{d} K[f]}{\mathrm{d} t}
	= - \int f \left\{
		\frac{\delta H}{\delta f}, \frac{\delta K}{\delta f}
	\right\} \mathrm{d} \Gamma
	= \left[K, H \right](f)
\end{equation}

Comme cela est lumineusement expliqué par Henry Kandrup  \cite{kandrupstability},
toute perturbation linéaire $f^{(1)}$ pouvant physiquement être reçue par l'état décrit par la fonction de distribution $f_0$ s'écrit
\[
	f^{(1)}\left(  {\Gamma},t \right) = -\left\{ g,f_{0}\right\}
\]
la fonction $g$ est appelée \emph{générateur} de la perturbation. Introduisons la quantité
\[
	G[f] := \int f g \mathrm{d} \Gamma
\]
A partir de cette perturbation, il est possible de construire la variation d'énergie correspondante. Au  premier ordre il vient
\begin{equation*}
	H^{(1)} [f_0] = \left[G, H \right](f_0)
	= - \int g \{ f_0, E \} \mathrm{d} \Gamma
	= 0
\end{equation*}
Si l'état $f_0$ est un équilibre, $\{ f_0, E \} = 0$ et la variation d'énergie est donc nulle à l'ordre 1. En d'autre terme La fonctionnelle $H[f]$ présente un extremum en $f=f_0$, c'est bien la définition d'un état d'équilibre.

La variation de l'énergie au second ordre  
$
	H^{(2)} [f_{0}] = \left[G, [G,H] \right](f_0)
$
ne résiste pas à quelques lignes de calculs, il vient
\begin{eqnarray}
	H^{(2)}[f_{0}]
	& = & - \int \left.\frac{\delta [G,H]}{\delta f}\right|_{f=f_{0}} \{g,f_{0}\} \mathrm{d} \Gamma
	\nonumber \\
	& = & - \int \left(
		\{g,E\} + \int \frac{Gm^2}{|\mathbf{q}-\mathbf{q'}|}
		\{g',f'_{0}\} \mathrm{d} \Gamma'
	\right) \{g,f_{0}\} \mathrm{d} \Gamma
	\nonumber \\
	& = & - \int \{g,E\} \{g,f_{0}\} \mathrm{d} \Gamma
	- G m^2 \int\!\!\!\int \frac{\{g,f_{0}\}\{g',f'_{0}\}}{|\mathbf{q} - \mathbf{q'}|}
	\mathrm{d} \Gamma \mathrm{d} \Gamma'
\end{eqnarray}

L'étude du signe de cette quantité s'est révélé un outil efficace d'investigation de la stabilité des systèmes autogravitants.

%%%%%%%%%%%%%%%%%%%%%%%%%%%%%%%%%%%%%%%

\subsubsection{Des critères de stabilité}

Dans le cas classique d'une particule soumise à l'influence de forces conservatives, la stabilité locale d'un état d'équilibre est directement reliée au signe de la variation d'énergie à l'ordre 2 au voisinage de cet état : une variation positive laisse l'équilbre stable alors qu'une variation négative de l'énergie conduit à une instabilité.

L'extension d'un tel résultat à des situations plus compliquées (forces non conservatives, systèmes de particules, etc...) n'est cependant pas triviale.

Dans le cas qui nous intéresse de la physique statistique des systèmes non collisionnels évoluant dans un champ moyen la situation n'est cependant pas  désespérée.

Si la variation $H^{(2)}$ est positive pour tous les générateurs $g$,
Bartholomew \cite{bartho} montre que le système est stable. Cette méthode a été largement mise en application dans tous les cas favorables par J. Perez et J.-J. Aly \cite{perezaly}. 
Par contre, s'il existe des générateurs $g$ conduisant à des variations $H^{(2)}$ négatives, il n'existe pas de résultat général assurant l'instabilité du système; du moins sans hypothèse complémentaire...


Dans ce contexte, les résultats  \cite{blochmarsden} mais surtout  \cite{krechet} de l'équipe de J. Marsden, permet l'étude de cas assez généraux.

Ces résultats affirment que l'existence de modes d'énergie négative ($g$ tels que $H^{(2)}<0$ dans notre contexte) rendent instables des systèmes hamiltoniens dès lors qu'ils ont la possibilité de dissiper leur énergie. Ces résultats sont d'ailleurs connus sous l'appellation \og d'instabilités dissipatives \fg.


%%%%%%%%%%%%%%%%%%%%%%%%%%%%%%%%%%%%%%%

\subsubsection{L'application à l'instabilité d'orbites radiales}

Il n'est pas question ici de reprendre le détail des calculs présentés par l'article de J. Perez et L. Maréchal \cite{future}, nous présenterons simplement l'idée du résultat.

Un état d'équilibre composé exclusivement de particules en orbite radiale est décrit par une fonction de distribution $f_0(E,L^2) = \varphi(E) \delta(L^2)$ : si $\varphi$ est une fonction acceptable mais quelconque, la distribution de Dirac, qui sélectionne les valeurs $L^2=0$,  assure en effet le caractère purement radial de toutes les orbites.
L'idée mise en \oe uvre dans la preuve est alors claire. Après avoir rappelé le caractère hamiltonien de la dynamique de Vlasov-Poisson 
gravitationnelle, il suffit de calculer la variation de 
\begin{equation}
	H^{(2)} =
	\underbrace{- \int \{g,E\} \{g,f\} \mathrm{d} \Gamma}_{(A)}
	\underbrace{- G m^2 %
		\int\!\!\!\int \frac{\{g,f\}\{g',f'\}}{|\mathbf{q} - \mathbf{q'}|}
		\mathrm{d} \Gamma \mathrm{d} \Gamma'}_{(B)}
		\label{H2AB}
\end{equation}
 pour un équilibre de la forme $f_{0,a}(E,L^2) = \varphi(E) \delta_a (L^2)$ avec une fonction $\delta_a (L^2)$ admettant $\delta(L^2)$ lorsque $a$ tend vers 0. Le terme $(A)$ dans la variation \ref{H2AB} correspond à la variation d'énergie cinétique alors que $(B)$ rend compte de la variation d'énergie potentielle. Cette dernière est toujours négative compte-tenu des propriétés du laplacien.
Il est alors un peu laborieux de montrer qu'il existe toute une classe de perturbations, non radiales, associées à des variations d'énergie négative dès que $a$ est suffisamment petit et donc que le système est suffisamment radial.

Dans ce contexte la présence de dissipation est donc irrémédiablement associée à une instabilité du système. Il est clair qu'il est impossible de garantir la totale conservation de l'énergie pour un système physique réel ou numérique. L'instabilité d'orbite radiale peut donc se développer dans les simulations ou dans la réalité! Le mécanisme est celui par initié par Lynden-Bell (voir section {ROIadvances}) : une perturbation non radiale étire ou compresse certaines régions du système initialement sphérique, le couplage avec les orbites proches conduit à une avalanche dans la direction sélectionnée et une barre se forme. Sans la dissipation, un tel couplage est impossible, l'instabilité d'orbite radiale est bien d'origine dissipative. 

On remarque dans ce processus qu'un état d'équilibre radial est nécessaire afin que puisse se développer cette instabilité. 
C'est l'accumulation d'orbites radiales couplées par une légère dissipation qui permet le déclenchement de l'avalanche.   

 

%%%%%%%%%%%%%%%%%%%%%%%%%%%%%%%%%%%%%%%%%%%%%%%%%%%%%%%%%%%

