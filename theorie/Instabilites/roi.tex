\section{L'instabilité d'orbites radiales}

L'un des vieux problèmes de la dynamique galactique est l'étude des systèmes autogravitants formés de particules majoritairement en orbite radiale,
très allongées et passant donc près du centre du système. Il est très vite apparu que de tels systèmes initialement sphériques pourraient être
instables et perdre leur symétrie dans ce qu'il est convenu d'appeler l'instabilité d'orbite radiale \textsc{ior} ou \textsc{roi} en anglais. Ce
mécanisme pourrait même être à l'origine de la forme de certains objets à l'échelle galactique attendu que la gravitation peine à former des
structures possédant une direction privilégiée.


\subsection{L'histoire de l'instabilité d'orbites radiales}

\subsubsection{Les pionniers}

Le premier résultat publié est l'œuvre de Vadim \cite{antonov}. Il s'agit d'un résultat analytique concernant un système constitué de $N$ particules.
L'équation de Poisson correspondante est étudiée en perturbation dans un cas limite correspondant aux orbites radiales. Un système différentiel est
alors produit pour un certain \og déplacement\fg des orbites. Le résultat est obtenu en utilisant une fonction de Lyapunov stricte pour ce système
différentiel.

La même année, Michel \cite{henon} publie l'un des tout premiers résultats numériques dans ce domaine. Utilisant $N=1000$ coquilles concentriques
distribuées selon un modèle polytropique $f\left(  E\right)  \propto E^{n}$, il vérifie la stabilité obtenue analytiquement par Antonov dès les années
60 pour ce type de système. Puis, en étendant son modèle au cas anisotrope (dans l'espace des vitesses) des polytropes généralisés  $f\left(  E\right)
\propto E^{n}L^{2m}$, il montre numériquement que le système devient instable lorsque $m\rightarrow-1$, qu'il identifie à la fonction de distribution
$f\left(  E\right)  \propto E^{n}\delta\left(  L^{2}\right) $ et donc aux systèmes présentant de plus en plus d'orbites radiales. L'instabilité est
identifiée dans l'espace des phases qui montre une évolution du système dès que celui-ci est trop anisotrope dans l'espace des vitesses. Le mécanisme
de cette instabilité n'est pas précisé, l'effet de cette instabilité dans l'espace des positions (la fameuse barre) n'est pas non plus abordé. Cet
article ne fait pas référence au travail d'\cite{antonov}. L'utilisation de coquilles sphériques ne permet pas de prendre en compte d'éventuelles
interactions non radiales.

En utilisant les méthodes de Water-Bag, qui consistent à décomposer la fonction de distribution sur une base de fonctions continues par morceaux,
une équipe française \cite{waterbag} publie un résultat affirmant la stabilité de tous les systèmes autogravitants $f\left(  E,L^{2}\right)  $ contre
des perturbations non-sphériques\footnote{La seule condition requise est le fait que $\partial f/\partial E<0$ et $\partial f/\partial L^{2}<0$, ce
qui correspond au cas physique.}. Le même Antonov ayant déjà réglé le cas des perturbations à symétrie sphérique pour ces même systèmes anisotropes, tout les
cas étaient traités: tout système autogravitant sphérique (isotrope ou non dans l'espace des vitesses) est stable contre toutes les formes de
perturbations. Le cas des systèmes constitués d'orbites de plus en plus radiales, inclus dans le champ d'action du résultat de l'équipe française est
donc prédit stable par cette analyse, en désaccord patent avec le résultat analytique d'\cite{antonov}, et les simulations de \cite{henon}.

\subsubsection{Des avancées conséquentes\label{roiadvances}}

Sans faire référence au résultat de l'équipe française, \citet{polyach} proposent une formulation matricielle du problème de
la stabilité (basée sur une décomposition en série de Fourier des pertubations) qui leur permet de prouver qu'une fonction de distribution de la forme
$f\left(  E-\lambda L^{2}/r_{a}^{2}\right)$, modèle que l'on appelera plus tard Ossipkov-Merritt, décrit un modèle instable si $r_{a}^{2}$ est
suffisamment petit. Leurs résultat est contraire à celui proposé par \cite{waterbag}, et va dans
le sens d'Antonov et Hénon. Cet article contient de plus un critère de stabilité qui sera repris dans le livre de \citet{1984pgs1.book.....F,1984pgs2.book.....F}
% Friedmann et Polyachenko
, et qui
affirme qu'un système sphérique déclenche une instabilité d'orbite radiale dès que l'on a $2T_{r}/T_{\perp}>1,75\pm0,25$ o\`{u} $T_{r}$ et
$T_{\perp}$ représentent respectivement les énergies cinétiques radiale et perpendiculaire totales contenues dans le système.

La première étude numérique globale et à peu près réaliste du problème de l'effondrement gravitationnel est effectuée par  \cite{albada}. Cette étude
considère des ensembles de $N=5000$ particules de même masse dont les conditions initiales sont réparties en 2 catégories : les sphères homogènes de
taille 1 et des systèmes composés de 20 sphères homogènes (clumps) contenant chacune 250 particules. Ces clumps possèdent initialement un rayon égal à
0,4 et leurs centres sont positionnés uniformément dans une boule de rayon 1$.$ Ces différentes conditions initiales sont abandonnées à leur gravité
dans 3 conditions initiales de vitesse déterminées par le rapport du viriel initial : $-2T/U=0.5,~0.2$ et $0.1$. Le résultat est clair : les sphères
homogènes ne souffrent pas (dans les cas d'effondrement condidérés) d'instabilité d'orbite radiale. Par contre, les assemblages de grumeaux qui
s'éffondrent violemment ($-2T/U=0,1$)~produisent un équilibre triaxial. Même si cet article ne parle pas de l'instabilité d'orbite radiale, il confirme
que les profils (luminosité et densité) obtenus numériquement sont compatibles avec ceux qui sont observés pour les galaxies (loi en $r^{1/4}$).

Dans un article à la fois numérique et analytique, \cite{barneshut} abordent l'une des premières études globale de l'instabilité. L'étude numérique
consiste à refaire les expériences avec des coquilles de Michel Hénon en utilisant maintenant des techniques à $N$ corps pour des valeurs de $N$
comprises entre $10^{3}$ et $10^{4}$. Ils confirment les résultats de leur prédécesseur en faisant une étude plus exhaustive dans l'espace des
paramètres des modèles utilisés par Hénon, Polyachenko et Shukhman;  ils confirment d'ailleurs le critère de stabilité des russes et proposent une
explication analytique qui ferait passer l'instabilité d'orbite radiale pour une sorte d'instabilité de
Jeans\footnote{Instabilité qui survient lorsque la pression cinétique ne suffit plus à compenser la tendance qu'à le système à s'effondrer sous
l'effet de son propre poids.} : la pression stellaire dans la direction tangentielle devient insuffisante pour compenser la tendance naturelle qu'ont
les orbites radiales à se condenser. Il est à noter que l'école russe propose le même style d'interprétation pour l'instabilité dans le livre de
\citet{1984pgs2.book.....F} (p. 148).
% Polyachenko et Friedmann (Vol2, p. 148).

Le travail suivant sur le sujet est publié par \cite{merritt_aguilar}.
% Il fait référence aux travaux précédents de Barnes et al., en le résumant en quelques lignes.
Il se concentre sur les aspects numériques et
sur l'opportunité que représente cette instabilité dans le contexte de la formation des galaxies, c'est la première fois que cette idée surgit dans la
littérature. Les auteurs utilisent une première famille de systèmes dont le profil de densité est \og de type galactique\fg: $\rho\left(  r\right)
$ $\propto (r/r_{o})^{-2}(1+r/r_{o})^{-2}$ (c'est le modèle de Jaffe qui est compatible avec le profil de luminosité en $r^{1/4}$ ). Il possède
en outre la bonne propriété d'être facilement transposable en un modèle anisotrope en suivant l'algorithme d'Ossipkov-Merritt\footnote{Il s'agit de
modèles sphériques dont la fonction de distribution étend un modèle isotrope à un système présentant une anisotropie radiale de plus en plus forte en
s'éloignant du centre du système.}. Il s'agit d'expériences à $N$ corps avec $N=5\times10^{3}$, l'état initial est un équilibre dont le degré
d'anisotropie radiale est contrôlé par la valeur de $r_{o}$. Les conclusions sont les suivantes : la transition stable/instable est très rapide et
elle se produit pour un rapport $2T_{r}/T_{\perp}\approx2,5$, soit un peu plus que ce qui est prévu par le critère russe. L'étude de deux familles
complémentaires (l'une avec une anisotropie indépendante du rayon et l'autre avec une fonction de distribution décroissante en $E$ \emph{et} en
$L^{2}$) semble indiquer d'une part que la valeur de $2T_{r}/T_{\perp}$ n'est pas un critère de stabilité et d'autre part que le résultat
de \cite{waterbag} est définitivement infirmé. L'idée de l'importance de cette instabilité dans le processus de formation des galaxies
est avancée en conclusion de l'article, en reprenant ses termes \og elle ne doit pas être écartée\fg.

Une étude analytique d'une équipe anglaise (\cite{palmerpapa}) basée sur une analyse spectrale de la perturbation d'un système sphérique anisotrope
conclut à l'instabilité. C'est, depuis les résultats de \cite{polyach} et hormis le résultat water-bag de \cite{waterbag}, la seule approche
analytique de ce problème et toujours par des méthodes consistant à décomposer les perturbations sur des bases de fonctions orthogonales.
Deux autres aspects importants de cet article sont la démonstration de la non validité du critère de stabilité russe déjà écorné par Merritt
et Aguilar et la présentation d'un nouveau mécanisme pour la croissance de l'instabilité inspiré d'un travail de \cite{lyndenbell}. Ce dernier point
mérite une attention particulière. Le travail de \cite{lyndenbell} étudie l'influence d'une perturbation axisymétrique dans le plan de symétrie d'un
potentiel de galaxie spirale sur l'orbite d'une étoile. Il tend à montrer un effet d'élongation des orbites qui ont alors tendance à s'aligner le long
de la perturbation. Cet effet serait à l'œuvre dans la formation des barres des galaxies spirales. Pour la première fois dans l'histoire de
l'instabilité d'orbites radiales, \cite{palmerpapa} suggèrent que c'est le mécanisme de Lynden-Bell qui est à l'œuvre.

Une synthèse de tous ces résultats est effectuée par \cite{merritt1987}. Les 2 mécanismes sont détaillés et expliqués, l'instabilité de Jeans est
critiquée car elle nécessite un système homogène ce qui n'est pas le cas, \cite{merritt1987} met donc en avant le mécanisme de Lynden Bell.

Un article de \cite{katz} propose alors de nouvelles simulations cosmologiques montrant que le processus hiérarchique de
formation des structures cosmologiques avec ses effondrements successifs tend à gommer les traces possibles d'une instabilité d'orbite radiale qui
aurait pu se produire dans les phases initiales de cette formation.

Un article de \cite{saha} vient étendre la portée des méthodes spectrales de modes normaux aux systèmes d'extension
infinie. %, ce qui n'était apparement pas le cas des études précédentes.
Une étude de \cite{weinberg}, reprend les méthodes matricielles initiées par l'école russe de Polyachenko, retrouve des résultats et présente dans sa
section IV-c une analyse détaillée du mécanisme de Lynden-Bell appliqué à l'instabilité d'orbites radiales. Cette analyse fait l'objet d'un
article complet de \cite{cincotta} qui étudie la transformation d'orbites de type boucle en type boite, ce qui est dans la veine du mécanisme de
Lynden-Bell et confirme l'intuition de \cite{merritt1987}.

% \subsubsection{Un regain d'intérêt}
\subsubsection{De nouveaux travaux}


Mettant à profit certains de leurs résultats analytiques \cite{JPerez96} et \cite{perez_et_al} proposent et testent un critère de stabilité
pour les systèmes auto-gravitants construit sur la nature des perturbations qu'ils reçoivent. Ce critère est validé sur des modèle Ossipkov-Merritt
appliqués à des polytropes. Le nombre de particules mis en jeu devient pour la première fois raisonnable $N\approx10^{4}$ pour l'ensemble des
simulations. Dans leurs résultats analytiques, ils expliquent la carence des méthodes de Water-Bag dans le domaine des orbites radiales, ce qui
pourrait expliquer le résultat de \cite{waterbag}.

Une étude numérique systématique de l'instabilité d'orbite radiale, utilisant les machines dédiées \og{}GRAPE\fg, est produite par \cite{theis}. Les
simulations effectuées sont des effondrements de sphères de Plummer de température initiale variable. Le taux de croissance de \textsc{roi} est
grandement affecté par le softening $\epsilon$ du potentiel et très peu par des variations du nombre de particules. Ces simulations ont mis en
évidence une évolution à très long terme (de l'ordre du temps de relaxation à 2 corps) du système triaxial produit par \textsc{roi} vers un système
plus ou moins sphérique, selon les auteurs cette transformation est due aux interactions à deux particules.

Un étude systématique de l'effondrement gravitationnel (collapse) avec test des paramètres numériques ($N,\varepsilon$, ...) par
\cite{roy}, permet, entre autres résultats, de mettre en évidence un aspect pressenti de l'instabilité d'orbites radiales. Son déclenchement dans un
collapse est subordonné à la présence d'inhomogénéités robustes. Ce ne sont en effet que les effondrements en au moins deux phases successives qui sont le
siège de \textsc{roi}: l'effondrement d'une sphère homogène ne remplit pas cette condition. Ces résultats sont complétés et raffinés par
\cite{boily} qui montrent un léger effet du nombre de particules sur l'état final de \textsc{roi}.

Bien que le rôle de \textsc{roi} dans la formation des structures ait été atténué par le travail de \cite{katz}, deux analyses
complémentaires par \cite{huss} et \cite{macmillan} observent le résultat de la formation de structures à moyenne
échelle par des expériences de collapse en se donnant la possibilité de supprimer numériquement l'instabilité d'orbite radiale (en retirant la
composante radiale de la force de gravitation). Ils remarquent que si l'on empêche \textsc{roi} dans les phases primordiales, le profil
de densité final de ces structures formées dans un contexte cosmologique est modifié: on obtient un profil à deux pentes au lieu de trois dans NFW par exemple.
La forme allongée que prend le système à cause de l'instabilité d'orbite radiale serait donc progressivement gommée par le processus de formation
hiérarchique comme l'a remarqué Katz, mais le profil de densité final garderait subtilement sa trace. Les différents résultats observationnels ou
numériques indiquent la présence de cette marque.

Un regain d'activité dans le domaine se manifeste sous l'impulsion d'une équipe dirigée par E. Barnes depuis 2005. Ce qu'il est possible de retenir des
articles de cette équipe, notamment \cite{barnes2005} et \cite{ROI_Moderne} est la chose suivante: depuis longtemps nous savons que \textsc{roi} produit
un système triaxial dans l'espace des positions, mais il crée aussi une ségrégation spatiale dans l'espace des vitesse (centre isotrope et halo
radial). C'est cette ségrégation qui serait à l'origine du profil universel observé dans les grandes structures. Dans le cadre de ce
renouveau, deux articles étudient \textsc{roi} (\cite{barneslanzel}, \cite{trenti}) avec de nombreux détails, des
comparaisons de codes. L'équipe de \cite{trenti} s'étonne de ne pas pouvoir produire d'instabilité d'orbite radiale lors de
l'effondrement d'une sphère homogène -- dite de Hénon. Leur explication à ce sujet est discutable: l'effondrement serait trop rapide ou l'anisotropie
ne serait pas suffisante. L'explication de ce phénomène avait déjà été proposée par \cite{roy}: un effondrement monolithique
(sphère de Hénon) se produit en une seule phase. Le germe dont à besoin \textsc{roi} pour se développer n'est donc pas présent dans le cas générique
de ce type d'effondrement.

Enfin \textsc{roi} aurait été observée dans un équilibre triaxial selon \cite{antonini}: elle se produirait lorsque ce dernier serait trop peuplé
d'orbites en forme de boites. Le système deviendrait alors plus prolate et toujours triaxial.

Comme le montre cette perspective historique, l'instabilité d'orbite radiale a connu des développements controversés. Elle demeure cependant
fondamentale dans le processus de formation hiérarchique des structures gravitationnelles. La dernière avancée dans la compréhension de son mécanisme
ainsi que la preuve de l'instabilité par des méthodes d'énergie se trouve dans le travail de \cite{future}. Les principaux éléments
de ce travail sont présentés dans la section suivante.

%%%%%%%%%%%%%%%%%%%%%%%%%%%%%%%%%%%%%%%%%%%%%%%%%%%%%%%%%%%%%%%%
\subsection{La méthode symplectique}
%%%%%%%%%%%%%%%%%%%%%%%%%%%%%%%%%%%%%%%%%%%%%%%%%%%%%%%%%%%%%%%%

\subsubsection{Présentation de la méthode}

La méthode symplectique a été introduite dans le contexte de la stabilité des systèmes autogravitants par \cite{bartho}, elle est issue de
la physique des plasmas. Elle fut popularisée par \cite{kandrupstability}.

Il s'agit de tirer parti de la structure hamiltonienne associée au système de Vlasov-Poisson. Dans ce contexte, nous remarquons tout d'abord que
l'énergie totale contenue dans le système autogravitant décrit par la fonction de distribution $f$, i.e.:

\begin{align*}
	H \left[ f \right]  =
	\int \mathrm{d} {\Gamma}
		\frac{\vec{p}^{\,2}}{2m} f \left( {\Gamma},t \right)
	- \frac{1}{2} Gm^2 \int \mathrm{d} {\Gamma} \int\mathrm{d}{\Gamma}^{\,\prime}
		\frac{f \left({\Gamma},t \right) f\left( {\Gamma}^{\,\prime},t\right)}%
		{\left\vert \vec{q} - \vec{q}^{\,\prime} \right\vert }
\end{align*}
est telle que sa dérivée fonctionnelle est l'énergie moyenne d'une particule test:
\begin{align*}
	\frac{\delta H}{\delta f}=\lim_{\delta f \to 0}\dfrac{H\left[ f +\delta f\right]-H \left[ f \right]}{\delta f}
	= \frac{\vec{p}^{\,2}}{2m} + m \psi=E
\end{align*}
Si $K[f]$ est une fonctionnelle dérivable de la fonction de distribution nous pouvons donc écrire:
\begin{align}
	\frac{\mathrm{d} K[f]}{\mathrm{d} t}
	= \int \frac{\delta K}{\delta f}
		\frac{\partial f}{\partial t} \mathrm{d} \Gamma
	= \int \frac{\delta K}{\delta f} \left\{ E, f \right\} \mathrm{d} \Gamma
	\label{derivk}
\end{align}
où nous avons utilisé la forme canonique de l'équation de Vlasov $\dot f=\left\{ E, f \right\}$.
Nous pouvons alors introduire les crochets popularisés par \cite{morrison}: pour deux fonctionnelles $A$ et $B$ de $f$, nous avons:
\begin{align*}
	\left[ A, B \right](f) :=
	\int f \left\{
		\frac{\delta A}{\delta f}, \frac{\delta B}{\delta f}
	\right\} \mathrm{d} \Gamma
\end{align*}
Ainsi, après une intégration par partie, la relation~\refeq{derivk} s'écrit:
\begin{align}
	\frac{\mathrm{d} K[f]}{\mathrm{d} t}
	= - \int f \left\{
		\frac{\delta H}{\delta f}, \frac{\delta K}{\delta f}
	\right\} \mathrm{d} \Gamma
	= \left[K, H \right](f)
\end{align}

Comme cela est expliqué par \cite{kandrupstability}, toute perturbation linéaire $f^{(1)}$ pouvant physiquement être reçue par l'état décrit par la
fonction de distribution $f_0$ s'écrit:
\begin{align*}
	f^{(1)}\left(  {\Gamma},t \right) = -\left\{ g,f_{0}\right\}
\end{align*}
la fonction $g$ est appelée \emph{générateur} de la perturbation. Introduisons la quantité:
\begin{align*}
	G[f] := \int f g \mathrm{d} \Gamma
\end{align*}
À partir de cette perturbation, il est possible de construire la variation d'énergie correspondante. Au  premier ordre il vient:
\begin{align*}
	H^{(1)} [f_0] = \left[G, H \right](f_0)
	= - \int g \{ f_0, E \} \mathrm{d} \Gamma
	= 0
\end{align*}
Si l'état $f_0$ est un équilibre, $\{ f_0, E \} = 0$ et la variation d'énergie est donc nulle à l'ordre 1. En d'autre terme La fonctionnelle $H[f]$
présente un extremum en $f=f_0$, c'est bien la définition d'un état d'équilibre.

La variation de l'énergie au second ordre
$
	H^{(2)} [f_{0}] = \left[G, [G,H] \right](f_0)
$
donne alors:
\begin{align}
	H^{(2)}[f_{0}]
	& = - \int \left.\frac{\delta [G,H]}{\delta f}\right|_{f=f_{0}} \{g,f_{0}\} \mathrm{d} \Gamma
	\nonumber \\
	& = - \int \left(
		\{g,E\} + \int \frac{Gm^2}{|\vec{q}-\vec{q}\,'|}
		\{g',f'_{0}\} \mathrm{d} \Gamma'
	\right) \{g,f_{0}\} \mathrm{d} \Gamma
	\nonumber \\
	& = - \int \{g,E\} \{g,f_{0}\} \mathrm{d} \Gamma
	- G m^2 \int\!\!\!\int \frac{\{g,f_{0}\}\{g',f'_{0}\}}{|\vec{q} - \vec{q}\,'|}
	\mathrm{d} \Gamma \mathrm{d} \Gamma'
\end{align}

L'étude du signe de cette quantité s'est révélé un outil efficace d'investigation de la stabilité des systèmes autogravitants (voir par exemple
\citet{JPerez96}).

%%%%%%%%%%%%%%%%%%%%%%%%%%%%%%%%%%%%%%%

\subsubsection{Des critères de stabilité}

Dans le cas classique d'une particule soumise à l'influence de forces conservatives, la stabilité locale d'un état d'équilibre est directement reliée
au signe de la variation d'énergie à l'ordre 2 au voisinage de cet état: une variation positive laisse l'équilibre stable alors qu'une variation
négative de l'énergie conduit à une instabilité.

L'extension d'un tel résultat à des situations plus compliquées (forces non conservatives, systèmes de particules, etc...) n'est cependant pas
triviale.

Dans le cas qui nous intéresse de la physique statistique des systèmes non collisionnels évoluant dans un champ moyen la situation n'est cependant pas
désespérée.

Si la variation $H^{(2)}$ est positive pour tous les générateurs $g$, \cite{bartho} montre que le système est stable. Cette méthode a été
largement mise en application dans tous les cas favorables par \cite{perezaly}. Par contre, s'il existe des générateurs $g$
conduisant à des variations $H^{(2)}$ négatives, il n'existe pas de résultat général assurant l'instabilité du système; du moins sans hypothèse
complémentaire.


Dans ce contexte, les résultats de \cite{blochmarsden} mais surtout de \cite{krechet}, permettent l'étude de cas assez généraux.

Ces résultats affirment que l'existence de modes d'énergie négative ($g$ tels que $H^{(2)}<0$ dans notre contexte) rendent instables des systèmes
hamiltoniens dès lors qu'ils ont la possibilité de dissiper leur énergie. Ces résultats sont d'ailleurs connus sous l'appellation \og d'instabilités
dissipatives\fg.


%%%%%%%%%%%%%%%%%%%%%%%%%%%%%%%%%%%%%%%

\subsubsection{L'application à l'instabilité d'orbites radiales}

Il n'est pas question ici de reprendre le détail des calculs présentés par l'article de \cite{future}, nous présenterons
simplement l'idée du résultat.

Un état d'équilibre composé exclusivement de particules en orbite radiale est décrit par une fonction de distribution $f_0(E,L^2) = \varphi(E)
\delta(L^2)$: si $\varphi$ est une fonction acceptable mais quelconque, la distribution de Dirac, qui sélectionne les valeurs $L^2=0$,  assure en
effet le caractère purement radial de toutes les orbites. L'idée mise en œuvre dans la preuve est alors claire. Après avoir rappelé le caractère
hamiltonien de la dynamique de Vlasov-Poisson gravitationnelle, il suffit de calculer la variation de:
\begin{align}
	H^{(2)} =
	\underbrace{- \int \{g,E\} \{g,f\} \mathrm{d} \Gamma}_{(A)}
	\underbrace{- G m^2 %
		\int\!\!\!\int \frac{\{g,f\}\{g',f'\}}{|\vec{q} - \vec{q}\,'|}
		\mathrm{d} \Gamma \mathrm{d} \Gamma'}_{(B)}
		\label{H2AB}
\end{align}
pour un équilibre de la forme $f_{0,a}(E,L^2) = \varphi(E) \delta_a (L^2)$ avec une fonction $\delta_a (L^2)$ admettant $\delta(L^2)$ comme limite lorsque $a$ tend
vers $0$. Le terme $(A)$ dans la variation~\refeq{H2AB} correspond à la variation d'énergie cinétique alors que $(B)$ rend compte de la variation
d'énergie potentielle. Cette dernière est toujours négative compte-tenu des propriétés du laplacien. Il est alors possible de montrer qu'il %un peu laborieux de montrer qu'il
existe toute une classe de perturbations, non radiales, associées à des variations d'énergie négative dès que $a$ est suffisamment petit et donc que
le système est suffisamment radial.

Dans ce contexte la présence de dissipation est donc irrémédiablement associée à une instabilité du système. Il est clair qu'il est impossible de
garantir la totale conservation de l'énergie pour un système physique réel ou numérique. L'instabilité d'orbite radiale peut donc se développer dans
les simulations ou dans la réalité. Le mécanisme est celui par initié par Lynden-Bell (voir section~\ref{roiadvances}): une perturbation non
radiale étire ou compresse certaines régions du système initialement sphérique, le couplage avec les orbites proches conduit à une avalanche dans la
direction sélectionnée et une barre se forme. Sans la dissipation, un tel couplage est impossible, l'instabilité d'orbite radiale est bien d'origine
dissipative.

Nous remarquons dans ce processus qu'un état d'équilibre radial est nécessaire afin que cette instabilité puisse se développer. C'est l'accumulation
d'orbites radiales couplées par une légère dissipation qui en permet le déclenchement.



%%%%%%%%%%%%%%%%%%%%%%%%%%%%%%%%%%%%%%%%%%%%%%%%%%%%%%%%%%%

