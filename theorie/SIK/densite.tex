%\subsection{Calcul v2.0}

La fonction de distribution associée à ce modèle de King s'écrit (~voir~\cite{King-1966AJ}~) :
\begin{equation}
	f_K(E) = \begin{cases}
		\rho_0 \(2\pi m\sigma^2\)^{-3/2}\( e^{\frac{E_l-E}{\sigma^2}} - 1\) & \text{si $E < E_l$} \\
		0 & \text{si $E > E_l$}
	\end{cases}\label{King::Eq::DistribFunc}
\end{equation}
%\begin{equation}
%	f_K(E) = \left\{\begin{array}{l} \rho_0 \(2\pi m\sigma^2\)^{-3/2}\( e^{\frac{E_l-E}{\sigma^2}} - 1\)\ \mathrm{si}\ E < E_l \\
%		\\
%		0\ \mathrm{si}\ E > E_l
%	\end{array}\right.
%\end{equation}
où la dimension de $\sigma^2$ est celle d'une énergie. Ce paramètre représente la dispersion d'énergie du système.
Le paramètre $\rho_0$ est la densité de masse au centre du système. L'énergie $E_l$  de libération d'une particule s'écrit : $E_l = \frac{p_l^2}{2m} + m\psi(r)$, où $\psi(r)$ est le potentiel gravitationnel de la sphère de King et  $p_l$ l'impulsion de libération fonction de $r$.

La densité de masse s'écrit toujours à partir de la fonction de distribution :
\begin{eqnarray*}
	\rho(r) &=& \int^{p_l}_0\,f_K(E)4\pi p^2dp \\
		&=& \frac{\rho_0}{\(2\pi m\sigma^2\)^{3/2}}\int_0^{p_l}\,\left\{e^{\frac{E_l-E}{\sigma^2}} -1\right\} 4\pi p^2 dp\\
\end{eqnarray*}
L'énergie d'une particule test s'écrit  $E = \frac{p^2}{2m} + m\psi$, on a donc :
\begin{eqnarray*}
	\rho(r) &=& \frac{\rho_0}{\(2\pi m\sigma^2\)^{3/2}} \int_0^{p_l}\,\left\{e^{\frac{E_l - \frac{p^2}{2m} - m\psi}{\sigma^2}} -1\right\} 4\pi p^2 dp\\
		&=& \frac{4\pi \rho_0}{\(2\pi m\sigma^2\)^{3/2}} \(e^{\frac{E_l - m\psi}{\sigma^2}} \int_0^{p_l}\,e^{-\frac{p^2}{2m\sigma^2}} p^2 dp - \int_0^{p_l}\,p^2 dp\)\\
\end{eqnarray*}
Une intégration par partie de la première intégrale permet alors d'écrire
\begin{equation*}
	\rho(r) = \frac{4\pi\rho_0}{\(2\pi m\sigma^2\)^{3/2}} 
	\(e^{\frac{E_l - m\psi}{\sigma^2}} 
	\left[
	-m\sigma^2p_l e^{-\frac{p_l^2}{2m\sigma^2}} + m\sigma^2\int_0^{p_l}\,e^{-\frac{p^2}{2m\sigma^2}} dp
	\right] - \frac{p_l^3}{3}\)
\end{equation*}
En introduisant un nouveau potentiel $\phi(r)$ tel que :
\begin{equation}
	p_l^2 = 2m\(E_l - m\psi(r)\) = 2m\phi(r)
\end{equation}
il vient  maintenant :
\begin{equation*}
	\rho(r) = 
	\rho_0 \(-\sqrt{\frac{4\phi}{\pi\sigma^2}}\(1 + \frac{ 2\phi }{3\sigma^2}\) 
	+ 
	\frac{2e^{\frac{\phi}{\sigma^2}}}{\sqrt{2m\pi\sigma^2}} \int_0^{p_l}\,e^{-\frac{p^2}{2m\sigma^2}} dp\)\\
\end{equation*}
la dernière intégrale s'exprime directement en utilisant la fonction d'erreur
$$\mathrm{erf}(t) = \displaystyle{\frac{2}{\sqrt{\pi}}\int_0^t e^{-u^2}du}$$
la densité du modèle de King s'écrit donc
\begin{eqnarray}
	\rho(r) = \rho_0 \(-\sqrt{\frac{4\phi}{\pi\sigma^2}}\(1 + \frac{ 2\phi }{3\sigma^2}\) + e^{\frac{\phi}{\sigma^2}}\mathrm{erf}\(\sqrt{\phi}/\sigma\)\)
	\label{rho_r}
\end{eqnarray}
L'équation de \textsc{Poisson} pour le potentiel $\phi(r) = E_l - m\psi(r)$ s'écrit donc :
\begin{eqnarray}
	\frac{d}{dr}\(r^2\frac{d\phi}{dr}\) = -4m\pi G r^2\rho_0 \left\{-\sqrt{\frac{4\phi}{\pi\sigma^2}}\(1 + \frac{ 2\phi }{3\sigma^2}\) + e^{\frac{\phi}{\sigma^2}}\mathrm{erf}\(\sqrt{\phi}/\sigma\)\right\} \label{King-Pois}
\end{eqnarray}
Malgré le fait que cette équation n'admette pas de solution explicite, son étude numérique ne pose pas de problème majeur.

