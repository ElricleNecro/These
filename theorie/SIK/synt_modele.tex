Nous avons maintenant étudié trois modèles pouvant décrire l'état d'équilibre d'un amas d'étoile:
\begin{description}
	\item[La sphère isotherme] (\textsc{si}) est un modèle thermodynamique étudié sur un support spatial infini. Ce modèle avait l'avantage de posséder une solution analytique,
		mais cette solution présente, comme nous l'avons montré, une masse infinie.
	\item[La sphère isotherme en boîte] (\textsc{sib}) est une variante du modèle de \textsc{si} avec une extension
		finie. Grâce à ce modèle, nous avons appris que l'étude de la stabilité était pilotée par un seul paramètre important:
		le contraste de densité $\R_c$.
	\item[Le modèle de King] est lui aussi une variante de \textsc{si}, mais avec une coupure plus physique, effectuée après coup. Il nous a permis de montrer que l'évolution
		de la pente du halo, et donc du contraste de densité, pouvait être liée à un seul paramètre: $W_0$.
		Une étude de sa température nous a confirmé que ce modèle pouvait être considéré comme isotherme dès lors qu'il est un peu évolué
		($W_0 > 5$) et qu'il correspond à un amas globulaire (pente du halo supérieur à $-5$).
		% Cette évolution peut-être décrite par une équation de la forme $\alpha = a e^{b W_0} + c$.
\end{description}
