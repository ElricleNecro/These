Nous avons maintenant étudié trois modèles pouvant décrire l'état d'équilibre d'un amas d'étoile :
\begin{description}
	\item[la sphère isotherme (~SI~)] est un modèle thermodynamique étudié sur un support spatial infini. Ce modèle avait l'avantage de posséder une solution analytique,
		mais cette solution possède, comme nous l'avons montré, une masse infinie.
	\item[la sphère isotherme en boîte] est une variante du modèle de SI avec une extension finie. Grâce à ce modèle, nous avons appris que la stabilité pouvait être ramené à un seul paramètre important :
		le contraste de densité $\R_c$.
	\item[le modèle de \textsc{King}] est lui aussi une variante de SI, mais avec une coupure plus physique, effectuée après coups. Il nous a permis de montrer que l'évolution
		de la pente du halo, et donc du contraste de densité, pouvait être liée à un seul paramètre : $W_0$. Cette évolution peut-être décrite par une équation de la forme $\alpha = a e^{b W_0} + c$.
\end{description}
