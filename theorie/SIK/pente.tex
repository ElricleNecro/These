	Nous nous intéressons maintenant à l'existence d'un lien entre le rayon du cœur et la pente
du halo. Il n'y a apparemment aucun moyen de faire ça analytiquement (~en tout cas, je n'ai pas encore trouvé~).
Par contre, lorsque je traçais des graphes, obtenu avec le code abordé ci-dessus,
j'ai remarqué une forte dépendance\footnote{logique~!?!} entre la pente, en $\log-\log$, et $W_0$
(~voir graphe de droite de la figure~\ref{King_Modele-test}~). De même pour le rayon à $10\%$
de l'objet étudié. L'idée que j'ai utilisé pour faire le lien entre les deux quantités a été de
regarder leur comportement en fonction des conditions initiales puis de combiner ensuite les 2
comportements pour obtenir une relation empirique entre ces deux paramètres. %courbe les reliant.

\subsection{Calcul des pentes pour différentes conditions initiales\label{pente-critére}}

	Pour obtenir les pentes, nous traçons dans un diagramme $\log-\log$ la densité, puis, à l'aide
du logiciel \textsc{GNUPlot}\footnote{\url{http://www.gnuplot.info/}} nous ajustons une équation du type $a x+b$ à la partie linéaire de la
courbe, le coefficient $a$ représentant la pente.

	Le problème avec cette méthode, c'est que, pour certaines conditions initiales, il n'y a
presque pas de halo : la densité d'étoile chute brutalement (~voir figure~\ref{ci-pente_1}
page~\pageref{ci-pente_1} en annexe~). Les valeurs des pentes pour des systèmes ayant une énergie de
libération et une énergie minimale très proche, ou une grande dispersion d'énergie, sont donc assez
peu fiables. Cette brusque pente vient de la faible quantité d'énergie à fournir pour faire
s'échapper une ou plusieurs particules du système.

	Le traitement a ensuite été automatisé en considérant que le halo correspond à la zone :
	\begin{align}
		10^{-4} < \dfrac{\rho(r)}{\rho(0)} < 0.1 \label{pente::critere}
	\end{align}

	Pour obtenir les points rouge de la figure~\ref{coeff_evo}, nous avons fait varier $\gamma(0) = W_0 = 1$ à $W_0 = 21$.
	La courbe verte est obtenu en ajustant aux points rouge, et en respectant le critère~\ref{pente::critere}, une équation du type $a e^{b x} + c$. %$a x^2 + b x + c$.
	Les coefficients obtenus sont donnés sur la courbe~\ref{coeff_evo}.
%	\begin{table}[hbt!]
%		\begin{center}
%			\begin{tabular}{|c|c|c|}
%				\hline
%				Coefficient & Valeur & Erreur \\
%				\hline
%				\hline
%				$a$       &         -10.0698      &  $\pm 0.2423$       (~$2.406\%$~) \\
%				\hline
%				$b$       &         0.220152      &  $\pm 0.01075$      (~$4.883\%$~) \\
%				\hline
%				$c$       &         -1.63409      &  $\pm 0.09393$      (~$5.748\%$~) \\
%				$a$       &        $-0.0157022$   &   $\pm 0.002226$ (~$14.18\%$~) \\
%				$b$       &        $0.443128$     &   $\pm 0.03663$  (~$8.266\%$~) \\
%				$c$       &        $-5.33431$     &   $\pm 0.1274$   (~$2.388\%$~) \\
%				\hline
%			\end{tabular}
%		\end{center}
%		\caption{Valeur des coefficients donnée par l'ajustement pour les pentes}
%		\label{pente-fit}
%	\end{table}
	\begin{figure}[hbt!]
%		\centering \includegraphics[scale=1.00]{../Resol_King/img-king/pente-w0.pdf} %{graphe/evo-coeff_ci.pdf}
		\centering \includegraphics[scale=1.00]{graphe/pente-w0.pdf}%{img-king/pente-w0.pdf} %{graphe/evo-coeff_ci.pdf}
		\caption{Évolution des pentes pour différentes conditions initiales}
		\label{coeff_evo}
	\end{figure}

	Il est réconfortant de remarquer sur cette courbe la présence d'une asymptote horizontale autour de $-2$ correspondant à la pente d'une sphère isotherme singulière.
	En effet, pour $W_0 \to \infty$ le modèle de \textsc{King} se rapproche d'une sphère isotherme de rayon infini dont le halo est très proche de celui d'une SIS.
%	\FloatBarrier

\subsection{Calcul du rayon à $10\%$ pour différentes conditions initiales\label{r_10}}

	L'un des paramètres libres les plus utilisés du modèle de \textsc{King} est le rayon du cœur (formule~\ref{r_c}, page~\pageref{r_c}~).
%	Il serait ainsi intéressant de prendre cette valeur et de la tracer en fonction de la condition initiale $W_0$,
%	ce qui nous permettrait ainsi de la relier à la pente.
	Il est donc intéressant de relier les paramètres $r_c$ et $W_0$. Ainsi, en utilisant la relation que nous avons déterminé entre $W_0$ et la pente $\alpha$ du halo, nous obtiendrons une relation entre $r_c$ et $\alpha$

	Mais deux problèmes se posent :
	\begin{enumerate}
		\item de la même manière que dans le paragraphe précédent, la détermination de $r_c$ a un problème : quand
	$W_0$ devient trop grand, le système se \og~dilue~\fg, c'est-à-dire que la différence entre
	cœur et halo est de moins en moins évidente, comme sur la figure~\ref{w_0-5_10}
	page~\pageref{w_0-5_10} (~nous sommes dans le cas d'un amas effondré~),
		\item le programme de résolution qui nous permet d'avoir ces courbes est adimensionné par rapport à ce rayon, rendant ainsi très difficile
			le redimensionnement des résultats et leur comparaison avec les données observationnelles.
	\end{enumerate}

	Pour palier à ce dernier problème, il est possible de prendre, plutôt que le rayon du cœur, le rayon à $10\%$.
%	Le rayon à $10\%$ est le rayon à partir duquel $10\%$ de la masse totale de l'amas se trouve à l'intérieur.
	Il est défini comme :
	\begin{quote}
		Le rayon à $10\%$ du système est atteint lorsque, du centre vers le bord du
		système, la densité a diminué de $10\%$ :
		\begin{align}
			\frac{\rho(x_{10})}{\rho(0)} = 0.1
		\end{align}
	\end{quote}

%	Le rayon à $10\%$ est plus délicat à obtenir.
%	Pour obtenir la figure~\ref{coeur_evo}, nous avons considéré que le rayon à $10\%$ correspondait à la distance pour laquelle la densité
%	ne pouvait plus être considérée comme constante : nous avons placé, à la main, la valeur du
%	rayon au moment où la pente devenait nette\footnote{cette contrainte dépend des gens et n'est donc
%	pas très adapté à ce que nous faisons}.

	Une fois ces valeurs de rayon mesurées, nous avons tracé la courbe rouge de la
	figure~\ref{coeur_evo2} que nous avons ajustée avec la fonction $a e^{b x} + c$ et avons
	obtenu les coefficients indiqués sur la courbe.

%	\begin{table}[hbt!]
%		\begin{center}
%			\begin{tabular}{|c|c|c|}
%				\hline
%				Coefficient & Valeur & Erreur \\
%				\hline
%				\hline
%				$d$       &       $1.86332$      &   $\pm 0.04078$ 	(~$2.189\%$~)\\
%				\hline
%				$e$       &       $-0.635746$    &   $\pm 0.01585$    	(~$2.494\%$~)\\
%				\hline
%				$f$       &       $0.0049355$    &   $\pm 0.003953$     (~$80.1\%$~)\\
%				\hline
%			\end{tabular}
%		\end{center}
%		\caption{Valeur des coefficients donnée par l'ajustement pour les rayons à $10\%$}
%		\label{coeur-fit}
%	\end{table}
%	\begin{figure}[hbt!]
%		\centering \includegraphics[scale=1.00]{graphe/evo-coeur_ci.pdf}
%		\caption{Évolution du rayon à $10\%$ calculé à la main pour différents $W_0$}
%		\label{coeur_evo}
%	\end{figure}

%	\textcolor{red}{\underline{Attention :}} il peut être intéressant de refaire ces courbes en
%	incluant le calcul du rayon à $10\%$ directement dans le code (~selon la définition choisi~).

%	Le calcul a été refait en sortant le rayon à $10\%$ directement du code résolvant les
%	équations. De cette manière, la courbe reste sensiblement la même, et les coefficients ont par contre
%	bien changé, comme nous pouvons le voir sur la courbe~\ref{coeur_evo2}.
	\begin{figure}[hbt!]
		\centering \includegraphics[scale=1.00]{graphe/evo-coeur_ci2.pdf}
		\caption{Évolution du rayon à $10\%$ calculé lors de la résolution numérique pour différents $W_0$}
		\label{coeur_evo2}
	\end{figure}
%	\begin{table}[hbt!]
%		\begin{center}
%			\begin{tabular}{|c|c|c|}
%				\hline
%				Coefficient & Valeur & Erreur \\
%				\hline
%				\hline
%				$d$        &       $7.36877$      &   $\pm 0.2352$       (~$3.191\%$~)\\
%				\hline
%				$e$        &       $-0.633846$    &   $\pm 0.02309$      (~$3.642\%$~)\\
%				\hline
%				$f$        &       $0.0419222$    &   $\pm 0.02291$      (~$54.66\%$~)\\
%				\hline
%			\end{tabular}
%		\end{center}
%		\caption{Valeur des coefficients donnée par l'ajustement pour les rayons à $10\%$
%		(~v2.0~)}
%		\label{coeur-fit2}
%	\end{table}
	\FloatBarrier

\subsection{Lien entre les deux\label{ssec::LinkBetween}}

	Maintenant que nous avons la dépendance des rayons à $10\%$ et des pentes en fonction de la
	condition initiale $W_0$, nous pouvons regarder comment varie la pente en fonction de la
	taille de l'amas. Sur la figure~\ref{coeff-coeur2}, nous avons tracé en vert les données, et
	en rouge la courbe formée par nos deux expressions obtenues par ajustement (~les coefficients
	utilisés sont les mêmes que ceux donnés sur les figures~\ref{coeff_evo} et~\ref{coeur_evo2}~).
%	\begin{figure}[hbt!]
%		\centering \includegraphics[scale=1.00]{graphe/evo-coeff_coeur.pdf}
%		\caption{Évolution de la pente calculé à la main en fonction du rayon à $10\%$}
%		\label{coeff-coeur}
%	\end{figure}

	En supposant que nos ajustements sont valables pour toutes les valeurs possibles de $W_0$,
	nous pouvons tenter d'obtenir une expression analytique reliant nos deux quantités. En
	partant de :
	\begin{align}
		\left\{\begin{array}{l}
			f(W_0) = \alpha = a e^{b W_0} + c \\ %a W_0^2 + b W_0 + c \\
			g(W_0) = x_{10\%} = d e^{e W_0} + f
		\end{array}\right.
	\end{align}
	Et en les combinant :
	\begin{align}
		\Rightarrow e^{e W_0} &= \frac{x_{10\%} - f}{d} \notag\\
				W_0   &= \frac{1}{e}\ln\(\frac{x_{10\%} - f}{d}\) \\
		\Rightarrow \alpha &= a \exp\(\frac{1}{e}\ln\(\frac{x_{10\%} - f}{d}\)\) + c %\frac{a}{e^2}\ln^2\(\frac{x_{10\%} - f}{d}\) +
					%\frac{b}{e}\ln\(\frac{x_{10\%} - f}{d}\) + c \notag \\
%		\intertext{Nous obtenons :}
%		\beta &= a e^{1/e} %\frac{1}{e}\ln\(\frac{x_{10\%} - f}{d}\)\(\frac{a}{e}\ln\(\frac{x_{10\%} - f}{d}\) + b\) + c
	\end{align}

	Cette expression s'ajuste \og~bien~\fg~à la courbe, mais \textsc{GNUPlot} donne des erreurs
	assez élevées.
%	\begin{table}[hbt!]
%		\begin{center}
%			\begin{tabular}{|c|c|c|}
%				\hline
%				Coefficient & Valeur & Erreur \\
%				\hline
%				\hline
%				$q=a/e^2$       &      $-0.0394991$   &   $\pm 0.02115$ (~$53.55\%$~)\\
%				\hline
%				$s=1/d$         &      $0.524033$     &   $\pm 4.538$   (~$865.9\%$~)\\
%				\hline
%				$j=f/d$         &      $0.00254127$   &   $\pm 0.02122$ (~$835.2\%$~)\\
%				\hline
%				$k=b/e$         &      $-0.688944$    &   $\pm 0.8111$  (~$117.7\%$~)\\
%				\hline
%				$l=c$           &      $-5.3267$      &   $\pm 6.127$   (~$115\%$~)\\
%				\hline
%			\end{tabular}
%		\end{center}
%		\caption{Valeur des coefficients donnée par l'ajustement pour les pentes en
%		fonctions des rayons à $10\%$}
%		\label{param-fit}
%	\end{table}
%	Les corrections dans la détermination du rayon à $10\%$ ont amélioré l'ajustement de cette courbe, en plus d'avoir changé l'échelle des $x$. Les changements sont
%	donnés sur la figure~\ref{coeff-coeur2}.
%	\begin{table}[hbt!]
%		\begin{center}
%			\begin{tabular}{|c|c|c|}
%				\hline
%				Coefficient & Valeur & Erreur \\
%				\hline
%				\hline
%				$q$       &       $-0.0526849$    &  $\pm 0.02245$      (~$42.6\%$~)\\
%				\hline
%				$s$       &       $0.211927$      &  $\pm 0.4648$       (~$219.3\%$~)\\
%				\hline
%				$j$       &       $0.0081136$     &  $\pm 0.02144$      (~$264.2\%$~)\\
%				\hline
%				$k$       &       $-0.716362$     &  $\pm 0.1606$       (~$22.42\%$~)\\
%				\hline
%				$l$       &       $-5.01452$      &  $\pm 1.478$        (~$29.47\%$~)\\
%				\hline
%			\end{tabular}
%		\end{center}
%		\caption{Valeur des coefficients donnée par l'ajustement pour les pentes en
%		fonctions des rayons à $10\%$ (~v2.0~)}
%		\label{param-fit2}
%	\end{table}
	\begin{figure}[hbt!]
		\centering \includegraphics[scale=1.00]{graphe/evo-coeff_coeur3.pdf}
		\caption{Évolution de la pente en fonction du rayon à $10\%$}
		\label{coeff-coeur2}
	\end{figure}

%	Ensuite, il est intéressant de comparer les données observationnelle à notre travail. Le problème auquel auquel nous avons fait face est
%	que nous avons adimensionné le problème par rapport au rayon de cœur, qu'il n'est donc pas possible d'obtenir par la résolution numérique.
%	Ce travail n'est donc pas utilisable directement pour une comparaison avec les observations. Par contre, le rayon de cœur est mesurable, comme indiqué dans~\cite{Djo-rc}.
	\FloatBarrier

\subsection{Paramètre de concentration : $c$}
	Le dernier paramètre d'un modèle de \textsc{King} sur lequel nous pouvons jouer est le paramètre de concentration $c$.
	Dans le modèle de King, la concentration s'écrit :
	\begin{align}
		c = \log_{10}\(\frac{R}{r_c}\)
	\end{align}
	avec $R$ la taille du système (~tel que $\rho(R) = 0$~) et $r_c$ le rayon de cœur. Ce paramètre est infini pour la sphère isotherme singulière, comme $W_0$.
	Notons que dans le cas de la SIS $W_0$, $R$ et $c$ sont infini.

	Ce paramètre permet aussi de fixer l'état du système. Trouver une fonction ou une équation permettant de faire le lien entre la concentration et la condition
	initiale $W_0$ nous permettrait de trouver le modèle de King correspondant à un amas (~les paramètres $c$, $r_c$ et $\sigma^2$ le représentant~). Nous avons donc inclus
	dans la résolution des équations un calcul de ce paramètre de concentration. Nous avons ainsi pu obtenir la courbe~\ref{concentre}
	\begin{figure}[hbt!]
		\centering \includegraphics[scale=1.00]{graphe/concentration-king.pdf}
		\caption{Évolution de la concentration avec la condition initiale $W_0$}
		\label{concentre}
	\end{figure}

	Le paramètre de concentration est mesurable pour chaque amas (~en tout cas, c'est un des paramètres du catalogue de \textsc{Harris}~\cite{Harris}~).
	Mais le fait que la courbe ne soit pas bijective nous impose plusieurs modèles pour un seul paramètre de concentration.
	\FloatBarrier
