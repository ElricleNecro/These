L'une des premières questions que l'on peut se poser, par rapport au modèle précèdent~: le modèle de \King est il isotherme~?

%Dans un premier temps, nous rappelons les expressions des fonctions de distribution et densité pour ce modèle~:
%\begin{align}
%	f_K(E) &= \begin{cases}
%		\rho_0 \(2\pi m\sigma^2\)^{-3/2}\( e^{\frac{E_l-E}{\sigma^2}} - 1\) & \text{si $E < E_l$} \\
%		0 & \text{si $E > E_l$}
%	\end{cases} \\
%	\rho(r) &= \rho_0 \(-\sqrt{\frac{4\phi}{\pi\sigma^2}}\(1 + \frac{ 2\phi }{3\sigma^2}\) + e^{\frac{\phi}{\sigma^2}}\mathrm{erf}\(\sqrt{\phi}/\sigma\)\)
%\end{align}

La température est définit comme étant proportionnelle à la dispersion de vitesse tel que~:
\begin{align}
	T(r) &\propto \dfrac{\int p^2 f_K(E) \vdp}{m^2 \int f_K(E) \vdp} & v^2 = \frac{p^2}{m^2}
\end{align}

\subsection{Évolution de la température avec le rayon}

Le numérateur s'écrit~:
\begin{align}
	\int_0^{p_l} \dfrac{p^2}{m^2} f_K\(E\) \mathrm{d}\vec{p} &= \int_0^{p_l} \dfrac{p^2}{m^2} \rho_0 \(2\pi m\sigma^2\)^{-3/2}\( e^{\frac{E_l-E}{\sigma^2}} - 1\) 4\pi p^2 \ddp \notag \\
	    &= \dfrac{4\pi\rho_0}{\(2\pi m\sigma^2\)^{3/2}}\(\int_0^{p_l} \frac{p^4}{m^2} e^{\frac{E_l-E}{\sigma^2}}\ddp -
	    \int_0^{p_l} \frac{p^4}{m^2}\ddp \) \notag
\intertext{En utilisant $E = \dfrac{p^2}{2m} + m\psi$, et à l'aide d'intégration par partie, il est possible de simplifier plusieurs étapes de l'intégration~:}
	    &= \frac{4\pi\rho_0}{m^2\(2\pi m\sigma^2\)^{3/2}}
	    	\(
			e^{\frac{E_l - m\psi}{\sigma^2}}
			\left[
				3m\sigma^2\int_0^{p_l} p^2 e^{-\frac{p^2}{2m\sigma^2}}\ddp
				- m\sigma^2 p_l^3 e^{-\frac{p_l^2}{2m\sigma^2}}
			\right]
			- \dfrac{p_l^5}{5}
	    	\) \notag \\
	    &= \frac{4\pi\rho_0}{m^2\(2\pi m\sigma^2\)^{3/2}}
		\(
			e^{\frac{E_l - m\psi}{\sigma^2}}
	    		\left[
				  3\(m\sigma^2\)^2 \int_0^{p_l} e^{-\frac{p^2}{2m\sigma^2}} \ddp
				- 3\(m\sigma^2\)^2p_le^{-\frac{p_l^2}{2m\sigma^2}}
			\right. \right. \notag \\
				&\qquad \left.\vphantom{ e^{-\frac{p_l^2}{2m\sigma^2}}}\left.\vphantom{e^{-\frac{p_l^2}{2m\sigma^2}}}
				- m\sigma^2 p_l^3 e^{-\frac{p_l^2}{2m\sigma^2}}
			\right]
	    		- \dfrac{p_l^5}{5}
		\) \notag \\
	    &= \frac{4\pi\rho_0}{m^2\(2\pi m\sigma^2\)^{3/2}}
		\(
			e^{\frac{E_l - m\psi}{\sigma^2}}\(m\sigma^2\)^2
	    		\left[
				  3 \int_0^{p_l} e^{-\frac{p^2}{2m\sigma^2}} \ddp
				- 3p_le^{-\frac{p_l^2}{2m\sigma^2}}
				\(1 + \dfrac{p_l^2}{3m\sigma^2}\)
			\right] \right. \notag \\
	    &\qquad \left.\vphantom{ e^{-\frac{p_l^2}{2m\sigma^2}}}
	    		- \dfrac{p_l^5}{5}
		\) \notag \\
	    &= \frac{4\pi\rho_0}{m^2\(2\pi m\sigma^2\)^{3/2}}
		\(
			e^{\frac{E_l - m\psi}{\sigma^2}}\(m\sigma^2\)^2
	    		\left[
				3 \dfrac{\sqrt{\pi}}{2}\sqrt{2m\sigma^2}\erf\(\dfrac{p_l}{\sqrt{2m\sigma^2}}\)
				- 3p_le^{-\frac{p_l^2}{2m\sigma^2}}
				\(1 + \dfrac{p_l^2}{3m\sigma^2}\)
			\right] \right. \notag \\
	    &\qquad \left.\vphantom{ e^{-\frac{p_l^2}{2m\sigma^2}}}
	    		- \dfrac{p_l^5}{5}
		\) \notag \\
%	    &= \frac{12\pi\rho_0 \sigma^4}{\(2\pi m\sigma^2\)^{3/2}} e^{\frac{E_l - m\psi\(r\)}{\sigma^2}} h\(p_l\)\dr
\intertext{avec $p_l = \sqrt{2m\(E_l - m\psi(r)\)} = \sqrt{2m\phi(r)}$~:}
	    &= \frac{4\pi\rho_0}{m^2\(2\pi m\sigma^2\)^{3/2}}
		\(
			e^{\frac{\phi}{\sigma^2}}\(m\sigma^2\)^2
	    		\left[
				3\dfrac{\sqrt{\pi}}{2} \sqrt{2m\sigma^2}\erf\(\dfrac{\sqrt{2m\phi(r)}}{\sqrt{2m\sigma^2}}\)
			\right. \right. \notag \\
	    &\qquad \left.\vphantom{ e^{-\frac{2m\phi(r)}{2m\sigma^2}}}\left.\vphantom{ e^{-\frac{2m\phi(r)}{2m\sigma^2}}}
			        - 3\sqrt{2m\phi(r)}e^{-\frac{2m\phi(r)}{2m\sigma^2}}
				\(1 + \dfrac{2m\phi(r)}{3m\sigma^2}\)
			\right]
			- \dfrac{\(2m\phi(r)\)^{5/2}}{5}
		\) \notag \\
	    &= \frac{4\pi\rho_0}{m^2\(2\pi m\sigma^2\)^{3/2}}
		\(
			\dfrac{3}{2} \(m\sigma^2\)^2\sqrt{2m\pi\sigma^2}
	    		\left[
				e^{\frac{\phi}{\sigma^2}}\erf\(\sqrt{\dfrac{\phi(r)}{\sigma^2}}\)
			\right. \right. \notag \\
	    &\qquad \left.\vphantom{\(\sqrt{\dfrac{\phi(r)}{\sigma^2}}\) }\left.\vphantom{\(\sqrt{\dfrac{\phi(r)}{\sigma^2}}\)}
				- \sqrt{\dfrac{4\phi(r)}{ \pi\sigma^2}}
				\(1 + \dfrac{2\phi(r)}{3\sigma^2}\)
			\right]
			- \dfrac{\(2m\phi(r)\)^{5/2}}{5}
		\) \notag
\end{align}

La température va alors s'écrire~:
\begin{align}
	T(r) &\propto \frac{4\pi\rho_0}{m^2\(2\pi m\sigma^2\)^{3/2}}
		\(
		\dfrac{
			\dfrac{3}{2} \(m\sigma^2\)^2\sqrt{2m\pi\sigma^2}
	    		\left[
				e^{\frac{\phi}{\sigma^2}}\erf\(\sqrt{\dfrac{\phi(r)}{\sigma^2}}\)
				- \sqrt{\dfrac{4\phi(r)}{ \pi\sigma^2}}
				\(1 + \dfrac{2\phi(r)}{3\sigma^2}\)
			\right]
		}
		{
			\rho_0 \(-\sqrt{\frac{4\phi}{\pi\sigma^2}}\(1 + \frac{ 2\phi }{3\sigma^2}\) + e^{\frac{\phi}{\sigma^2}}\mathrm{erf}\(\sqrt{\phi}/\sigma\)\)
		}
			\right. \notag \\
	    &\qquad \left.
		- \dfrac{
			\(2m\phi(r)\)^{5/2}
		}
		{
			5 \rho_0 \(-\sqrt{\frac{4\phi(r)}{\pi\sigma^2}}\(1 + \frac{ 2\phi(r) }{3\sigma^2}\) + e^{\frac{\phi(r)}{\sigma^2}}\mathrm{erf}\(\sqrt{\phi(r)}/\sigma\)\)
		}
		\) \notag \\
	&\propto \frac{4\pi}{m^2\(2\pi m\sigma^2\)^{3/2}} \(m\sigma^2\)^2
		\(
			\dfrac{3}{2}\sqrt{2m\pi\sigma^2}
		- \dfrac{
			\(2m\phi(r)\)^{5/2} \(m\sigma^2\)^{-2}
		}
		{
			5 \(-\sqrt{\frac{4\phi(r)}{\pi\sigma^2}}\(1 + \frac{ 2\phi(r) }{3\sigma^2}\) + e^{\frac{\phi(r)}{\sigma^2}}\mathrm{erf}\(\sqrt{\phi(r)}/\sigma\)\)
		}
		\) \notag \\
	&\propto \sqrt{\dfrac{2\sigma^2}{m^3\pi}}							%\frac{4\pi}{m^2\(2\pi m\sigma^2\)^{3/2}} \(m\sigma^2\)^2
		\(
			\dfrac{3}{2}\sqrt{2m\pi\sigma^2}
		- \dfrac{
			\(2m\phi(r)\)^{5/2} \(m\sigma^2\)^{-2}
		}
		{
			5 \(-\sqrt{\frac{4\phi(r)}{\pi\sigma^2}}\(1 + \frac{ 2\phi(r) }{3\sigma^2}\) + e^{\frac{\phi(r)}{\sigma^2}}\mathrm{erf}\(\sqrt{\phi(r)}/\sigma\)\)
		}
		\)
\end{align}
Nous obtenons alors une expression de la température qui n'est pas analytique. Pour obtenir la température, nous reprenons le noyau développé autour
du modèle de \King pendant le stage, puis nous appliquons la formule obtenue.
Les courbes obtenues sont données tracé sur les graphes~\ref{courbe::Temp}.
\begin{figure}[H]
	\begin{minipage}[b]{0.40\linewidth}
		\centering \includegraphics[scale=0.60]{theorie/graphe/temperature_0-997636.pdf}
	\end{minipage}\hfill
	\begin{minipage}[b]{0.48\linewidth}
		\centering \includegraphics[scale=0.60]{theorie/graphe/temperature_3-98531.pdf}
	\end{minipage}
	\centering \includegraphics[scale=0.60]{theorie/graphe/temperature_11-1096.pdf}
	\caption{Courbes de température pour différents $W_0$\label{courbe::Temp}}
\end{figure}
%\FloatBarrier

Toutes ces courbes sont normalisées par rapport à la température centrale donnée par~:
\begin{align}
	T(0) &\propto \sqrt{\dfrac{2\sigma^2}{m^3\pi}}								%\frac{4\pi\rho_0}{m^2\(2\pi m\sigma^2\)^{3/2}} \(m\sigma^2\)^2
		\(
			\dfrac{3}{2}\sqrt{2m\pi\sigma^2}
		- \dfrac{
			\(2m\phi(0)\)^{5/2} \(m\sigma^2\)^{-2}
		}
		{
			5 \(-\sqrt{\frac{4\phi(0)}{\pi\sigma^2}}\(1 + \frac{ 2\phi(0) }{3\sigma^2}\) +
			e^{\frac{\phi(0)}{\sigma^2}}\mathrm{erf}\(\sqrt{\phi(0)}/\sigma\)\)
		}
		\) \notag \\
	   &\propto \sqrt{\dfrac{2\sigma^2}{m^3\pi}}								%\frac{4\pi\rho_0}{m^2\(2\pi m\sigma^2\)^{3/2}} \(m\sigma^2\)^2
		\(
			\dfrac{3}{2}\sqrt{2m\pi\sigma^2}
		- \dfrac{
			\(2m\sigma^2W_0\)^{5/2} \(m\sigma^2\)^{-2}
		}
		{
			5 \(-\sqrt{\frac{4W_0}{\pi}}\(1 + \frac{ 2W_0 }{3}\) +
			e^{W_0}\mathrm{erf}\(\sqrt{W_0}\)\)
		}
		\)
\end{align}
%\FloatBarrier

\subsection{Température moyenne d'un modèle de \King}
\label{Calc::Temp}

Au vu de ces courbes, nous pouvons considérer que la sphère de \King est bien isotherme. Il est alors possible de déterminer une température moyenne~:
\begin{align}
%	T_{\mathrm{moy}} &= \dfrac{\displaystyle{\int}_0^R 4\pi r^2 T(r) f_K(E) \dr}{\displaystyle{\int}_0^R 4\pi r^2 \dr} \\
%			 &= \dfrac{\displaystyle{\int}_0^R r^2 T(r) \rho_0 \(2\pi m\sigma^2\)^{-3/2}\( e^{\frac{E_l-E}{\sigma^2}} - 1\)}
%			 	  {\displaystyle{\int}_0^R r^2      \rho_0 \(2\pi m\sigma^2\)^{-3/2}\( e^{\frac{E_l-E}{\sigma^2}} - 1\)}
T_{\mathrm{moy}} &= \dfrac{\displaystyle{\int}\displaystyle{\int}\frac{p^2}{m^2}f_K(E)\vdp\vdr}{\displaystyle{\int}\displaystyle{\int}f_K(E)\vdp\vdr} \notag \\ %\int_0^R 4\pi r^2 T(r) \dr \\
	    &= \frac{4\pi\rho_0}{Mm^2\(2\pi m\sigma^2\)^{3/2}}
		    \displaystyle{\int}_0^R 4\pi r^2\(
			\dfrac{3}{2} \(m\sigma^2\)^2\sqrt{2m\pi\sigma^2}
	    		\left[
				e^{\frac{\phi}{\sigma^2}}\erf\(\sqrt{\dfrac{\phi(r)}{\sigma^2}}\)
			\right. \right. \notag \\
	    &\qquad \left.\vphantom{\(\sqrt{\dfrac{\phi(r)}{\sigma^2}}\) }\left.\vphantom{\(\sqrt{\dfrac{\phi(r)}{\sigma^2}}\)}
				- \sqrt{\dfrac{4\phi(r)}{ \pi\sigma^2}}
				\(1 + \dfrac{2\phi(r)}{3\sigma^2}\)
			\right]
			- \dfrac{\(2m\phi(r)\)^{5/2}}{5}
		\) \dr \notag \\
	    &= \frac{4\pi}{Mm^2\(2\pi m\sigma^2\)^{3/2}}
		    \(
			\dfrac{3}{2} \(m\sigma^2\)^2\sqrt{2m\pi\sigma^2}
	    		\displaystyle{\int}_0^R 4\pi r^2 \rho_0\left[
				e^{\frac{\phi}{\sigma^2}}\erf\(\sqrt{\dfrac{\phi(r)}{\sigma^2}}\)
			\right. \right. \notag \\
	    &\qquad \left.\vphantom{\(\sqrt{\dfrac{\phi(r)}{\sigma^2}}\) }\left.\vphantom{\(\sqrt{\dfrac{\phi(r)}{\sigma^2}}\)}
				- \sqrt{\dfrac{4\phi(r)}{ \pi\sigma^2}}
				\(1 + \dfrac{2\phi(r)}{3\sigma^2}\)
			\right] \dr
			-\displaystyle{\int}_0^R 4\pi r^2 \rho_0 \dfrac{\(2m\phi(r)\)^{5/2}}{5} \dr
		\) \notag \\
	    &= \frac{4\pi}{Mm^2\(2\pi m\sigma^2\)^{3/2}}
		    \(
			\dfrac{3}{2} \(m\sigma^2\)^2\sqrt{2m\pi\sigma^2}
	    		\displaystyle{\int}_0^R 4\pi r^2 \rho(r) \dr
			-\displaystyle{\int}_0^R 4\pi r^2 \rho_0 \dfrac{\(2m\phi(r)\)^{5/2}}{5} \dr
		\) \notag \\
	    &= \frac{4\pi}{Mm^2\(2\pi m\sigma^2\)^{3/2}}
		    \(
			\dfrac{3}{2} \(m\sigma^2\)^2\sqrt{2m\pi\sigma^2}
	    		M
			-\displaystyle{\int}_0^R 4\pi r^2 \rho_0 \dfrac{\(2m\phi(r)\)^{5/2}}{5} \dr
		\) \notag \\
	    &= \frac{4\pi}{Mm^2\(2\pi m\sigma^2\)^{3/2}}
			\dfrac{3}{2} \(m\sigma^2\)^2\sqrt{2m\pi\sigma^2}
	    		M
			-\frac{4\pi}{Mm^2\(2\pi m\sigma^2\)^{3/2}}\displaystyle{\int}_0^R 4\pi r^2 \rho_0 \dfrac{\(2m\phi(r)\)^{5/2}}{5}\dr \notag \\
	    &= \frac{3\sigma^2}{m} - \frac{4\pi}{Mm^2\(2\pi m\sigma^2\)^{3/2}}\displaystyle{\int}_0^R 4\pi r^2 \rho_0
	    \dfrac{\(2m\phi(r)\)^{5/2}}{5}\dr \notag \\
	    &= \frac{3\sigma^2}{m} - \frac{32\pi^2\rho_0}{5 M m\(\pi \sigma^2\)^{3/2}}\displaystyle{\int}_0^R r^2 \(\phi(r)\)^{5/2}\dr
\end{align}
\begin{figure}[h!]
	\centering \includegraphics[width=1.00\textwidth]{theorie/graphe/Temperature_centre-moyenne.pdf}
	\caption{Températures maximums et moyennes en fonction de $W_0$\label{courbe::Moy}}
\end{figure}
La courbe~\ref{courbe::Moy} représente la variation des températures moyennes et centrales pour notre amas. Sur la courbe verte représentant la
température moyenne, un régime assez étrange commence à apparaître vers les $W_0$. Ce nouveau régime peut avoir 2 source possible~:
\begin{enumerate}
	\item une instabilité numérique apparaissant à cause de valeurs trop importante ou trop faible (~présence de \textsc{Not A Number} dans le
		fichier de donnée~),
	\item le $W_0$ à une valeur maximum au-delà de laquelle les équations ne sont plus valable.
\end{enumerate}
La première chose à faire dans ce cas est de tracer le maximum de profil de densité pour ces valeurs, de même il va falloir étudier la variation de
$\rho(r=0)$ en fonction de $W_0$.

Il est utile de rappeler que le but de ces travaux est de trouver les quantités à utiliser pour tracer un diagramme d'énergie similaire à celui de la
sphère isotherme en boîte, et ceci afin d'en étudier la stabilité.

Pour ceci nous avons d'avoir chercher à vérifier que le modèle est isotherme, ce qui est vérifié par les courbes~\ref{courbe::Temp}. Puis nous devions
savoir si la quantités à utiliser comme température est la température moyenne ou la température centrale. Ce dernier point est assez délicat
\FloatBarrier

