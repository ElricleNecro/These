
La température cinétique d'un système autogravitant est la valeur moyenne dans l'espace des positions du carré de la vitesse d'une particule test. Comme $\vec p = m\vec v $, si $f(\vec r, \vec p, t)$ est la fonction de distribution de ce système et $\rho(\vec r , t)=m \int f(\vec r, \vec p, t) d\vec p$ sa densité, alors sa température cinétique $T(\vec r , t)$ est définie par la relation
\begin{equation}
	T(\vec r , t)=\dfrac{\int v^2 f(\vec r, \vec p, t) d\vec p}{\int f(\vec r, \vec p, t) d\vec p}=\dfrac{1}{m\rho(\vec r, t)} \int p^2 f(\vec r, \vec p, t) d\vec p
\end{equation}
Un rapide calcul montre que dans la cas de la sphère isotherme (\ref{sph-iso}), la température $T(\vec r , t)=(k_B\beta)^{-1}$ est constante en chaque point, d'où le nom de ce système. Nous avons vu que la sphère de King est une sphère isotherme tronquée en énergie, étudions l'influence de cette troncature sur la température.
 
\subsection{Calcul de la température en chaque point}

Dans le cas de la fonction de distribution de King, nous avons
\begin{equation}
	\int_0^{p_l} p^2 f_K\(E\) d\vec{p} 
	=
	4\pi \dfrac{\rho_0}{\(2\pi m\sigma^2\)^{3/2}} \int_0^{p_l}\( e^{\frac{E_l-E}{\sigma^2}} - 1\)  p^4 \ddp 
\end{equation}
Puisque l'é
nergie moyenne par particule $E = \dfrac{p^2}{2m} + m\psi$ ne dépend que des modules $r$ de $\vec r$ et $p$ de $\vec p$, la température ne dépend spatialement que de $r$. Nous avions vu que $p_l = \sqrt{2m\(E_l - m\psi(r)\)} = \sqrt{2m\phi(r)}$, une intégration par parties et quelques lignes de calcul donnent alors directement
\begin{equation}
\int_0^{p_l} p^2 f_K\(E\) d\vec{p} =
\frac{4\pi\rho_0}{m^2\(2\pi m\sigma^2\)^{3/2}}
		\(
			\dfrac{3}{2} \(m\sigma^2\)^2\sqrt{2m\pi\sigma^2}
	    		\left[
				e^{\frac{\phi}{\sigma^2}}\erf\(\sqrt{\dfrac{\phi(r)}{\sigma^2}}\)
			\right. \right. \notag \\
	    \left.\vphantom{\(\sqrt{\dfrac{\phi(r)}{\sigma^2}}\) }\left.\vphantom{\(\sqrt{\dfrac{\phi(r)}{\sigma^2}}\)}
				- \sqrt{\dfrac{4\phi(r)}{ \pi\sigma^2}}
				\(1 + \dfrac{2\phi(r)}{3\sigma^2}\)
			\right]
			- \dfrac{\(2m\phi(r)\)^{5/2}}{5}
		\) 
\end{equation}

Nous en déduisons la température cinétique de la sphère de King en divisant par le produit $m\rho$; après quelques simplifications et en utilisant la fonction $\gamma(r)=\phi/\sigma^2$,  il vient
\begin{equation}
	T_K (r) 
	= \dfrac{3\sigma^2}{m}	
		\(
			1
		- 
		\dfrac{8}{15 \sqrt{\pi}}
		\(\dfrac{\rho(r)}{\rho_0}\)^{-1}\(\gamma(r)\)^{5/2}
		\)
\end{equation}
Cette expression n'est pas très explicite mais son calcul numérique ne présente pas de difficulté dès lors que l'on a déterminé, numériquement aussi, la fonction $\phi(r)$. Il s'avère intéressant de tracer la fonction $$\zeta(r)=\(\dfrac{\rho(r)}{\rho_0}\)^{-1}\(\gamma(r)\)^{5/2}$$ et de la comparer avec la valeur $1$. C'est l'objet de la figure \ref{kingisotherme}. Outre le facteur $\dfrac{8}{15 \sqrt{\pi}}\approx 0.3$, on remarque que pour des valeurs de $W_0\leq15$, la température $T(r)$ est quasiment constante sur dans les régions ou la densité est importante ($\frac{\rho(r)}{\rho_0}>10^{-5}$ pour $W_0=15$ et bien mieux pour les valeurs de $W_0$ plus faibles).

\begin{figure}[H]
	\begin{minipage}[b]{0.40\linewidth}
		\centering \includegraphics[scale=0.40]{theorie/graphe/Courbes_temperature_KING.pdf}
	\end{minipage}\hfill
	\begin{minipage}[b]{0.48\linewidth}
		\centering \includegraphics[scale=0.40]{theorie/graphe/Courbes_densite_KING.pdf}
	\end{minipage}
	\caption{Température d'un modèle de King : les fonctions $\zeta(r)$ et $\rho(r)$.}
	\label{kingisotherme}
\end{figure}


\subsection{Température moyenne d'un modèle de \King}
\label{Calc::Temp}

Hormis le cas particulier  de la sphère isotherme, la température cinétique est une fonction de la position\footnote{Il s'agit même d'une fonction du temps si le système est hors de l'équilibre. Dans ce cas la notion de température perd un peu de son sens...}. On peut tout de même définir une température moyenne en intégrant la température cinétique dans l'espace des positions. On prendra par exemple et par définition
\begin{equation}
T_{\mathrm{moy}} = \dfrac{\displaystyle{\int}\frac{p^2}{m^2}f(\vec r, \vec p, t) d\vec p d\vec r}
                                        {\displaystyle{\int} f(\vec r, \vec p, t) d\vec p d\vec r}
\end{equation}
IL est clair que pour la sphère isotherme $T_{\mathrm{moy}}=T(r)=T$; pour la sphère de King un calcul final montre que 
\begin{equation}
	    T_{K,\mathrm{moy}}
	    = \frac{3\sigma^2}{m} - \frac{32\pi^2\rho_0}{5 M m\(\pi \sigma^2\)^{3/2}}\displaystyle{\int}_0^R r^2 \(\phi(r)\)^{5/2}
			dr
\end{equation}
\begin{figure}[h!]
	\centering \includegraphics[width=1.00\textwidth]{theorie/graphe/Temperature_centre-moyenne.pdf}
	\caption{Températures maximums et moyennes en fonction de $W_0$\label{courbe::Moy}}
\end{figure}
La courbe~\ref{courbe::Moy} représente la variation des températures moyennes et centrales pour notre amas. 

DANS LA CAPTION DE LA FIGURE \ref{courbe::Moy} ON PARLE DE TMAX ???
CONCLUSION !!!!!