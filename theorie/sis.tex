\section{La sphère isotherme et son problème intrinsèque}
	La sphère isotherme est la solution classique fournie par une approche thermodynamique du problème de l'équilibre d'un système autogravitant.
	Elle apparait lorsque nous cherchons un maximum de l'entropie statistique de Boltzmann $S(f) = -k_B \int f\ln(f) d\vec{r}d\vec{p}$, où $k_B$ est la constante de Boltzmann.
	En imposant les contraintes d'une masse fixée:
	\begin{align}
		M = \int \rho(\vec{r},t) d\vec{r}
	\end{align}
	et une énergie totale fixée:
	\begin{align}
		H = \int \frac{\vec{p}\,^2}{2m} f d\vec{r}d\vec{p}-\frac{Gm^2}{2}\int \int 
		\frac{f(\vec{r},\vec{p})f(\vec{r}\,^{\prime},\vec{p}\,^{\prime})}
		      {\left|\vec{r}-\vec{r}\,^{\prime}\right|}
		d^3\vec{r}\,^{\prime}d^3\vec{p}\,^{\prime}
	\end{align}
	Pour le système sa fonction de distribution s'écrit :
	\begin{align}
		f(\vec{r},\vec{p})=f(E) = \(\frac{2\pi \alpha^2 m}{\beta}\)^{-3/2} e^{-\beta E}
		\label{sph-iso}
	\end{align}
	La fonction $E=\frac{\vec{p}\,^2}{2m}+m\psi(\vec{r})$ caractérise l'énergie d'une particule test.
	Les constantes $\beta = \frac{1}{k_B T}$ et $\alpha$ sont des multiplicateurs de \textsc{Lagrange} associés aux contraintes imposées lors de la recherche de l'extremum de l'entropie :
	$\beta \Leftrightarrow H = \mathrm{cte}$ et $\alpha \Leftrightarrow M = \mathrm{cte}$.

	Les différents problèmes posés par cette solution ont émaillé la recherche dans ce domaine tout au long du
	\textsc{xx}$^{e}$ siècle d'\cite{emden07} à \cite{1989ApJS...71..651P}  en passant par
	\cite{chandra39}, la revue de \cite{2006IJMPB..20.3113C} pourra être consulté à ce sujet.
	% on pourra consulter à ce sujet la revue de \cite{2006IJMPB..20.3113C}.
	Nous en reprendrons les éléments principaux.
	
\section{Formulation générale du problème}
	
	La densité de la sphère isotherme se calcule directement à partir de la définition~\ref{sph-iso}:
	\begin{align}
		\rho(r) &= m \int \(\frac{2\pi \alpha^2 m}{\beta}\)^{-3/2} e^{-\beta \(\frac{p^2}{2m} + m\psi\)} d^3 p \notag \\
			&= 4\pi m \(\frac{2\pi \alpha^2 m}{\beta}\)^{-3/2} \int_0^\infty e^{-\beta \(\frac{p^2}{2m} + m\psi\)} p^2 dp \notag \\
			&= \frac{m}{\alpha^3} e^{-m\beta\psi}
	\end{align}
	Nous avons $\rho(\vec{r}) = \rho(\psi)$, le théorème Gidas-Ni-Niremberg s'applique (voir~\cite{CoursJP}) et le système est donc à symétrie sphérique $\rho(\vec{r}) = \rho(r)$  dans l'espace des positions.
	Avec cette symétrie radiale l'équation de Poisson s'écrit:
	\begin{align}
		\frac{1}{r^2}\frac{d}{dr}\(r^2\frac{d\psi(r)}{dr}\) &= 4\pi G \rho(r) = 4\pi G \frac{m}{\alpha^3} e^{-m\beta\psi(r)} \notag \\
		\intertext{En introduisant les variables $y = m\beta\psi$ et $x = r/r_0$, nous obtenons :}
		\frac{1}{x^2}\frac{d}{dx}\(x^2\frac{d y}{dx}\) &=  \frac{4\pi G m r_0^2}{\alpha^3} e^{-y} \notag \\
		\intertext{Nous pouvons alors choisir $r_0^2 = \frac{\alpha^3}{4\pi G m r_0^2}$ afin d'adimensionner l'équation de Poisson sous la forme:}
		\frac{1}{x^2}\frac{d}{dx}\(x^2\frac{d y}{dx}\) &= e^{-y} \label{Pois:sis}
	\end{align}
	
	Nous pouvons chercher dans un premier temps  des solutions autosimilaires pour cette équation, elles sont de la forme $\Tilde{y}$ telles que :
	\begin{align}
		\Tilde{\rho}(r) &= \frac{A}{r^2} = e^{-\Tilde{y}} \quad
		\Rightarrow \quad \Tilde{y} = - \ln\(\frac{A}{x^2}\)
	\end{align}
	Il est facile de vérifier que de telles solutions n'existent que si $A = 2$.
	% , nous avons donc 
	% \begin{align}
		% \Tilde{\rho}(r) &= \frac{m}{\alpha^3} e^{\ln\(\frac{2}{x^2}\)} = \frac{2 m r_0^2}{\alpha^3 r^2}
		% \intertext{La masse $M(r)$ contenue dans la sphère de rayon $r$ incluse dans ce système s'écrit}
		% \Tilde{M}(r)    &= \int_0^{\infty} \Tilde{\rho}(r) d^3 r = \frac{4\pi m r_0^2}{\alpha^3} \int_0^{r} r^2\frac{1}{r^2} dr = \frac{4\pi m r_0^2}{\alpha^3} r
	% \end{align}

	Les propriétés physiques de cette solution autosimilaire sont singulières :
	\begin{itemize}
		\item la densité diverge en zéro :
		\begin{align*}
			\lim\limits_{r \to 0} \Tilde{\rho}(r) &= \infty
		\end{align*}
		\item la masse est infinie si le support du système n'est pas limité (ce qui est inclu dans les hypothèses):
		\begin{align*}
			\lim\limits_{r \to \infty} \Tilde{M}(r) &= \infty
		\end{align*}
	\end{itemize}
	
	Cette solution forme ce que l'on appelle une sphère isotherme singulière (\textsc{sis}), elle ne peut en aucun cas
	correspondre à la solution thermodynamique recherchée qui doit posséder une masse finie.

	Étudions à présent l'existence de solutions plus générales pouvant avoir une densité et une masse qui ne
	diverge pas. Nous pratiquons pour cela un changement de fonction, en introduisant  $\zeta = y - \Tilde{y}$, la
	différence entre la solution générale $y$ du problème et la \textsc{sis}. Il vient successivement
	\begin{align}
		\frac{1}{x^2}\frac{d}{dx}\(x^2\frac{d\zeta}{dx}\) &= e^{-\zeta - \Tilde{y}} - \frac{1}{x^2}\frac{d}{dx}\(x^2\frac{d\Tilde{y}}{dx}\) \notag \\
								  &= e^{-\zeta - \Tilde{y}} - e^{-\Tilde{y}} \notag \\
								  &= \(e^{-\zeta} - 1\)\frac{2}{x^2}
			\end{align}
et en utilisant l'inconnue%
\begin{align*}
t=\ln\left(  x\right)
\end{align*}
il vient%
\begin{align}
\frac{d^{2}\zeta}{dt^{2}}+\frac{d\zeta}{dt}=2\left(  e^{-\zeta}-1\right)
\label{eq_diff_iso}%
\end{align}
Le problème consiste donc à étudier les propriétés générales de cette équation.

\section{Propriétés de la solution générale et conséquences}

Dans la littérature, le seul cas traité explicitement est celui de la linéarisation de l'équation autour de $\zeta=0$
(voir~\citet{chandra39}).
% Notons que les seuls résultats trouvés dans la littérature concernant le comportement asymtotique de ce système
% concernent son linéarisé au voisinage de l'origine (voir \cite{chandra39} qui demeure la référence absolue).
Nous proposons ici une étude plus générale dans la totalité du plan de phase. Posons $\vec{z}=\left[
\zeta,\dot{\zeta}=\frac{d\zeta}{dt}\right]^{\top}$, nous avons:
\begin{align}
\frac{d\vec{z}}{dt} =F\left(  \vec{z}\right)  =
\left[\dot{\zeta},2\left(  e^{-\zeta}-1\right)-\dot{\zeta}\right]^{\top} \label{sysdif}%
\end{align}
Le seul point d'équilibre est l'origine $\vec{z}_0=\left[  0,0\right]
^{\top}$. Considérons à présent:
\begin{align*}
\mathcal{E}(\zeta,\dot{\zeta}) = \frac{1}{2}\dot{\zeta}^2+2(e^{-\zeta}-\zeta-1)
\end{align*}
% Il est clair que :
Il est apparaît que :
\begin{itemize}
\item La fonction $\mathcal{E}$ est nulle en $\vec{z}=\vec{z}_0$ ;
\item La hessienne de $\mathcal{E}$ est définie positive sur $\mathbb{R}^2_*$: 
\begin{align*}
H_{\mathcal{E}}:=
\left[
\begin{array}
[c]{cc}%
\frac{\partial^2 \mathcal{E}}{\partial \zeta^2}         & \frac{\partial^2 \mathcal{E}}{\partial \zeta \partial \dot{\zeta}}\\
 & \\
\frac{\partial^2 \mathcal{E}}{\partial \dot{\zeta} \partial \zeta}& \frac{\partial^2 \mathcal{E}}{\partial \dot{\zeta}^2}
\end{array}
\right]
=\left[\begin{array}
[c]{cc}%
2e^{-\zeta}         & 0\\
0& 1
\end{array}
\right]
\end{align*}
et donc le point $\vec{z}_0$ est un minimum global de $\mathcal{E}$;
\item La fonction $\mathcal{E}$ est strictement décroissante de la variable $t$, en effet:
\begin{align*}
\frac{d\mathcal{E}}{dt}&=\frac{d\zeta}{dt}\frac{\partial \mathcal{E}}{\partial \zeta}+\frac{d\dot{\zeta}}{dt}\frac{\partial \mathcal{E}}{\partial \dot{\zeta}} \\
\\
&=-\dot{\zeta}^2 
\end{align*}

\end{itemize}
Les trois propriétés énumérées ci-dessus font de la fonction $\mathcal{E}$ une fonction de Lyapounov\index{Fonction! de Ljapounov}\index{Ljapounov, fonction de} stricte du système, l'équilibre $\vec{z}_0$ est donc globalement asymptotiquement stable. 

Le comportement asymptotique $\left(t\rightarrow+\infty\right)$ de la
solution $\zeta(t)$ s'obtient en considérant une combinaison linéaire
des exponentielles des valeurs propres de la matrice
\begin{align*}
	A:=D\left[F(\vec{z})\right](\vec{z}_0)=\left[
\begin{array}
[c]{cc}%
0     & 1\\
-2 & -1\\
\end{array}
\right]
\end{align*}
soit $\left(  -1\pm i\sqrt
{7}\right)  /2$, ainsi lorsque $x\rightarrow+\infty$ nous avons
\begin{align*}
\zeta\left(  x\right)
\sim\frac{k_{1}\cos\left[  \ln\left(  x^{\sqrt{7}/2}\right)  \right]
+k_{2}\sin\left[  \ln\left(  x^{\sqrt{7}/2}\right)  \right]  }{\sqrt{x}%
}\ \ \ \text{avec }k_{1},k_{2}\in\mathbb{R}%
\end{align*}
En majorant les fonctions trigonométriques, nous avons donc $\zeta\left(
x\right)  \sim k/x^{1/2}$ en $x\rightarrow+\infty$, soit $y\left(  x\right)
\sim k/x^{-1/2}-\ln\left(  2/x^{2}\right)  $ soit pour la densité en
variable $r$:
\begin{align}
\rho\left(  r\right)  \sim\frac{2mr_{o}^{2}}{\alpha^{3}r^{2}}\left(
1\pm\left(  \frac{r_{k}}{r}\right)  ^{1/2}\right)  \ \ \ \text{quand
}r\rightarrow+\infty\text{.}\label{asymp_sph_iso}%
\end{align}
% La longueur $r_{k}$ se calcule en écrivant proprement la majoration, en
% revenant aux variables physiques et en calculant le laplacien, elle n'a que
% peu d'intér\^{e}t. Le signe $\pm$ provient de l'encadrement des fonctions
% trigonométriques. Ce qu'il faut remarquer dans cette
% relation, c'est que la masse d'une sphère isotherme est \emph{toujours} infinie si
% celle-ci est d'extension infinie $\left(  r\rightarrow+\infty\right)$.
La longueur $r_{k}$ se calcule en écrivant proprement la majoration, en
revenant aux variables physiques et en calculant le laplacien. Le signe $\pm$ provient de l'encadrement des fonctions
trigonométriques. Ce qu'il faut remarquer dans cette
relation, c'est que la masse d'une sphère isotherme est \emph{toujours} infinie si
celle-ci est d'extension infinie $\left(  r\rightarrow+\infty\right)$.

La fonction de distribution~\ref{sph-iso} de la sphère isotherme n'est donc pas acceptable car elle est en contradiction avec les hypothèses posées pour l'obtenir.

Le problème de l'équilibre thermodynamique d'un système autogravitant est donc posé !

Pour palier à ce problème, plusieurs approches ont été développées au fil des ans :

\begin{enumerate}
	\item Une approche \og{}rigoureuse\fg consiste à chercher le maximum de l'entropie dans un ensemble de fonctions de distribution à support compact :
		% l'extension spatiale du système est alors finie. Ce problème est communément appelé sphère isotherme en boîte (~SIB~). Il sera étudié en détail dans le chapitre \label{SIB::Chapitre}.
		l'extension spatiale du système est alors finie. Ce problème est communément appelé sphère
		isotherme en boîte (\textsc{sib}). Il sera étudié en détail dans le chapitre~\ref{SIB::Chapitre}.
		
	\item Une solution \og{}pragmatique\fg: développée par \cite{King-1966AJ} consiste à tronquer à la main
		et après coup la sphère isotherme. Au-dessus d'une certaine énergie, la fonction de
		distribution est nulle. Bien que empirique cette solution est devenue un modèle de choix pour
		l'ajustement du profil de densité de nombreux amas globulaires et de galaxies naines. Ses trois
		paramètres libres rendent en effet son utilisation adaptée à la modélisation de tels objets.
		Nous y consacrerons le chapitre~\ref{King::Chapitre}.
	
	\item Plus récemment une prise en compte fine des spécificités des modèles gravitationnels semble avoir fait progresser les choses :
		il s'agit du paradigme Darkexp (voir~\citet{2010ApJ...722..851H}). En physique statistique, l'approximation $\ln x! = x\ln x
		-x$, valide pour $x\gg1$, est souvent utilisé. Dans ce type de calcul, $x$ est une fonction du nombre de particule $N$
		présentes dans le système. Les auteurs de ce paradigme remarque que dans notre cas (étude des galaxies ou des amas
		globulaire), ce nombre $N$ n'est pas suffisamment grand pour que nous puissions utiliser l'approximation de Stirling. Cette
		correction semble, selon les auteurs de ce paradigme, pouvoir effectuer une coupure rendant la sphère isotherme plus
		compatible avec un modèle physique acceptable.

\end{enumerate}
