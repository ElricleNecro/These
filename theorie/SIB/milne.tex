\subsection{Présentation du problème}
	Comme nous l'avons vu dans le chapitre précédent, les sphères isothermes ne sont pas une solution satisfaisante, du fait de leurs masses infinies.
	Nous allons présenter dans ce chapitre une solution de sphère isotherme intégrée sur un support borné. Cette solution est toujours obtenue en recherchant les maxima locaux de l'entropie statistique :
	\begin{align*}
		S(f) = - k_B \int f \ln f d\Gamma
	\end{align*}
	mais pour une fonction $f$ possédant un support borné. Ce dernier est par exemple une boîte (~d'où le nom du chapitre~) définie comme :
	\begin{align}
		B = \left\{ \vec{r} \in \mathbb{R}^3\ ;\ \left|\vec{r}\right| < R\right\}
	\end{align}
	avec $R$ le rayon de la sphère. Dans la suite, nous noterons $\mathbb{I}_{B_R}$ la fonction indicatrice sur cette sphère.

\subsection{Obtention des équations}
	La fonction de distribution d'une sphère isotherme en boîte, s'écrit :
	\begin{eqnarray}
		f^+(E) = \left(\frac{2\pi\alpha^2m}{\beta}\right)^{-3/2}e^{-\beta E}\times\mathbb{I}_{B_R}
	\end{eqnarray}
	où $\alpha$, en mètre, et $\beta$, en Joule, sont des multiplicateurs de \textsc{Lagrange}, \mbox{$\beta = \frac{1}{k_B T}$}.
	Ils servent à normaliser la fonction sur son intervalle de définition.
	La quantité \mbox{$E = \frac{p^2}{2m} - m\psi(r)$} est l'énergie d'une particule test de masse $m$ se déplacant dans
	le potentiel $\psi$ créé par la sphère isotherme.
	Le calcul de la densité de masse donne alors \mbox{$\rho(r) = \frac{m}{\alpha^3}e^{-\beta m \psi(r)}\times\mathbb{I}_{B_R}$}. % \psi'(r)}$}, dans la suite nous utiliserons :
	%\mbox{$\psi'(r) = \psi(r) - \psi(0)$}~\footnote{pour faciliter les calculs dans les variables \textsc{Milne} définies ci-dessous}.
	Nous injectons ensuite la densité dans l'équation de \textsc{Poisson} :
	\begin{align}
		\frac{1}{r^2}\frac{d}{dx}\left(r^2\frac{d \psi}{dx}\right) &= \frac{4\pi m G}{\alpha^3} e^{-m\beta \psi(r)} \times\mathbb{I}_{B_R} \Rightarrow \frac{1}{x^2}\frac{d}{dx}\left(x^2\frac{d h}{dx}\right) = e^{-h} \times\mathbb{I}_{B_R} \notag
		\intertext{où nous avons utilisé l'adimensionnement suivant :}
		x &= \frac{r}{r_0}\mathrm{,}\ r_0 = \sqrt{\frac{\alpha^3}{4\pi G m^2\beta}}e^{-\frac{m\beta\psi(0)}{2}} \notag \\
		h &= m\beta\psi(x) - m\beta\psi(0) \notag
		\intertext{il est important de noter que $h$ est différente du $y$ utilisé dans le chapitre précédent : ici $h(0) = 0$ alors que $y(0) = m\beta\psi(0)$. La transformation suivante a été introduite par \textsc{Milne}, avec $\dfrac{d h}{dx} = \x{h}$ :} %(~nous utiliserons la notation indicielle pour les dérivées selon $x$ : $\frac{d h}{dx} = \x{h}$~) :
		%	\left\{\begin{array}{l}
		&\begin{cases}
			v = x \dfrac{d h}{dx} = x \x{h} \\
			\\
			u = \dfrac{e^{-h} x}{h'} = \dfrac{e^{-h} x^2}{v}
		\end{cases} \label{syst_uv}
	%	\end{array}\right.\label{syst_uv}
	\intertext{Un calcul simple donne alors :}
	\dfrac{d(xv)}{dx} &= uv \notag \\
	\Rightarrow \x{v} &= \frac{v \left( u - 1\right)}{x} \notag
	\intertext{Pour obtenir l'équation sur $u$, nous dérivons l'expression de la transformation, laquelle nous donne finalement :}
	%	\left\{\begin{array}{l}
	&\begin{cases}
		\x{v} = \dfrac{v \left( u - 1\right)}{x} \\
		\x{u} = \dfrac{u}{x}\left(3 - v - u\right)
	\end{cases} \label{systdudv} \\
	%	\end{array}\right. \\
	\end{align}
	\begin{equation}
		\Rightarrow \fbox{$
		\dfrac{d v}{d u} = \dfrac{v \left( u - 1\right)}{u \left(3 - u - v\right)}
		$} \label{eqdudv}
	\end{equation}
	La courbe résultant de cette équation, tracée dans le plan $\left(u, v\right)$, forme le diagramme
	de \textsc{Milne}. Ce diagramme contient donc toutes les caractéristiques physiques d'une sphère isotherme
	en boîte. Ces équations sont résolues numériquement, à l'aide d'un \textsc{Runge-Kutta} d'ordre 4 et des développements donnés dans la section suivante.
	%C'est à partir de cette équation que nous allons pouvoir obtenir le diagramme de Milne.
%	La fonction de distribution d'une sphère isotherme en boîte, s'écrit :
\begin{eqnarray}
f^+(E) = \left(\frac{2\pi\alpha^2m}{\beta}\right)^{-3/2}e^{-\beta E}\times\mathbb{I}_{B_R}
\end{eqnarray}
où $\alpha$, en mètre, et $\beta$, en Joule, sont des multiplicateurs de \textsc{Lagrange}, \mbox{$\beta = \frac{1}{k_B T}$}.
Ils servent à normaliser la fonction sur son intervalle de définition.
La quantité \mbox{$E = \frac{p^2}{2m} - m\psi(r)$} est l'énergie d'une particule test de masse $m$ se déplacant dans
le potentiel $\psi$ créé par la sphère isotherme.
Le calcul de la densité de masse donne alors \mbox{$\rho(r) = \frac{m}{\alpha^3}e^{-\beta m \psi(r)}\times\mathbb{I}_{B_R}$}. % \psi'(r)}$}, dans la suite nous utiliserons :
%\mbox{$\psi'(r) = \psi(r) - \psi(0)$}~\footnote{pour faciliter les calculs dans les variables \textsc{Milne} définies ci-dessous}.
Nous injectons ensuite la densité dans l'équation de \textsc{Poisson} :
\begin{align}
	\frac{1}{r^2}\frac{d}{dx}\left(r^2\frac{d \psi}{dx}\right) &= \frac{4\pi m G}{\alpha^3} e^{-m\beta \psi(r)} \times\mathbb{I}_{B_R} \Rightarrow \frac{1}{x^2}\frac{d}{dx}\left(x^2\frac{d h}{dx}\right) = e^{-h} \times\mathbb{I}_{B_R} \notag
	\intertext{où nous avons utilisé l'adimensionnement suivant :}
	x &= \frac{r}{r_0}\mathrm{,}\ r_0 = \sqrt{\frac{\alpha^3}{4\pi G m^2\beta}}e^{-\frac{m\beta\psi(0)}{2}} \notag \\
	h &= m\beta\psi(x) - m\beta\psi(0) \notag
	\intertext{il est important de noter que $h$ est différente du $y$ utilisé dans le chapitre précédent : ici $h(0) = 0$ alors que $y(0) = m\beta\psi(0)$. La transformation suivante a été introduite par \textsc{Milne}, avec $\dfrac{d h}{dx} = \x{h}$ :} %(~nous utiliserons la notation indicielle pour les dérivées selon $x$ : $\frac{d h}{dx} = \x{h}$~) :
%	\left\{\begin{array}{l}
	&\begin{cases}
		v = x \dfrac{d h}{dx} = x \x{h} \\
		\\
		u = \dfrac{e^{-h} x}{h'} = \dfrac{e^{-h} x^2}{v}
	\end{cases} \label{syst_uv}
%	\end{array}\right.\label{syst_uv}
	\intertext{Un calcul simple donne alors :}
	\dfrac{d(xv)}{dx} &= uv \notag \\
	\Rightarrow \x{v} &= \frac{v \left( u - 1\right)}{x} \notag
	\intertext{Pour obtenir l'équation sur $u$, nous dérivons l'expression de la transformation, laquelle nous donne finalement :}
%	\left\{\begin{array}{l}
	&\begin{cases}
		\x{v} = \dfrac{v \left( u - 1\right)}{x} \\
		\x{u} = \dfrac{u}{x}\left(3 - v - u\right)
	\end{cases} \label{systdudv} \\
%	\end{array}\right. \\
\end{align}
\begin{equation}
	\Rightarrow \fbox{$
	\dfrac{d v}{d u} = \dfrac{v \left( u - 1\right)}{u \left(3 - u - v\right)}
	$} \label{eqdudv}
\end{equation}
La courbe résultant de cette équation, tracée dans le plan $\left(u, v\right)$, forme le diagramme
de \textsc{Milne}. Ce diagramme contient donc toutes les caractéristiques physiques d'une sphère isotherme
en boîte. Ces équations sont résolues numériquement, à l'aide d'un \textsc{Runge-Kutta} d'ordre 4 et des développements donnés dans la section suivante.
%C'est à partir de cette équation que nous allons pouvoir obtenir le diagramme de Milne.



\subsection{Conditions sur les bords de la sphère}
	Pour obtenir les conditions initiales permettant la résolution de l'équation~\ref{eqdudv}, nous devons obtenir les conditions aux limites pour la sphère isotherme en boîte.
\subsubsection{En $x = 0$}
	Pour ce faire, nous nous plaçons dans le voisinage de $x=0$, et faisons donc des développements limités de la variable $h$ représentant le potentiel.
	Cela nous permettra d'obtenir le comportement approché de $u(x)$ et de $v(x)$ au centre.
	Nous avons donc :
	\begin{eqnarray*}
		h(x) = h(0) + \x{h}(0)x + \frac{d^2 h}{dx^2}(0)\frac{x^2}{2!} + \frac{d^3 h}{dx^3}(0)\frac{x^3}{3!} + o(x^3) = ax^2 + bx^3 + o(x^3)
	\end{eqnarray*}
	En effet, \mbox{$h(0) = m\beta\left(\psi(0) - \psi(0)\right) = 0$} et \mbox{$\x{h}(0) = m\beta\x{\psi}(0) = 0$} car \mbox{$\x{\psi} = r_0\frac{d \psi}{dr}\propto F$},
	$F$ étant la force s'appliquant sur la particule test.
	Au centre de la sphère, les forces qui s'appliquent à une particule test s'opposent les unes aux autres, d'où $\x{h}(0) = 0$.

	Les variables $u$ et $v$ vont alors se développer comme :
	\begin{eqnarray}
		v &=& x\x{h} = 2ax^2 + 3bx^3 + o(x^3)\label{vDL} \\
		u &=& \frac{x^2e^{-ax^2 - bx^3 + o(x^3)}}{2ax^2 + 3bx^3 + o(x^3)} \\
		  &=& \frac{1 - ax^2 - bx^3 + o(x^3)}{2a + 3bx +o(x)} \\
		  &=& \frac{1}{2a} - \frac{3b}{4a^2}x + o(x)\label{uDL}
	\end{eqnarray}
	Nous utilisons ensuite l'équation de Poisson pour déterminer, par identification, les valeurs de $a$ et de $b$ :
	\begin{eqnarray}
		\x{\left(xv\right)} = 6ax^2 + 12bx^3 + o(x^3) = uv = x^2 + o(x^3)\label{PoisDL}
	\end{eqnarray}
	D'où :
	\begin{eqnarray}
		\left\{\begin{array}{l}
			a = \frac{1}{6} \\
			b = 0
		\end{array}\right.
	\end{eqnarray}
	Maintenant que nous avons nos développements, nous en déduisons les conditions initiales :
	$$(u,v) \substack{\longrightarrow \\ r\to 0} (3,0)$$

\subsubsection{Sur le bords de la sphère : $r = R$}
	Soit $X = \frac{R}{r_0}$. Nous allons tenter d'exprimer nos variables $u$ et $v$ au point $X$ en fonction des quantités connues du problème,
	telles que l'énergie totale $H$, la masse totale $M$, le rayon de la sphère $R$.
	Et, selon la description canonique ou micro canonique choisie, en fonction de la température cinétique du système.

	On introduit les constantes adimensionnées suivantes :
	\begin{eqnarray*}
		\left\{\begin{array}{l}
			\lambda = - \frac{H R}{G M^2} \\
			\\
			\mu     = \frac{m\beta GM}{R}
		\end{array}\right.
	\end{eqnarray*}
	où $\lambda$ représente l'énergie adimensionnée et $\mu$ l'inverse de la température adimensionnée.

	Au bord du système, nous pouvons écrire que $\frac{d \psi}{dr} \equiv \frac{GM}{R^2}$~\footnote{Théorème de Gauss}. De plus :
	\begin{eqnarray*}
		\x{h}(X) = m\beta r_0 \frac{d \psi}{dr} = m\beta r_0 \frac{GM}{R^2} = \frac{\mu}{X} \Rightarrow \mu = X\x{h}(X)
	\end{eqnarray*}
	Or $v(X) = X\x{h}(X)$, d'où :
	\begin{eqnarray}
		v\left(X\right) := v_m = \mu\label{vmmu}
	\end{eqnarray}

	Il nous reste l'énergie à exprimer.
	Pour cela, nous allons commencer par calculer l'énergie totale $H$ à l'aide du théorème du Viriel adapté à une sphère isotherme en boîte~\footnote{Pour un système tronqué, le calcul permettant d'arriver au théorème du Viriel fait apparaître des termes dû aux intégrations par partie qui vont rester.} :
	\begin{eqnarray}
		2T + W = \frac{4}{3}\pi R^3 P_e
	\end{eqnarray}
	avec $P_e$ la pression qui s'exerce sur le bord de la sphère.
	L'énergie cinétique s'écrit \mbox{$K = \frac{1}{2}mv^2 = \frac{3}{2} N k_B T = \frac{3}{2} \frac{M}{m} k_B T$}.
	Donc :
	\begin{eqnarray*}
		H &=& K + W = \frac{3}{2} \frac{M}{m\beta} + \frac{4}{3}\pi R^3 P_e - 2K \\
		  &=& -\frac{3}{2} \frac{M}{m\beta} + \frac{4}{3}\pi R^3 P_e \\
		\Rightarrow \lambda &=& \frac{3}{2}\frac{MR}{m\beta GM^2} - \frac{4}{3}\pi R^3 P_e \frac{R}{GM^2}
	\end{eqnarray*}
	Une sphère isotherme est un système barotropique. Elle a donc une équation d'état polytropique d'indice 1 : $P = \frac{\rho(r)}{m\beta} \Rightarrow P_e = \frac{\rho(R)}{m\beta}$ que nous remplaçons :
	\begin{eqnarray}
		\lambda &=& \frac{3}{2}\frac{1}{X\x{h}(R)} - \frac{4}{3}\pi R^3 \frac{\rho(R)}{m\beta} \frac{R}{GM^2} \notag \\
			&=& \frac{3}{2}\frac{1}{X\x{h}(R)} - \frac{4}{3}\pi R^3 \frac{e^{-h(R) + m\beta\psi(0)}}{\alpha^3\beta} \frac{R}{GM^2} \notag \\
			&=& \frac{3}{2}\frac{1}{X\x{h}(R)} - \frac{4}{3}\pi R^3 \frac{4\pi G \beta m^2 e^{\beta m \psi(0)}}{\alpha^3} r_0^2\frac{e^{-h(R)}}{(\x{h}(R))^2} \notag \\
			&=& \frac{3}{2}\frac{1}{X\x{h}(R)} - \frac{e^{-h(R)}}{(\x{h}(R))^2}\label{lamum}
	\end{eqnarray}

	Nous avons ensuite, par substitution, les valeurs maximales $u_m$ et $v_m$ de $u$ et $v$ en fonction des paramètres du problème :
	\begin{eqnarray}
		\label{uv_max}
		\fbox{$
		\left\{\begin{array}{l}
			v_m = \mu \\
			u_m = \frac{3}{2} - \lambda \mu
		\end{array}\right.
		$}
	\end{eqnarray}



\subsection{Droite de \textsc{Padmanabhan}}
	Il est possible d'inverser la relation n°~\ref{uv_max} pour obtenir une relation entre $u_m$, $v_m$ et l'énergie $H$ du système :
	\begin{equation}
		v_m = \frac{3}{2\lambda} - \frac{u_m}{\lambda}\label{droitePb}
	\end{equation}
	Ces droites, dites de \textsc{Padmanabhan}, passent forcément par le dernier point du diagramme représentant la sphère isotherme, du fait de leur définition.
	Ces droites vont nous permettre d'imposer des contraintes sur $\lambda$ : si jamais $\lambda$ dépasse une valeur $\lambda_c$,
	alors il n'y a plus d'intersection entre la courbe solution de l'équation~\ref{eqdudv} et la droite~\ref{droitePb}. Donc il n'existe aucune sphère isotherme pouvant avoir les valeurs d'énergie,
	de température et de rayon correspondantes.
	Cette valeur $\lambda_c$ est obtenue numériquement.

	Comme la courbe de \textsc{Milne} est une spirale (~voir diagramme n°~\ref{Milne}~),
	nous pouvons aussi trouver une valeur $\lambda_0$ à partir de laquelle il existe deux solutions pour l'intersection des deux courbes.

	Ainsi :
	\begin{itemize}
		\item si $\lambda < \lambda_0$ : il n'existe qu'une seule et unique solution de sphère isotherme.
		\item si $\lambda \in \left[\lambda_0,\lambda_c\right]$ : il existe plusieurs solutions possibles.
			Ce sont les paramètres du système, comme l'énergie totale $H$ de la sphère, qui vont discriminer entre ces solutions.
		\item si $\lambda > \lambda_c$ : il n'y a plus d'intersection, et donc plus de sphère isotherme en boîte stable.
	\end{itemize}


	À ce point, nous avons tout ce qu'il nous faut pour caractériser une sphère isotherme. Nous résolvons alors, à l'aide d'un \textsc{Runge-Kutta} d'ordre 4, le
	système~\ref{systdudv}. Choisir de résoudre le système
	\begin{align}
		\begin{cases}
			\x{v} = \dfrac{v \left( u - 1\right)}{x} \\
			\x{u} = \dfrac{u}{x}\left(3 - v - u\right)
		\end{cases} \notag
	\end{align}
	plutôt que l'équation
	\begin{align*}
		\dfrac{d v}{d u} = \dfrac{v \left( u - 1\right)}{u \left(3 - u - v\right)}
	\end{align*}
	nous permet de remonter plus facilement à l'évolution des quantités physiques comme la densité ou le potentiel de la sphère.
	Nous traçons alors les diagrammes de \textsc{Milne} correspondants aux sphères voulues. Il est intéressant de remarquer
	que tous les paramètres adimensionnés peuvent s'obtenir en connaissant $X = \frac{R}{r_0}$.

\subsection{Diagramme}
	Le diagramme de \textsc{Milne} s'obtient donc en résolvant, numériquement,
	l'équation n°~\ref{eqdudv} ou le système d'équations~\ref{syst_uv} puis en traçant la solution obtenue dans le plan $\left(u,v\right)$.
	Nous obtenons alors une spirale, comme sur la figure~\ref{Milne}.
	La courbe ainsi obtenue est une représentation de la sphère isotherme en boîte.

	Nous avons résolu et tracé, sur les quatre graphiques qui suivent, la spirale de \textsc{Milne} pour différents rayons de la sphère (~nous avons fait varier le rapport $\frac{R}{r_0}$~).
	Sont aussi tracés les droites de \textsc{Padmanabhan} correspondantes.
	Nous pouvons alors remarquer que, sur ces diagrammes, la sphère isotherme est représentée, non pas par un point, mais par la totalité de la courbe~\footnote{Résultat semblant cohérent avec le fait que les variables de \textsc{Milne} sont directement reliées au potentiel de la sphère et donc à sa densité}.
	Nous remarquons aussi que plus le rayon $R$ de la sphère est grand, plus on tend vers le point $(1,2)$ correspondant à la SIS.
	De plus, la longueur de cette courbe augmente très rapidement pour de petite valeur de $X$, puis ralentit quand elle commence à s'enrouler.
	\begin{figure}[h!]
		\begin{minipage}[b]{0.40\linewidth}
			\centering \includegraphics[scale=0.60]{graphe/r_max-1.pdf}
		\end{minipage}\hfill
		\begin{minipage}[b]{0.48\linewidth}
			\centering \includegraphics[scale=0.60]{graphe/r_max-5.pdf}
		\end{minipage}
		\begin{minipage}[b]{0.40\linewidth}
			\centering \includegraphics[scale=0.60]{graphe/r_max-10.pdf}
		\end{minipage}\hfill
		\begin{minipage}[b]{0.48\linewidth}
			\centering \includegraphics[scale=0.60]{graphe/r_max-300.pdf}
		\end{minipage}
		\caption{Diagramme de \textsc{Milne} pour $X=1$, $X=5$, $X=10$ et pour $X=300$}
		\label{Milne}
	\end{figure}

	Le diagramme de \textsc{Milne} représente donc l'état d'une sphère isotherme. La longueur de la spirale est directement liée à son rayon, et une fois atteint le point situé à l'extrémité de la courbe,
	nous pouvons déterminer la température et l'énergie totale de cette sphère. Ce diagramme contient donc toutes les informations physiques utiles, mais il
	ne nous apprend rien sur l'éventuelle stabilité de la sphère ; pour avoir cette dernière information, il nous faut étudier un autre type de diagramme : le diagramme d'énergie.
%	\FloatBarrier
