\subsection{Diagramme}
	Le diagramme n°~\ref{Ener} a été tracé à l'aide de la même résolution numérique que le diagramme de \textsc{Milne}.
	En effet, à la sortie de la résolution, nous connaissions les coordonnées du point $\left(u_m,v_m\right)$.
	Connaissant ces valeurs et leurs expressions théoriques en fonction des paramètres du système,
	nous pouvons remonter aux variables adimensionnées $\lambda$ et $\mu$ à l'aide des relations \ref{vmmu} et \ref{lamum}.
	Ces variables représentant physiquement l'inverse de la température cinétique du système (~$\mu$~) et l'énergie totale du système (~$-\lambda$~),
	nous les utilisons pour remonter à ce diagramme, constituant un diagramme de stabilité.
	\textcolor{red}{\underline{Attention :}} sur ce graphe est tracée la courbe $\mu\left(-\lambda\varpropto H\right)$.
	\begin{figure}[h!]
		\centering \includegraphics[scale=1.00]{graphe/energie.pdf}
		\caption{Diagramme d'énergie/stabilité}
		\label{Ener}
	\end{figure}

	Dans le diagramme de \textsc{Milne}, la sphère isotherme singulière correspondait au point $\left(1,2\right)$.
	Dans ce diagramme, elle va correspondre au point $\left(\lambda = \frac{u_m - 3/2}{v_m} = -1/4, \mu = v_m = 2\right)$.
	De la même manière que sur le diagramme précédent, la courbe tend en spiralant vers la sphère isotherme singulière, mais avant de l'atteindre,
	elle passe par un minimum de température, puis un maximum d'énergie. Ces deux diagrammes sont équivalents et complémentaires.

\subsection{Stabilité de la sphère isotherme en boîte}
	En physique statistique, il existe deux ensembles capables de décrire la sphère isotherme telle que nous l'avons construite :
	\begin{enumerate}
		\item l'ensemble micro canonique : nous fixons l'énergie totale de la sphère, et nous en déduisons la température cinétique.
		\item l'ensemble canonique : cette fois, l'énergie est laissée libre mais la
			température est fixée.
			A l'équilibre, nous déduisons l'énergie de la température.
	\end{enumerate}

	Le diagramme d'énergie (~figure~\ref{Ener}~) montre deux tangentes (~une verticale et une horizontale~) :
	au-delà de ces tangentes, des questions sur la stabilité commence à se poser : nous avons tronqué la SIS en la mettant dans une boîte ; ce faisant, nous avons introduit
	des contraintes sur les valeurs d'énergie et de température possible, contraintes représentées par les tangentes horizontale et verticale.
%	Si la sphère à une température tel que l'on soit au-dessus de la tangente horizontale, il n'y a plus
%	de sphère isotherme en boîte possible.
	Nous comprenons alors qu'il n'est pas possible de choisir arbitrairement la température d'une sphère isotherme en boîte et qu'il existe une température minimale pour ce type de système.
	\textsc{Katz} a montré qu'il s'agissait en fait d'une limite d'énergie (~voir~\cite{Katz-Stab}~).
	La tangente horizontale nous donne la température minimale permettant de construire une sphère isotherme en boîte
	stable. Le raisonnement est identique pour la tangente verticale conduisant à une énergie minimale.
	Ces tangentes sont tracées sur la figure~\ref{Ener-tg}.
	\begin{figure}[h!]
		\centering \includegraphics[scale=1.00]{graphe/Energie_tg.pdf}
		\caption{Les tangentes : limites de stabilité}
		\label{Ener-tg}
	\end{figure}

	Nous avons maintenant les énergies et températures limites pour la sphère isotherme en boîte.
	Pour comprendre un peu mieux ce qui se passe, voyons si nous pouvons obtenir la densité de la sphère pour ces valeurs limites.

\subsection{Contraste de densité\label{contraste-dens-SIB}}
	Nous rappelons que la densité de la sphère isotherme s'écrit :
	\begin{equation*}
		\rho(r) = \frac{m}{\alpha^3}e^{-\beta m\psi(r)}
	\end{equation*}
	Commençons par adimensionner la densité.
	Nous allons réutiliser le changement de variable sur le potentiel fait plus haut :
	$h(r) = m\beta\(\psi(r) - \psi(0)\)$,
	puis nous allons poser $\rho^s(r) = \frac{\rho(r)}{\rho_0}$, où $\rho_0 = \frac{m}{\alpha^3}e^{-m\beta\psi(0)}$ est la densité au cœur de la sphère.
	Nous obtenons alors l'équation :
	\begin{equation*}
		\rho^s(r) = e^{-h(r)}
	\end{equation*}
	qui se réécrit facilement dans les variables de \textsc{Milne} :
	\begin{equation}
		\rho^s(x) = \frac{u(x) v(x)}{x^2}
	\end{equation}

	Examinons le comportement de la densité au centre de la sphère et sur son bord externe :
	\begin{itemize}
		\item en $x=0$ : nous réutilisons les développements limités de $u$, $v$ et $uv$ fait plus haut (~équations~\ref{uDL}, \ref{vDL} et \ref{PoisDL}~) :
			\begin{align}
				\rho^s(0) &= \left(x^2 + o\left(x^3\right)\right)x^{-2} \notag \\
					  &= 1 + o\left(x\right) \notag
			\end{align}
		\item en $x=X$ : le résultat arrive assez naturellement :
			\begin{align}
				\rho^s(X) &= u_mv_mX^{-2} \notag \\
					  &= \mu X^{-2}\left(\frac{3}{2} - \lambda\mu\right) \notag
			\end{align}
	\end{itemize}

	Nous en déduisons ainsi l'expression du contraste de densité en fonction des paramètres de la sphère :
	\begin{align}
		\R = \frac{\rho(0)}{\rho(R)} &= \frac{\rho^s(0)}{\rho^s(X)} \\
			   &= \frac{X^2}{\mu\left(\frac{3}{2} - \lambda\mu\right)}
	\end{align}

	Lors de la résolution numérique du système, il a été obtenu que les valeurs de $\mu$ et $\lambda$ pour ces limites sont :
	\begin{itemize}
		\item Pour la tangente verticale : $\(\lambda, \mu\) = \left(-0.33457, 2.03152\right)$ pour $X = 34.2999$.
			%$\left(\mu, \lambda\right) = \left( 2.034753, -0.334562 \right)$ et $X = 33.99$.
			D'où un contraste de densité de $\R^H_c \thickapprox 705.967$.
		\item Pour la tangente horizontale : $\(\lambda, \mu\) = \left(-0.19876, 2.51755\right)$ pour $X = 8.9999$.
			%$\left(\mu, \lambda\right) = \left( 2.51755, -0.19874 \right)$ et $X = 8.99$.
			D'où un contraste de densité de $\R^\beta_c \thickapprox 32.186$.%16.081$.
	\end{itemize}
	L'instabilité intervient quand le système dépasse l'une de ces valeurs de contraste, selon que nous ayons imposé la température ou l'énergie.
	Ainsi, les descriptions canonique et microcanonique ne sont pas équivalentes.

%	Nous avons maintenant réduit l'étude d'une sphère isotherme en boîte à une seule quantité : le contraste de densité.
	Nous avons maintenant exprimé les limites de stabilité s'appliquant à la sphère isotherme en boîte en terme d'énergie, de température, et de contraste de densité.
	Après tous ces développements, nous résumons les paramètres utiles d'une sphère isotherme et le passage des quantités utilisées ici
	à des quantités plus physiques dans les tableaux~\ref{SIB:important} et~\ref{SIB:lien}.
	\begin{table}[h]
		\begin{center}
			\begin{tabular}{|c|c|}
				\hline
				Paramètres	&	Description \\
				\hline
				\hline
				$\R_c$		&	\textbf{Contraste de densité :} dépend du rayon de la sphère isotherme. \\
						&	Il nous donne des informations sur la stabilité de l'objet. \\
				\hline
				$X = R/r_0$	&	\textbf{Rayon de la sphère :} paramètre dont dépend toutes les informations physiques \\
						&	sur l'objet (~position dans les différents diagrammes, contraste de densité~). \\
				\hline
			\end{tabular}
			\caption{Paramètres importants\label{SIB:important}}
		\end{center}
	\end{table}

	\begin{table}[h]
		\begin{center}
			\begin{tabular}{|c|c|}
				\hline
				Quantité physique	&	Lien avec le modèle \\
				\hline
				\hline
				Distance au centre $r$	&	$r = r_0 x$ \\
				\hline
					&	\\
				Densité $\rho(r)$	&	$\rho(r) = \dfrac{m}{\alpha^3}\dfrac{u v}{x^2}$ \\
					&	\\
				\hline
				Potentiel $\psi(r)$	&	\mbox{$\psi(r) = h + m\beta\psi(0)$} \\
							&	\mbox{$ h = -\ln\(\dfrac{u v}{x^2}\) $} \\
							&	\mbox{$ \Rightarrow \psi(r) = -\ln\(\dfrac{u v}{x^2}\) + m\beta\psi(0)$} \\
				\hline
			\end{tabular}
			\caption{Lien entre les paramètres physiques et les variables de \textsc{Milne}\label{SIB:lien}}
		\end{center}
	\end{table}


	L'état d'une sphère isotherme en boîte (~diagramme de \textsc{Milne}, densité, position sur le diagramme d'énergie et contraste de densité~)
	est tracée sur la figure~\ref{Sphere}.
	La courbe de densité fait clairement apparaître un cœur (~$\rho(r) \simeq \mathrm{cte}$~) et un halo (~$\rho(r) \varpropto r^{-\alpha}$~) dont la pente est $\alpha \simeq 2$, qui est celle d'une SIS.
	\begin{figure}[ht!]
		\begin{minipage}[b]{0.40\linewidth}
			\centering \includegraphics[scale=0.60]{graphe/Milne_etat.pdf}
		\end{minipage}\hfill
		\begin{minipage}[b]{0.48\linewidth}
			\centering \includegraphics[scale=0.60]{graphe/Densite_etat.pdf}
		\end{minipage}
		\begin{minipage}[b]{0.40\linewidth}
			\centering \includegraphics[scale=0.60]{graphe/Energie_etat.pdf}
		\end{minipage}\hfill
		\begin{minipage}[b]{0.48\linewidth}
			\centering \includegraphics[scale=0.60]{graphe/Contraste_etat.pdf}
		\end{minipage}
		\caption{État de la sphère isotherme pour $X = 30$}
		\label{Sphere}
	\end{figure}

	Nous allons maintenant étudier la solution proposé par \textsc{King}.
