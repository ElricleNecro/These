	Nous avons dans le chapitre~\ref{King::Chapitre}, section~\ref{sec::temp}, le modèle de \King était un modèle
	quasi-isotherme ; il est de plus sphérique. Nous pouvons alors considérer ce modèle comme une sphère isotherme, en première
	approximation du moins. Par conséquent, il est possible de lui appliquer le problème d'Antonov. La première chose à
	faire est un rappel de ce qu'est l'instabilité d'Antonov.

\subsection{Problème/Instabilité d'Antonov}
	Le problème d'Antonov se pose lorsque l'on étudie la stabilité d'une sphère isotherme en boîte (voir
chapitre~\ref{SIB::Chapitre}) plongée dans un bain thermique.
%Ceci décale les tangentes de la spirale de stabilité.


	Cette explication de l'instabilité d'Antonov est grandement basée sur celle donnée
	dans~\cite{2008gady.book.....B}.

	Le problème que Antonov a posé est simple: supposons que le bord de la sphère isotherme en boîte soit un mur
conducteur et plaçons autour un bain thermique avec une température égale à celle de la sphère, pour commencer.
Si nous diminuons la température, nous déplaçons la sphère sur son diagramme Énergie--Température.
Par conséquent, nous changeons aussi le contraste de densité, qui est le paramètre de contrôle de la stabilité du
système.
Quand la température devient trop \og faible\fg, le contraste de densité devient trop important, et la sphère devient
instable.

	CHERCHER DES ARTICLES QUI PARLENT DE CATASTROPHE GRAVOTHERMALE (LYNDELL, BELL \& WOOD) ET D'INSTABILITE
	D'ANTONOV.

\subsection{Hypothèses simplificatrices}
	Nous avons étudié dans le chapitre~\ref{SIB::Chapitre} un modèle de sphère isotherme tronquée qui nous
	permettait une analyse de la stabilité de l'objet. Le problème est que ce modèle n'a rien d'analytique.
	Nous avons, à la fin de ce même chapitre, vu que l'objet avait un profil de densité $\rho(r)$ évoluant comme un
	cœur--halo, avec un halo ayant une pente $\alpha = 2$. L'idée de ce \og Toy-Model\fg est d'approximer le profil
	de densité comme :
	\begin{align}
		\rho(r) = \begin{cases}
			\rho_0	&	\text{si $<r_0$}\\
			\rho_0 \(\dfrac{R}{rx}\)^2	&	\text{si $r_0 < r \leq R$}\\
			0	&	\text{si $R < r$}
		\end{cases}
	\end{align}
	où $x = \frac{R}{r_0}$ est le contraste de densité, $\rho$ est la densité centrale de l'objet, $r_0$ le rayon de cœur, et $R$ le rayon de
	l'objet.

	Cette approximation permet de rendre compte des 2 tangentes discutées dans le chapitre~\ref{SIB::Chapitre}.

	Nous souhaitons appliquer ce problème à des fonctions de distributions pouvant être considérer comme des sphères
	isothermes (par exemple: la sphère de \King, en première approximation, section~\ref{sec::temp} et figure~\ref{King_Modele-test}).
	Nous allons donc approximer leur densité par:
	\begin{align}
		\rho(r) = \begin{cases} \rho_0 & \text{si $r < r_0$} \\
					\rho_0 \(\dfrac{R}{r}\)^\alpha \dfrac{1}{x^\alpha} & \text{si $R \geq r > r_0$, et
					$x=\frac{R}{r_0}$} \\
					0	&	\text{si $r > R$}
			  \end{cases}
	\end{align}
	où $\alpha$ est un entier donnant la pente du halo. Cette expression analytique nous permet de mener à bien la
	plupart des calculs, selon la valeur de $\alpha$.

	Les différentes valeurs de $\alpha$ vont permettre d'imiter la forme du profile de \King, et peuvent donc être
	rapprochées du paramètre $W_0$, de la même façon que dans~\ref{ssec::LinkBetween}.
