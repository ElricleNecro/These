Il est utile de préciser avant de commencer que tous ces calculs généraux sont valides pour $\alpha\geq 4$.
Les valeurs $\alpha = 3,\, 4$ faisant intervenir des logarithmes, il est nécessaire des les étudier à part.
\subsection{Normalisation}
	À partir de ces hypothèses, nous pouvons commencer par calculer $\rho_0$ en utilisant:
	\begin{align*}
		\int_0^R \rho(r) = M
	\end{align*}
	Un calcul rapide nous permet d'obtenir:
	\begin{align}
		\rho_0 = \frac{3 M x^{3+\alpha } (-3+\alpha )}{4 \pi  R^3 \left(-3 x^3+x^{\alpha } \alpha
\right)}
	\end{align}

\subsection{L'énergie potentielle}
	Elle se calcul bien, en utilisant:
	\begin{align}
		E_p &= -4\pi G \int_0^R r\rho(r)\mu(r) \dr \notag
		\intertext{Avec:}
		\mu(r) &= 4\pi\int_0^r r^2\rho(r)\dr \notag
	\end{align}
	Un calcul rapide nous donne alors:
	\begin{align}
%			E_p(x) = \begin{cases}
%					\dfrac{ r_0^2 M^2 }{60\pi \( \dfrac{1}{\alpha-3}\( 1-x^{3-\alpha} \) + \dfrac{1}{3} \)^2} & \text{si $r < r_0$} \\
%					\\
%					\dfrac	{M^2 \left( \frac		{\left(2 \alpha^2-8 \alpha+6 \right) x^{\alpha}}
%										{\left(2 \alpha^3-9 \alpha^2+10 \alpha\right) x^{\alpha}+\left(-6 \alpha^2+27 \alpha-30\right) x^3}
%								-    \frac	{\left(\left(2 \alpha^2-5 \alpha\right) x^{\alpha+2}+\left(6-3 \alpha\right) x^5\right)}
%										{\left(-6 \alpha^2+27 \alpha-30\right) x^{\alpha+3}+ \left(2 \alpha^3-9 \alpha^2+10 \alpha\right) x^{2 \alpha}}
%							      \right)}
%						{4 \pi r_0\left(\frac{1}{\alpha-3} \left(1 - x^{3-\alpha}\right)+\frac{1}{3}  \right)} & \text{si $r_0<r$}
%			\end{cases}
		E_p(x) &= \frac{3 G M^2 x (-3+\alpha ) \left(-15 x^5 (-2+\alpha )+5 x^{2+\alpha } \alpha  (-5+2 \alpha
)-x^{2 \alpha } (-3+\alpha ) \alpha  (1+2 \alpha )\right)}{5 R (-2+\alpha ) (-5+2 \alpha ) \left(-3
x^3+x^{\alpha } \alpha \right)^2}
	\end{align}


\subsection{Pression}
	Nous allons avoir besoin de la pression pour étudier le problème d'Antonov. Selon les conditions que nous souhaitons mettre sur le bord de la sphère, nous
	avons deux manière de calculer. Nous allons présenter les deux méthodes, mais les calculs ne
	seront donné que pour une méthode en particulier.
	%: celle qui s'applique à notre problème: un bain thermique, et donc un extérieur isotherme.

	\paragraph{Cas \og général\fg:} Dans le cas \og général\fg, nous appliquons
		l'équation de l'hydrostatique:
		\begin{align}
			\deriv{P}{r} = -\rho(r) \deriv{\psi}{r}
		\end{align}
		Le potentiel étant obtenu par résolution de l'équation de Poisson:
		\begin{align}
			\Delta\psi(r) = 4\pi G \rho(r)
		\end{align}
		En appliquant ces équations à notre problème simplifié, nous obtenons:
		\begin{align}
			P(R) &= \frac{3 G M^2 R^{-4-\alpha } x^3 (-3+\alpha ) \left(2 (R x)^{\alpha }
(-1+\alpha ) \alpha +R^{\alpha } x \left(-3 x^2 (1+\alpha )-x^{2 \alpha } (-3+\alpha ) \alpha  (2+\alpha
)\right)\right)}{8 \pi  \left(-3 x^3+x^{\alpha } \alpha \right)^2 \left(-1+\alpha ^2\right)}
		\end{align}
	\paragraph{Cas isotherme:} dans ce cas ci, l'équation de l'hydrostatique se
		ramène à l'équation d'un polytrope:
		\begin{align}
			P(r) &= \dfrac{\rho(r)}{m\beta}\\
			P(R) &= \frac{3 M \left(\frac{1}{x}\right)^{\alpha } x^{3+\alpha } (-3+\alpha )}{4 m \pi
R^3 \left(-3 x^3+x^{\alpha } \alpha \right) \beta }
		\end{align}

	\paragraph{Problème de définition:}
		Il nous faut maintenant définir de quelle pression nous parlons. En effet, dans un cas
		d'équilibre où les deux objets sont à la même température, le problème ne se pose pas: la
		continuité en pression impose qu'elle soit les mêmes au bord de la sphère. D'après la
		démonstration du théorème du Viriel pour une sphère (voir l'annexe~\ref{Demo::Viriel}, page~\pageref{Demo::Viriel}).

\subsection{Énergie cinétique}
	Ici encore, il y a deux méthodes de calculs, et ici encore, une seule
	sera décrite.
	\paragraph{Cas non-isotherme:} Dans ce cas, les calculs peuvent être très compliqué. En
		effet, ils nécessitent de connaître la fonction de distribution. Si elle n'est pas
		connue, il est possible de l'avoir en utilisant la formule d'Abel:
		\begin{align*}
			f(E) =
			\dfrac{1}{2\sqrt{2}\pi^2m^{7/2}}\(\int_0^E\deriv{\rho}{\phi}\dfrac{\mathrm{d\phi}}{\sqrt{E -
			m\phi}} + \dfrac{1}{\sqrt{E}} \left.\deriv{\rho}{\phi}\right\vert_{\phi=0}\)
		\end{align*}
		avec $\phi$ le potentiel et $\rho$ la densité.
		Ensuite, l'énergie cinétique se calcul alors simplement, pour peu que toutes les
		fonctions présentes soient analytique:
		\begin{align}
			E_c = \int_0^R\(\dfrac{1}{\rho(r)}\int \dfrac{p^2}{2m}f(E)\vdp\)\vdr
		\end{align}
	\paragraph{Cas isotherme:} Avec cette hypothèse, le calcul est \og on ne peut plus simple\fg:
		\begin{align}
			E_c = \dfrac{3M}{2m\beta}
		\end{align}

\subsection{Spirale de stabilité}
	Nous avons maintenant tous les ingrédients nécessaire, il va être nécessaire de les adimensionner. Nous allons utiliser
	les mêmes changements que pour la SIB:
	\begin{align}
		\mu     &= \dfrac{GMm\beta}{r_0x} \\
		\lambda &= \dfrac{HR}{GM^2}
	\end{align}
	où $H = E_c + E_p$ est l'énergie totale du système.

	Notre but étant d'obtenir ces deux quantités en fonction de $x$. L'énergie totale s'obtient facilement, en
	adimensionnant, mais nous auront besoin de la température. Pour la température, les calculs sont plus lourd: nous devons
	utiliser le théorème du Viriel adapté aux sphères. Ce qui nous donne l'équation suivante à résoudre:
	\begin{align}
		2E_c + E_p = \dfrac{4}{3}\pi R^3P_e
	\end{align}
	avec $P_e = P(R)$.

	Une fois ces deux quantités obtenues, il ne reste plus qu'à les tracer dans le plan $\(\lambda, \mu\)$.

	Ce qui nous amène à des courbes comme~\ref{ToyModel::AllAlpha}.
	\begin{figure}
		\centering \includegraphics[width=1\linewidth]{theorie/graphe/ToyModel-alpha_spirale.pdf}
		\caption{Diagramme Énergie-Température pour $\alpha=4, 5, 6, 7, 8$\label{ToyModel::AllAlpha}}
	\end{figure}

\subsection{Recherche des instabilité} % (fold)
	\label{sub:Recherche des instabilité}
	La recherche des instabilité se fait en recherchant les tangentes verticale et horizontal (si elles existent).
	Nous recherchons alors les points où :
	\begin{align}
		\deriv{\mu}{\lambda} &= 0
		\intertext{Soit :}
		\dfrac{\deriv{\mu(x)}{x}}{\deriv{\lambda(x)}{x}} &= 0
		\intertext{Donc :}
		\deriv{\mu(x)}{x} &= 0
	\end{align}
	Si il y a plusieurs solutions, il faut un moyen de les discriminer.
	En observant bien les courbes~\ref{ToyModel::AllAlpha}, nous nous rendons compte facilement que la tangente
	horizontale correspond toujours au maximum de la fonction $\mu(x)$. Nous devons donc chercher la valeur de $x$
	qui annule $\deriv{\mu(x)}{x}$ mais qui maximise $\mu(x)$.
% subsection Recherche des instabilité (end)
